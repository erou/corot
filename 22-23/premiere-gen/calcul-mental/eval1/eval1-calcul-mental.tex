%%%%%%%%%%%%%%%%%%%%%%%%%%%%%%%%%%%%%%%%%%%%%%%%%%%%%%%%%%%%
%%  This Beamer template was created by Cameron Bracken.
%%  Anyone can freely use or modify it for any purpose
%%  without attribution.
%%
%%  Last Modified: January 9, 2009
%%

%\documentclass[20pt,xcolor=x11names,compress, aspectratio=43]{beamer}
\documentclass[17pt,xcolor=x11names]{beamer}
% change to aspectratio=169 to obtain 16:9 ratio (standard on many computers)

\usepackage[utf8]{inputenc}
\usepackage[T1]{fontenc}
\usepackage[french]{babel}

%\usepackage[utf8]{inputenc}
\usepackage[T1]{fontenc}
\usepackage[english]{babel}
\usepackage{graphicx}
\usepackage{amsmath,amssymb,amsthm,amsopn}
\usepackage{mathrsfs}
\usepackage{graphicx}
\usepackage{array}
\usepackage{makecell}
\usepackage{bm}
\usepackage{hyperref}
\hypersetup{
    colorlinks=true,
    linkcolor=blue,
    citecolor=red,
}
\usepackage{smartdiagram}
\usepackage{algorithm}
\usepackage{algorithmic}
\usepackage{animate}
\usepackage{xmpmulti}

%\usepackage[top=1cm,bottom=1cm]{geometry}
%\usepackage{listings}
%\usepackage{xcolor}

\usepackage{wasysym}

\usepackage{fancyvrb}

%\usepackage{tikz}
\usetikzlibrary{arrows}
\usetikzlibrary{math}
\usetikzlibrary{calc}
\usetikzlibrary{decorations.text}

% Tikz style

\tikzset{round/.style={circle, draw=black, very thick, scale = 0.7}}
\tikzset{arrow/.style={->, >=latex}}
\tikzset{possible-arrow/.style={->, >=latex, color=gray!35, thick}}
\tikzset{dashed-arrow/.style={->, >=latex, dashed}}
\tikzset{dotstyle/.style={circle, inner sep = 1.2pt, outer sep = 4pt, fill = gray},
         edgetower/.style={thick},
         edgecomp/.style={thick, lightgray}}
\tikzset{additive-structure/.style={very thick, red!50}}
\tikzset{multiplicative-structure/.style={very thick, blue!50}}
\tikzset{curved text/.style={decorate,
        decoration={text effects along path,
            text={#1}, text align=center,
            text effects/.cd, text along path}}}

% New command
\newcommand{\drawrec}[3]{
  \ifthenelse{\equal{#1}{1}}{
    \fill (#2,#3) -- ++(1,0) -- ++(0, 1) -- ++(-1, 0) -- cycle;
    }
    {
      \tikzmath{\x = #2; \y = #3; \a = #1;
                \b =\a/2; \xx = \x+\b; \yy = \y+\b; }

      \drawrec{\b}{\x}{\y}
      \drawrec{\b}{\xx}{\y}
      \drawrec{\b}{\x}{\yy}
    }
}

%\newtheoremstyle{break}%
{}{}%
{\itshape}{}%
{\bfseries}{}%  % Note that final punctuation is omitted.
{\newline}{}

\newtheoremstyle{sc}%
{}{}%
{}{}%
{\scshape}{}%  % Note that final punctuation is omitted.
{\newline}{}

\theoremstyle{break}
\newtheorem{thm}{Theorem}[section]
\newtheorem{lm}[thm]{Lemma}
\newtheorem{prop}[thm]{Proposition}
\newtheorem{cor}[thm]{Corollary}

\theoremstyle{sc}
\newtheorem{exo}{Exercice}

\theoremstyle{definition}
\newtheorem{defi}[thm]{Definition}
\newtheorem{ex}[thm]{Example}

\theoremstyle{remark}
\newtheorem{rem}[thm]{Remark}

%% Math Operators

\DeclareMathOperator{\Card}{Card}
\DeclareMathOperator{\Gal}{Gal}
\DeclareMathOperator{\Id}{Id}
\DeclareMathOperator{\Img}{Im}
\DeclareMathOperator{\Ker}{Ker}
\DeclareMathOperator{\minpoly}{minpoly}
\DeclareMathOperator{\Mod}{mod}
\DeclareMathOperator{\Ord}{Ord}
\DeclareMathOperator{\ppcm}{ppcm}
\DeclareMathOperator{\lcm}{lcm}
\DeclareMathOperator{\Tr}{Tr}
\DeclareMathOperator{\Vect}{Vect}

% Shortcuts

\newcommand{\dE}{\partial(E)}
\newcommand{\dF}{\partial(F)}
\newcommand{\dG}{\partial(G)}
\newcommand{\diff}{\mathop{}\!\mathrm{d}}
\newcommand{\eg}{\emph{e.g. }}
\newcommand{\emb}{\hookrightarrow}
\newcommand{\embed}[2]{\phi_{#1\hookrightarrow#2}}
\newcommand{\ent}[2]{[\![#1,#2]\!]}
\newcommand{\ie}{\emph{i.e. }}
\newcommand{\ps}[2]{\left\langle#1,#2\right\rangle}
\newcommand{\eqdef}{\overset{\text{def}}{=}}
\newcommand{\first}[2]{\left\lfloor #1 \right\rfloor_{#2}}
\newcommand{\embedalg}[2]{\Phi_{A_{#1}\hookrightarrow A_{#2}}}
\newcommand{\A}{\mathcal{A}}
\newcommand{\G}{\mathcal{G}}
\newcommand{\Gsym}{\G_{\text{sym}}}
\newcommand{\B}{\mathrm{B}}

\newcommand{\mutri}{\mu^{\textrm{tri}}}

\def\norm {\ensuremath{\mathcal{N}}}

% trick to make comments
\newcommand{\comment}[1]{}

\useoutertheme[subsection=false,shadow]{miniframes}
\useinnertheme{default}
\usefonttheme{serif}
\usepackage{palatino}

\setbeamerfont{title like}{shape=\scshape}
\setbeamerfont{frametitle}{shape=\scshape}
\setbeamertemplate{navigation symbols}{}
\setbeamertemplate{footline}[frame number]

\setbeamercolor*{lower separation line head}{bg=blue!80!white} 
\setbeamercolor*{page number in head/foot}{fg=gray} 

% \setbeamercolor*{normal text}{fg=black,bg=white} 
% \setbeamercolor*{alerted text}{fg=red} 
% \setbeamercolor*{example text}{fg=black} 
% \setbeamercolor*{structure}{fg=black} 
  
\setbeamercolor*{palette tertiary}{fg=black,bg=black!10} 
%\setbeamercolor*{palette quaternary}{fg=black,bg=black!10} 

\definecolor{mygreen}{rgb}{0.20,0.43,0.09}
\newcommand{\bib}[2]{\textcolor{blue}{[#1 '#2]}}
\newcommand{\fvb}[1]{\textcolor{violet}{\textbf{#1}}}
\newcommand{\frb}[1]{\textcolor{red}{\mathbf{#1}}}
\newcommand{\comp}[1]{\textcolor{purple}{$O(#1)$}}
\newcommand{\softcomp}[1]{\textcolor{purple}{$\tilde O(#1)$}}
\newcommand{\good}{\textcolor{mygreen}{\smiley{}}}
\newcommand{\bad}{\textcolor{red}{\frownie{}}}
\newcommand{\openquestion}{\includegraphics[scale=.03]{../logos/open-question.png}\fvb{Open
question: }}

% In order to have a slide dedicated to announcing the section
%\AtBeginSection[]
%{
%    \begin{frame}
%        \frametitle{Table of Contents}
%        \tableofcontents[currentsection]
%    \end{frame}
%}

\AtBeginSection[]{
  \begin{frame}
  \vfill
  \centering
  \begin{beamercolorbox}[sep=8pt,center,shadow=false,rounded=true]{title}
    \usebeamerfont{title}\secname\par%
  \end{beamercolorbox}
  \vfill
  \end{frame}
}

% So that alerted text is bold
% \setbeamerfont{alerted text}{series=\bfseries}

\usepackage{multicol}
\usepackage{enumerate}
\usepackage{enumitem}


\begin{document}
\begin{frame}
  \title{Évaluation 1}
  \author{Calcul mental}
\date{17 Novembre}
\titlepage
\end{frame}

\section{Calcul mental 1}

\begin{frame}
  \begin{center}
    Calculer
    \[
      \frac{2}{3}+\frac{1}{5}
    \]
  \end{center}~\\
  \begin{multicols}{2}
    \begin{enumerate}[label=(\Alph*)]
      \item $\frac{11}{15}$
      \item $\frac{17}{15}$
      \item $\frac{13}{15}$
      \item $\frac{9}{15}$
    \end{enumerate}
  \end{multicols}
\end{frame}

\begin{frame}
  \begin{center}
    Calculer
    \[
      (\sqrt2-1)^2
    \]
  \end{center}~\\
  \begin{multicols}{2}
    \begin{enumerate}[label=(\Alph*)]
      \item $3-2\sqrt2$
      \item $3+2\sqrt2$
      \item $1+2\sqrt2$
      \item $1-2\sqrt2$
    \end{enumerate}
  \end{multicols}
\end{frame}

\begin{frame}
  \begin{center}
    Développer
    \[
      (2-3x)^2
    \]
  \end{center}~\\
  \begin{multicols}{2}
    \begin{enumerate}[label=(\Alph*)]
      \item $4-12x+9x^2$
      \item $4-6x+9x^2$
      \item $12+6x+9x^2$
      \item $4-2x-9x^2$
    \end{enumerate}
  \end{multicols}
\end{frame}

\begin{frame}
  \begin{center}
    Soit $f$ la fonction définie sur $\mathbb{R}$ par
    \[
      f(x)=-x^2+2x-1.
    \]
    Combien vaut $f(3)$ ?
  \end{center}~\\
  \begin{multicols}{2}
    \begin{enumerate}[label=(\Alph*)]
      \item $14$
      \item $-2$
      \item $8$
      \item $-4$
    \end{enumerate}
  \end{multicols}
\end{frame}

\begin{frame}
  \begin{center}
    Factoriser
    \[
      4x^2+4x+1.
    \]
  \end{center}~\\
  \begin{multicols}{2}
    \begin{enumerate}[label=(\Alph*)]
      \item $(x+2)^2$
      \item $(2x+1)^2$
      \item $2(x+1)^2$
      \item $(\frac{x}{2}+2)^2$
    \end{enumerate}
  \end{multicols}
\end{frame}

\begin{frame}
  \begin{center}
    Simplifier 
    \[
      \sqrt{72}
    \]
  \end{center}~\\
  \begin{multicols}{2}
    \begin{enumerate}[label=(\Alph*)]
      \item $4\sqrt{18}$
      \item $2\sqrt6$
      \item $6\sqrt2$
      \item $12\sqrt{6}$
    \end{enumerate}
  \end{multicols}
\end{frame}

\begin{frame}
  \begin{center}
    Calculer
    \[
      30\text{\% de }570
    \]
  \end{center}~\\
  \begin{multicols}{2}
    \begin{enumerate}[label=(\Alph*)]
      \item $151$
      \item $147$
      \item $191$
      \item $171$
    \end{enumerate}
  \end{multicols}
\end{frame}

\begin{frame}
  \begin{center}
    Dériver
    \[
      f(x) = -3x^3 + 5x^2 - 10
    \]
  \end{center}~\\
  \begin{multicols}{2}
    \begin{enumerate}[label=(\Alph*)]
      \item $9x^2+5x$
      \item $-3x^2+10x^2-10$
      \item $-9x^2+10x$
      \item $3x^2+10$
    \end{enumerate}
  \end{multicols}
\end{frame}

\begin{frame}
  \begin{center}
    Calculer
    \[
     \left( 0,9 \right)^2 
    \]
  \end{center}~\\
  \begin{multicols}{2}
    \begin{enumerate}[label=(\Alph*)]
      \item $0,18$
      \item $1,8$
      \item $0,81$
      \item $8,1$
    \end{enumerate}
  \end{multicols}
\end{frame}

\begin{frame}
  \begin{center}
    Calculer
    \[
      3-\left( \frac{1}{3}-1 \right)
    \]
  \end{center}~\\
  \begin{multicols}{2}
    \begin{enumerate}[label=(\Alph*)]
      \item $\frac{7}{3}$
      \item $\frac{5}{3}$
      \item $\frac{11}{3}$
      \item $\frac{8}{3}$
    \end{enumerate}
  \end{multicols}
\end{frame}

\end{document}
