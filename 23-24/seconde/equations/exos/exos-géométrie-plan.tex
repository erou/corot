\documentclass[11pt]{article}

\newcommand{\titrechapitre}{Géométrie dans le plan -- Exercices}
\newcommand{\titreclasse}{Lycée Jean-Baptiste \textsc{Corot}}
\newcommand{\pagination}{\thepage/\pageref{LastPage}}
\newcommand{\topbotmargins}{2cm}%\thepage/\pageref{LastPage}}
\newcommand{\spacebelowexo}{3mm}%\thepage/\pageref{LastPage}}

%%%%%%%%%%%%%%%%%%%%%%%%%%%%%%%%%%%%%%%%%%%%%%%%%%%%%%%%%%%%%%%%%%%%%%%%%%%%%%%%
%
% PACKAGES
% ========
%
%%%%%%%%%%%%%%%%%%%%%%%%%%%%%%%%%%%%%%%%%%%%%%%%%%%%%%%%%%%%%%%%%%%%%%%%%%%%%%%%

\usepackage[english, french]{babel}
\usepackage[utf8]{inputenc}
\usepackage[T1]{fontenc}
\usepackage{graphicx}
\usepackage{amsmath,amssymb,amsthm,amsopn}
\usepackage{hyperref}

% Pour avoir l'écriture \mathscr (math script)
% ============================================

\usepackage{mathrsfs}

% Deal with coma as a decimal separator
% =====================================

\usepackage{icomma}

% Package Geometry
% ================

\usepackage[a4paper, lmargin=2cm, rmargin=2cm, top=\topbotmargins, bottom=\topbotmargins]{geometry}

% Package multicol
% ================

\usepackage{multicol}

% Redefine abstract
% =================

% Note
% ====
%
% Le reste a été commenté pour ne pas charger trop de choses au démarrage. On
% verra si on en a besoin plus tard.
%
% --------
%
%\usepackage{mathrsfs}
%\usepackage{multirow}
%\usepackage{bm}
%\hypersetup{
%    colorlinks=true,
%    linkcolor=blue,
%    citecolor=red,
%}
%\usepackage{diagbox}
%
%\usepackage{algorithm}
%\usepackage{algpseudocode}
%
%\renewcommand{\algorithmicrequire}{\textbf{Input:}}
%\renewcommand{\algorithmicensure}{\textbf{Output:}}


%%%%%%%%%%%%%%%%%%%%%%%%%%%%%%%%%%%%%%%%%%%%%%%%%%%%%%%%%%%%%%%%%%%%%%%%%%%%%%%%
%
% TIKZ
% ====
%
%%%%%%%%%%%%%%%%%%%%%%%%%%%%%%%%%%%%%%%%%%%%%%%%%%%%%%%%%%%%%%%%%%%%%%%%%%%%%%%%

\usepackage{tikz}
\usetikzlibrary{arrows}

\usepackage{tkz-tab} % Variation tables

\usepackage{pgfplots}
%\usepackage{pgf-pie} % Pie charts

\pgfplotsset{
%\newcommand{\settingsgraph}{
x=.5cm,y=.5cm,
xticklabel style = {font=\scriptsize, yshift=.1cm},
yticklabel style = {font=\scriptsize, xshift=.1cm},
axis lines=middle,
ymajorgrids=true,
xmajorgrids=true,
major grid style = {color=white!80!blue},
xmin=-5.5,
xmax=5.5,
ymin=-5.5,
ymax=5.5,
xtick={-5.0,-4.0,...,5.0},
ytick={-5.0,-4.0,...,5.0},
}

% Tikz style

\tikzset{round/.style={circle, draw=black, very thick, scale = 0.7}}
\tikzset{arrow/.style={->, >=latex}}
\tikzset{dashed-arrow/.style={->, >=latex, dashed}}

\newcommand{\point}[3]{\draw[very thick, #3] (#1-.1, #2)--(#1+.1, #2)
(#1, #2-.1)--(#1, #2+.1)}

%%%%%%%%%%%%%%%%%%%%%%%%%%%%%%%%%%%%%%%%%%%%%%%%%%%%%%%%%%%%%%%%%%%%%%%%%%%%%%%%
%
% FANCY HEADER
% ============
%
%%%%%%%%%%%%%%%%%%%%%%%%%%%%%%%%%%%%%%%%%%%%%%%%%%%%%%%%%%%%%%%%%%%%%%%%%%%%%%%%


\usepackage{fancyhdr}
\usepackage{lastpage}

\pagestyle{fancy}
\newcommand{\changefont}{\fontsize{9}{9}\selectfont}
\renewcommand{\headrulewidth}{0mm}
\renewcommand{\footrulewidth}{0mm}

\fancyhead[C]{}
\fancyhead[L]{\titreclasse}
\fancyhead[R]{\titrechapitre}
\fancyfoot[C]{}
\fancyfoot[L]{}
\fancyfoot[R]{\pagination}
\addtolength{\skip\footins}{20pt} % distance between text and footnotes

%%%%%%%%%%%%%%%%%%%%%%%%%%%%%%%%%%%%%%%%%%%%%%%%%%%%%%%%%%%%%%%%%%%%%%%%%%%%%%%%
%
% THEOREM STYLE
% =============
%
%%%%%%%%%%%%%%%%%%%%%%%%%%%%%%%%%%%%%%%%%%%%%%%%%%%%%%%%%%%%%%%%%%%%%%%%%%%%%%%%

\usepackage[tikz]{bclogo}
\usepackage{mdframed}

\usepackage{tcolorbox}
\tcbuselibrary{listings, breakable, theorems, skins}

%\newtheoremstyle{break}%
%{}{}%
%{\itshape}{}%
%{\bfseries}{}%  % Note that final punctuation is omitted.
%{\newline}{}

\newtheoremstyle{scbf}%
{}{}%
{}{}%
%{\scshape}{}%  % Note that final punctuation is omitted.
{\bfseries\scshape}{}%  % Note that final punctuation is omitted.
{\newline}{}

%\theoremstyle{break}
%\theoremstyle{plain}
%\newtheorem{thm}{Theorem}[section]
%\newtheorem{lm}[thm]{Lemma}
%\newtheorem{prop}[thm]{Proposition}
%\newtheorem{cor}[thm]{Corollary}

%\theoremstyle{scbf}
%\newtheorem{exo}{$\star$ Exercice}

%\theoremstyle{definition}
%\newtheorem{defi}[thm]{Definition}
%\newtheorem{ex}[thm]{Example}

%\theoremstyle{remark}
%\newtheorem{rem}[thm]{Remark}

% Defining the Remark environment
% ===============================

\newenvironment{rmq}
  {
    \begin{bclogo}[logo=\bcinfo, noborder=true]{Remarque}
  }
  {
    \end{bclogo}
  }

% Defining the exercise environment
% =================================

\newcounter{exos}
\setcounter{exos}{1}

\newenvironment{exo}
  {
    \begin{bclogo}[logo=\bccrayon, noborder=true]{Exercice \theexos}
  }
  {
    \end{bclogo}
    \addtocounter{exos}{1}
  }


% Redefining the proof environment from amsthm
% ============================================

\tcolorboxenvironment{proof}{
  blanker, breakable, before skip=10pt,after skip=10pt,
  borderline west={1mm}{0pt}{red},
  left=5mm,
}

% Defining the definition environment
% ===================================

\colorlet{coldef}{black!50!green}

\newcounter{defis}
\setcounter{defis}{1}

\newenvironment{defi}[1]
  {
    \begin{defihid}{{#1}}{\thedefis}
  }
  {
    \end{defihid}
    \addtocounter{defis}{1}
  }

\newtcolorbox{defihid}[2]{%
  empty,title={ {\bfseries Définition {#2}} ({#1})},attach boxed title to top left,
boxed title style={empty,size=minimal,toprule=2pt,top=4pt,
overlay={\draw[coldef,line width=2pt]
([yshift=-1pt]frame.north west)--([yshift=-1pt]frame.north east);}},
coltitle=coldef,
before=\par\medskip\noindent,parbox=false,boxsep=0pt,left=0pt,right=3mm,top=4pt,
breakable,pad at break*=0mm,vfill before first,
overlay unbroken={\draw[coldef,line width=1pt]
([yshift=-1pt]title.north east)--([xshift=-0.5pt,yshift=-1pt]title.north-|frame.east)
--([xshift=-0.5pt]frame.south east)--(frame.south west); },
overlay first={\draw[coldef,line width=1pt]
([yshift=-1pt]title.north east)--([xshift=-0.5pt,yshift=-1pt]title.north-|frame.east)
--([xshift=-0.5pt]frame.south east); },
overlay middle={\draw[coldef,line width=1pt] ([xshift=-0.5pt]frame.north east)
--([xshift=-0.5pt]frame.south east); },
overlay last={\draw[coldef,line width=1pt] ([xshift=-0.5pt]frame.north east)
--([xshift=-0.5pt]frame.south east)--(frame.south west);},%
}

\newenvironment{notation}
  {
    \begin{notationhid}{\thedefis}
  }
  {
    \end{notationhid}
    \addtocounter{defis}{1}
  }

\newtcolorbox{notationhid}[1]{%
  empty,title={Notation {#1}},attach boxed title to top left,
boxed title style={empty,size=minimal,toprule=2pt,top=4pt,
overlay={\draw[coldef,line width=2pt]
([yshift=-1pt]frame.north west)--([yshift=-1pt]frame.north east);}},
coltitle=coldef,fonttitle=\bfseries,
before=\par\medskip\noindent,parbox=false,boxsep=0pt,left=0pt,right=3mm,top=4pt,
breakable,pad at break*=0mm,vfill before first,
overlay unbroken={\draw[coldef,line width=1pt]
([yshift=-1pt]title.north east)--([xshift=-0.5pt,yshift=-1pt]title.north-|frame.east)
--([xshift=-0.5pt]frame.south east)--(frame.south west); },
overlay first={\draw[coldef,line width=1pt]
([yshift=-1pt]title.north east)--([xshift=-0.5pt,yshift=-1pt]title.north-|frame.east)
--([xshift=-0.5pt]frame.south east); },
overlay middle={\draw[coldef,line width=1pt] ([xshift=-0.5pt]frame.north east)
--([xshift=-0.5pt]frame.south east); },
overlay last={\draw[coldef,line width=1pt] ([xshift=-0.5pt]frame.north east)
--([xshift=-0.5pt]frame.south east)--(frame.south west);},%
}


% Defining the proposition, theorem, etc. environment
% ===================================================

\colorlet{colprop}{red!75!black}

\newcounter{props}
\setcounter{props}{1}

\newenvironment{prop}
  {
    \begin{prophid}{\theprops}
  }
  {
    \end{prophid}
    \refstepcounter{props}
  }

\newtcolorbox{prophid}[1]{%
empty,title={Propriété {#1}},attach boxed title to top left,
boxed title style={empty,size=minimal,toprule=2pt,top=4pt,
overlay={\draw[colprop,line width=2pt]
([yshift=-1pt]frame.north west)--([yshift=-1pt]frame.north east);}},
coltitle=colprop,fonttitle=\bfseries,
before=\par\medskip\noindent,parbox=false,boxsep=0pt,left=0pt,right=3mm,top=4pt,
breakable,pad at break*=0mm,vfill before first,
overlay unbroken={\draw[colprop,line width=1pt]
([yshift=-1pt]title.north east)--([xshift=-0.5pt,yshift=-1pt]title.north-|frame.east)
--([xshift=-0.5pt]frame.south east)--(frame.south west); },
overlay first={\draw[colprop,line width=1pt]
([yshift=-1pt]title.north east)--([xshift=-0.5pt,yshift=-1pt]title.north-|frame.east)
--([xshift=-0.5pt]frame.south east); },
overlay middle={\draw[colprop,line width=1pt] ([xshift=-0.5pt]frame.north east)
--([xshift=-0.5pt]frame.south east); },
overlay last={\draw[colprop,line width=1pt] ([xshift=-0.5pt]frame.north east)
--([xshift=-0.5pt]frame.south east)--(frame.south west);},%
}

\newenvironment{propadm}
  {
    \begin{propadmhid}{\theprops}
  }
  {
    \end{propadmhid}
    \refstepcounter{props}
  }

  \newtcolorbox{propadmhid}[1]{%
    empty,title={{\bfseries Propriété {#1}} (admise)},attach boxed title to top left,
boxed title style={empty,size=minimal,toprule=2pt,top=4pt,
overlay={\draw[colprop,line width=2pt]
([yshift=-1pt]frame.north west)--([yshift=-1pt]frame.north east);}},
coltitle=colprop,%fonttitle=\bfseries,
before=\par\medskip\noindent,parbox=false,boxsep=0pt,left=0pt,right=3mm,top=4pt,
breakable,pad at break*=0mm,vfill before first,
overlay unbroken={\draw[colprop,line width=1pt]
([yshift=-1pt]title.north east)--([xshift=-0.5pt,yshift=-1pt]title.north-|frame.east)
--([xshift=-0.5pt]frame.south east)--(frame.south west); },
overlay first={\draw[colprop,line width=1pt]
([yshift=-1pt]title.north east)--([xshift=-0.5pt,yshift=-1pt]title.north-|frame.east)
--([xshift=-0.5pt]frame.south east); },
overlay middle={\draw[colprop,line width=1pt] ([xshift=-0.5pt]frame.north east)
--([xshift=-0.5pt]frame.south east); },
overlay last={\draw[colprop,line width=1pt] ([xshift=-0.5pt]frame.north east)
--([xshift=-0.5pt]frame.south east)--(frame.south west);},%
}

\newenvironment{propnom}[1]
  {
    \begin{propnomhid}{#1}{\theprops}
  }
  {
    \end{propnomhid}
    \refstepcounter{props}
  }

\newtcolorbox{propnomhid}[2]{%
empty,title={{\bfseries Propriété {#2}} ({#1})},attach boxed title to top left,
boxed title style={empty,size=minimal,toprule=2pt,top=4pt,
overlay={\draw[colprop,line width=2pt]
([yshift=-1pt]frame.north west)--([yshift=-1pt]frame.north east);}},
coltitle=colprop,
before=\par\medskip\noindent,parbox=false,boxsep=0pt,left=0pt,right=3mm,top=4pt,
breakable,pad at break*=0mm,vfill before first,
overlay unbroken={\draw[colprop,line width=1pt]
([yshift=-1pt]title.north east)--([xshift=-0.5pt,yshift=-1pt]title.north-|frame.east)
--([xshift=-0.5pt]frame.south east)--(frame.south west); },
overlay first={\draw[colprop,line width=1pt]
([yshift=-1pt]title.north east)--([xshift=-0.5pt,yshift=-1pt]title.north-|frame.east)
--([xshift=-0.5pt]frame.south east); },
overlay middle={\draw[colprop,line width=1pt] ([xshift=-0.5pt]frame.north east)
--([xshift=-0.5pt]frame.south east); },
overlay last={\draw[colprop,line width=1pt] ([xshift=-0.5pt]frame.north east)
--([xshift=-0.5pt]frame.south east)--(frame.south west);},%
}




\newenvironment{thm}
  {
    \begin{thmhid}{\theprops}
  }
  {
    \end{thmhid}
    \refstepcounter{props}
  }

\newtcolorbox{thmhid}[1]{%
empty,title={Théorème {#1}},attach boxed title to top left,
boxed title style={empty,size=minimal,toprule=2pt,top=4pt,
overlay={\draw[colprop,line width=2pt]
([yshift=-1pt]frame.north west)--([yshift=-1pt]frame.north east);}},
coltitle=colprop,fonttitle=\bfseries,
before=\par\medskip\noindent,parbox=false,boxsep=0pt,left=0pt,right=3mm,top=4pt,
breakable,pad at break*=0mm,vfill before first,
overlay unbroken={\draw[colprop,line width=1pt]
([yshift=-1pt]title.north east)--([xshift=-0.5pt,yshift=-1pt]title.north-|frame.east)
--([xshift=-0.5pt]frame.south east)--(frame.south west); },
overlay first={\draw[colprop,line width=1pt]
([yshift=-1pt]title.north east)--([xshift=-0.5pt,yshift=-1pt]title.north-|frame.east)
--([xshift=-0.5pt]frame.south east); },
overlay middle={\draw[colprop,line width=1pt] ([xshift=-0.5pt]frame.north east)
--([xshift=-0.5pt]frame.south east); },
overlay last={\draw[colprop,line width=1pt] ([xshift=-0.5pt]frame.north east)
--([xshift=-0.5pt]frame.south east)--(frame.south west);},%
}

\newenvironment{thmadm}
  {
    \begin{thmadmhid}{\theprops}
  }
  {
    \end{thmadmhid}
    \refstepcounter{props}
  }

  \newtcolorbox{thmadmhid}[1]{%
    empty,title={{\bfseries Théorème {#1}} (admis)},attach boxed title to top left,
boxed title style={empty,size=minimal,toprule=2pt,top=4pt,
overlay={\draw[colprop,line width=2pt]
([yshift=-1pt]frame.north west)--([yshift=-1pt]frame.north east);}},
coltitle=colprop,%fonttitle=\bfseries,
before=\par\medskip\noindent,parbox=false,boxsep=0pt,left=0pt,right=3mm,top=4pt,
breakable,pad at break*=0mm,vfill before first,
overlay unbroken={\draw[colprop,line width=1pt]
([yshift=-1pt]title.north east)--([xshift=-0.5pt,yshift=-1pt]title.north-|frame.east)
--([xshift=-0.5pt]frame.south east)--(frame.south west); },
overlay first={\draw[colprop,line width=1pt]
([yshift=-1pt]title.north east)--([xshift=-0.5pt,yshift=-1pt]title.north-|frame.east)
--([xshift=-0.5pt]frame.south east); },
overlay middle={\draw[colprop,line width=1pt] ([xshift=-0.5pt]frame.north east)
--([xshift=-0.5pt]frame.south east); },
overlay last={\draw[colprop,line width=1pt] ([xshift=-0.5pt]frame.north east)
--([xshift=-0.5pt]frame.south east)--(frame.south west);},%
}

\newenvironment{thmnom}[1]
  {
    \begin{thmnomhid}{#1}{\theprops}
  }
  {
    \end{thmnomhid}
    \refstepcounter{props}
  }

\newtcolorbox{thmnomhid}[2]{%
empty,title={{\bfseries Théorème {#2}} ({#1})},attach boxed title to top left,
boxed title style={empty,size=minimal,toprule=2pt,top=4pt,
overlay={\draw[colprop,line width=2pt]
([yshift=-1pt]frame.north west)--([yshift=-1pt]frame.north east);}},
coltitle=colprop,
before=\par\medskip\noindent,parbox=false,boxsep=0pt,left=0pt,right=3mm,top=4pt,
breakable,pad at break*=0mm,vfill before first,
overlay unbroken={\draw[colprop,line width=1pt]
([yshift=-1pt]title.north east)--([xshift=-0.5pt,yshift=-1pt]title.north-|frame.east)
--([xshift=-0.5pt]frame.south east)--(frame.south west); },
overlay first={\draw[colprop,line width=1pt]
([yshift=-1pt]title.north east)--([xshift=-0.5pt,yshift=-1pt]title.north-|frame.east)
--([xshift=-0.5pt]frame.south east); },
overlay middle={\draw[colprop,line width=1pt] ([xshift=-0.5pt]frame.north east)
--([xshift=-0.5pt]frame.south east); },
overlay last={\draw[colprop,line width=1pt] ([xshift=-0.5pt]frame.north east)
--([xshift=-0.5pt]frame.south east)--(frame.south west);},%
}

\newenvironment{coro}
  {
    \begin{corohid}{\theprops}
  }
  {
    \end{corohid}
    \refstepcounter{props}
  }

  \newtcolorbox{corohid}[1]{%
  empty,title={Corollaire {#1}},attach boxed title to top left,
boxed title style={empty,size=minimal,toprule=2pt,top=4pt,
overlay={\draw[colprop,line width=2pt]
([yshift=-1pt]frame.north west)--([yshift=-1pt]frame.north east);}},
coltitle=colprop,fonttitle=\bfseries,
before=\par\medskip\noindent,parbox=false,boxsep=0pt,left=0pt,right=3mm,top=4pt,
breakable,pad at break*=0mm,vfill before first,
overlay unbroken={\draw[colprop,line width=1pt]
([yshift=-1pt]title.north east)--([xshift=-0.5pt,yshift=-1pt]title.north-|frame.east)
--([xshift=-0.5pt]frame.south east)--(frame.south west); },
overlay first={\draw[colprop,line width=1pt]
([yshift=-1pt]title.north east)--([xshift=-0.5pt,yshift=-1pt]title.north-|frame.east)
--([xshift=-0.5pt]frame.south east); },
overlay middle={\draw[colprop,line width=1pt] ([xshift=-0.5pt]frame.north east)
--([xshift=-0.5pt]frame.south east); },
overlay last={\draw[colprop,line width=1pt] ([xshift=-0.5pt]frame.north east)
--([xshift=-0.5pt]frame.south east)--(frame.south west);},%
}

\newenvironment{lemme}
  {
    \begin{lemmehid}{\theprops}
  }
  {
    \end{lemmehid}
    \refstepcounter{props}
  }

  \newtcolorbox{lemmehid}[1]{%
  empty,title={Lemme {#1}},attach boxed title to top left,
boxed title style={empty,size=minimal,toprule=2pt,top=4pt,
overlay={\draw[colprop,line width=2pt]
([yshift=-1pt]frame.north west)--([yshift=-1pt]frame.north east);}},
coltitle=colprop,fonttitle=\bfseries,
before=\par\medskip\noindent,parbox=false,boxsep=0pt,left=0pt,right=3mm,top=4pt,
breakable,pad at break*=0mm,vfill before first,
overlay unbroken={\draw[colprop,line width=1pt]
([yshift=-1pt]title.north east)--([xshift=-0.5pt,yshift=-1pt]title.north-|frame.east)
--([xshift=-0.5pt]frame.south east)--(frame.south west); },
overlay first={\draw[colprop,line width=1pt]
([yshift=-1pt]title.north east)--([xshift=-0.5pt,yshift=-1pt]title.north-|frame.east)
--([xshift=-0.5pt]frame.south east); },
overlay middle={\draw[colprop,line width=1pt] ([xshift=-0.5pt]frame.north east)
--([xshift=-0.5pt]frame.south east); },
overlay last={\draw[colprop,line width=1pt] ([xshift=-0.5pt]frame.north east)
--([xshift=-0.5pt]frame.south east)--(frame.south west);},%
}

\colorlet{colexemple}{blue!50!black}
%\newtcolorbox{exemple}{empty, title=Exemple, attach boxed title to top left,
%  boxed title style={empty, size=minimal, toprule=2pt, top=4pt,
%    overlay={\draw[colexemple,line width=2pt]
%([yshift=-1pt]frame.north west)--([yshift=-1pt]frame.north east);}},
%coltitle=colexemple,fonttitle=\bfseries,%\large\bfseries,
%before=\par\medskip\noindent,parbox=false,boxsep=0pt,left=0pt,right=3mm,top=4pt,
%overlay={\draw[colexemple,line width=1pt]
%([yshift=-1pt]title.north east)--([xshift=-0.5pt,yshift=-1pt]title.north-|frame.east)
%--([xshift=-0.5pt]frame.south east)--(frame.south west); },
%}

\newcounter{exemples}
\setcounter{exemples}{1}

\newenvironment{exemple}
  {
    \begin{exemplehid}{\theexemples}
  }
  {
    \end{exemplehid}
    \addtocounter{exemples}{1}
  }

\newtcolorbox{exemplehid}[1]{%
empty,title={Exemple {#1}},attach boxed title to top left,
boxed title style={empty,size=minimal,toprule=2pt,top=4pt,
overlay={\draw[colexemple,line width=2pt]
([yshift=-1pt]frame.north west)--([yshift=-1pt]frame.north east);}},
coltitle=colexemple,fonttitle=\bfseries,
before=\par\medskip\noindent,parbox=false,boxsep=0pt,left=0pt,right=3mm,top=4pt,
breakable,pad at break*=0mm,vfill before first,
overlay unbroken={\draw[colexemple,line width=1pt]
([yshift=-1pt]title.north east)--([xshift=-0.5pt,yshift=-1pt]title.north-|frame.east)
--([xshift=-0.5pt]frame.south east)--(frame.south west); },
overlay first={\draw[colexemple,line width=1pt]
([yshift=-1pt]title.north east)--([xshift=-0.5pt,yshift=-1pt]title.north-|frame.east)
--([xshift=-0.5pt]frame.south east); },
overlay middle={\draw[colexemple,line width=1pt] ([xshift=-0.5pt]frame.north east)
--([xshift=-0.5pt]frame.south east); },
overlay last={\draw[colexemple,line width=1pt] ([xshift=-0.5pt]frame.north east)
--([xshift=-0.5pt]frame.south east)--(frame.south west);},%
}

\newenvironment{contrex}
  {
    \begin{contrexhid}{\theexemples}
  }
  {
    \end{contrexhid}
    \addtocounter{exemples}{1}
  }

\newtcolorbox{contrexhid}[1]{%
empty,title={Contre-exemple {#1}},attach boxed title to top left,
boxed title style={empty,size=minimal,toprule=2pt,top=4pt,
overlay={\draw[colexemple,line width=2pt]
([yshift=-1pt]frame.north west)--([yshift=-1pt]frame.north east);}},
coltitle=colexemple,fonttitle=\bfseries,
before=\par\medskip\noindent,parbox=false,boxsep=0pt,left=0pt,right=3mm,top=4pt,
breakable,pad at break*=0mm,vfill before first,
overlay unbroken={\draw[colexemple,line width=1pt]
([yshift=-1pt]title.north east)--([xshift=-0.5pt,yshift=-1pt]title.north-|frame.east)
--([xshift=-0.5pt]frame.south east)--(frame.south west); },
overlay first={\draw[colexemple,line width=1pt]
([yshift=-1pt]title.north east)--([xshift=-0.5pt,yshift=-1pt]title.north-|frame.east)
--([xshift=-0.5pt]frame.south east); },
overlay middle={\draw[colexemple,line width=1pt] ([xshift=-0.5pt]frame.north east)
--([xshift=-0.5pt]frame.south east); },
overlay last={\draw[colexemple,line width=1pt] ([xshift=-0.5pt]frame.north east)
--([xshift=-0.5pt]frame.south east)--(frame.south west);},%
}

\newenvironment{app}
  {
    \begin{apphid}{\theexemples}
  }
  {
    \end{apphid}
    \addtocounter{exemples}{1}
  }

\newtcolorbox{apphid}[1]{%
empty,title={Application {#1}},attach boxed title to top left,
boxed title style={empty,size=minimal,toprule=2pt,top=4pt,
overlay={\draw[colexemple,line width=2pt]
([yshift=-1pt]frame.north west)--([yshift=-1pt]frame.north east);}},
coltitle=colexemple,fonttitle=\bfseries,
before=\par\medskip\noindent,parbox=false,boxsep=0pt,left=0pt,right=3mm,top=4pt,
breakable,pad at break*=0mm,vfill before first,
overlay unbroken={\draw[colexemple,line width=1pt]
([yshift=-1pt]title.north east)--([xshift=-0.5pt,yshift=-1pt]title.north-|frame.east)
--([xshift=-0.5pt]frame.south east)--(frame.south west); },
overlay first={\draw[colexemple,line width=1pt]
([yshift=-1pt]title.north east)--([xshift=-0.5pt,yshift=-1pt]title.north-|frame.east)
--([xshift=-0.5pt]frame.south east); },
overlay middle={\draw[colexemple,line width=1pt] ([xshift=-0.5pt]frame.north east)
--([xshift=-0.5pt]frame.south east); },
overlay last={\draw[colexemple,line width=1pt] ([xshift=-0.5pt]frame.north east)
--([xshift=-0.5pt]frame.south east)--(frame.south west);},%
}

%%%%%%%%%%%%%%%%%%%%%%%%%%%%%%%%%%%%%%%%%%%%%%%%%%%%%%%%%%%%%%%%%%%%%%%%%%%%%%%%
%
% ENUMERATE
% =========
%
%%%%%%%%%%%%%%%%%%%%%%%%%%%%%%%%%%%%%%%%%%%%%%%%%%%%%%%%%%%%%%%%%%%%%%%%%%%%%%%%

\usepackage{enumerate}
\usepackage{enumitem}

% To have special enumerate items like
%
% 1/
% 2/
% 3/

%%%%%%%%%%%%%%%%%%%%%%%%%%%%%%%%%%%%%%%%%%%%%%%%%%%%%%%%%%%%%%%%%%%%%%%%%%%%%%%%
%
% ARRAYS
% ======
%
%%%%%%%%%%%%%%%%%%%%%%%%%%%%%%%%%%%%%%%%%%%%%%%%%%%%%%%%%%%%%%%%%%%%%%%%%%%%%%%%


\usepackage{array}
\usepackage{makecell} % Used to break lines within arrays
\usepackage{multirow}
\usepackage{booktabs} % Used to have nice arrays with headrules

%%%%%%%%%%%%%%%%%%%%%%%%%%%%%%%%%%%%%%%%%%%%%%%%%%%%%%%%%%%%%%%%%%%%%%%%%%%%%%%%
%
% WRITE CODE
% ==========
%
%%%%%%%%%%%%%%%%%%%%%%%%%%%%%%%%%%%%%%%%%%%%%%%%%%%%%%%%%%%%%%%%%%%%%%%%%%%%%%%%

\usepackage{listings}
\usepackage{xcolor}

%New colors defined below
\definecolor{codegreen}{rgb}{0,0.6,0}
\definecolor{codegray}{rgb}{0.5,0.5,0.5}
\definecolor{codepurple}{rgb}{0.58,0,0.82}
\definecolor{backcolour}{rgb}{0.95,0.95,0.92}

%Code listing style named "mystyle"
\lstdefinestyle{python}{
  %backgroundcolor=\color{backcolour},
  commentstyle=\color{codegreen},
  keywordstyle=\color{magenta},
  numberstyle=\tiny\color{codegray},
  stringstyle=\color{codepurple},
  basicstyle=\ttfamily\footnotesize,
  breakatwhitespace=false,
  breaklines=true,
  captionpos=b,
  keepspaces=true,
  numbers=left,
  numbersep=5pt,
  showspaces=false,
  showstringspaces=false,
  showtabs=false,
  tabsize=2
}

\lstset{style=python}

%%%%%%%%%%%%%%%%%%%%%%%%%%%%%%%%%%%%%%%%%%%%%%%%%%%%%%%%%%%%%%%%%%%%%%%%%%%%%%%%
%
% Tabular 
% =======
%
%%%%%%%%%%%%%%%%%%%%%%%%%%%%%%%%%%%%%%%%%%%%%%%%%%%%%%%%%%%%%%%%%%%%%%%%%%%%%%%%

% In order to obtain a tabular with given width.

\usepackage{tabularx}
\newcolumntype{Y}{>{\centering\arraybackslash}X}
\newcolumntype{R}{>{\raggedright\arraybackslash}X}
\newcolumntype{L}{>{\raggedleft\arraybackslash}X}
% \usepackage{tabulary} % younger brother

%%%%%%%%%%%%%%%%%%%%%%%%%%%%%%%%%%%%%%%%%%%%%%%%%%%%%%%%%%%%%%%%%%%%%%%%%%%%%%%%
%
% MACROS
% ======
%
%%%%%%%%%%%%%%%%%%%%%%%%%%%%%%%%%%%%%%%%%%%%%%%%%%%%%%%%%%%%%%%%%%%%%%%%%%%%%%%%

% Math Operators

\DeclareMathOperator{\Card}{Card}
\DeclareMathOperator{\Gal}{Gal}
\DeclareMathOperator{\Id}{Id}
\DeclareMathOperator{\Img}{Im}
\DeclareMathOperator{\Ker}{Ker}
\DeclareMathOperator{\Minpoly}{Minpoly}
\DeclareMathOperator{\Mod}{mod}
\DeclareMathOperator{\Ord}{Ord}
\DeclareMathOperator{\ppcm}{ppcm}
\DeclareMathOperator{\pgcd}{pgcd}
\DeclareMathOperator{\tr}{Tr}
\DeclareMathOperator{\Vect}{Vect}
\DeclareMathOperator{\Span}{Span}
\DeclareMathOperator{\rank}{rank}
\DeclareMathOperator{\rg}{rg}
\DeclareMathOperator{\ev}{ev}

% Shortcuts

\newcommand{\eg}{\emph{e.g. }}
\newcommand{\ent}[2]{[\![#1,#2]\!]}
\newcommand{\ie}{\emph{i.e. }}
\newcommand{\ps}[2]{\left\langle#1,#2\right\rangle}
\newcommand{\eqdef}{\overset{\text{def}}{=}}
\newcommand{\E}{\mathcal{E}}
\newcommand{\M}{\mathcal{M}}
\newcommand{\A}{\mathcal{A}}
\newcommand{\B}{\mathcal{B}}
\newcommand{\R}{\mathcal{R}}
\newcommand{\D}{\mathcal{D}}
\newcommand{\Pcal}{\mathcal{P}}
\newcommand{\K}{\mathbf{k}}


\begin{document}

\begin{exo}~\\[-4mm]
  \begin{minipage}{.5\textwidth}
   On considère le repère $(O; I, J)$ ci-contre.
   \begin{enumerate}
     \item Le repère $(O; I, J)$ est-il orthonormé ? Orthogonal ?
     \item Lire les coordonnées des points $A$, $B$ et $C$ dans le repère
       $(O; I, J)$.
     \item Placer les points $D(1; 1)$ et $E(-1;0)$.
     \item Déterminer les coordonnées de tous les points dans le repère
       $(D; A, I)$.
   \end{enumerate}
  \end{minipage}
  \begin{minipage}{.5\textwidth}
  \begin{center}
  \begin{tikzpicture}
    \def\absmin{-2.3}
    \def\absmax{4.3}
    \def\ordmin{-2.3}
    \def\ordmax{3.3}

    \draw[semithick] (\absmin, 0) -- (\absmax, 0) (0, \ordmin) -- (0, \ordmax);
      \foreach \x in {-2, ..., 4}{
      \draw[blue, thin, opacity=.3] (\x, \ordmin) -- (\x, \ordmax);
      \draw[semithick] (\x, -.1) -- (\x, .1);
      }
      \foreach \x in {-2, ..., 3}{
      \draw[blue, thin, opacity=.3] (\absmin, \x) -- (\absmax, \x);
      \draw[semithick] (-.1, \x) -- (.1, \x);
      }
      \draw[thick] (-1-.1, -1) -- (-1+.1, -1) (-1, -1-.1) -- (-1, -1+.1);
      \draw[thick] (3-.1, 1) -- (3+.1, 1) (3, 1-.1) -- (3, 1+.1);

      \node (O) at (-.25, -.25) {$O$};
      \node (I) at (1, -.5) {$I$};
      \node (J) at (-.5, 1) {$J$};
      \node (C) at (.25, 2.2) {$C$};
      \node (A) at (3.2, 1.25) {$A$};
      \node (B) at (-1.35, -1) {$B$};
  \end{tikzpicture}
\end{center}
\end{minipage}
\end{exo}

\begin{exo}
Dans un repère $(O; I, J)$ du plan, on considère les
points $A(3;1)$, $B(-4; 2)$ et $C(-1; 4)$.
\begin{enumerate}
  \item Déterminer les coordonnées du point $D$, symétrique de $C$ par rapport à
    $B$.
  \item On note $E$ le point du plan tel que les segments $\left[ AC
    \right]$ et $\left[ BE \right]$ aient le même milieu. Déterminer les
    coordonnées du point $E$.
\end{enumerate}
\end{exo}

\begin{exo}
Soit $(O; I, J)$ un repère orthonormé du plan. On
considère les trois points $A(1; 3)$, $B(1,5 ; 8)$ et $C(4; 5)$.
\begin{enumerate}
  \item Faire une figure et y placer les points cités.
  \item Calculer les coordonnées du milieu $K$ de $\left[ BC \right]$.
  \item Calculer les coordonnées de $D$, symétrique du point $A$ par rapport à
    $K$.
  \item Déterminer la nature du quadrilatère $ABCD$.
\end{enumerate}
\end{exo}

\begin{exo}
Dans un repère orthonormé $(O; I, J)$, on donne les
points $A(-2; 1)$, $B(4; 3)$ et $C(2; -3)$.
\begin{enumerate}
  \item Calculer les coordonnées du point :\begin{enumerate}
      \item $D$ tel que $ABCD$ soit un parallélogramme.
      \item $E$ tel que $ACBE$ soit un parallélogramme.
    \end{enumerate}
  \item Faire une figure et vérifier les résultats.
  \item Montrer que $A$ est le milieu du segment $\left[ DE \right]$.
\end{enumerate}
\end{exo}

\begin{exo}
Le plan est muni d'un repère orthonormé $(O; I, J)$
d'unité $1$ cm. On considère trois points du plan $A(-5; 2)$, $B(4; -1)$ et
$C(-2; 5)$.
\begin{enumerate}
  \item Placer les points $A, B$ et $C$ dans le repère $(O; I, J)$.
  \item Calculer les distances $AB$, $AC$ et $BC$.
  \item En déduire la nature du triangle $ABC$. Justifier.
\end{enumerate}
\end{exo}

\begin{exo}
Dans un repère orthonormé $(O; I, J)$, on considère les
points $A(\frac{-1}{2}; -1)$, $B(\frac{1}{2}; 2)$, $C(\frac{3}{2}; -1)$ et
$D(\frac{1}{2}; -4)$.
\begin{enumerate}
  \item Faire une figure.
  \item Conjecturer la nature du quadrilatère $ABCD$.
  \item Démontrer cette conjecture.
\end{enumerate}
\end{exo}

\begin{exo}
Soit $ABCD$ un carré de centre $O$. On considère les
points $E$ et $F$, milieux respectifs de $\left[ DC \right]$ et $\left[ OB
\right]$.
\begin{enumerate}
  \item Faire une figure.
  \item Lire les coordonnées de tous les points dans le repère $(A; B, D)$.
  \item Calculer $EF$, $EA$ et $FA$.
  \item En déduire la nature du triangle $EFA$.
\end{enumerate}
\end{exo}

    
\begin{exo}
Les points $R, S$ et $T$ sont-ils alignés ? Justifier.
\begin{center}
  \begin{tikzpicture}
    \draw (-.5, 0) -- (14.5, 0) (0, -.5) -- (0, 5.5);
    \foreach \x in {1, ..., 14}{
      \draw[blue!70!black, opacity=.3] (\x, -.5) -- (\x, 5.5);
      \draw[very thick] (\x, -.1) -- (\x, .1);
    }
    \foreach \x in {1, ..., 5}{
      \draw[blue!70!black, opacity=.3] (-.5, \x) -- (14.5, \x);
      \draw[very thick] (-.1, \x) -- (.1, \x);
    }
    \point{2}{1}{red};
    \point{9}{3}{red};
    \point{13}{4}{red};

    \node[red] at (2.3, 1.3) {$R$};
    \node[red] at (9.3, 3.3) {$S$};
    \node[red] at (13.3, 4.3) {$T$};
    
    \node at (-.3, -.3) {$O$};
    \node at (1, -.4) {$I$};
    \node at (-.4, 1) {$J$};
  \end{tikzpicture}
\end{center}
\end{exo}

\begin{comment}
\begin{exo}~\\
\begin{minipage}{.6\textwidth}
  Une famille de touristes visite New-York et se trouve au niveau du point rouge
  indiqué. Dans le repère, une unité est égale à $180$ mètres. La famille
  souhaite se rendre à la salle de concert Carnegie Hall puis au musée d'art
  moderne en suivant les rues. En utilisant le repère et l'échelle indiquée,
  donner une estimation de la distance parcourue en mètres.
\end{minipage}
\begin{minipage}{.4\textwidth}
  \begin{center}
  \includegraphics[scale=.25]{map.png}
  \end{center}
\end{minipage}
\end{exo}
\end{comment}


\begin{exo}
On considère le triangle $RST$ tel que $RS=4,8$ cm,
$ST=5,2$ cm et $RT=2$ cm.
\begin{enumerate}
  \item Démontrer que le triangle est rectangle en $R$.
  \item Calculer alors la mesure de tous les angles dans ce triangle.
\end{enumerate}
\end{exo}

\begin{exo}
On considère un repère orthonormé $(O; I, J)$ du plan.
On donne les points $A(-1; 6)$, $B(7; -2)$, $C(1; -2)$ et $D(9; 6)$.
\begin{enumerate}
  \item Faire une figure.
  \item Construire le centre $\Omega$ du cercle circonscrit au triangle $ABC$.
  \item Donner, sans justification, les coordonnées de $\Omega$ et calculer le
    rayon du cercle.
  \item Montrer que les points $A$, $B$, $C$ et $D$ appartiennent à un même
    cercle : on dira qu'ils sont \textbf{cocycliques}.
\end{enumerate}
\end{exo}

\begin{exo}
On considère un triangle $LMN$ rectangle en $N$ tel que
$\cos\widehat{MLN}=0,6$.
\begin{enumerate}
  \item Calculer la valeur exacte de $\sin(\widehat{MLN})$.
  \item Sachant que $LM=10$ cm, calculer la longueur des autres côtés du
    triangle. Arrondir au dixième.
\end{enumerate}
\end{exo}


\begin{exo}
Le plan est muni d'un repère orthonormé $(O; I, J)$
d'unité $2$ cm. On considère les points $A(2; 1)$, $B(5; 1)$, $C(5; -2)$ et
$D(2; -2)$.
\begin{enumerate}
  \item Faire une figure
  \item \begin{enumerate}
      \item Déterminer les coordonnées de $K$, milieu de $\left[ AC
        \right]$.
      \item Déterminer les coordonnés de $L$, milieu de $\left[ BD \right]$.
      \item En déduire que $ABCD$ est un parallélogramme.
    \end{enumerate}
  \item \begin{enumerate}
      \item Calculer les distances $AC$, $AD$ et $DC$.
      \item En déduire la nature du triangle $ADC$.
    \end{enumerate}
  \item Conclure sur la nature du parallélogramme $ABCD$.
\end{enumerate}
\end{exo}

\begin{exo}
Dans un repère orthonormé $(O; I, J)$, on considère les
points $A(-2;1)$, $B(-1; 4)$ et $C(5;2)$.
\begin{enumerate}
  \item Faire une figure.
  \item Calculer les valeurs exactes des longueurs $AB$, $AC$ et $BC$.
  \item En déduire la nature du triangle $ABC$
  \item Calculer les coordonnées du point $M$, milieu de $\left[ AC
    \right]$.
  \item Déterminer les coordonnées du point $D$ tel que $ABCD$ soit un
    rectangle.
\end{enumerate}
\end{exo}

\begin{exo}
$ABCD$ est un carré de côté $10$. On trace le cercle de
centre $A$ passant par $C$. Le point $E$ est l'intersection du cercle avec la
droite $(AB)$. On construit un carré $DEFG$.
\begin{enumerate}
  \item Faire une figure.
  \item Calculer la longueur $AC$.
  \item En déduire la longueur $DE$.
  \item Montrer que l'aire du carré $DEFG$ est le triple de l'aire du carré
    $ABCD$.
\end{enumerate}
\end{exo}

\begin{center}
  \LARGE
  \textbf{Exercices d'approfondissement}
\end{center}

\begin{exo}
Soit $ABCD$ un carré dont les quatre côtés ont été
partagés en quatre part égales.
\begin{minipage}{.7\textwidth}
On munit le plan du repère orthonormé $(A; B, D)$.
\begin{enumerate}
  \item Déterminer les coordonnées des points $A, B, C$ et $D$.
  \item Reproduire la figure et placer les points $J$ et $L$, milieux respectifs
    de $\left[ CD \right]$ et $\left[ AB \right]$.
  \item Calculer les coordonnées des points $J$ et $L$.
  \item Placer les points $I(0; \frac{3}{4})$ et $K(1; \frac{1}{4})$.
  \item Montrer que le quadrilatère $IJKL$ est un parallélogramme.
\end{enumerate}
\end{minipage}
\begin{minipage}{.3\textwidth}
  \begin{center}
\begin{tikzpicture}
  \draw (0,0)--(4,0)--(4,4)--(0,4)--(0,0);
  \foreach \x in {1, 2, 3}{
    \draw (\x, -.1) -- (\x, .1) (\x, 3.9) -- (\x, 4.1) (-.1, \x) -- (.1, \x)
    (3.9, \x) -- (4.1, \x);
  }
  \node at (.25, .25) {$A$};
  \node at (3.75, .25) {$B$};
  \node at (3.75, 3.75) {$C$};
  \node at (.25, 3.75) {$D$};
\end{tikzpicture}
  \end{center}
\end{minipage}
\end{exo}

\begin{exo}
Dans le plan muni d'un repère orthonormé $(O; I, J)$, on
considère les trois points $A(-3; 3)$, $B(2; 4)$ et $C(1; -4)$.
\begin{enumerate}
  \item Faire une figure.
  \item Conjecturer la nature du triangle $ABC$.
  \item Démontrer cette conjecture.
\end{enumerate}
\end{exo}

\begin{exo}
On veut démontrer la propriété suivante : « Le projeté
orthogonal d'un point $M$ sur une droite $\Delta$ est le point de $\Delta$ le
plus proche de $M$. ».
\begin{enumerate}
  \item Réaliser la figure suivante.
    \begin{enumerate}
      \item Tracer une droite $\Delta$ et placer un point $M$ m'appartenant pas
        à cette droite ;
      \item construire le point $H$, projeté orthogonal de $M$ sur $\Delta$;
      \item placer deux points distincts $A$ et $B$ sur $\Delta$ et différents
        de $H$.
    \end{enumerate}
  \item Traduire la propriété que l'on cherche à démontrer en utilisant les
    points de la figure construite.
    \item \begin{enumerate}
        \item Que peut-on dire des triangles $AMH$ et $MBH$ ?
        \item En déduire le plus grand côté de chacun de ses triangles.
        \item La position des points $A$ et $B$ influence-t-elle la réponse à la
          question précédente ?
      \end{enumerate}
    \item Conclure.
\end{enumerate}
\end{exo}

\begin{exo}
Soit $(O; I, J)$ un repère orthonormé du plan. On
considère le quart de cercle de centre $O$ et de rayon $1$ unité. Le point $M$
de coordonnées $(x_m; y_M)$ est un point mobile dans le quart de cercle. On note
$H$ le projeté orthogonal de $M$ sur $(OI)$ et $L$ le projeté orthogonal de $M$
sur $(OJ)$.
\begin{minipage}[]{.6\textwidth}
\begin{enumerate}
  \item Montrer que $IJ=\sqrt 2$.
  \item Expliquer pourquoi $0\leq x_m\leq 1$ et $0\leq y_M\leq 1$.
  \item Déterminer la distance $OM$.
  \item Calculer les coordonnées de $S$, milieu de $\left[ OI \right]$.
  \item On considère dans cette question que $M$ est aussi sur la médiatrice de
    $\left[ OI \right]$.
    \begin{enumerate}
      \item Montrer que $x_M=x_H=\frac{1}{2}$.
      \item En déduire la distance $MH$ et les coordonnées du point $M$.
      \item Déterminer la mesure de l'angle $\widehat{HOM}$.
    \end{enumerate}
\end{enumerate}
\end{minipage}
\begin{minipage}[]{.4\textwidth}
  \begin{center}
    \begin{tikzpicture}[x=4cm, y=4cm]
      \draw (-.2, 0) -- (1.2, 0);
      \draw (0, -.2) -- (0, 1.2);
      \draw (1, -.05) -- (1, .05);
      \draw (-.05, 1) -- (.05, 1);
      \draw (1,0) arc (0:90:1);
      \node at (-.1, -.1) {$O$};
      \draw[thick] (.574 -.05, .819)--(.574 +.05, .819) (.574, .819-.05)--(.574,.819+.05);
      \draw[loosely dashed, thick] (.574, .819)--(0, .819) (.574, .819)--(.574,0);

      \node at (.7, .9) {$M$};
      \node at (.574, -.1) {$H$};
      \node at (-.1, .819) {$L$};
      \node at (-.1, 1) {$J$};
      \node at (1, -.1) {$I$};
    \end{tikzpicture}
  \end{center}
\end{minipage}
\begin{enumerate}
    \setcounter{enumi}{5}
  \item[]
    \begin{enumerate}
        \setcounter{enumii}{3}
      \item Conclure en donnant les valeurs exactes de $\cos(60^\circ)$ et
        $\sin(60^\circ)$.
    \end{enumerate}
  \item En raisonnant de manière analogue, déterminer les valeurs exactes de
    $\cos(30^\circ)$ et $\sin(30^\circ)$.
\end{enumerate}
\end{exo}



\end{document}
