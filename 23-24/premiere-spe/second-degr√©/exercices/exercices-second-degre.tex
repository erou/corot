\documentclass[11pt]{article}

\newcommand{\titrechapitre}{Second degré -- Exercices}
\newcommand{\titreclasse}{Lycée Jean-Baptiste \textsc{Corot}}
\newcommand{\pagination}{\thepage}%/\pageref{LastPage}}
\newcommand{\topbotmargins}{2cm}
\newcommand{\spacebelowexo}{5mm}
%%%%%%%%%%%%%%%%%%%%%%%%%%%%%%%%%%%%%%%%%%%%%%%%%%%%%%%%%%%%%%%%%%%%%%%%%%%%%%%%
%
% PACKAGES
% ========
%
%%%%%%%%%%%%%%%%%%%%%%%%%%%%%%%%%%%%%%%%%%%%%%%%%%%%%%%%%%%%%%%%%%%%%%%%%%%%%%%%

\usepackage[english, french]{babel}
\usepackage[utf8]{inputenc}
\usepackage[T1]{fontenc}
\usepackage{graphicx}
\usepackage{amsmath,amssymb,amsthm,amsopn}
\usepackage{hyperref}

% Pour avoir l'écriture \mathscr (math script)
% ============================================

\usepackage{mathrsfs}

% Deal with coma as a decimal separator
% =====================================

\usepackage{icomma}

% Package Geometry
% ================

\usepackage[a4paper, lmargin=2cm, rmargin=2cm, top=\topbotmargins, bottom=\topbotmargins]{geometry}

% Package multicol
% ================

\usepackage{multicol}

% Redefine abstract
% =================

% Note
% ====
%
% Le reste a été commenté pour ne pas charger trop de choses au démarrage. On
% verra si on en a besoin plus tard.
%
% --------
%
%\usepackage{mathrsfs}
%\usepackage{multirow}
%\usepackage{bm}
%\hypersetup{
%    colorlinks=true,
%    linkcolor=blue,
%    citecolor=red,
%}
%\usepackage{diagbox}
%
%\usepackage{algorithm}
%\usepackage{algpseudocode}
%
%\renewcommand{\algorithmicrequire}{\textbf{Input:}}
%\renewcommand{\algorithmicensure}{\textbf{Output:}}


%%%%%%%%%%%%%%%%%%%%%%%%%%%%%%%%%%%%%%%%%%%%%%%%%%%%%%%%%%%%%%%%%%%%%%%%%%%%%%%%
%
% TIKZ
% ====
%
%%%%%%%%%%%%%%%%%%%%%%%%%%%%%%%%%%%%%%%%%%%%%%%%%%%%%%%%%%%%%%%%%%%%%%%%%%%%%%%%

\usepackage{tikz}
\usetikzlibrary{arrows}

\usepackage{tkz-tab} % Variation tables

\usepackage{pgfplots}
%\usepackage{pgf-pie} % Pie charts

\pgfplotsset{
%\newcommand{\settingsgraph}{
x=.5cm,y=.5cm,
xticklabel style = {font=\scriptsize, yshift=.1cm},
yticklabel style = {font=\scriptsize, xshift=.1cm},
axis lines=middle,
ymajorgrids=true,
xmajorgrids=true,
major grid style = {color=white!80!blue},
xmin=-5.5,
xmax=5.5,
ymin=-5.5,
ymax=5.5,
xtick={-5.0,-4.0,...,5.0},
ytick={-5.0,-4.0,...,5.0},
}

% Tikz style

\tikzset{round/.style={circle, draw=black, very thick, scale = 0.7}}
\tikzset{arrow/.style={->, >=latex}}
\tikzset{dashed-arrow/.style={->, >=latex, dashed}}

\newcommand{\point}[3]{\draw[very thick, #3] (#1-.1, #2)--(#1+.1, #2)
(#1, #2-.1)--(#1, #2+.1)}

%%%%%%%%%%%%%%%%%%%%%%%%%%%%%%%%%%%%%%%%%%%%%%%%%%%%%%%%%%%%%%%%%%%%%%%%%%%%%%%%
%
% FANCY HEADER
% ============
%
%%%%%%%%%%%%%%%%%%%%%%%%%%%%%%%%%%%%%%%%%%%%%%%%%%%%%%%%%%%%%%%%%%%%%%%%%%%%%%%%


\usepackage{fancyhdr}
\usepackage{lastpage}

\pagestyle{fancy}
\newcommand{\changefont}{\fontsize{9}{9}\selectfont}
\renewcommand{\headrulewidth}{0mm}
\renewcommand{\footrulewidth}{0mm}

\fancyhead[C]{}
\fancyhead[L]{\titreclasse}
\fancyhead[R]{\titrechapitre}
\fancyfoot[C]{}
\fancyfoot[L]{}
\fancyfoot[R]{\pagination}
\addtolength{\skip\footins}{20pt} % distance between text and footnotes

%%%%%%%%%%%%%%%%%%%%%%%%%%%%%%%%%%%%%%%%%%%%%%%%%%%%%%%%%%%%%%%%%%%%%%%%%%%%%%%%
%
% THEOREM STYLE
% =============
%
%%%%%%%%%%%%%%%%%%%%%%%%%%%%%%%%%%%%%%%%%%%%%%%%%%%%%%%%%%%%%%%%%%%%%%%%%%%%%%%%

\usepackage[tikz]{bclogo}
\usepackage{mdframed}

\usepackage{tcolorbox}
\tcbuselibrary{listings, breakable, theorems, skins}

%\newtheoremstyle{break}%
%{}{}%
%{\itshape}{}%
%{\bfseries}{}%  % Note that final punctuation is omitted.
%{\newline}{}

\newtheoremstyle{scbf}%
{}{}%
{}{}%
%{\scshape}{}%  % Note that final punctuation is omitted.
{\bfseries\scshape}{}%  % Note that final punctuation is omitted.
{\newline}{}

%\theoremstyle{break}
%\theoremstyle{plain}
%\newtheorem{thm}{Theorem}[section]
%\newtheorem{lm}[thm]{Lemma}
%\newtheorem{prop}[thm]{Proposition}
%\newtheorem{cor}[thm]{Corollary}

%\theoremstyle{scbf}
%\newtheorem{exo}{$\star$ Exercice}

%\theoremstyle{definition}
%\newtheorem{defi}[thm]{Definition}
%\newtheorem{ex}[thm]{Example}

%\theoremstyle{remark}
%\newtheorem{rem}[thm]{Remark}

% Defining the Remark environment
% ===============================

\newenvironment{rmq}
  {
    \begin{bclogo}[logo=\bcinfo, noborder=true]{Remarque}
  }
  {
    \end{bclogo}
  }

% Defining the exercise environment
% =================================

\newcounter{exos}
\setcounter{exos}{1}

\newenvironment{exo}
  {
    \begin{bclogo}[logo=\bccrayon, noborder=true]{Exercice \theexos}
  }
  {
    \end{bclogo}
    \addtocounter{exos}{1}
  }


% Redefining the proof environment from amsthm
% ============================================

\tcolorboxenvironment{proof}{
  blanker, breakable, before skip=10pt,after skip=10pt,
  borderline west={1mm}{0pt}{red},
  left=5mm,
}

% Defining the definition environment
% ===================================

\colorlet{coldef}{black!50!green}

\newcounter{defis}
\setcounter{defis}{1}

\newenvironment{defi}[1]
  {
    \begin{defihid}{{#1}}{\thedefis}
  }
  {
    \end{defihid}
    \addtocounter{defis}{1}
  }

\newtcolorbox{defihid}[2]{%
  empty,title={ {\bfseries Définition {#2}} ({#1})},attach boxed title to top left,
boxed title style={empty,size=minimal,toprule=2pt,top=4pt,
overlay={\draw[coldef,line width=2pt]
([yshift=-1pt]frame.north west)--([yshift=-1pt]frame.north east);}},
coltitle=coldef,
before=\par\medskip\noindent,parbox=false,boxsep=0pt,left=0pt,right=3mm,top=4pt,
breakable,pad at break*=0mm,vfill before first,
overlay unbroken={\draw[coldef,line width=1pt]
([yshift=-1pt]title.north east)--([xshift=-0.5pt,yshift=-1pt]title.north-|frame.east)
--([xshift=-0.5pt]frame.south east)--(frame.south west); },
overlay first={\draw[coldef,line width=1pt]
([yshift=-1pt]title.north east)--([xshift=-0.5pt,yshift=-1pt]title.north-|frame.east)
--([xshift=-0.5pt]frame.south east); },
overlay middle={\draw[coldef,line width=1pt] ([xshift=-0.5pt]frame.north east)
--([xshift=-0.5pt]frame.south east); },
overlay last={\draw[coldef,line width=1pt] ([xshift=-0.5pt]frame.north east)
--([xshift=-0.5pt]frame.south east)--(frame.south west);},%
}

\newenvironment{notation}
  {
    \begin{notationhid}{\thedefis}
  }
  {
    \end{notationhid}
    \addtocounter{defis}{1}
  }

\newtcolorbox{notationhid}[1]{%
  empty,title={Notation {#1}},attach boxed title to top left,
boxed title style={empty,size=minimal,toprule=2pt,top=4pt,
overlay={\draw[coldef,line width=2pt]
([yshift=-1pt]frame.north west)--([yshift=-1pt]frame.north east);}},
coltitle=coldef,fonttitle=\bfseries,
before=\par\medskip\noindent,parbox=false,boxsep=0pt,left=0pt,right=3mm,top=4pt,
breakable,pad at break*=0mm,vfill before first,
overlay unbroken={\draw[coldef,line width=1pt]
([yshift=-1pt]title.north east)--([xshift=-0.5pt,yshift=-1pt]title.north-|frame.east)
--([xshift=-0.5pt]frame.south east)--(frame.south west); },
overlay first={\draw[coldef,line width=1pt]
([yshift=-1pt]title.north east)--([xshift=-0.5pt,yshift=-1pt]title.north-|frame.east)
--([xshift=-0.5pt]frame.south east); },
overlay middle={\draw[coldef,line width=1pt] ([xshift=-0.5pt]frame.north east)
--([xshift=-0.5pt]frame.south east); },
overlay last={\draw[coldef,line width=1pt] ([xshift=-0.5pt]frame.north east)
--([xshift=-0.5pt]frame.south east)--(frame.south west);},%
}


% Defining the proposition, theorem, etc. environment
% ===================================================

\colorlet{colprop}{red!75!black}

\newcounter{props}
\setcounter{props}{1}

\newenvironment{prop}
  {
    \begin{prophid}{\theprops}
  }
  {
    \end{prophid}
    \refstepcounter{props}
  }

\newtcolorbox{prophid}[1]{%
empty,title={Propriété {#1}},attach boxed title to top left,
boxed title style={empty,size=minimal,toprule=2pt,top=4pt,
overlay={\draw[colprop,line width=2pt]
([yshift=-1pt]frame.north west)--([yshift=-1pt]frame.north east);}},
coltitle=colprop,fonttitle=\bfseries,
before=\par\medskip\noindent,parbox=false,boxsep=0pt,left=0pt,right=3mm,top=4pt,
breakable,pad at break*=0mm,vfill before first,
overlay unbroken={\draw[colprop,line width=1pt]
([yshift=-1pt]title.north east)--([xshift=-0.5pt,yshift=-1pt]title.north-|frame.east)
--([xshift=-0.5pt]frame.south east)--(frame.south west); },
overlay first={\draw[colprop,line width=1pt]
([yshift=-1pt]title.north east)--([xshift=-0.5pt,yshift=-1pt]title.north-|frame.east)
--([xshift=-0.5pt]frame.south east); },
overlay middle={\draw[colprop,line width=1pt] ([xshift=-0.5pt]frame.north east)
--([xshift=-0.5pt]frame.south east); },
overlay last={\draw[colprop,line width=1pt] ([xshift=-0.5pt]frame.north east)
--([xshift=-0.5pt]frame.south east)--(frame.south west);},%
}

\newenvironment{propadm}
  {
    \begin{propadmhid}{\theprops}
  }
  {
    \end{propadmhid}
    \refstepcounter{props}
  }

  \newtcolorbox{propadmhid}[1]{%
    empty,title={{\bfseries Propriété {#1}} (admise)},attach boxed title to top left,
boxed title style={empty,size=minimal,toprule=2pt,top=4pt,
overlay={\draw[colprop,line width=2pt]
([yshift=-1pt]frame.north west)--([yshift=-1pt]frame.north east);}},
coltitle=colprop,%fonttitle=\bfseries,
before=\par\medskip\noindent,parbox=false,boxsep=0pt,left=0pt,right=3mm,top=4pt,
breakable,pad at break*=0mm,vfill before first,
overlay unbroken={\draw[colprop,line width=1pt]
([yshift=-1pt]title.north east)--([xshift=-0.5pt,yshift=-1pt]title.north-|frame.east)
--([xshift=-0.5pt]frame.south east)--(frame.south west); },
overlay first={\draw[colprop,line width=1pt]
([yshift=-1pt]title.north east)--([xshift=-0.5pt,yshift=-1pt]title.north-|frame.east)
--([xshift=-0.5pt]frame.south east); },
overlay middle={\draw[colprop,line width=1pt] ([xshift=-0.5pt]frame.north east)
--([xshift=-0.5pt]frame.south east); },
overlay last={\draw[colprop,line width=1pt] ([xshift=-0.5pt]frame.north east)
--([xshift=-0.5pt]frame.south east)--(frame.south west);},%
}

\newenvironment{propnom}[1]
  {
    \begin{propnomhid}{#1}{\theprops}
  }
  {
    \end{propnomhid}
    \refstepcounter{props}
  }

\newtcolorbox{propnomhid}[2]{%
empty,title={{\bfseries Propriété {#2}} ({#1})},attach boxed title to top left,
boxed title style={empty,size=minimal,toprule=2pt,top=4pt,
overlay={\draw[colprop,line width=2pt]
([yshift=-1pt]frame.north west)--([yshift=-1pt]frame.north east);}},
coltitle=colprop,
before=\par\medskip\noindent,parbox=false,boxsep=0pt,left=0pt,right=3mm,top=4pt,
breakable,pad at break*=0mm,vfill before first,
overlay unbroken={\draw[colprop,line width=1pt]
([yshift=-1pt]title.north east)--([xshift=-0.5pt,yshift=-1pt]title.north-|frame.east)
--([xshift=-0.5pt]frame.south east)--(frame.south west); },
overlay first={\draw[colprop,line width=1pt]
([yshift=-1pt]title.north east)--([xshift=-0.5pt,yshift=-1pt]title.north-|frame.east)
--([xshift=-0.5pt]frame.south east); },
overlay middle={\draw[colprop,line width=1pt] ([xshift=-0.5pt]frame.north east)
--([xshift=-0.5pt]frame.south east); },
overlay last={\draw[colprop,line width=1pt] ([xshift=-0.5pt]frame.north east)
--([xshift=-0.5pt]frame.south east)--(frame.south west);},%
}




\newenvironment{thm}
  {
    \begin{thmhid}{\theprops}
  }
  {
    \end{thmhid}
    \refstepcounter{props}
  }

\newtcolorbox{thmhid}[1]{%
empty,title={Théorème {#1}},attach boxed title to top left,
boxed title style={empty,size=minimal,toprule=2pt,top=4pt,
overlay={\draw[colprop,line width=2pt]
([yshift=-1pt]frame.north west)--([yshift=-1pt]frame.north east);}},
coltitle=colprop,fonttitle=\bfseries,
before=\par\medskip\noindent,parbox=false,boxsep=0pt,left=0pt,right=3mm,top=4pt,
breakable,pad at break*=0mm,vfill before first,
overlay unbroken={\draw[colprop,line width=1pt]
([yshift=-1pt]title.north east)--([xshift=-0.5pt,yshift=-1pt]title.north-|frame.east)
--([xshift=-0.5pt]frame.south east)--(frame.south west); },
overlay first={\draw[colprop,line width=1pt]
([yshift=-1pt]title.north east)--([xshift=-0.5pt,yshift=-1pt]title.north-|frame.east)
--([xshift=-0.5pt]frame.south east); },
overlay middle={\draw[colprop,line width=1pt] ([xshift=-0.5pt]frame.north east)
--([xshift=-0.5pt]frame.south east); },
overlay last={\draw[colprop,line width=1pt] ([xshift=-0.5pt]frame.north east)
--([xshift=-0.5pt]frame.south east)--(frame.south west);},%
}

\newenvironment{thmadm}
  {
    \begin{thmadmhid}{\theprops}
  }
  {
    \end{thmadmhid}
    \refstepcounter{props}
  }

  \newtcolorbox{thmadmhid}[1]{%
    empty,title={{\bfseries Théorème {#1}} (admis)},attach boxed title to top left,
boxed title style={empty,size=minimal,toprule=2pt,top=4pt,
overlay={\draw[colprop,line width=2pt]
([yshift=-1pt]frame.north west)--([yshift=-1pt]frame.north east);}},
coltitle=colprop,%fonttitle=\bfseries,
before=\par\medskip\noindent,parbox=false,boxsep=0pt,left=0pt,right=3mm,top=4pt,
breakable,pad at break*=0mm,vfill before first,
overlay unbroken={\draw[colprop,line width=1pt]
([yshift=-1pt]title.north east)--([xshift=-0.5pt,yshift=-1pt]title.north-|frame.east)
--([xshift=-0.5pt]frame.south east)--(frame.south west); },
overlay first={\draw[colprop,line width=1pt]
([yshift=-1pt]title.north east)--([xshift=-0.5pt,yshift=-1pt]title.north-|frame.east)
--([xshift=-0.5pt]frame.south east); },
overlay middle={\draw[colprop,line width=1pt] ([xshift=-0.5pt]frame.north east)
--([xshift=-0.5pt]frame.south east); },
overlay last={\draw[colprop,line width=1pt] ([xshift=-0.5pt]frame.north east)
--([xshift=-0.5pt]frame.south east)--(frame.south west);},%
}

\newenvironment{thmnom}[1]
  {
    \begin{thmnomhid}{#1}{\theprops}
  }
  {
    \end{thmnomhid}
    \refstepcounter{props}
  }

\newtcolorbox{thmnomhid}[2]{%
empty,title={{\bfseries Théorème {#2}} ({#1})},attach boxed title to top left,
boxed title style={empty,size=minimal,toprule=2pt,top=4pt,
overlay={\draw[colprop,line width=2pt]
([yshift=-1pt]frame.north west)--([yshift=-1pt]frame.north east);}},
coltitle=colprop,
before=\par\medskip\noindent,parbox=false,boxsep=0pt,left=0pt,right=3mm,top=4pt,
breakable,pad at break*=0mm,vfill before first,
overlay unbroken={\draw[colprop,line width=1pt]
([yshift=-1pt]title.north east)--([xshift=-0.5pt,yshift=-1pt]title.north-|frame.east)
--([xshift=-0.5pt]frame.south east)--(frame.south west); },
overlay first={\draw[colprop,line width=1pt]
([yshift=-1pt]title.north east)--([xshift=-0.5pt,yshift=-1pt]title.north-|frame.east)
--([xshift=-0.5pt]frame.south east); },
overlay middle={\draw[colprop,line width=1pt] ([xshift=-0.5pt]frame.north east)
--([xshift=-0.5pt]frame.south east); },
overlay last={\draw[colprop,line width=1pt] ([xshift=-0.5pt]frame.north east)
--([xshift=-0.5pt]frame.south east)--(frame.south west);},%
}

\newenvironment{coro}
  {
    \begin{corohid}{\theprops}
  }
  {
    \end{corohid}
    \refstepcounter{props}
  }

  \newtcolorbox{corohid}[1]{%
  empty,title={Corollaire {#1}},attach boxed title to top left,
boxed title style={empty,size=minimal,toprule=2pt,top=4pt,
overlay={\draw[colprop,line width=2pt]
([yshift=-1pt]frame.north west)--([yshift=-1pt]frame.north east);}},
coltitle=colprop,fonttitle=\bfseries,
before=\par\medskip\noindent,parbox=false,boxsep=0pt,left=0pt,right=3mm,top=4pt,
breakable,pad at break*=0mm,vfill before first,
overlay unbroken={\draw[colprop,line width=1pt]
([yshift=-1pt]title.north east)--([xshift=-0.5pt,yshift=-1pt]title.north-|frame.east)
--([xshift=-0.5pt]frame.south east)--(frame.south west); },
overlay first={\draw[colprop,line width=1pt]
([yshift=-1pt]title.north east)--([xshift=-0.5pt,yshift=-1pt]title.north-|frame.east)
--([xshift=-0.5pt]frame.south east); },
overlay middle={\draw[colprop,line width=1pt] ([xshift=-0.5pt]frame.north east)
--([xshift=-0.5pt]frame.south east); },
overlay last={\draw[colprop,line width=1pt] ([xshift=-0.5pt]frame.north east)
--([xshift=-0.5pt]frame.south east)--(frame.south west);},%
}

\newenvironment{lemme}
  {
    \begin{lemmehid}{\theprops}
  }
  {
    \end{lemmehid}
    \refstepcounter{props}
  }

  \newtcolorbox{lemmehid}[1]{%
  empty,title={Lemme {#1}},attach boxed title to top left,
boxed title style={empty,size=minimal,toprule=2pt,top=4pt,
overlay={\draw[colprop,line width=2pt]
([yshift=-1pt]frame.north west)--([yshift=-1pt]frame.north east);}},
coltitle=colprop,fonttitle=\bfseries,
before=\par\medskip\noindent,parbox=false,boxsep=0pt,left=0pt,right=3mm,top=4pt,
breakable,pad at break*=0mm,vfill before first,
overlay unbroken={\draw[colprop,line width=1pt]
([yshift=-1pt]title.north east)--([xshift=-0.5pt,yshift=-1pt]title.north-|frame.east)
--([xshift=-0.5pt]frame.south east)--(frame.south west); },
overlay first={\draw[colprop,line width=1pt]
([yshift=-1pt]title.north east)--([xshift=-0.5pt,yshift=-1pt]title.north-|frame.east)
--([xshift=-0.5pt]frame.south east); },
overlay middle={\draw[colprop,line width=1pt] ([xshift=-0.5pt]frame.north east)
--([xshift=-0.5pt]frame.south east); },
overlay last={\draw[colprop,line width=1pt] ([xshift=-0.5pt]frame.north east)
--([xshift=-0.5pt]frame.south east)--(frame.south west);},%
}

\colorlet{colexemple}{blue!50!black}
%\newtcolorbox{exemple}{empty, title=Exemple, attach boxed title to top left,
%  boxed title style={empty, size=minimal, toprule=2pt, top=4pt,
%    overlay={\draw[colexemple,line width=2pt]
%([yshift=-1pt]frame.north west)--([yshift=-1pt]frame.north east);}},
%coltitle=colexemple,fonttitle=\bfseries,%\large\bfseries,
%before=\par\medskip\noindent,parbox=false,boxsep=0pt,left=0pt,right=3mm,top=4pt,
%overlay={\draw[colexemple,line width=1pt]
%([yshift=-1pt]title.north east)--([xshift=-0.5pt,yshift=-1pt]title.north-|frame.east)
%--([xshift=-0.5pt]frame.south east)--(frame.south west); },
%}

\newcounter{exemples}
\setcounter{exemples}{1}

\newenvironment{exemple}
  {
    \begin{exemplehid}{\theexemples}
  }
  {
    \end{exemplehid}
    \addtocounter{exemples}{1}
  }

\newtcolorbox{exemplehid}[1]{%
empty,title={Exemple {#1}},attach boxed title to top left,
boxed title style={empty,size=minimal,toprule=2pt,top=4pt,
overlay={\draw[colexemple,line width=2pt]
([yshift=-1pt]frame.north west)--([yshift=-1pt]frame.north east);}},
coltitle=colexemple,fonttitle=\bfseries,
before=\par\medskip\noindent,parbox=false,boxsep=0pt,left=0pt,right=3mm,top=4pt,
breakable,pad at break*=0mm,vfill before first,
overlay unbroken={\draw[colexemple,line width=1pt]
([yshift=-1pt]title.north east)--([xshift=-0.5pt,yshift=-1pt]title.north-|frame.east)
--([xshift=-0.5pt]frame.south east)--(frame.south west); },
overlay first={\draw[colexemple,line width=1pt]
([yshift=-1pt]title.north east)--([xshift=-0.5pt,yshift=-1pt]title.north-|frame.east)
--([xshift=-0.5pt]frame.south east); },
overlay middle={\draw[colexemple,line width=1pt] ([xshift=-0.5pt]frame.north east)
--([xshift=-0.5pt]frame.south east); },
overlay last={\draw[colexemple,line width=1pt] ([xshift=-0.5pt]frame.north east)
--([xshift=-0.5pt]frame.south east)--(frame.south west);},%
}

\newenvironment{contrex}
  {
    \begin{contrexhid}{\theexemples}
  }
  {
    \end{contrexhid}
    \addtocounter{exemples}{1}
  }

\newtcolorbox{contrexhid}[1]{%
empty,title={Contre-exemple {#1}},attach boxed title to top left,
boxed title style={empty,size=minimal,toprule=2pt,top=4pt,
overlay={\draw[colexemple,line width=2pt]
([yshift=-1pt]frame.north west)--([yshift=-1pt]frame.north east);}},
coltitle=colexemple,fonttitle=\bfseries,
before=\par\medskip\noindent,parbox=false,boxsep=0pt,left=0pt,right=3mm,top=4pt,
breakable,pad at break*=0mm,vfill before first,
overlay unbroken={\draw[colexemple,line width=1pt]
([yshift=-1pt]title.north east)--([xshift=-0.5pt,yshift=-1pt]title.north-|frame.east)
--([xshift=-0.5pt]frame.south east)--(frame.south west); },
overlay first={\draw[colexemple,line width=1pt]
([yshift=-1pt]title.north east)--([xshift=-0.5pt,yshift=-1pt]title.north-|frame.east)
--([xshift=-0.5pt]frame.south east); },
overlay middle={\draw[colexemple,line width=1pt] ([xshift=-0.5pt]frame.north east)
--([xshift=-0.5pt]frame.south east); },
overlay last={\draw[colexemple,line width=1pt] ([xshift=-0.5pt]frame.north east)
--([xshift=-0.5pt]frame.south east)--(frame.south west);},%
}

\newenvironment{app}
  {
    \begin{apphid}{\theexemples}
  }
  {
    \end{apphid}
    \addtocounter{exemples}{1}
  }

\newtcolorbox{apphid}[1]{%
empty,title={Application {#1}},attach boxed title to top left,
boxed title style={empty,size=minimal,toprule=2pt,top=4pt,
overlay={\draw[colexemple,line width=2pt]
([yshift=-1pt]frame.north west)--([yshift=-1pt]frame.north east);}},
coltitle=colexemple,fonttitle=\bfseries,
before=\par\medskip\noindent,parbox=false,boxsep=0pt,left=0pt,right=3mm,top=4pt,
breakable,pad at break*=0mm,vfill before first,
overlay unbroken={\draw[colexemple,line width=1pt]
([yshift=-1pt]title.north east)--([xshift=-0.5pt,yshift=-1pt]title.north-|frame.east)
--([xshift=-0.5pt]frame.south east)--(frame.south west); },
overlay first={\draw[colexemple,line width=1pt]
([yshift=-1pt]title.north east)--([xshift=-0.5pt,yshift=-1pt]title.north-|frame.east)
--([xshift=-0.5pt]frame.south east); },
overlay middle={\draw[colexemple,line width=1pt] ([xshift=-0.5pt]frame.north east)
--([xshift=-0.5pt]frame.south east); },
overlay last={\draw[colexemple,line width=1pt] ([xshift=-0.5pt]frame.north east)
--([xshift=-0.5pt]frame.south east)--(frame.south west);},%
}

%%%%%%%%%%%%%%%%%%%%%%%%%%%%%%%%%%%%%%%%%%%%%%%%%%%%%%%%%%%%%%%%%%%%%%%%%%%%%%%%
%
% ENUMERATE
% =========
%
%%%%%%%%%%%%%%%%%%%%%%%%%%%%%%%%%%%%%%%%%%%%%%%%%%%%%%%%%%%%%%%%%%%%%%%%%%%%%%%%

\usepackage{enumerate}
\usepackage{enumitem}

% To have special enumerate items like
%
% 1/
% 2/
% 3/

%%%%%%%%%%%%%%%%%%%%%%%%%%%%%%%%%%%%%%%%%%%%%%%%%%%%%%%%%%%%%%%%%%%%%%%%%%%%%%%%
%
% ARRAYS
% ======
%
%%%%%%%%%%%%%%%%%%%%%%%%%%%%%%%%%%%%%%%%%%%%%%%%%%%%%%%%%%%%%%%%%%%%%%%%%%%%%%%%


\usepackage{array}
\usepackage{makecell} % Used to break lines within arrays
\usepackage{multirow}
\usepackage{booktabs} % Used to have nice arrays with headrules

%%%%%%%%%%%%%%%%%%%%%%%%%%%%%%%%%%%%%%%%%%%%%%%%%%%%%%%%%%%%%%%%%%%%%%%%%%%%%%%%
%
% WRITE CODE
% ==========
%
%%%%%%%%%%%%%%%%%%%%%%%%%%%%%%%%%%%%%%%%%%%%%%%%%%%%%%%%%%%%%%%%%%%%%%%%%%%%%%%%

\usepackage{listings}
\usepackage{xcolor}

%New colors defined below
\definecolor{codegreen}{rgb}{0,0.6,0}
\definecolor{codegray}{rgb}{0.5,0.5,0.5}
\definecolor{codepurple}{rgb}{0.58,0,0.82}
\definecolor{backcolour}{rgb}{0.95,0.95,0.92}

%Code listing style named "mystyle"
\lstdefinestyle{python}{
  %backgroundcolor=\color{backcolour},
  commentstyle=\color{codegreen},
  keywordstyle=\color{magenta},
  numberstyle=\tiny\color{codegray},
  stringstyle=\color{codepurple},
  basicstyle=\ttfamily\footnotesize,
  breakatwhitespace=false,
  breaklines=true,
  captionpos=b,
  keepspaces=true,
  numbers=left,
  numbersep=5pt,
  showspaces=false,
  showstringspaces=false,
  showtabs=false,
  tabsize=2
}

\lstset{style=python}

%%%%%%%%%%%%%%%%%%%%%%%%%%%%%%%%%%%%%%%%%%%%%%%%%%%%%%%%%%%%%%%%%%%%%%%%%%%%%%%%
%
% Tabular 
% =======
%
%%%%%%%%%%%%%%%%%%%%%%%%%%%%%%%%%%%%%%%%%%%%%%%%%%%%%%%%%%%%%%%%%%%%%%%%%%%%%%%%

% In order to obtain a tabular with given width.

\usepackage{tabularx}
\newcolumntype{Y}{>{\centering\arraybackslash}X}
\newcolumntype{R}{>{\raggedright\arraybackslash}X}
\newcolumntype{L}{>{\raggedleft\arraybackslash}X}
% \usepackage{tabulary} % younger brother

%%%%%%%%%%%%%%%%%%%%%%%%%%%%%%%%%%%%%%%%%%%%%%%%%%%%%%%%%%%%%%%%%%%%%%%%%%%%%%%%
%
% MACROS
% ======
%
%%%%%%%%%%%%%%%%%%%%%%%%%%%%%%%%%%%%%%%%%%%%%%%%%%%%%%%%%%%%%%%%%%%%%%%%%%%%%%%%

% Math Operators

\DeclareMathOperator{\Card}{Card}
\DeclareMathOperator{\Gal}{Gal}
\DeclareMathOperator{\Id}{Id}
\DeclareMathOperator{\Img}{Im}
\DeclareMathOperator{\Ker}{Ker}
\DeclareMathOperator{\Minpoly}{Minpoly}
\DeclareMathOperator{\Mod}{mod}
\DeclareMathOperator{\Ord}{Ord}
\DeclareMathOperator{\ppcm}{ppcm}
\DeclareMathOperator{\pgcd}{pgcd}
\DeclareMathOperator{\tr}{Tr}
\DeclareMathOperator{\Vect}{Vect}
\DeclareMathOperator{\Span}{Span}
\DeclareMathOperator{\rank}{rank}
\DeclareMathOperator{\rg}{rg}
\DeclareMathOperator{\ev}{ev}
\DeclareMathOperator{\Var}{Var}

% Shortcuts

\newcommand{\eg}{\emph{e.g. }}
\newcommand{\ent}[2]{[\![#1,#2]\!]}
\newcommand{\ie}{\emph{i.e. }}
\newcommand{\ps}[2]{\left\langle#1,#2\right\rangle}
\newcommand{\eqdef}{\overset{\text{def}}{=}}
\newcommand{\E}{\mathcal{E}}
\newcommand{\M}{\mathcal{M}}
\newcommand{\A}{\mathcal{A}}
\newcommand{\B}{\mathcal{B}}
\newcommand{\R}{\mathcal{R}}
\newcommand{\D}{\mathcal{D}}
\newcommand{\Pcal}{\mathcal{P}}
\newcommand{\K}{\mathbf{k}}
\newcommand{\vect}[1]{\overrightarrow{#1}}


%\input{layout-nb.tex}

% TODO: ajouter un ou deux exercices sur un problème avec des ventes/bénéfices

\begin{document}
\begin{exo}
  Préciser si la fonction définie sur $\mathbb{R}$ est une fonction polynôme du
  second degré. Si oui, identifier les coefficients $a, b, c$ dans l'expression
  $ax^2+bx+c$.
  \begin{align*}
    \textbf{a)}&\; f(x) = 3x(x+2)-5x &
    \textbf{b)}&\; g(x) = (2x+1)^2-4x^2 \\
    \textbf{c)}&\; h(x) = (x-2)^2-(x+2)^2 &
    \textbf{d)}&\; k(x) = 5(x^2-3)
  \end{align*}
\end{exo}

\begin{exo}
  Compléter pour mettre sous forme canonique.
  \begin{align*}
    \textbf{a)}&\; x^2-2x+3 = (x-\ldots)^2+\ldots &
    \textbf{b)}&\; x^2+2x+3 = (x-\ldots)^2+\ldots \\
    \textbf{c)}&\; x^2+2x-3 = (x-\ldots)^2-\ldots & 
    \textbf{d)}&\; 3x^2-6x+1 = \ldots(x-\ldots)^2+\ldots \\
    \textbf{e)}&\; 3x^2+6x+1 = \ldots(x-\ldots)^2+\ldots &
    \textbf{f)}&\; 3x^2-6x-1 = \ldots(x-\ldots)^2+\ldots
  \end{align*}
\end{exo}

\begin{exo}
  Pour les fonctions suivantes, déterminez la forme canonique, puis les
  variations et le maximum ou minimum. Vérifiez vos résultats en regardant la
  représentation graphique de la fonction.
  \begin{enumerate}
    \item La fonction $f_1$ définie sur $\mathbb{R}$ par $f_1(x)=x^2-2x+1$.
    \item La fonction $f_2$ définie sur $\mathbb{R}$ par $f_2(x)=-x^2+4x-5$.
  \end{enumerate}
\end{exo}

\begin{exo}~
  \begin{enumerate}
    \item Montrer que $4(x-1,5)^2-9$ est la forme canonique de la fonction $g$ définie
  sur $\mathbb{R}$ par
  \[
    g(x) = 4x^2-12x.
  \]
\item En déduire le tableau de variations de $g$.
  \end{enumerate}
\end{exo}

\begin{exo}~\\[-10mm]
  \begin{minipage}{.47\textwidth}
  Une fonction $f$ polynôme du second degré est représentée graphiquement
  ci-contre sur l'intervalle $[0; 5]$.
  \begin{itemize}
    \item Déduire de cette représentation graphique la forme canonique de de la
      fonction $f$.
  \end{itemize}
\end{minipage}
\begin{minipage}{.47\textwidth}
  \begin{center}
\begin{tikzpicture}
\begin{axis}
\addplot[red, very thick, samples=201, domain=0:5]{-.5*(x-2)^2+3};
\end{axis}
\end{tikzpicture}
\end{center}
\end{minipage}
\end{exo}

\begin{exo}
  La quantité de sucre $q(x)$ (en kg) présente dans $100$ kg de betteraves
  sucrières est donnée par
  \[
    q(x) = -0,004x^2+x-40
  \]
  où $x$ est la masse (en kg) d'engrais répandue à l'hectare, avec $x\in[60;
  180]$.
  \begin{enumerate}
    \item Montrer que, pour tout $x\in[60; 180]$, on a
      \[
        q(x) = -0,004(x-125)^2+22,5.
      \]
    \item En déduire, à l'aide du tableau de variations de $q$, la masse $x$
      d'engrais répandue à l'hectare pour que la quantité du sucre soit
      maximale.
  \end{enumerate}
\end{exo}

\begin{exo}
  Résoudre les équations suivantes dans $\mathbb{R}$ en utilisant la méthode la
  plus pertinente.
  \begin{align*}
    \textbf{a)}\;& -2x^2 - 5x +3 = 0 &
    \textbf{b)}\;& x^2 + 7x = 0 \\
    \textbf{c)}\;& 5x^2 + 7x +18 = 0 &
    \textbf{d)}\;& x^2 +x +1 = 0
  \end{align*}
\end{exo}
\newpage
\begin{exo}~
  \begin{enumerate}
    \item Soit $f$ la fonction définie sur $\mathbb{R}$ par
      \[
        f(x) = x^2+3x-5.
      \]
      \begin{enumerate}
        \item Tracer la courbe représentative de $f$ avec la calculatrice.
        \item Donner les valeurs approchées des éventuelles solutions dans
          $\mathbb{R}$ de l'équation $f(x)=0$.
        \item Résoudre dans $\mathbb{R}$ l'équation $f(x)=0$.
      \end{enumerate}
    \item Soit $g$ la fonction définie sur $\mathbb{R}$ par
      \[
        g(x) = 2x^2+3.
      \]
      \begin{enumerate}
        \item Résoudre dans $\mathbb{R}$ l'équation $f(x)=g(x)$.
        \item Que peut-on déduire pour les courbes représentatives de $f$ et $g$
          ? Vérifier à l'aide de la calculatrice.
      \end{enumerate}
  \end{enumerate}
\end{exo}

\begin{exo}
  Pour chacune des fonctions polynôme du second degré, déterminer ses racines
  éventuelles et une forme factorisée le cas échéant.
  \begin{align*}
    \textbf{a)}&\; f:x\mapsto 4x^2+x+9 &
    \textbf{b)}&\; g:x\mapsto 4x^2+13x+9
  \end{align*}
\end{exo}

\begin{exo}~
  \begin{enumerate}
    \item Dresser le tableau de variation des fonctions $f$ et $g$ suivantes définies
  sur $\mathbb{R}$.
\item Résoudre dans $\mathbb{R}$ les inéquations $f(x)\geq0$ et $g(x)<0$.
  \end{enumerate}
  \begin{align*}
    \textbf{a)}&\; f(x) = \frac{1}{2}x^2-6x+18 &
    \textbf{b)}&\; g(x) = -2x^2+8x-6
  \end{align*}
\end{exo}

\begin{exo}
  Résoudre les inéquations suivantes dans $\mathbb{R}$.
  \begin{align*}
    \textbf{a)}&\; 4x^2-7\leq 0 &
    \textbf{b)}&\; 3x^2-5x < 4x+5
  \end{align*}
\end{exo}

\begin{exo}
  Soit $f$ la fonction définie sur $\mathbb{R}$ par
  \[
    f(x) = \frac{10x}{x^2+2x+4}.
  \]
  \begin{enumerate}
    \item Justifier que $f$ est définie sur $\mathbb{R}$.
    \item À l'aide de la calculatrice, conjecturer le signe de la fonction $f$.
    \item Dresser le tableau de signes de $f(x)$, puis valider ou corriger la
      conjecture émise à la question précédente.
  \end{enumerate}
\end{exo}

\begin{exo}~
  \begin{enumerate}
    \item Soit $f$ la fonction définie sur $\mathbb{R}$ par
      \[
        f(x) = 3x^2-5x+2.
      \]
      On note $\mathscr C_f$ sa courbe représentative dans le plan muni d'un
      repère.
      \begin{enumerate}
        \item Conjecturer, à l'aide de la calculatrice, la position relative de
          la courbe $\mathscr C_f$ par rapport à l'axe des abscisses.
        \item Dresser par le calcul le tableau de signes de la fonction $f$,
          puis valider ou corriger la conjecture précédente.
      \end{enumerate}
    \item On considère maintenant la fonction $g$ définie sur $\mathbb{R}$ par
      \[
        g(x) = -x^2+x+1.
      \]
      \begin{enumerate}
        \item Étudier le signe de $f(x)-g(x)$ suivant les valeurs du nombre réel
          $x$.
        \item Que peut-on déduire pour les représentations graphiques de $f$ et
          $g$ ?
        \item Vérifier à l'aide de la calculatrice.
      \end{enumerate}
  \end{enumerate}
\end{exo}

\begin{exo}[$\star$]~\\[-8mm]
  \begin{minipage}{.7\textwidth}
  La figure ci-contre représente le logo d'une entreprise. Le quadrilatère $ABCD$ est un carré
  de côté $4$ cm. Les quadrilatères $AFKE$ et $KHCG$ sont aussi des carrés. Le
  créateur du logo souhaite que l'aire de la surface en bleu soit la plus petite
  possible.
  \begin{enumerate}
    \item À quel intervalle appartient $x$ ?
    \item Déterminer les longueurs $FK$ et $EK$ en fonction de $x$. En déduire
      l'aire du triangle $EFK$ en fonction de $x$.
  \end{enumerate}
\end{minipage}
\begin{minipage}{.3\textwidth}
\begin{center}
  \begin{tikzpicture}[scale=1]
  \node[label={$A$}] (A) at (0, 4) {};
  \node[label={$B$}] (B) at (4, 4) {};
  \node[label={[shift={(0, -.8)}]$C$}] (C) at (4, 0) {};
  \node[label={[shift={(0,-.8)}]$D$}] (D) at (0, 0) {};


  \node[blue!70!black, label={$F$}] (F) at (1, 4) {};
  \node[blue!70!black, label={[shift={(0, -.8)}]$G$}] (G) at (1, 0) {};
  \node[blue!70!black, label={[shift={(-.3, -.6)}]$E$}] (E) at (0, 3) {};
  \node[blue!70!black, label={[shift={(-.3, -.6)}]$K$}] (K) at (1, 3) {};
  \node[blue!70!black, label={[shift={(.3, -.6)}]$H$}] (H) at (4, 3) {};

  \draw[thick] (A.center) -- (B.center) -- (C.center) -- (D.center) -- (A.center);
  \draw[thick, blue!70!black] (G.center) -- (F.center) -- (E.center) -- (H.center);
  \fill[blue!70!black, opacity=.1] (E.center) -- (F.center) -- (K.center);
  \fill[blue!70!black, opacity=.1] (G.center) -- (K.center) -- (H.center) --
  (C.center);

  \draw[stealth-stealth] (A.north) -- node[above] {$x$} (F.north);
\end{tikzpicture}
\end{center}
\end{minipage}
\begin{enumerate}
    \setcounter{enumi}{2}
    \item Déterminer la longueur $KH$ en fonction de $x$ puis l'aire du
      carré $KHCG$ en fonction de $x$.
    \item En déduire l'aire de la figure bleue en fonction de $x$.
    \item Pour quelle valeur de $x$ la partie bleue a-t-elle la plus petite aire
      ?
\end{enumerate}
\end{exo}


\begin{exo}[$\star$]
  Une entreprise produit chaque jour une quantité $x$ de
chargeurs d'ordinateur comprise entre $0$ et $50$. Une étude a montré que le
coût total de production de $x$ chargeurs d'ordinateur est donné, en euro, par
\[
  C(x) = 3x^2-100x+900.
\]
Un chargeur d'ordinateur est vendu $20$ euros.
\begin{enumerate}
  \item Combien coûte la fabrication de $10$ chargeurs ?
  \item Combien rapporte la vente de $x$ chargeurs ?
  \item Montrer que le bénéfice (c'est-à-dire l'argent obtenu par la vente,
    moins les coûts de production) de $x$ chargeurs est
    \[
      B(x) = -3x^2+120x-900.
    \]
  \item \begin{enumerate}
      \item Déterminer la forme canonique de $B$.
      \item Dresser le tableau de variation de la fonction $B$ sur l'intervalle
        $\left[ 0; 50 \right]$.
      \item En déduire le bénéfice maximal de l'entreprise. Pour combien de
        chargeurs a-t-il lieu ?
    \end{enumerate}
  \item \begin{enumerate}
      \item Dresser le tableau de signes de la fonction $B$ sur l'intervalle $\left[
    0;50 \right]$.
  \item Combien de chargeurs l'entreprise doit-elle vendre pour être rentable
    \emph{(c'est-à-dire avoir un bénéfice positif)} ?
    \end{enumerate}
\end{enumerate}
\end{exo}

\begin{exo}[$\star\star$]
  Un club de vacances organise un weekend avec des
activités de plein air. Le nombre maximum de participants est fixé à $60$. Le
prix par personne est de $50$ euros pour les $30$ premiers. Pour tout participant
supplémentaire, chaque personne bénéficie d'une remise de $1$ euro. Par exemple, si
$35$ personnes s'inscrivent à ce weekend, le prix par personne sera de $45$
euros.\\
Pour quel nombre de participants le club gagnera-t-il le plus d'argent ?
\end{exo}

\begin{exo}[$\star\star$]~\\[-16mm]
  \begin{minipage}{.7\textwidth}
  $ABCD$ est un carré de côté $4$. Soit $x\in\left[ 0; 4 \right]$ et $E$ le
  point de $\left[ AB \right]$ tel que $AE=x$. Soit aussi $F$ le point de
  $\left[ AD \right]$ tel que $DF=x$.\\
  Déterminer la valeur de $x$ pour que l'aire du triangle $FEC$ soit minimale.
\end{minipage}
\begin{minipage}{.3\textwidth}
\begin{center}
  \begin{tikzpicture}[scale=1]
  \node[label={$A$}] (A) at (0, 4) {};
  \node[label={$B$}] (B) at (4, 4) {};
  \node[label={[shift={(0, -.8)}]$C$}] (C) at (4, 0) {};
  \node[label={[shift={(0,-.8)}]$D$}] (D) at (0, 0) {};


  \node[blue!70!black, label={$E$}] (E) at (1, 4) {};
  \node[blue!70!black, label={[shift={(-.2, -.2)}]$F$}] (F) at (0, 1) {};

  \draw[thick] (A.center) -- (B.center) -- (C.center) -- (D.center) -- (A.center);
  \draw[thick, blue!70!black] (C.center) -- (F.center) -- (E.center) -- (C.center);
  \fill[blue!70!black, opacity=.1] (C.center) -- (F.center) -- (E.center);

  \draw[stealth-stealth] (A.north) -- node[above] {$x$} (E.north);
  \draw[stealth-stealth] (F.west) -- node[left] {$x$} (D.west);
\end{tikzpicture}
\end{center}
\end{minipage}
\end{exo}

\begin{comment}
\begin{exo}[$\star\star$]
  Grégoire, $10$ ans, veut délimiter dans son jardin un
enclos rectangulaire pour son lapin nain. Son père lui donne $18$m de grillage.
\begin{enumerate}
  \item Déterminer les dimensions de cet enclos rectangulaire qui donnent une
    aire maximale.
  \item Quelle est alors la valeur de cette aire ?
\end{enumerate}
\end{exo}

\begin{exo}[$\star\star$]
  Soit $f$ la fonction définie sur $\mathbb{R}$ par
  \[
    f(x) = 15x^3-34x^2-47x+42.
  \]
  \begin{enumerate}
    \item À l'aide de la calculatrice, conjecturer une solution entière de
      l'équation $f(x)=0$.
    \item Déterminer les valeurs des nombres réels $a,b,c$ tels que, pour tout
      réel $x$, on ait
      \[
        f(x) = (x-3)(ax^2+bx+c).
      \]
    \item Résoudre dans $\mathbb{R}$ l'équation $f(x)=0$.
    \item Rechercher (sur internet, par exemple) s'il existe une méthode générale de résolution des
      équations du troisième degré.
  \end{enumerate}
\end{exo}
\end{comment}

\end{document}
