\documentclass[11pt]{article}

\newcommand{\titrechapitre}{Probabilités conditionnelles -- Exercices}
\newcommand{\titreclasse}{Lycée Jean-Baptiste \textsc{Corot}}
\newcommand{\pagination}{\thepage/\pageref{LastPage}}
\newcommand{\topbotmargins}{1.6cm}
\newcommand{\spacebelowexo}{3mm}
%%%%%%%%%%%%%%%%%%%%%%%%%%%%%%%%%%%%%%%%%%%%%%%%%%%%%%%%%%%%%%%%%%%%%%%%%%%%%%%%
%
% PACKAGES
% ========
%
%%%%%%%%%%%%%%%%%%%%%%%%%%%%%%%%%%%%%%%%%%%%%%%%%%%%%%%%%%%%%%%%%%%%%%%%%%%%%%%%

\usepackage[english, french]{babel}
\usepackage[utf8]{inputenc}
\usepackage[T1]{fontenc}
\usepackage{graphicx}
\usepackage{amsmath,amssymb,amsthm,amsopn}
\usepackage{hyperref}

% Pour avoir l'écriture \mathscr (math script)
% ============================================

\usepackage{mathrsfs}

% Deal with coma as a decimal separator
% =====================================

\usepackage{icomma}

% Package Geometry
% ================

\usepackage[a4paper, lmargin=2cm, rmargin=2cm, top=\topbotmargins, bottom=\topbotmargins]{geometry}

% Package multicol
% ================

\usepackage{multicol}

% Redefine abstract
% =================

% Note
% ====
%
% Le reste a été commenté pour ne pas charger trop de choses au démarrage. On
% verra si on en a besoin plus tard.
%
% --------
%
%\usepackage{mathrsfs}
%\usepackage{multirow}
%\usepackage{bm}
%\hypersetup{
%    colorlinks=true,
%    linkcolor=blue,
%    citecolor=red,
%}
%\usepackage{diagbox}
%
%\usepackage{algorithm}
%\usepackage{algpseudocode}
%
%\renewcommand{\algorithmicrequire}{\textbf{Input:}}
%\renewcommand{\algorithmicensure}{\textbf{Output:}}


%%%%%%%%%%%%%%%%%%%%%%%%%%%%%%%%%%%%%%%%%%%%%%%%%%%%%%%%%%%%%%%%%%%%%%%%%%%%%%%%
%
% TIKZ
% ====
%
%%%%%%%%%%%%%%%%%%%%%%%%%%%%%%%%%%%%%%%%%%%%%%%%%%%%%%%%%%%%%%%%%%%%%%%%%%%%%%%%

\usepackage{tikz}
\usetikzlibrary{arrows}

\usepackage{tkz-tab} % Variation tables

\usepackage{pgfplots}
%\usepackage{pgf-pie} % Pie charts

\pgfplotsset{
%\newcommand{\settingsgraph}{
x=.5cm,y=.5cm,
xticklabel style = {font=\scriptsize, yshift=.1cm},
yticklabel style = {font=\scriptsize, xshift=.1cm},
axis lines=middle,
ymajorgrids=true,
xmajorgrids=true,
major grid style = {color=white!80!blue},
xmin=-5.5,
xmax=5.5,
ymin=-5.5,
ymax=5.5,
xtick={-5.0,-4.0,...,5.0},
ytick={-5.0,-4.0,...,5.0},
}

% Tikz style

\tikzset{round/.style={circle, draw=black, very thick, scale = 0.7}}
\tikzset{arrow/.style={->, >=latex}}
\tikzset{dashed-arrow/.style={->, >=latex, dashed}}

\newcommand{\point}[3]{\draw[very thick, #3] (#1-.1, #2)--(#1+.1, #2)
(#1, #2-.1)--(#1, #2+.1)}

%%%%%%%%%%%%%%%%%%%%%%%%%%%%%%%%%%%%%%%%%%%%%%%%%%%%%%%%%%%%%%%%%%%%%%%%%%%%%%%%
%
% FANCY HEADER
% ============
%
%%%%%%%%%%%%%%%%%%%%%%%%%%%%%%%%%%%%%%%%%%%%%%%%%%%%%%%%%%%%%%%%%%%%%%%%%%%%%%%%


\usepackage{fancyhdr}
\usepackage{lastpage}

\pagestyle{fancy}
\newcommand{\changefont}{\fontsize{9}{9}\selectfont}
\renewcommand{\headrulewidth}{0mm}
\renewcommand{\footrulewidth}{0mm}

\fancyhead[C]{}
\fancyhead[L]{\titreclasse}
\fancyhead[R]{\titrechapitre}
\fancyfoot[C]{}
\fancyfoot[L]{}
\fancyfoot[R]{\pagination}
\addtolength{\skip\footins}{20pt} % distance between text and footnotes

%%%%%%%%%%%%%%%%%%%%%%%%%%%%%%%%%%%%%%%%%%%%%%%%%%%%%%%%%%%%%%%%%%%%%%%%%%%%%%%%
%
% THEOREM STYLE
% =============
%
%%%%%%%%%%%%%%%%%%%%%%%%%%%%%%%%%%%%%%%%%%%%%%%%%%%%%%%%%%%%%%%%%%%%%%%%%%%%%%%%

\usepackage[tikz]{bclogo}
\usepackage{mdframed}

\usepackage{tcolorbox}
\tcbuselibrary{listings, breakable, theorems, skins}

%\newtheoremstyle{break}%
%{}{}%
%{\itshape}{}%
%{\bfseries}{}%  % Note that final punctuation is omitted.
%{\newline}{}

\newtheoremstyle{scbf}%
{}{}%
{}{}%
%{\scshape}{}%  % Note that final punctuation is omitted.
{\bfseries\scshape}{}%  % Note that final punctuation is omitted.
{\newline}{}

%\theoremstyle{break}
%\theoremstyle{plain}
%\newtheorem{thm}{Theorem}[section]
%\newtheorem{lm}[thm]{Lemma}
%\newtheorem{prop}[thm]{Proposition}
%\newtheorem{cor}[thm]{Corollary}

%\theoremstyle{scbf}
%\newtheorem{exo}{$\star$ Exercice}

%\theoremstyle{definition}
%\newtheorem{defi}[thm]{Definition}
%\newtheorem{ex}[thm]{Example}

%\theoremstyle{remark}
%\newtheorem{rem}[thm]{Remark}

% Defining the Remark environment
% ===============================

\newenvironment{rmq}
  {
    \begin{bclogo}[logo=\bcinfo, noborder=true]{Remarque}
  }
  {
    \end{bclogo}
  }

% Defining the exercise environment
% =================================

\newcounter{exos}
\setcounter{exos}{1}

\newenvironment{exo}
  {
    \begin{bclogo}[logo=\bccrayon, noborder=true]{Exercice \theexos}
  }
  {
    \end{bclogo}
    \addtocounter{exos}{1}
  }


% Redefining the proof environment from amsthm
% ============================================

\tcolorboxenvironment{proof}{
  blanker, breakable, before skip=10pt,after skip=10pt,
  borderline west={1mm}{0pt}{red},
  left=5mm,
}

% Defining the definition environment
% ===================================

\colorlet{coldef}{black!50!green}

\newcounter{defis}
\setcounter{defis}{1}

\newenvironment{defi}[1]
  {
    \begin{defihid}{{#1}}{\thedefis}
  }
  {
    \end{defihid}
    \addtocounter{defis}{1}
  }

\newtcolorbox{defihid}[2]{%
  empty,title={ {\bfseries Définition {#2}} ({#1})},attach boxed title to top left,
boxed title style={empty,size=minimal,toprule=2pt,top=4pt,
overlay={\draw[coldef,line width=2pt]
([yshift=-1pt]frame.north west)--([yshift=-1pt]frame.north east);}},
coltitle=coldef,
before=\par\medskip\noindent,parbox=false,boxsep=0pt,left=0pt,right=3mm,top=4pt,
breakable,pad at break*=0mm,vfill before first,
overlay unbroken={\draw[coldef,line width=1pt]
([yshift=-1pt]title.north east)--([xshift=-0.5pt,yshift=-1pt]title.north-|frame.east)
--([xshift=-0.5pt]frame.south east)--(frame.south west); },
overlay first={\draw[coldef,line width=1pt]
([yshift=-1pt]title.north east)--([xshift=-0.5pt,yshift=-1pt]title.north-|frame.east)
--([xshift=-0.5pt]frame.south east); },
overlay middle={\draw[coldef,line width=1pt] ([xshift=-0.5pt]frame.north east)
--([xshift=-0.5pt]frame.south east); },
overlay last={\draw[coldef,line width=1pt] ([xshift=-0.5pt]frame.north east)
--([xshift=-0.5pt]frame.south east)--(frame.south west);},%
}

\newenvironment{notation}
  {
    \begin{notationhid}{\thedefis}
  }
  {
    \end{notationhid}
    \addtocounter{defis}{1}
  }

\newtcolorbox{notationhid}[1]{%
  empty,title={Notation {#1}},attach boxed title to top left,
boxed title style={empty,size=minimal,toprule=2pt,top=4pt,
overlay={\draw[coldef,line width=2pt]
([yshift=-1pt]frame.north west)--([yshift=-1pt]frame.north east);}},
coltitle=coldef,fonttitle=\bfseries,
before=\par\medskip\noindent,parbox=false,boxsep=0pt,left=0pt,right=3mm,top=4pt,
breakable,pad at break*=0mm,vfill before first,
overlay unbroken={\draw[coldef,line width=1pt]
([yshift=-1pt]title.north east)--([xshift=-0.5pt,yshift=-1pt]title.north-|frame.east)
--([xshift=-0.5pt]frame.south east)--(frame.south west); },
overlay first={\draw[coldef,line width=1pt]
([yshift=-1pt]title.north east)--([xshift=-0.5pt,yshift=-1pt]title.north-|frame.east)
--([xshift=-0.5pt]frame.south east); },
overlay middle={\draw[coldef,line width=1pt] ([xshift=-0.5pt]frame.north east)
--([xshift=-0.5pt]frame.south east); },
overlay last={\draw[coldef,line width=1pt] ([xshift=-0.5pt]frame.north east)
--([xshift=-0.5pt]frame.south east)--(frame.south west);},%
}


% Defining the proposition, theorem, etc. environment
% ===================================================

\colorlet{colprop}{red!75!black}

\newcounter{props}
\setcounter{props}{1}

\newenvironment{prop}
  {
    \begin{prophid}{\theprops}
  }
  {
    \end{prophid}
    \refstepcounter{props}
  }

\newtcolorbox{prophid}[1]{%
empty,title={Propriété {#1}},attach boxed title to top left,
boxed title style={empty,size=minimal,toprule=2pt,top=4pt,
overlay={\draw[colprop,line width=2pt]
([yshift=-1pt]frame.north west)--([yshift=-1pt]frame.north east);}},
coltitle=colprop,fonttitle=\bfseries,
before=\par\medskip\noindent,parbox=false,boxsep=0pt,left=0pt,right=3mm,top=4pt,
breakable,pad at break*=0mm,vfill before first,
overlay unbroken={\draw[colprop,line width=1pt]
([yshift=-1pt]title.north east)--([xshift=-0.5pt,yshift=-1pt]title.north-|frame.east)
--([xshift=-0.5pt]frame.south east)--(frame.south west); },
overlay first={\draw[colprop,line width=1pt]
([yshift=-1pt]title.north east)--([xshift=-0.5pt,yshift=-1pt]title.north-|frame.east)
--([xshift=-0.5pt]frame.south east); },
overlay middle={\draw[colprop,line width=1pt] ([xshift=-0.5pt]frame.north east)
--([xshift=-0.5pt]frame.south east); },
overlay last={\draw[colprop,line width=1pt] ([xshift=-0.5pt]frame.north east)
--([xshift=-0.5pt]frame.south east)--(frame.south west);},%
}

\newenvironment{propadm}
  {
    \begin{propadmhid}{\theprops}
  }
  {
    \end{propadmhid}
    \refstepcounter{props}
  }

  \newtcolorbox{propadmhid}[1]{%
    empty,title={{\bfseries Propriété {#1}} (admise)},attach boxed title to top left,
boxed title style={empty,size=minimal,toprule=2pt,top=4pt,
overlay={\draw[colprop,line width=2pt]
([yshift=-1pt]frame.north west)--([yshift=-1pt]frame.north east);}},
coltitle=colprop,%fonttitle=\bfseries,
before=\par\medskip\noindent,parbox=false,boxsep=0pt,left=0pt,right=3mm,top=4pt,
breakable,pad at break*=0mm,vfill before first,
overlay unbroken={\draw[colprop,line width=1pt]
([yshift=-1pt]title.north east)--([xshift=-0.5pt,yshift=-1pt]title.north-|frame.east)
--([xshift=-0.5pt]frame.south east)--(frame.south west); },
overlay first={\draw[colprop,line width=1pt]
([yshift=-1pt]title.north east)--([xshift=-0.5pt,yshift=-1pt]title.north-|frame.east)
--([xshift=-0.5pt]frame.south east); },
overlay middle={\draw[colprop,line width=1pt] ([xshift=-0.5pt]frame.north east)
--([xshift=-0.5pt]frame.south east); },
overlay last={\draw[colprop,line width=1pt] ([xshift=-0.5pt]frame.north east)
--([xshift=-0.5pt]frame.south east)--(frame.south west);},%
}

\newenvironment{propnom}[1]
  {
    \begin{propnomhid}{#1}{\theprops}
  }
  {
    \end{propnomhid}
    \refstepcounter{props}
  }

\newtcolorbox{propnomhid}[2]{%
empty,title={{\bfseries Propriété {#2}} ({#1})},attach boxed title to top left,
boxed title style={empty,size=minimal,toprule=2pt,top=4pt,
overlay={\draw[colprop,line width=2pt]
([yshift=-1pt]frame.north west)--([yshift=-1pt]frame.north east);}},
coltitle=colprop,
before=\par\medskip\noindent,parbox=false,boxsep=0pt,left=0pt,right=3mm,top=4pt,
breakable,pad at break*=0mm,vfill before first,
overlay unbroken={\draw[colprop,line width=1pt]
([yshift=-1pt]title.north east)--([xshift=-0.5pt,yshift=-1pt]title.north-|frame.east)
--([xshift=-0.5pt]frame.south east)--(frame.south west); },
overlay first={\draw[colprop,line width=1pt]
([yshift=-1pt]title.north east)--([xshift=-0.5pt,yshift=-1pt]title.north-|frame.east)
--([xshift=-0.5pt]frame.south east); },
overlay middle={\draw[colprop,line width=1pt] ([xshift=-0.5pt]frame.north east)
--([xshift=-0.5pt]frame.south east); },
overlay last={\draw[colprop,line width=1pt] ([xshift=-0.5pt]frame.north east)
--([xshift=-0.5pt]frame.south east)--(frame.south west);},%
}




\newenvironment{thm}
  {
    \begin{thmhid}{\theprops}
  }
  {
    \end{thmhid}
    \refstepcounter{props}
  }

\newtcolorbox{thmhid}[1]{%
empty,title={Théorème {#1}},attach boxed title to top left,
boxed title style={empty,size=minimal,toprule=2pt,top=4pt,
overlay={\draw[colprop,line width=2pt]
([yshift=-1pt]frame.north west)--([yshift=-1pt]frame.north east);}},
coltitle=colprop,fonttitle=\bfseries,
before=\par\medskip\noindent,parbox=false,boxsep=0pt,left=0pt,right=3mm,top=4pt,
breakable,pad at break*=0mm,vfill before first,
overlay unbroken={\draw[colprop,line width=1pt]
([yshift=-1pt]title.north east)--([xshift=-0.5pt,yshift=-1pt]title.north-|frame.east)
--([xshift=-0.5pt]frame.south east)--(frame.south west); },
overlay first={\draw[colprop,line width=1pt]
([yshift=-1pt]title.north east)--([xshift=-0.5pt,yshift=-1pt]title.north-|frame.east)
--([xshift=-0.5pt]frame.south east); },
overlay middle={\draw[colprop,line width=1pt] ([xshift=-0.5pt]frame.north east)
--([xshift=-0.5pt]frame.south east); },
overlay last={\draw[colprop,line width=1pt] ([xshift=-0.5pt]frame.north east)
--([xshift=-0.5pt]frame.south east)--(frame.south west);},%
}

\newenvironment{thmadm}
  {
    \begin{thmadmhid}{\theprops}
  }
  {
    \end{thmadmhid}
    \refstepcounter{props}
  }

  \newtcolorbox{thmadmhid}[1]{%
    empty,title={{\bfseries Théorème {#1}} (admis)},attach boxed title to top left,
boxed title style={empty,size=minimal,toprule=2pt,top=4pt,
overlay={\draw[colprop,line width=2pt]
([yshift=-1pt]frame.north west)--([yshift=-1pt]frame.north east);}},
coltitle=colprop,%fonttitle=\bfseries,
before=\par\medskip\noindent,parbox=false,boxsep=0pt,left=0pt,right=3mm,top=4pt,
breakable,pad at break*=0mm,vfill before first,
overlay unbroken={\draw[colprop,line width=1pt]
([yshift=-1pt]title.north east)--([xshift=-0.5pt,yshift=-1pt]title.north-|frame.east)
--([xshift=-0.5pt]frame.south east)--(frame.south west); },
overlay first={\draw[colprop,line width=1pt]
([yshift=-1pt]title.north east)--([xshift=-0.5pt,yshift=-1pt]title.north-|frame.east)
--([xshift=-0.5pt]frame.south east); },
overlay middle={\draw[colprop,line width=1pt] ([xshift=-0.5pt]frame.north east)
--([xshift=-0.5pt]frame.south east); },
overlay last={\draw[colprop,line width=1pt] ([xshift=-0.5pt]frame.north east)
--([xshift=-0.5pt]frame.south east)--(frame.south west);},%
}

\newenvironment{thmnom}[1]
  {
    \begin{thmnomhid}{#1}{\theprops}
  }
  {
    \end{thmnomhid}
    \refstepcounter{props}
  }

\newtcolorbox{thmnomhid}[2]{%
empty,title={{\bfseries Théorème {#2}} ({#1})},attach boxed title to top left,
boxed title style={empty,size=minimal,toprule=2pt,top=4pt,
overlay={\draw[colprop,line width=2pt]
([yshift=-1pt]frame.north west)--([yshift=-1pt]frame.north east);}},
coltitle=colprop,
before=\par\medskip\noindent,parbox=false,boxsep=0pt,left=0pt,right=3mm,top=4pt,
breakable,pad at break*=0mm,vfill before first,
overlay unbroken={\draw[colprop,line width=1pt]
([yshift=-1pt]title.north east)--([xshift=-0.5pt,yshift=-1pt]title.north-|frame.east)
--([xshift=-0.5pt]frame.south east)--(frame.south west); },
overlay first={\draw[colprop,line width=1pt]
([yshift=-1pt]title.north east)--([xshift=-0.5pt,yshift=-1pt]title.north-|frame.east)
--([xshift=-0.5pt]frame.south east); },
overlay middle={\draw[colprop,line width=1pt] ([xshift=-0.5pt]frame.north east)
--([xshift=-0.5pt]frame.south east); },
overlay last={\draw[colprop,line width=1pt] ([xshift=-0.5pt]frame.north east)
--([xshift=-0.5pt]frame.south east)--(frame.south west);},%
}

\newenvironment{coro}
  {
    \begin{corohid}{\theprops}
  }
  {
    \end{corohid}
    \refstepcounter{props}
  }

  \newtcolorbox{corohid}[1]{%
  empty,title={Corollaire {#1}},attach boxed title to top left,
boxed title style={empty,size=minimal,toprule=2pt,top=4pt,
overlay={\draw[colprop,line width=2pt]
([yshift=-1pt]frame.north west)--([yshift=-1pt]frame.north east);}},
coltitle=colprop,fonttitle=\bfseries,
before=\par\medskip\noindent,parbox=false,boxsep=0pt,left=0pt,right=3mm,top=4pt,
breakable,pad at break*=0mm,vfill before first,
overlay unbroken={\draw[colprop,line width=1pt]
([yshift=-1pt]title.north east)--([xshift=-0.5pt,yshift=-1pt]title.north-|frame.east)
--([xshift=-0.5pt]frame.south east)--(frame.south west); },
overlay first={\draw[colprop,line width=1pt]
([yshift=-1pt]title.north east)--([xshift=-0.5pt,yshift=-1pt]title.north-|frame.east)
--([xshift=-0.5pt]frame.south east); },
overlay middle={\draw[colprop,line width=1pt] ([xshift=-0.5pt]frame.north east)
--([xshift=-0.5pt]frame.south east); },
overlay last={\draw[colprop,line width=1pt] ([xshift=-0.5pt]frame.north east)
--([xshift=-0.5pt]frame.south east)--(frame.south west);},%
}

\newenvironment{lemme}
  {
    \begin{lemmehid}{\theprops}
  }
  {
    \end{lemmehid}
    \refstepcounter{props}
  }

  \newtcolorbox{lemmehid}[1]{%
  empty,title={Lemme {#1}},attach boxed title to top left,
boxed title style={empty,size=minimal,toprule=2pt,top=4pt,
overlay={\draw[colprop,line width=2pt]
([yshift=-1pt]frame.north west)--([yshift=-1pt]frame.north east);}},
coltitle=colprop,fonttitle=\bfseries,
before=\par\medskip\noindent,parbox=false,boxsep=0pt,left=0pt,right=3mm,top=4pt,
breakable,pad at break*=0mm,vfill before first,
overlay unbroken={\draw[colprop,line width=1pt]
([yshift=-1pt]title.north east)--([xshift=-0.5pt,yshift=-1pt]title.north-|frame.east)
--([xshift=-0.5pt]frame.south east)--(frame.south west); },
overlay first={\draw[colprop,line width=1pt]
([yshift=-1pt]title.north east)--([xshift=-0.5pt,yshift=-1pt]title.north-|frame.east)
--([xshift=-0.5pt]frame.south east); },
overlay middle={\draw[colprop,line width=1pt] ([xshift=-0.5pt]frame.north east)
--([xshift=-0.5pt]frame.south east); },
overlay last={\draw[colprop,line width=1pt] ([xshift=-0.5pt]frame.north east)
--([xshift=-0.5pt]frame.south east)--(frame.south west);},%
}

\colorlet{colexemple}{blue!50!black}
%\newtcolorbox{exemple}{empty, title=Exemple, attach boxed title to top left,
%  boxed title style={empty, size=minimal, toprule=2pt, top=4pt,
%    overlay={\draw[colexemple,line width=2pt]
%([yshift=-1pt]frame.north west)--([yshift=-1pt]frame.north east);}},
%coltitle=colexemple,fonttitle=\bfseries,%\large\bfseries,
%before=\par\medskip\noindent,parbox=false,boxsep=0pt,left=0pt,right=3mm,top=4pt,
%overlay={\draw[colexemple,line width=1pt]
%([yshift=-1pt]title.north east)--([xshift=-0.5pt,yshift=-1pt]title.north-|frame.east)
%--([xshift=-0.5pt]frame.south east)--(frame.south west); },
%}

\newcounter{exemples}
\setcounter{exemples}{1}

\newenvironment{exemple}
  {
    \begin{exemplehid}{\theexemples}
  }
  {
    \end{exemplehid}
    \addtocounter{exemples}{1}
  }

\newtcolorbox{exemplehid}[1]{%
empty,title={Exemple {#1}},attach boxed title to top left,
boxed title style={empty,size=minimal,toprule=2pt,top=4pt,
overlay={\draw[colexemple,line width=2pt]
([yshift=-1pt]frame.north west)--([yshift=-1pt]frame.north east);}},
coltitle=colexemple,fonttitle=\bfseries,
before=\par\medskip\noindent,parbox=false,boxsep=0pt,left=0pt,right=3mm,top=4pt,
breakable,pad at break*=0mm,vfill before first,
overlay unbroken={\draw[colexemple,line width=1pt]
([yshift=-1pt]title.north east)--([xshift=-0.5pt,yshift=-1pt]title.north-|frame.east)
--([xshift=-0.5pt]frame.south east)--(frame.south west); },
overlay first={\draw[colexemple,line width=1pt]
([yshift=-1pt]title.north east)--([xshift=-0.5pt,yshift=-1pt]title.north-|frame.east)
--([xshift=-0.5pt]frame.south east); },
overlay middle={\draw[colexemple,line width=1pt] ([xshift=-0.5pt]frame.north east)
--([xshift=-0.5pt]frame.south east); },
overlay last={\draw[colexemple,line width=1pt] ([xshift=-0.5pt]frame.north east)
--([xshift=-0.5pt]frame.south east)--(frame.south west);},%
}

\newenvironment{contrex}
  {
    \begin{contrexhid}{\theexemples}
  }
  {
    \end{contrexhid}
    \addtocounter{exemples}{1}
  }

\newtcolorbox{contrexhid}[1]{%
empty,title={Contre-exemple {#1}},attach boxed title to top left,
boxed title style={empty,size=minimal,toprule=2pt,top=4pt,
overlay={\draw[colexemple,line width=2pt]
([yshift=-1pt]frame.north west)--([yshift=-1pt]frame.north east);}},
coltitle=colexemple,fonttitle=\bfseries,
before=\par\medskip\noindent,parbox=false,boxsep=0pt,left=0pt,right=3mm,top=4pt,
breakable,pad at break*=0mm,vfill before first,
overlay unbroken={\draw[colexemple,line width=1pt]
([yshift=-1pt]title.north east)--([xshift=-0.5pt,yshift=-1pt]title.north-|frame.east)
--([xshift=-0.5pt]frame.south east)--(frame.south west); },
overlay first={\draw[colexemple,line width=1pt]
([yshift=-1pt]title.north east)--([xshift=-0.5pt,yshift=-1pt]title.north-|frame.east)
--([xshift=-0.5pt]frame.south east); },
overlay middle={\draw[colexemple,line width=1pt] ([xshift=-0.5pt]frame.north east)
--([xshift=-0.5pt]frame.south east); },
overlay last={\draw[colexemple,line width=1pt] ([xshift=-0.5pt]frame.north east)
--([xshift=-0.5pt]frame.south east)--(frame.south west);},%
}

\newenvironment{app}
  {
    \begin{apphid}{\theexemples}
  }
  {
    \end{apphid}
    \addtocounter{exemples}{1}
  }

\newtcolorbox{apphid}[1]{%
empty,title={Application {#1}},attach boxed title to top left,
boxed title style={empty,size=minimal,toprule=2pt,top=4pt,
overlay={\draw[colexemple,line width=2pt]
([yshift=-1pt]frame.north west)--([yshift=-1pt]frame.north east);}},
coltitle=colexemple,fonttitle=\bfseries,
before=\par\medskip\noindent,parbox=false,boxsep=0pt,left=0pt,right=3mm,top=4pt,
breakable,pad at break*=0mm,vfill before first,
overlay unbroken={\draw[colexemple,line width=1pt]
([yshift=-1pt]title.north east)--([xshift=-0.5pt,yshift=-1pt]title.north-|frame.east)
--([xshift=-0.5pt]frame.south east)--(frame.south west); },
overlay first={\draw[colexemple,line width=1pt]
([yshift=-1pt]title.north east)--([xshift=-0.5pt,yshift=-1pt]title.north-|frame.east)
--([xshift=-0.5pt]frame.south east); },
overlay middle={\draw[colexemple,line width=1pt] ([xshift=-0.5pt]frame.north east)
--([xshift=-0.5pt]frame.south east); },
overlay last={\draw[colexemple,line width=1pt] ([xshift=-0.5pt]frame.north east)
--([xshift=-0.5pt]frame.south east)--(frame.south west);},%
}

%%%%%%%%%%%%%%%%%%%%%%%%%%%%%%%%%%%%%%%%%%%%%%%%%%%%%%%%%%%%%%%%%%%%%%%%%%%%%%%%
%
% ENUMERATE
% =========
%
%%%%%%%%%%%%%%%%%%%%%%%%%%%%%%%%%%%%%%%%%%%%%%%%%%%%%%%%%%%%%%%%%%%%%%%%%%%%%%%%

\usepackage{enumerate}
\usepackage{enumitem}

% To have special enumerate items like
%
% 1/
% 2/
% 3/

%%%%%%%%%%%%%%%%%%%%%%%%%%%%%%%%%%%%%%%%%%%%%%%%%%%%%%%%%%%%%%%%%%%%%%%%%%%%%%%%
%
% ARRAYS
% ======
%
%%%%%%%%%%%%%%%%%%%%%%%%%%%%%%%%%%%%%%%%%%%%%%%%%%%%%%%%%%%%%%%%%%%%%%%%%%%%%%%%


\usepackage{array}
\usepackage{makecell} % Used to break lines within arrays
\usepackage{multirow}
\usepackage{booktabs} % Used to have nice arrays with headrules

%%%%%%%%%%%%%%%%%%%%%%%%%%%%%%%%%%%%%%%%%%%%%%%%%%%%%%%%%%%%%%%%%%%%%%%%%%%%%%%%
%
% WRITE CODE
% ==========
%
%%%%%%%%%%%%%%%%%%%%%%%%%%%%%%%%%%%%%%%%%%%%%%%%%%%%%%%%%%%%%%%%%%%%%%%%%%%%%%%%

\usepackage{listings}
\usepackage{xcolor}

%New colors defined below
\definecolor{codegreen}{rgb}{0,0.6,0}
\definecolor{codegray}{rgb}{0.5,0.5,0.5}
\definecolor{codepurple}{rgb}{0.58,0,0.82}
\definecolor{backcolour}{rgb}{0.95,0.95,0.92}

%Code listing style named "mystyle"
\lstdefinestyle{python}{
  %backgroundcolor=\color{backcolour},
  commentstyle=\color{codegreen},
  keywordstyle=\color{magenta},
  numberstyle=\tiny\color{codegray},
  stringstyle=\color{codepurple},
  basicstyle=\ttfamily\footnotesize,
  breakatwhitespace=false,
  breaklines=true,
  captionpos=b,
  keepspaces=true,
  numbers=left,
  numbersep=5pt,
  showspaces=false,
  showstringspaces=false,
  showtabs=false,
  tabsize=2
}

\lstset{style=python}

%%%%%%%%%%%%%%%%%%%%%%%%%%%%%%%%%%%%%%%%%%%%%%%%%%%%%%%%%%%%%%%%%%%%%%%%%%%%%%%%
%
% Tabular 
% =======
%
%%%%%%%%%%%%%%%%%%%%%%%%%%%%%%%%%%%%%%%%%%%%%%%%%%%%%%%%%%%%%%%%%%%%%%%%%%%%%%%%

% In order to obtain a tabular with given width.

\usepackage{tabularx}
\newcolumntype{Y}{>{\centering\arraybackslash}X}
\newcolumntype{R}{>{\raggedright\arraybackslash}X}
\newcolumntype{L}{>{\raggedleft\arraybackslash}X}
% \usepackage{tabulary} % younger brother

%%%%%%%%%%%%%%%%%%%%%%%%%%%%%%%%%%%%%%%%%%%%%%%%%%%%%%%%%%%%%%%%%%%%%%%%%%%%%%%%
%
% MACROS
% ======
%
%%%%%%%%%%%%%%%%%%%%%%%%%%%%%%%%%%%%%%%%%%%%%%%%%%%%%%%%%%%%%%%%%%%%%%%%%%%%%%%%

% Math Operators

\DeclareMathOperator{\Card}{Card}
\DeclareMathOperator{\Gal}{Gal}
\DeclareMathOperator{\Id}{Id}
\DeclareMathOperator{\Img}{Im}
\DeclareMathOperator{\Ker}{Ker}
\DeclareMathOperator{\Minpoly}{Minpoly}
\DeclareMathOperator{\Mod}{mod}
\DeclareMathOperator{\Ord}{Ord}
\DeclareMathOperator{\ppcm}{ppcm}
\DeclareMathOperator{\pgcd}{pgcd}
\DeclareMathOperator{\tr}{Tr}
\DeclareMathOperator{\Vect}{Vect}
\DeclareMathOperator{\Span}{Span}
\DeclareMathOperator{\rank}{rank}
\DeclareMathOperator{\rg}{rg}
\DeclareMathOperator{\ev}{ev}
\DeclareMathOperator{\Var}{Var}

% Shortcuts

\newcommand{\eg}{\emph{e.g. }}
\newcommand{\ent}[2]{[\![#1,#2]\!]}
\newcommand{\ie}{\emph{i.e. }}
\newcommand{\ps}[2]{\left\langle#1,#2\right\rangle}
\newcommand{\eqdef}{\overset{\text{def}}{=}}
\newcommand{\E}{\mathcal{E}}
\newcommand{\M}{\mathcal{M}}
\newcommand{\A}{\mathcal{A}}
\newcommand{\B}{\mathcal{B}}
\newcommand{\R}{\mathcal{R}}
\newcommand{\D}{\mathcal{D}}
\newcommand{\Pcal}{\mathcal{P}}
\newcommand{\K}{\mathbf{k}}
\newcommand{\vect}[1]{\overrightarrow{#1}}



% TODO: ajouter un exercice avec trois branches, genre celui du feu tricolore.
% Ça manque un peu en l'état.

\begin{document}

\begin{exo}
On considère $A$ et $B$ deux événements. On rappelle que
\[
  P(A\cup B)=P(A)+P(B)-P(A\cap B).
\]
\begin{enumerate}
  \item Si $P(A)=0,5$, $P(\overline B)=0,7$ et $P(A\cap B)=0,12$, calculer
    $P(B)$ puis $P(A\cup B)$.
  \item Si $P(A)=\frac{1}{4}$, $P(B)=\frac{3}{5}$ et $P(A\cup
    B)=\frac{7}{10}$, que vaut $P(A\cap B)$ ?
\end{enumerate}
\end{exo}

\begin{exo}
Quand il commande une pizza à emporter, Jonas a remarqué
que le temps d'attente annoncé est $5$, $10$, $15$ ou $20$ minutes avec les
probabilités $p_5=0,3$, $p_{10}=0,2$, $p_{15} = 0,1$ et
$p_{20}$.
\begin{enumerate}
  \item Déterminer la probabilité, notée $p_{20}$, que le temps d'attente soit de $20$ minutes.
  \item Quelle est la probabilité d'attendre $10$ minutes ou moins ?
\end{enumerate}
\end{exo}

\begin{exo}
{\small Dans son placard, Valérie a des bols et des tasses avec
ou sans anse selon la répartition ci-dessous.}\\
\begin{minipage}{.6\textwidth}
Le matin, elle prend un de ces
récipients au hasard pour prendre son café et on considère les événements
\begin{itemize}
  \item $A$ : « Le récipient a une anse. »
  \item $B$ : « Le récipient est un bol. »
\end{itemize}
\end{minipage}
\begin{minipage}{.4\textwidth}
  \begin{center}
 \begin{tabular}{cccc}
  \toprule
  & \textbf{Bol} & \textbf{Tasse} & \textbf{Total} \\
  \midrule
  \textbf{Avec anse} & $2$ & $9$ & $11$ \\
  \textbf{Sans anse} & $6$ & $3$ & $9$ \\
  \textbf{Total} & $8$ & $12$ & $20$ \\
  \bottomrule
\end{tabular}
  \end{center}
\end{minipage}
\begin{enumerate}
  \item Déterminer les probabilités $P(A)$ et $P(B)$.
  \item Décrire chacun des événements $A\cap B$, $A\cup B$, $\overline A\cap B$
    par une phrase et donner sa probabilité.
  \item Écrire l'événement « Le récipient est une tasse sans anse » à l'aise des
    événements $A$ et $B$.
  \item Associer chacune des phrases suivantes à la valeur qui lui correspond.\\
    \begin{minipage}[]{.6\textwidth}
      \begin{center}
        \begin{enumerate}
          \item Probabilité qu'une tasse ait une anse
          \item Probabilité qu'un récipient à anse soit une tasse
          \item Probabilité qu'un récipient soit une tasse à anse
        \end{enumerate}
      \end{center}
    \end{minipage}
    \hfill
    \begin{minipage}[]{.2\textwidth}
      \begin{center}
        \begin{enumerate}[label=(\arabic*)]
          \item $\frac{9}{20}$ 
          \item $\frac{9}{11}$ 
          \item $\frac{9}{12}$ 
        \end{enumerate}
      \end{center}
    \end{minipage}
\end{enumerate}
\end{exo}

\begin{exo}~\\
\begin{minipage}{.5\textwidth}
  Lors d'une enquête portant sur les $2000$ salariés d'une entreprise, on a
  obtenu les informations suivantes :
\begin{itemize}
  \item $30$\% des salariés ont $40$ ans ou plus;
\end{itemize}
\end{minipage}
\begin{minipage}{.5\textwidth}
  \begin{center}
 \begin{tabular}{cccc}
  \toprule
  & \textbf{$<40$ ans} & \textbf{$\geq40$ ans} & \textbf{Total} \\
  \midrule
  \textbf{Cadres} &  &  &  \\
  \textbf{Non cadres} &  &  &  \\
  \textbf{Total} &  & & $2000$ \\
  \bottomrule
\end{tabular}
  \end{center}
\end{minipage}
\begin{itemize}
  \item $40$\% des salariés de plus de $40$ ans sont des cadres;
  \item $25$\% des salariés de moins de $40$ ans sont des cadres;
\end{itemize}
\begin{enumerate}
  \item Compléter le tableau ci-contre.
  \item On interroge au hasard un employé de cette entreprise. On considère les
    événements $A$ : « la personne interrogée a $40$ ans ou plus » et $B$ : « la
    personne interrogée est cadre ».
    \begin{enumerate}
      \item Calculer les probabilités $P(A)$ et $P(B)$ des événements $A$ et
        $B$.
      \item Définir par une phrase chacun des événements $A\cap B$ et $A\cup B$.
    \item Calculer les probabilités $P(A\cap B)$ et $P(A\cup B)$.
    \end{enumerate}
  \item Sachant que la personne a $40$ ans ou plus, quelle est la probabilité
    que ce ne soit pas un cadre ?
\end{enumerate}
\end{exo}

\begin{exo}
Soient $A$ et $B$ deux événements tels que $P_A(B)=0,8$,
$P_B(A)=0,6$ et $P(A)=0,4$.
\begin{enumerate}
  \item Calculer $P(A\cap B)$.
  \item En déduire $P(B)$.
  \item Calculer alors $P(A\cup B)$.
\end{enumerate}
\end{exo}

%\newpage
\begin{exo}
On lance un dé équilibré à six faces et on considère les
événements $A$ : «~obtenir $4, 5, 6$~» et $B$ : «~obtenir un nombre pair~».
\begin{enumerate}
  \item Calculer $P_{B}(A)$.
  \item Calculer $P_A(B)$.
  \item Calculer $P_{A\cap B}(A\cup B)$.
\end{enumerate}
\end{exo}

\begin{exo}
Dans une forêt, il y a $30$\% d'épicéas et $70$\% de sapins. Un parasite infecte
$10$\% des arbres. Les épicéas représentent $20$\% des arbres touchés.
\begin{enumerate}
  \item Quelle est la probabilité qu'un épicéa soit touché par le parasite ?
  \item Faire un tableau pour résumer l'ensemble de la situation.
\end{enumerate}
\end{exo}

\begin{exo}
Vincent et Anne sont haltérophiles. La probabilité pour que Vincent soulève plus
de $100$ kg est égale à $0,75$, alors que la probabilité pour qu'Anne soulève
plus de $100$ kg est égale à $0,6$. La probabilité pour qu'au moins l'un des
deux soulève plus de $100$ kg est égale à $0,85$.
\begin{enumerate}
  \item Quelle est la probabilité qu'ils soulèvent $100$ kg tous les deux ?
  \item Anne vient de voir Vincent soulever $100$ kg. Quelle est la probabilité
    qu'elle soulève $100$ kg ?
\end{enumerate}
\end{exo}

\begin{exo}
Une urne opaque contient trois boules rouges, une boule
noire et une boule verte, toutes indiscernables au toucher. On procède au
tirage, sans remise, de trois boules dont on note la couleur.\\
Déterminer la probabilité d'obtenir trois boules de couleurs différentes.
\end{exo}

\begin{exo}
Dans une crèche, chaque matin, Hapsatou fait la sieste
avec une probabilité de $0,7$. Si elle a fait la sieste le matin, elle fera à
nouveau la sieste l'après-midi avec une probabilité de $0,2$. Sinon, elle fera
la sieste l'après-midi avec une probabilité de $0,9$.
\begin{enumerate}
  \item Représenter la situation avec un arbre de probabilité que l'on
    complètera entièrement.
  \item Calculer la probabilité qu'elle de fasse pas du tout la sieste dans la
    journée.
  \item Calculer la probabilité qu'elle fasse la sieste l'après-midi.
\end{enumerate}
\end{exo}

\begin{exo}
Une maladie est
présente dans la population, dans la proportion d'une personne malade sur
$10\;000$. Un responsable d'un grand laboratoire pharmaceutique vient
vanter son nouveau test de dépistage : si une personne est malade, le test est
positif à $99\%$. Si une personne n'est pas malade, le test est positif à
$0,1\%$.
\begin{enumerate}
  \item Sans faire de calculs (intuitivement), ce test semble-t-il performant ?
  \item Représenter la situation par un arbre de probabilité.
  \item Calculer la probabilité qu'une personne soit malade sachant que son test
    est positif.
  \item Que penser de ce test ?
\end{enumerate}
\end{exo}

\begin{exo}
On jette un dé non truqué à $20$ faces numérotées de $1$
à $20$. On note
\begin{itemize}
  \item $A$ : « Le résultat est pair. »
  \item $B$ : « Le résultat est l'un des nombres $1; 3; 5; 7; 11; 13; 17; 19$. »
  \item $C$ : « Le résultat est impair. »
  \item $D$ : « Le résultat est l'un des nombres $9$ ou $15$. »
  \item $E$ : « Le résultat est un nombre premier. »
\end{itemize}
\begin{enumerate}
  \item Les événements $E$ et $A$ forment-ils une partition de l'univers ?
  \item Les événements $C$ et $\overline B$ forment-ils une partition de l'univers ?
  \item Parmi ces cinq événements, en donner deux qui forment une partition de
    l'univers.
  \item Parmi ces cinq événements, en donner trois qui forment une partition de
    l'univers.
\end{enumerate}
\end{exo}

\begin{exo}
Dans chacun des cas suivants, dire si les événements
$A$ et $B$ sont indépendants.
\begin{enumerate}
  \item $P(A)=0,2$, $P(B)=0,8$, et $P(A\cap B)=0,2$.
  \item $P(A)=0,4$, $P(B)=0,8$, et $P(A\cap B)=0,32$.
  \item $P(A)=0,5$, $P(B)=0,3$, et $P(A\cup B)=0,65$.
  \item $P(A)=0,48$, $P(B)=0,8$, et $P(A\cap B)=0$.
\end{enumerate}
\end{exo}

\begin{exo}
On tire une carte dans un jeu de $32$ cartes. Dans
chacun des cas suivants, dire si les événements sont indépendants.
\begin{enumerate}
  \item $A$ : « tirer un roi » et $B$ : « tirer un rouge »
  \item $A$ : « tirer un roi » et $B$ : « ne pas tirer un as »
  \item $A$ : « tirer un roi ou tirer une dame rouge » et $B$ : « tirer un rouge »
\end{enumerate}
\end{exo}

\begin{center}
  \Large
  Exercices bilan
\end{center}

\begin{exo}
  À l’occasion d’une cérémonie, un pâtissier confectionne
un assortiment de $180$ pâtisseries composé d’éclairs et de religieuses qui sont
soit au chocolat, soit au café. Les deux tiers de ces pâtisseries sont des
éclairs. On sait également qu’il y a $100$ pâtisseries au chocolat parmi
lesquels un quart sont des religieuses.
Soit $E$ l'événement : « la pâtisserie est un éclair » et $C$ l'événement : « la
pâtisserie est au chocolat ».
\begin{enumerate}
  \item À partir des indications de l'énoncé, compléter le tableau suivant.
    \begin{center}
      \def\arraystretch{1.5}
    \begin{tabular}{|c|c|c|c|}
      \hline
      & \textbf{Chocolat} & \textbf{Café} & \textbf{Total} \\
      \hline
      \textbf{Éclairs} & & & \\
      \hline
      \textbf{Religieuses} & & & \\
      \hline
      \textbf{Total} & & & \\
      \hline
    \end{tabular}
    \end{center}
  \item Antoine choisit au hasard un g\^ateau parmi toutes les p\^atisseries.
    Calculer la probabilité qu'il s'agisse :
    \begin{enumerate}
      \item d'une religieuse ;
      \item d'une p\^atisserie au café.
    \end{enumerate}
  \item Décrire en français l'événement $E\cap C$, puis calculer sa probabilité.
  \item Calculer la probabilité de l'événement : « la p\^atisserie est un éclair
    ou est une p\^atisserie au chocolat ».
  \item Calculer $P_E(\overline C)$. Interpréter ce résultat.
  \item Les événements $E$ et $C$ sont-ils indépendants ? Justifier.
\end{enumerate}
\end{exo}
\begin{exo}
En prévision d’une élection entre deux candidats A et B, un institut de
sondage recueille les intentions de vote de futurs électeurs. Parmi les $1\;200$
personnes qui ont répondu au sondage, $47$\% affirment vouloir voter pour le
candidat A et les autres pour le candidat B.
Compte-tenu du profil des candidats, l’institut de sondage estime que $10$\% des personnes déclarant
vouloir voter pour le candidat A ne disent pas la vérité et votent en réalité
pour le candidat B, tandis que $20$\% des personnes déclarant vouloir voter
pour le candidat B ne disent pas la vérité et votent en réalité
pour le candidat A.\\[5mm]
On choisit au hasard une personne ayant répondu au sondage et on note :
\begin{itemize}
  \item $A$ l'événement « la personne interrogée affirme vouloir voter pour le
    candidat A » ;
  \item $B$ l'événement « la personne interrogée affirme vouloir voter pour le
    candidat B » ;
  \item $V$ l'événement « la personne interrogée dit la vérité ».
\end{itemize}
\begin{enumerate}
  \item Construire un arbre de probabilités traduisant la situation.
  \item Donner $P_A(V)$ et $P_B(\overline V)$.
  \item Calculer la probabilité qu'une personne interrogée vote pour le candidat
    A et dise la vérité.
  \item \begin{enumerate}
      \item Calculer la probabilité que la personne interrogée dise la vérité.
      \item En déduire le nombre de personnes qui disent la vérité à ce sondage.
      \item Sachant que la personne interrogée dit la vérité, calculer la
        probabilité qu'elle affirme vouloir voter pour le candidat A.
    \end{enumerate}
  \item \begin{enumerate}
      \item On note $E$ l'événement « la personne choisie vote effectivement
        pour le candidat A ». Exprimer $E$ en fonction des événements $A$, $B$
        et $V$.
      \item Montrer que la probabilité de l'événement $E$ est $0,529$.
    \end{enumerate}
\end{enumerate}
\end{exo}

\begin{center}
  \Large
  Exercices plus théoriques
\end{center}

\begin{exo}[$\star$]
Soient $A$ et $B$ deux événements incompatibles
($P(A\cap B)=0$) de probabilités non nulles. Démontrer que $A$ et $B$ ne sont pas
indépendants.
\end{exo}

\begin{exo}[$\star$]
On considère $A$ un événement indépendant de lui-même. Démontrer que $P(A)=0$ ou
$P(A)=1$.
\end{exo}

\begin{exo}[$\star\star$]
On considère deux événements $A$ et $B$ tels que $P(A\cap B)=0,8$ et $P(A\cup
B)=0,9$.
\begin{enumerate}
  \item Montrer que, pour tout $x\in\mathbb{R}$,
    $x^2-1,7x+0,8=(x-0,85)^2+0,0775$.
  \item Montrer que $A$ et $B$ ne peuvent pas être indépendants.
\end{enumerate}
\end{exo}

\begin{center}
  \Large
  Des paradoxes
\end{center}

\begin{exo}
On considère que la probabilité de donner naissance à
une fille est la même que celle de donner naissance à un garçon. De plus, on
admet que le sexe d'un enfant à la naissance est indépendant du sexe des enfants
nés avant.\\
Manuel a deux enfants. Lorsque le facteur est venu sonner à sa porte pour lui
apporter un colis, c'est une fille qui a répondu. On ne sait pas si cette fille
est l'aînée des deux enfants ou pas. Quelle est la probabilité que l'autre
enfant de Manuel soit un garçon : $\frac{1}{2}$ ou $\frac{2}{3}$ ? Justifier.
\end{exo}

\begin{exo}[Paradoxe de Monty-Hall]
Lors d'un jeu télévisé, une
candidate est placée devant trois portes fermées. Derrière l'une d'elle se trouve
une voiture, tandis que derrière chacune des deux autres portes il n'y a rien.
Le jeu se déroule en trois étapes :
\begin{enumerate}[label=(\alph*)]
  \item La candidate choisit d'abord une première porte.
  \item\label{choix} Le présentateur va alors ouvrir une autre porte, différente de celle
    choisie par la candidate, derrière laquelle il n'y a rien.
  \item Le présentateur propose à la candidate de modifier son choix, et ouvre
    la porte finalement choisie par la candidate, qui gagne la voiture si elle
    est derrière la porte choisie.
\end{enumerate}
\begin{enumerate}
  \item Intuitivement, quelle est la probabilité que la voiture soit derrière la
    porte choisie par la candidate au début du jeu ? À l'issue de l'étape
    \ref{choix} ? Vérifier par le calcul (un arbre peut être utile !).
  \item La candidate a-t-elle intérêt à modifier son choix lorsque le présentateur lui
propose de le faire ?
\end{enumerate}
\end{exo}

\end{document}
