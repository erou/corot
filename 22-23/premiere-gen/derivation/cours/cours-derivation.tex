\documentclass[11pt]{article}

%%%%%%%%%%%%%%%%%%%%%%%%%%%%%%%%%%%%%%%%%%%%%%%%%%%%%%%%%%%%%%%%%%%%%%%%%%%%%%%%
%
% PACKAGES
% ========
%
%%%%%%%%%%%%%%%%%%%%%%%%%%%%%%%%%%%%%%%%%%%%%%%%%%%%%%%%%%%%%%%%%%%%%%%%%%%%%%%%

\usepackage[english, french]{babel}
\usepackage[utf8]{inputenc}
\usepackage[T1]{fontenc}
\usepackage{graphicx}
\usepackage{amsmath,amssymb,amsthm,amsopn}
\usepackage{hyperref}

% Pour avoir l'écriture \mathscr (math script)
% ============================================

\usepackage{mathrsfs}

% Deal with coma as a decimal separator
% =====================================

\usepackage{icomma}

% Package Geometry
% ================

\usepackage[a4paper, lmargin=2cm, rmargin=2cm, top=\topbotmargins, bottom=\topbotmargins]{geometry}

% Package multicol
% ================

\usepackage{multicol}

% Redefine abstract
% =================

% Note
% ====
%
% Le reste a été commenté pour ne pas charger trop de choses au démarrage. On
% verra si on en a besoin plus tard.
%
% --------
%
%\usepackage{mathrsfs}
%\usepackage{multirow}
%\usepackage{bm}
%\hypersetup{
%    colorlinks=true,
%    linkcolor=blue,
%    citecolor=red,
%}
%\usepackage{diagbox}
%
%\usepackage{algorithm}
%\usepackage{algpseudocode}
%
%\renewcommand{\algorithmicrequire}{\textbf{Input:}}
%\renewcommand{\algorithmicensure}{\textbf{Output:}}


%%%%%%%%%%%%%%%%%%%%%%%%%%%%%%%%%%%%%%%%%%%%%%%%%%%%%%%%%%%%%%%%%%%%%%%%%%%%%%%%
%
% TIKZ
% ====
%
%%%%%%%%%%%%%%%%%%%%%%%%%%%%%%%%%%%%%%%%%%%%%%%%%%%%%%%%%%%%%%%%%%%%%%%%%%%%%%%%

\usepackage{tikz}
\usetikzlibrary{arrows}

\usepackage{tkz-tab} % Variation tables

\usepackage{pgfplots}
%\usepackage{pgf-pie} % Pie charts

\pgfplotsset{
%\newcommand{\settingsgraph}{
x=.5cm,y=.5cm,
xticklabel style = {font=\scriptsize, yshift=.1cm},
yticklabel style = {font=\scriptsize, xshift=.1cm},
axis lines=middle,
ymajorgrids=true,
xmajorgrids=true,
major grid style = {color=white!80!blue},
xmin=-5.5,
xmax=5.5,
ymin=-5.5,
ymax=5.5,
xtick={-5.0,-4.0,...,5.0},
ytick={-5.0,-4.0,...,5.0},
}

% Tikz style

\tikzset{round/.style={circle, draw=black, very thick, scale = 0.7}}
\tikzset{arrow/.style={->, >=latex}}
\tikzset{dashed-arrow/.style={->, >=latex, dashed}}

\newcommand{\point}[3]{\draw[very thick, #3] (#1-.1, #2)--(#1+.1, #2)
(#1, #2-.1)--(#1, #2+.1)}

%%%%%%%%%%%%%%%%%%%%%%%%%%%%%%%%%%%%%%%%%%%%%%%%%%%%%%%%%%%%%%%%%%%%%%%%%%%%%%%%
%
% FANCY HEADER
% ============
%
%%%%%%%%%%%%%%%%%%%%%%%%%%%%%%%%%%%%%%%%%%%%%%%%%%%%%%%%%%%%%%%%%%%%%%%%%%%%%%%%


\usepackage{fancyhdr}
\usepackage{lastpage}

\pagestyle{fancy}
\newcommand{\changefont}{\fontsize{9}{9}\selectfont}
\renewcommand{\headrulewidth}{0mm}
\renewcommand{\footrulewidth}{0mm}

\fancyhead[C]{}
\fancyhead[L]{\titreclasse}
\fancyhead[R]{\titrechapitre}
\fancyfoot[C]{}
\fancyfoot[L]{}
\fancyfoot[R]{\pagination}
\addtolength{\skip\footins}{20pt} % distance between text and footnotes

%%%%%%%%%%%%%%%%%%%%%%%%%%%%%%%%%%%%%%%%%%%%%%%%%%%%%%%%%%%%%%%%%%%%%%%%%%%%%%%%
%
% THEOREM STYLE
% =============
%
%%%%%%%%%%%%%%%%%%%%%%%%%%%%%%%%%%%%%%%%%%%%%%%%%%%%%%%%%%%%%%%%%%%%%%%%%%%%%%%%

\usepackage[tikz]{bclogo}
\usepackage{mdframed}

\usepackage{tcolorbox}
\tcbuselibrary{listings, breakable, theorems, skins}

%\newtheoremstyle{break}%
%{}{}%
%{\itshape}{}%
%{\bfseries}{}%  % Note that final punctuation is omitted.
%{\newline}{}

\newtheoremstyle{scbf}%
{}{}%
{}{}%
%{\scshape}{}%  % Note that final punctuation is omitted.
{\bfseries\scshape}{}%  % Note that final punctuation is omitted.
{\newline}{}

%\theoremstyle{break}
%\theoremstyle{plain}
%\newtheorem{thm}{Theorem}[section]
%\newtheorem{lm}[thm]{Lemma}
%\newtheorem{prop}[thm]{Proposition}
%\newtheorem{cor}[thm]{Corollary}

%\theoremstyle{scbf}
%\newtheorem{exo}{$\star$ Exercice}

%\theoremstyle{definition}
%\newtheorem{defi}[thm]{Definition}
%\newtheorem{ex}[thm]{Example}

%\theoremstyle{remark}
%\newtheorem{rem}[thm]{Remark}

% Defining the Remark environment
% ===============================

\newenvironment{rmq}
  {
    \begin{bclogo}[logo=\bcinfo, noborder=true]{Remarque}
  }
  {
    \end{bclogo}
  }

% Defining the exercise environment
% =================================

\newcounter{exos}
\setcounter{exos}{1}

\newenvironment{exo}
  {
    \begin{bclogo}[logo=\bccrayon, noborder=true]{Exercice \theexos}
  }
  {
    \end{bclogo}
    \addtocounter{exos}{1}
  }


% Redefining the proof environment from amsthm
% ============================================

\tcolorboxenvironment{proof}{
  blanker, breakable, before skip=10pt,after skip=10pt,
  borderline west={1mm}{0pt}{red},
  left=5mm,
}

% Defining the definition environment
% ===================================

\colorlet{coldef}{black!50!green}

\newcounter{defis}
\setcounter{defis}{1}

\newenvironment{defi}[1]
  {
    \begin{defihid}{{#1}}{\thedefis}
  }
  {
    \end{defihid}
    \addtocounter{defis}{1}
  }

\newtcolorbox{defihid}[2]{%
  empty,title={ {\bfseries Définition {#2}} ({#1})},attach boxed title to top left,
boxed title style={empty,size=minimal,toprule=2pt,top=4pt,
overlay={\draw[coldef,line width=2pt]
([yshift=-1pt]frame.north west)--([yshift=-1pt]frame.north east);}},
coltitle=coldef,
before=\par\medskip\noindent,parbox=false,boxsep=0pt,left=0pt,right=3mm,top=4pt,
breakable,pad at break*=0mm,vfill before first,
overlay unbroken={\draw[coldef,line width=1pt]
([yshift=-1pt]title.north east)--([xshift=-0.5pt,yshift=-1pt]title.north-|frame.east)
--([xshift=-0.5pt]frame.south east)--(frame.south west); },
overlay first={\draw[coldef,line width=1pt]
([yshift=-1pt]title.north east)--([xshift=-0.5pt,yshift=-1pt]title.north-|frame.east)
--([xshift=-0.5pt]frame.south east); },
overlay middle={\draw[coldef,line width=1pt] ([xshift=-0.5pt]frame.north east)
--([xshift=-0.5pt]frame.south east); },
overlay last={\draw[coldef,line width=1pt] ([xshift=-0.5pt]frame.north east)
--([xshift=-0.5pt]frame.south east)--(frame.south west);},%
}

\newenvironment{notation}
  {
    \begin{notationhid}{\thedefis}
  }
  {
    \end{notationhid}
    \addtocounter{defis}{1}
  }

\newtcolorbox{notationhid}[1]{%
  empty,title={Notation {#1}},attach boxed title to top left,
boxed title style={empty,size=minimal,toprule=2pt,top=4pt,
overlay={\draw[coldef,line width=2pt]
([yshift=-1pt]frame.north west)--([yshift=-1pt]frame.north east);}},
coltitle=coldef,fonttitle=\bfseries,
before=\par\medskip\noindent,parbox=false,boxsep=0pt,left=0pt,right=3mm,top=4pt,
breakable,pad at break*=0mm,vfill before first,
overlay unbroken={\draw[coldef,line width=1pt]
([yshift=-1pt]title.north east)--([xshift=-0.5pt,yshift=-1pt]title.north-|frame.east)
--([xshift=-0.5pt]frame.south east)--(frame.south west); },
overlay first={\draw[coldef,line width=1pt]
([yshift=-1pt]title.north east)--([xshift=-0.5pt,yshift=-1pt]title.north-|frame.east)
--([xshift=-0.5pt]frame.south east); },
overlay middle={\draw[coldef,line width=1pt] ([xshift=-0.5pt]frame.north east)
--([xshift=-0.5pt]frame.south east); },
overlay last={\draw[coldef,line width=1pt] ([xshift=-0.5pt]frame.north east)
--([xshift=-0.5pt]frame.south east)--(frame.south west);},%
}


% Defining the proposition, theorem, etc. environment
% ===================================================

\colorlet{colprop}{red!75!black}

\newcounter{props}
\setcounter{props}{1}

\newenvironment{prop}
  {
    \begin{prophid}{\theprops}
  }
  {
    \end{prophid}
    \refstepcounter{props}
  }

\newtcolorbox{prophid}[1]{%
empty,title={Propriété {#1}},attach boxed title to top left,
boxed title style={empty,size=minimal,toprule=2pt,top=4pt,
overlay={\draw[colprop,line width=2pt]
([yshift=-1pt]frame.north west)--([yshift=-1pt]frame.north east);}},
coltitle=colprop,fonttitle=\bfseries,
before=\par\medskip\noindent,parbox=false,boxsep=0pt,left=0pt,right=3mm,top=4pt,
breakable,pad at break*=0mm,vfill before first,
overlay unbroken={\draw[colprop,line width=1pt]
([yshift=-1pt]title.north east)--([xshift=-0.5pt,yshift=-1pt]title.north-|frame.east)
--([xshift=-0.5pt]frame.south east)--(frame.south west); },
overlay first={\draw[colprop,line width=1pt]
([yshift=-1pt]title.north east)--([xshift=-0.5pt,yshift=-1pt]title.north-|frame.east)
--([xshift=-0.5pt]frame.south east); },
overlay middle={\draw[colprop,line width=1pt] ([xshift=-0.5pt]frame.north east)
--([xshift=-0.5pt]frame.south east); },
overlay last={\draw[colprop,line width=1pt] ([xshift=-0.5pt]frame.north east)
--([xshift=-0.5pt]frame.south east)--(frame.south west);},%
}

\newenvironment{propadm}
  {
    \begin{propadmhid}{\theprops}
  }
  {
    \end{propadmhid}
    \refstepcounter{props}
  }

  \newtcolorbox{propadmhid}[1]{%
    empty,title={{\bfseries Propriété {#1}} (admise)},attach boxed title to top left,
boxed title style={empty,size=minimal,toprule=2pt,top=4pt,
overlay={\draw[colprop,line width=2pt]
([yshift=-1pt]frame.north west)--([yshift=-1pt]frame.north east);}},
coltitle=colprop,%fonttitle=\bfseries,
before=\par\medskip\noindent,parbox=false,boxsep=0pt,left=0pt,right=3mm,top=4pt,
breakable,pad at break*=0mm,vfill before first,
overlay unbroken={\draw[colprop,line width=1pt]
([yshift=-1pt]title.north east)--([xshift=-0.5pt,yshift=-1pt]title.north-|frame.east)
--([xshift=-0.5pt]frame.south east)--(frame.south west); },
overlay first={\draw[colprop,line width=1pt]
([yshift=-1pt]title.north east)--([xshift=-0.5pt,yshift=-1pt]title.north-|frame.east)
--([xshift=-0.5pt]frame.south east); },
overlay middle={\draw[colprop,line width=1pt] ([xshift=-0.5pt]frame.north east)
--([xshift=-0.5pt]frame.south east); },
overlay last={\draw[colprop,line width=1pt] ([xshift=-0.5pt]frame.north east)
--([xshift=-0.5pt]frame.south east)--(frame.south west);},%
}

\newenvironment{propnom}[1]
  {
    \begin{propnomhid}{#1}{\theprops}
  }
  {
    \end{propnomhid}
    \refstepcounter{props}
  }

\newtcolorbox{propnomhid}[2]{%
empty,title={{\bfseries Propriété {#2}} ({#1})},attach boxed title to top left,
boxed title style={empty,size=minimal,toprule=2pt,top=4pt,
overlay={\draw[colprop,line width=2pt]
([yshift=-1pt]frame.north west)--([yshift=-1pt]frame.north east);}},
coltitle=colprop,
before=\par\medskip\noindent,parbox=false,boxsep=0pt,left=0pt,right=3mm,top=4pt,
breakable,pad at break*=0mm,vfill before first,
overlay unbroken={\draw[colprop,line width=1pt]
([yshift=-1pt]title.north east)--([xshift=-0.5pt,yshift=-1pt]title.north-|frame.east)
--([xshift=-0.5pt]frame.south east)--(frame.south west); },
overlay first={\draw[colprop,line width=1pt]
([yshift=-1pt]title.north east)--([xshift=-0.5pt,yshift=-1pt]title.north-|frame.east)
--([xshift=-0.5pt]frame.south east); },
overlay middle={\draw[colprop,line width=1pt] ([xshift=-0.5pt]frame.north east)
--([xshift=-0.5pt]frame.south east); },
overlay last={\draw[colprop,line width=1pt] ([xshift=-0.5pt]frame.north east)
--([xshift=-0.5pt]frame.south east)--(frame.south west);},%
}




\newenvironment{thm}
  {
    \begin{thmhid}{\theprops}
  }
  {
    \end{thmhid}
    \refstepcounter{props}
  }

\newtcolorbox{thmhid}[1]{%
empty,title={Théorème {#1}},attach boxed title to top left,
boxed title style={empty,size=minimal,toprule=2pt,top=4pt,
overlay={\draw[colprop,line width=2pt]
([yshift=-1pt]frame.north west)--([yshift=-1pt]frame.north east);}},
coltitle=colprop,fonttitle=\bfseries,
before=\par\medskip\noindent,parbox=false,boxsep=0pt,left=0pt,right=3mm,top=4pt,
breakable,pad at break*=0mm,vfill before first,
overlay unbroken={\draw[colprop,line width=1pt]
([yshift=-1pt]title.north east)--([xshift=-0.5pt,yshift=-1pt]title.north-|frame.east)
--([xshift=-0.5pt]frame.south east)--(frame.south west); },
overlay first={\draw[colprop,line width=1pt]
([yshift=-1pt]title.north east)--([xshift=-0.5pt,yshift=-1pt]title.north-|frame.east)
--([xshift=-0.5pt]frame.south east); },
overlay middle={\draw[colprop,line width=1pt] ([xshift=-0.5pt]frame.north east)
--([xshift=-0.5pt]frame.south east); },
overlay last={\draw[colprop,line width=1pt] ([xshift=-0.5pt]frame.north east)
--([xshift=-0.5pt]frame.south east)--(frame.south west);},%
}

\newenvironment{thmadm}
  {
    \begin{thmadmhid}{\theprops}
  }
  {
    \end{thmadmhid}
    \refstepcounter{props}
  }

  \newtcolorbox{thmadmhid}[1]{%
    empty,title={{\bfseries Théorème {#1}} (admis)},attach boxed title to top left,
boxed title style={empty,size=minimal,toprule=2pt,top=4pt,
overlay={\draw[colprop,line width=2pt]
([yshift=-1pt]frame.north west)--([yshift=-1pt]frame.north east);}},
coltitle=colprop,%fonttitle=\bfseries,
before=\par\medskip\noindent,parbox=false,boxsep=0pt,left=0pt,right=3mm,top=4pt,
breakable,pad at break*=0mm,vfill before first,
overlay unbroken={\draw[colprop,line width=1pt]
([yshift=-1pt]title.north east)--([xshift=-0.5pt,yshift=-1pt]title.north-|frame.east)
--([xshift=-0.5pt]frame.south east)--(frame.south west); },
overlay first={\draw[colprop,line width=1pt]
([yshift=-1pt]title.north east)--([xshift=-0.5pt,yshift=-1pt]title.north-|frame.east)
--([xshift=-0.5pt]frame.south east); },
overlay middle={\draw[colprop,line width=1pt] ([xshift=-0.5pt]frame.north east)
--([xshift=-0.5pt]frame.south east); },
overlay last={\draw[colprop,line width=1pt] ([xshift=-0.5pt]frame.north east)
--([xshift=-0.5pt]frame.south east)--(frame.south west);},%
}

\newenvironment{thmnom}[1]
  {
    \begin{thmnomhid}{#1}{\theprops}
  }
  {
    \end{thmnomhid}
    \refstepcounter{props}
  }

\newtcolorbox{thmnomhid}[2]{%
empty,title={{\bfseries Théorème {#2}} ({#1})},attach boxed title to top left,
boxed title style={empty,size=minimal,toprule=2pt,top=4pt,
overlay={\draw[colprop,line width=2pt]
([yshift=-1pt]frame.north west)--([yshift=-1pt]frame.north east);}},
coltitle=colprop,
before=\par\medskip\noindent,parbox=false,boxsep=0pt,left=0pt,right=3mm,top=4pt,
breakable,pad at break*=0mm,vfill before first,
overlay unbroken={\draw[colprop,line width=1pt]
([yshift=-1pt]title.north east)--([xshift=-0.5pt,yshift=-1pt]title.north-|frame.east)
--([xshift=-0.5pt]frame.south east)--(frame.south west); },
overlay first={\draw[colprop,line width=1pt]
([yshift=-1pt]title.north east)--([xshift=-0.5pt,yshift=-1pt]title.north-|frame.east)
--([xshift=-0.5pt]frame.south east); },
overlay middle={\draw[colprop,line width=1pt] ([xshift=-0.5pt]frame.north east)
--([xshift=-0.5pt]frame.south east); },
overlay last={\draw[colprop,line width=1pt] ([xshift=-0.5pt]frame.north east)
--([xshift=-0.5pt]frame.south east)--(frame.south west);},%
}

\newenvironment{coro}
  {
    \begin{corohid}{\theprops}
  }
  {
    \end{corohid}
    \refstepcounter{props}
  }

  \newtcolorbox{corohid}[1]{%
  empty,title={Corollaire {#1}},attach boxed title to top left,
boxed title style={empty,size=minimal,toprule=2pt,top=4pt,
overlay={\draw[colprop,line width=2pt]
([yshift=-1pt]frame.north west)--([yshift=-1pt]frame.north east);}},
coltitle=colprop,fonttitle=\bfseries,
before=\par\medskip\noindent,parbox=false,boxsep=0pt,left=0pt,right=3mm,top=4pt,
breakable,pad at break*=0mm,vfill before first,
overlay unbroken={\draw[colprop,line width=1pt]
([yshift=-1pt]title.north east)--([xshift=-0.5pt,yshift=-1pt]title.north-|frame.east)
--([xshift=-0.5pt]frame.south east)--(frame.south west); },
overlay first={\draw[colprop,line width=1pt]
([yshift=-1pt]title.north east)--([xshift=-0.5pt,yshift=-1pt]title.north-|frame.east)
--([xshift=-0.5pt]frame.south east); },
overlay middle={\draw[colprop,line width=1pt] ([xshift=-0.5pt]frame.north east)
--([xshift=-0.5pt]frame.south east); },
overlay last={\draw[colprop,line width=1pt] ([xshift=-0.5pt]frame.north east)
--([xshift=-0.5pt]frame.south east)--(frame.south west);},%
}

\newenvironment{lemme}
  {
    \begin{lemmehid}{\theprops}
  }
  {
    \end{lemmehid}
    \refstepcounter{props}
  }

  \newtcolorbox{lemmehid}[1]{%
  empty,title={Lemme {#1}},attach boxed title to top left,
boxed title style={empty,size=minimal,toprule=2pt,top=4pt,
overlay={\draw[colprop,line width=2pt]
([yshift=-1pt]frame.north west)--([yshift=-1pt]frame.north east);}},
coltitle=colprop,fonttitle=\bfseries,
before=\par\medskip\noindent,parbox=false,boxsep=0pt,left=0pt,right=3mm,top=4pt,
breakable,pad at break*=0mm,vfill before first,
overlay unbroken={\draw[colprop,line width=1pt]
([yshift=-1pt]title.north east)--([xshift=-0.5pt,yshift=-1pt]title.north-|frame.east)
--([xshift=-0.5pt]frame.south east)--(frame.south west); },
overlay first={\draw[colprop,line width=1pt]
([yshift=-1pt]title.north east)--([xshift=-0.5pt,yshift=-1pt]title.north-|frame.east)
--([xshift=-0.5pt]frame.south east); },
overlay middle={\draw[colprop,line width=1pt] ([xshift=-0.5pt]frame.north east)
--([xshift=-0.5pt]frame.south east); },
overlay last={\draw[colprop,line width=1pt] ([xshift=-0.5pt]frame.north east)
--([xshift=-0.5pt]frame.south east)--(frame.south west);},%
}

\colorlet{colexemple}{blue!50!black}
%\newtcolorbox{exemple}{empty, title=Exemple, attach boxed title to top left,
%  boxed title style={empty, size=minimal, toprule=2pt, top=4pt,
%    overlay={\draw[colexemple,line width=2pt]
%([yshift=-1pt]frame.north west)--([yshift=-1pt]frame.north east);}},
%coltitle=colexemple,fonttitle=\bfseries,%\large\bfseries,
%before=\par\medskip\noindent,parbox=false,boxsep=0pt,left=0pt,right=3mm,top=4pt,
%overlay={\draw[colexemple,line width=1pt]
%([yshift=-1pt]title.north east)--([xshift=-0.5pt,yshift=-1pt]title.north-|frame.east)
%--([xshift=-0.5pt]frame.south east)--(frame.south west); },
%}

\newcounter{exemples}
\setcounter{exemples}{1}

\newenvironment{exemple}
  {
    \begin{exemplehid}{\theexemples}
  }
  {
    \end{exemplehid}
    \addtocounter{exemples}{1}
  }

\newtcolorbox{exemplehid}[1]{%
empty,title={Exemple {#1}},attach boxed title to top left,
boxed title style={empty,size=minimal,toprule=2pt,top=4pt,
overlay={\draw[colexemple,line width=2pt]
([yshift=-1pt]frame.north west)--([yshift=-1pt]frame.north east);}},
coltitle=colexemple,fonttitle=\bfseries,
before=\par\medskip\noindent,parbox=false,boxsep=0pt,left=0pt,right=3mm,top=4pt,
breakable,pad at break*=0mm,vfill before first,
overlay unbroken={\draw[colexemple,line width=1pt]
([yshift=-1pt]title.north east)--([xshift=-0.5pt,yshift=-1pt]title.north-|frame.east)
--([xshift=-0.5pt]frame.south east)--(frame.south west); },
overlay first={\draw[colexemple,line width=1pt]
([yshift=-1pt]title.north east)--([xshift=-0.5pt,yshift=-1pt]title.north-|frame.east)
--([xshift=-0.5pt]frame.south east); },
overlay middle={\draw[colexemple,line width=1pt] ([xshift=-0.5pt]frame.north east)
--([xshift=-0.5pt]frame.south east); },
overlay last={\draw[colexemple,line width=1pt] ([xshift=-0.5pt]frame.north east)
--([xshift=-0.5pt]frame.south east)--(frame.south west);},%
}

\newenvironment{contrex}
  {
    \begin{contrexhid}{\theexemples}
  }
  {
    \end{contrexhid}
    \addtocounter{exemples}{1}
  }

\newtcolorbox{contrexhid}[1]{%
empty,title={Contre-exemple {#1}},attach boxed title to top left,
boxed title style={empty,size=minimal,toprule=2pt,top=4pt,
overlay={\draw[colexemple,line width=2pt]
([yshift=-1pt]frame.north west)--([yshift=-1pt]frame.north east);}},
coltitle=colexemple,fonttitle=\bfseries,
before=\par\medskip\noindent,parbox=false,boxsep=0pt,left=0pt,right=3mm,top=4pt,
breakable,pad at break*=0mm,vfill before first,
overlay unbroken={\draw[colexemple,line width=1pt]
([yshift=-1pt]title.north east)--([xshift=-0.5pt,yshift=-1pt]title.north-|frame.east)
--([xshift=-0.5pt]frame.south east)--(frame.south west); },
overlay first={\draw[colexemple,line width=1pt]
([yshift=-1pt]title.north east)--([xshift=-0.5pt,yshift=-1pt]title.north-|frame.east)
--([xshift=-0.5pt]frame.south east); },
overlay middle={\draw[colexemple,line width=1pt] ([xshift=-0.5pt]frame.north east)
--([xshift=-0.5pt]frame.south east); },
overlay last={\draw[colexemple,line width=1pt] ([xshift=-0.5pt]frame.north east)
--([xshift=-0.5pt]frame.south east)--(frame.south west);},%
}

\newenvironment{app}
  {
    \begin{apphid}{\theexemples}
  }
  {
    \end{apphid}
    \addtocounter{exemples}{1}
  }

\newtcolorbox{apphid}[1]{%
empty,title={Application {#1}},attach boxed title to top left,
boxed title style={empty,size=minimal,toprule=2pt,top=4pt,
overlay={\draw[colexemple,line width=2pt]
([yshift=-1pt]frame.north west)--([yshift=-1pt]frame.north east);}},
coltitle=colexemple,fonttitle=\bfseries,
before=\par\medskip\noindent,parbox=false,boxsep=0pt,left=0pt,right=3mm,top=4pt,
breakable,pad at break*=0mm,vfill before first,
overlay unbroken={\draw[colexemple,line width=1pt]
([yshift=-1pt]title.north east)--([xshift=-0.5pt,yshift=-1pt]title.north-|frame.east)
--([xshift=-0.5pt]frame.south east)--(frame.south west); },
overlay first={\draw[colexemple,line width=1pt]
([yshift=-1pt]title.north east)--([xshift=-0.5pt,yshift=-1pt]title.north-|frame.east)
--([xshift=-0.5pt]frame.south east); },
overlay middle={\draw[colexemple,line width=1pt] ([xshift=-0.5pt]frame.north east)
--([xshift=-0.5pt]frame.south east); },
overlay last={\draw[colexemple,line width=1pt] ([xshift=-0.5pt]frame.north east)
--([xshift=-0.5pt]frame.south east)--(frame.south west);},%
}

%%%%%%%%%%%%%%%%%%%%%%%%%%%%%%%%%%%%%%%%%%%%%%%%%%%%%%%%%%%%%%%%%%%%%%%%%%%%%%%%
%
% ENUMERATE
% =========
%
%%%%%%%%%%%%%%%%%%%%%%%%%%%%%%%%%%%%%%%%%%%%%%%%%%%%%%%%%%%%%%%%%%%%%%%%%%%%%%%%

\usepackage{enumerate}
\usepackage{enumitem}

% To have special enumerate items like
%
% 1/
% 2/
% 3/

%%%%%%%%%%%%%%%%%%%%%%%%%%%%%%%%%%%%%%%%%%%%%%%%%%%%%%%%%%%%%%%%%%%%%%%%%%%%%%%%
%
% ARRAYS
% ======
%
%%%%%%%%%%%%%%%%%%%%%%%%%%%%%%%%%%%%%%%%%%%%%%%%%%%%%%%%%%%%%%%%%%%%%%%%%%%%%%%%


\usepackage{array}
\usepackage{makecell} % Used to break lines within arrays
\usepackage{multirow}
\usepackage{booktabs} % Used to have nice arrays with headrules

%%%%%%%%%%%%%%%%%%%%%%%%%%%%%%%%%%%%%%%%%%%%%%%%%%%%%%%%%%%%%%%%%%%%%%%%%%%%%%%%
%
% WRITE CODE
% ==========
%
%%%%%%%%%%%%%%%%%%%%%%%%%%%%%%%%%%%%%%%%%%%%%%%%%%%%%%%%%%%%%%%%%%%%%%%%%%%%%%%%

\usepackage{listings}
\usepackage{xcolor}

%New colors defined below
\definecolor{codegreen}{rgb}{0,0.6,0}
\definecolor{codegray}{rgb}{0.5,0.5,0.5}
\definecolor{codepurple}{rgb}{0.58,0,0.82}
\definecolor{backcolour}{rgb}{0.95,0.95,0.92}

%Code listing style named "mystyle"
\lstdefinestyle{python}{
  %backgroundcolor=\color{backcolour},
  commentstyle=\color{codegreen},
  keywordstyle=\color{magenta},
  numberstyle=\tiny\color{codegray},
  stringstyle=\color{codepurple},
  basicstyle=\ttfamily\footnotesize,
  breakatwhitespace=false,
  breaklines=true,
  captionpos=b,
  keepspaces=true,
  numbers=left,
  numbersep=5pt,
  showspaces=false,
  showstringspaces=false,
  showtabs=false,
  tabsize=2
}

\lstset{style=python}

%%%%%%%%%%%%%%%%%%%%%%%%%%%%%%%%%%%%%%%%%%%%%%%%%%%%%%%%%%%%%%%%%%%%%%%%%%%%%%%%
%
% Tabular 
% =======
%
%%%%%%%%%%%%%%%%%%%%%%%%%%%%%%%%%%%%%%%%%%%%%%%%%%%%%%%%%%%%%%%%%%%%%%%%%%%%%%%%

% In order to obtain a tabular with given width.

\usepackage{tabularx}
\newcolumntype{Y}{>{\centering\arraybackslash}X}
\newcolumntype{R}{>{\raggedright\arraybackslash}X}
\newcolumntype{L}{>{\raggedleft\arraybackslash}X}
% \usepackage{tabulary} % younger brother

%%%%%%%%%%%%%%%%%%%%%%%%%%%%%%%%%%%%%%%%%%%%%%%%%%%%%%%%%%%%%%%%%%%%%%%%%%%%%%%%
%
% MACROS
% ======
%
%%%%%%%%%%%%%%%%%%%%%%%%%%%%%%%%%%%%%%%%%%%%%%%%%%%%%%%%%%%%%%%%%%%%%%%%%%%%%%%%

% Math Operators

\DeclareMathOperator{\Card}{Card}
\DeclareMathOperator{\Gal}{Gal}
\DeclareMathOperator{\Id}{Id}
\DeclareMathOperator{\Img}{Im}
\DeclareMathOperator{\Ker}{Ker}
\DeclareMathOperator{\Minpoly}{Minpoly}
\DeclareMathOperator{\Mod}{mod}
\DeclareMathOperator{\Ord}{Ord}
\DeclareMathOperator{\ppcm}{ppcm}
\DeclareMathOperator{\pgcd}{pgcd}
\DeclareMathOperator{\tr}{Tr}
\DeclareMathOperator{\Vect}{Vect}
\DeclareMathOperator{\Span}{Span}
\DeclareMathOperator{\rank}{rank}
\DeclareMathOperator{\rg}{rg}
\DeclareMathOperator{\ev}{ev}
\DeclareMathOperator{\Var}{Var}

% Shortcuts

\newcommand{\eg}{\emph{e.g. }}
\newcommand{\ent}[2]{[\![#1,#2]\!]}
\newcommand{\ie}{\emph{i.e. }}
\newcommand{\ps}[2]{\left\langle#1,#2\right\rangle}
\newcommand{\eqdef}{\overset{\text{def}}{=}}
\newcommand{\E}{\mathcal{E}}
\newcommand{\M}{\mathcal{M}}
\newcommand{\A}{\mathcal{A}}
\newcommand{\B}{\mathcal{B}}
\newcommand{\R}{\mathcal{R}}
\newcommand{\D}{\mathcal{D}}
\newcommand{\Pcal}{\mathcal{P}}
\newcommand{\K}{\mathbf{k}}
\newcommand{\vect}[1]{\overrightarrow{#1}}


%\input{layout-nb.tex}

\newcommand{\titrechapitre}{Dérivation -- Cours}
\newcommand{\titreclasse}{Lycée Jean-Baptiste \textsc{Corot}}
\newcommand{\pagination}{\thepage/\pageref{LastPage}}

\title{Chapitre 2 : Dérivation}
\date{}
\author{}

\begin{document}
\maketitle\thispagestyle{fancy}

\newcommand{\Cf}{\mathscr{C}_f}

\section{Nombre dérivé et tangente}
On considère $f$ une fonction définie sur un intervalle $I$ de $\mathbb{R}$ et
on note $\Cf$ sa courbe représentative dans un repère du plan. Soit $a$ un réel
appartenant à $I$ et $A$ le point de $\Cf$ d'abscisse $a$.

\subsection{Nombre dérivé et taux de variation}
Soient $h\in\mathbb{R}$ un réel non nul tel que $a+h\in I$ et $H$ le point de
$\Cf$ d'abscisse $a+h$. En particulier : $a\neq a+h$. 
\begin{defi}{Taux de variation}
  \begin{minipage}{.6\textwidth}
    Le nombre $\tau(h) =\frac{f(a+h)-f(a)}{h}$ est appelé \textbf{taux de
    variation} de $f$ entre $a$ et $a+h$. Sur la
    \href{https://www.geogebra.org/m/bj5pnzvr}{figure} ci-contre,
    le point $A$ a pour coordonnées $(a; f(a))$ et le point $H$ a pour coordonnés
    $(a+h; f(a+h))$.\\
    Le coefficient directeur de la droite $(AH)$ est
    \[
      \frac{f(a+h)-f(a)}{a+h-a};
    \]
    autrement dit, le coefficient directeur est $\tau(h)$. Le nombre $\tau(h)$
    dépend de $a$.
  \end{minipage}
  \begin{minipage}{.4\textwidth}
    \begin{center}
      \begin{tikzpicture}[scale=.85]
        \begin{axis}[x=1cm, y=1cm, xmin=-2.5, ymin=-.5, xmax=3.5]
          \addplot[red, semithick, samples=101]{x^2-x+1};
        \end{axis}
        \draw[blue!80!black] (2.75, 0) -- (5.75, 6);
        \node[blue!80!black] (H) at (4.8, 3.2) {$H$};
        \node[blue!80!black] (A) at (3.7, 1.3) {$A$};

        \draw[blue!80!black, semithick, dashed] (3.5, 1.5) -- (3.5, .5);
        \draw[blue!80!black, semithick, dashed] (4.5, 3.5) -- (4.5, .5);
        \draw[blue!80!black, semithick, dashed] (3.5, 1.5) -- (2.5, 1.5);
        \draw[blue!80!black, semithick, dashed] (4.5, 3.5) -- (2.5, 3.5);

        \node[blue!80!black] (a) at (3.5, 0) {\scriptsize $a$};
        \node[blue!80!black] (ah) at (4.5, 0) {\scriptsize $a+h$};
        \node[blue!80!black] (fa) at (1.6, 1.5) {\scriptsize $f(a)$};
        \node[blue!80!black] (fah) at (1.6, 3.5) {\scriptsize $f(a+h)$};
      \end{tikzpicture}
    \end{center}
  \end{minipage}
\end{defi}
\begin{rmq}
  Le taux de variation est aussi appelé \textbf{taux d'accroissement} entre $a$
  et $a+h$.
\end{rmq}

\begin{defi}{Nombre dérivée et dérivabilité}
  On dit que $f$ est \textbf{dérivable} en $a$ lorsque $\tau(h)$ tend vers un
  nombre réel quand $h$ prend des valeurs proches de $0$. Ce réel est appelé
  \textbf{nombre dérivée} de $f$ en $a$ et est noté $f'(a)$. On écrit alors
  \[
    f'(a) = \underset{h\to0}{\lim}\frac{f(a+h)-f(a)}{h}.
  \]
\end{defi}

\begin{exemple}
  Soit $f:x\mapsto 2x+1$. Soit $a\in\mathbb{R}$ un réel quelconque. Alors
  $f(a)=2a+1$ et $f(a+h)=2(a+h)+1=2a+2h+1$. Ainsi,
  pour tout $h\neq0$ et tout $a\in I$, on a
  \[
    \tau(h) = \frac{f(a+h)-f(a)}{h} = \frac{2a+2h+1-(2a+1)}{h} =
    \frac{2h}{h}=2.
  \]
  On a ainsi $\underset{h\to0}{\lim}\tau(h)=2$. On dit que la fonction $f$ est
  dérivable au point $a$ et que son nombre dérivé en $a$ vaut $f'(a)=2$.
\end{exemple}

\begin{exemple}
  Soit $f:x\mapsto x^2$. Pour $h\neq0$ et $a=0$,
  $\tau(h)=\frac{f(0+h)-f(0)}{h}=\frac{h^2}{h}=h$. On a
  $\underset{h\to0}{\lim}\tau(h)=0$ donc $f$ est dérivable en $0$ et $f'(0)=0$.
\end{exemple}

\begin{exemple}
  \begin{minipage}{.7\textwidth}
  Soit $g$ la fonction définie sur $\mathbb{R}$ par $g(x)=|x|$. Pour $h\neq0$ et
  $a=0$,
  \[
    \tau(h)=\frac{g(0+h)-g(0)}{h}=\frac{|h|-|0|}{h}=\frac{|h|-|0|}{h}=\frac{|h|}{h}.
  \]
  Pour $h>0$, $\tau(h)=\frac{h}{h}=1$, et pour $h<0$,
  $\tau(h)=\frac{-h}{h}=-1$. On obtient deux nombres différents quand $h$ prend
  des valeurs proches de $0$, donc $g$ n'est pas dérivable en $0$.
\end{minipage}
  \begin{minipage}{.3\textwidth}
  \begin{center}
    \begin{tikzpicture}[scale=.8]
      \begin{axis}
        \addplot[red, semithick, samples=101, domain=-5.5:5.5]{abs(x)};
      \end{axis}
    \end{tikzpicture}
  \end{center}
\end{minipage}
\end{exemple}
\begin{app}
  On considère la fonction $f$ définie sur $\mathbb{R}$ par
  $f(x)=-\frac{1}{3}x+1$.
  \begin{enumerate}
    \item Soit $h$ un réel non nul. Calculer $f(3+h)$ et $f(3)$.
    \item Montrer que $f$ est dérivable en $3$ et déterminer le nombre dérivé de
      $f$ en $3$.
  \end{enumerate}
\end{app}

\subsection{Tangente à une courbe}

\begin{defi}{Tangente}
  \begin{minipage}{.6\textwidth}
    Lorsque $f$ est dérivable en $a$, on appelle
    \href{https://www.geogebra.org/m/bunmvhxv}{\textbf{tangente}} à la courbe
    $\Cf$ au point d'abscisse $a$ la droite $T$ passant par $A(a; f(a))$ dont le
    coefficient directeur est le nombre dérivé $f'(a)$.
  \end{minipage}
  \begin{minipage}{.4\textwidth}
\begin{center}
      \begin{tikzpicture}[scale=.85]
        \begin{axis}[x=1cm, y=1cm, xmin=-2.5, ymin=-.5, xmax=3.5]
          \addplot[red, semithick, samples=101]{-x^2+2*x+1};
        \end{axis}
        \draw[blue!80!black] (1.46, 0) -- (5.46, 6);
        \node[blue!80!black] (A) at (3, 1.8) {$A$};

        \draw[blue!80!black, semithick, dashed] (2.75, 1.94) -- (2.75, .5);
        \draw[blue!80!black, semithick, densely dashed] (2.75, 1.94) -- (2.5, 1.94);

        \node[blue!80!black] (a) at (2.75, 0.25) {\scriptsize $a$};
        \node[blue!80!black] (fa) at (2, 1.94) {\scriptsize $f(a)$};

        \draw[blue!80!black, semithick, densely dashed] (3.5, 3) -- (4.5, 3) --
        (4.5, 4.5);

        \node[blue!80!black] (1) at (4, 2.8) {\scriptsize $1$};
        \node[blue!80!black] (f'a) at (5, 3.75) {\scriptsize $f'(a)$};
        \node[red!80!black] (Cf) at (5.2, 1) {$\Cf$};
        \node[blue!80!black] (T) at (4.7, 5.2) {$T$};
      \end{tikzpicture}
\end{center}
\end{minipage}
\end{defi}

\begin{exemple}
   \begin{minipage}{.6\textwidth}
     On donne la courbe d'une fonction $f$ dérivable en $3$, dont on a tracé la
     tangente $T_A$ au point d'abscisse $3$. Par lecture graphique, on a
     $f'(3)=2$.
  \end{minipage}
  \begin{minipage}{.4\textwidth}
\begin{center}
      \begin{tikzpicture}[scale=.85]
        \begin{axis}[x=1cm, y=1cm, xmin=-.5, ymin=-.5, xmax=5.5]
          \addplot[red, semithick, samples=101]{x^2-4*x+5};
        \end{axis}
        \draw[blue!80!black] (2.25, 0) -- (5.25, 6);
        \node[blue!80!black] (A) at (3.3, 2.8) {$A$};

        \draw[blue!80!black, semithick, densely dashed] (3.5, 2.5) -- (4.5, 2.5) --
        (4.5, 4.5);

        \node[blue!80!black] (1) at (4, 2.3) {\scriptsize $+1$};
        \node[blue!80!black] (f'a) at (4.8, 3.75) {\scriptsize $+2$};
        \node[blue!80!black] (T) at (5.3, 5.2) {$T_A$};
        \node[red!80!black] (Cf) at (1.2, 4.2) {$\Cf$};
      \end{tikzpicture}
    \end{center}
  \end{minipage}
\end{exemple}

\begin{app}
    \begin{minipage}{.6\textwidth}
    On considère la fonction $f$ définie sur $]0;+\infty[$ par
      $f(x)=\frac{1}{x}$ dont on a tracé la courbe représentative $\Cf$. La
      doite $(AB)$ est la tangente à $\Cf$ au point $A$ d'abscisse $2$.\\
      Déterminer graphiquement $f'(2)$.
  \end{minipage}
  \begin{minipage}{.4\textwidth}
\begin{center}
      \begin{tikzpicture}[scale=.85]
        \begin{axis}[x=1cm, y=1cm, xmin=-.5, ymin=-.5, xmax=5.5, ymax=2.5]
          \addplot[red, semithick, samples=101, domain=0:5.5]{1/x};
        \end{axis}

        \draw[blue!80!black] (0, 1.625) -- (6, .125);
        \node[blue!80!black] (A) at (2.7, 1.3) {$A$};
        \node[blue!80!black] (B) at (4.8, .8) {$B$};

        \draw[blue!80!black, thick] (2.5-.1, 1)--(2.5+.1, 1) (2.5, 1-.1)--(2.5,
        1+.1);
        \draw[blue!80!black, thick] (4.5-.1, .5)--(4.5+.1, .5) (4.5, .5-.1)--(4.5,
        .5+.1);

        \node[red!80!black] (Cf) at (1.7, 2.2) {$\Cf$};
      \end{tikzpicture}
    \end{center}
  \end{minipage}
\end{app}

\section{Équation de tangente}
\begin{prop}
  Soit $f$ une fonction dérivable en $a$. L'équation réduite de la tangente
  $T_A$ à la courbe de $f$ au point d'abscisse $a$ est
  \[
    y = f'(a)(x-a)+f(a).
  \]
\end{prop}
\begin{proof}
  Soit $T_A$ la tangente au point $A$ d'abscisse $a$ de $\Cf$. Par définition,
  le nombre dérivé $f'(a)$ est le coefficient directeur de cette tangente, et
  elle a donc pour équation
  \[
    y = f'(a)x + p,
  \]
  où le nombre $p$ est l'ordonnée à l'origine et reste à déterminer. Comme le
  point $A(a; f(a))$ appartient à $T_A$, ses coordonnées vérifient l'équation
  réduite de $T_A$. On a donc
  \[
    f(a) = f'(a)\times a+p\Longleftrightarrow p=f(a)-f'(a)\times
  a.
  \]
En remplaçant cette valeur dans l'équation réduite de $T_A$ et en
  factorisant par $f'(a)$, on a bien
  \[
    y = f'(a)(x-a)+f(a).
  \]
\end{proof}
\begin{exemple}
  Soit $f$ une fonction telle que $f(1)2$ et $f'(1)=\frac{1}{3}$. La tangente
  $T$ à la courbe de $f$ au point d'abscisse $1$ a donc pour équation réduite
  \[
    y = f'(1)(x-1)+f(1).
  \]
  Cela donne
  \begin{align*}
   y &= \frac{1}{3}(x-1)+2 \\
    \Leftrightarrow y &= \frac{1}{3}x-\frac{1}{3}+2 \\
    \Leftrightarrow y &= \frac{1}{3}x+\frac{5}{3}.
  \end{align*}
\end{exemple}
\begin{app}
  Soit $f$ la fonction définie sur $\mathbb{R}$ par $f(x) = x^3-x^2+x-1$. En
  remarquant que, pour tout $x\in\mathbb{R}$, $f(x)=(x-1)(x^2+1)$ :
  \begin{enumerate}
    \item déterminer la tangente $T$ à la courbe représentative de $f$ au point
      d'abscisse $1$;
    \item en déduire les coordonnées du point d'intersection de $T$ avec l'axe
      des ordonnées.
  \end{enumerate}
\end{app}

\section{Fonctions dérivées}
\begin{defi}{Fonction dérivée}
  On dit que $f$ est \textbf{dérivable sur un intervalle} $I$ lorsque $f$ est
  dérivable en tout réel $a$ de $I$. On appelle \textbf{fonction dérivée} de $f$
  la fonction qui, à tout réel $x$ de $I$, associe le réel $f'(x)$. On la note
  $f'$.
\end{defi}
\subsection{Fonctions dérivées des fonctions usuelles}
\begin{prop}
  \[
    \def\arraystretch{2}
    \begin{array}{|c|c|c|c|}
      \hline
      \text{Fonction }f\text{ définie par :} & \makecell{\text{Ensemble
      de}\\\text{définition }D_f} & \makecell{\text{Fonction dérivée}\\f'\text{
      définie par :}} & \makecell{\text{Ensemble de }\\\text{dérivabilité
      }D_{f'}}
      \\
      \hline
      f(x)=k\text{, avec }k\in\mathbb{R} & \mathbb{R} & f'(x)=0 & \mathbb{R} \\
      \hline
      \makecell{f(x)=mx+p\\\text{avec }m, p\in\mathbb{R}} & \mathbb{R} & f'(x)=m
      & \mathbb{R} \\
      \hline
      f(x)= x^2 & \mathbb{R} & f'(x)=2x & \mathbb{R} \\
      \hline
      f(x)= x^n\text{, avec }n\in\mathbb{N}^* & \mathbb{R} & f'(x)=nx^{n-1} & \mathbb{R} \\
      \hline
      f(x)= \frac{1}{x} & \mathbb{R}\setminus\left\{ 0 \right\} &
      f'(x)=-\frac{1}{x^2} & \mathbb{R}\setminus\left\{ 0 \right\} \\
      \hline
      f(x)= \frac{1}{x^n} & \mathbb{R}\setminus\left\{ 0 \right\} &
      f'(x)=-\frac{n}{x^{n+1}} & \mathbb{R}\setminus\left\{ 0 \right\} \\
      \hline
      f(x)= \sqrt x & [0; +\infty[ & f'(x)=\frac{1}{2\sqrt x} & ]0; +\infty[ \\
      \hline
    \end{array}
  \]
\end{prop}

\begin{exemple}
  La fonction $g$ est définie sur $\mathbb{R}\setminus\left\{ 0 \right\}$ par
  $g(x)=\frac{1}{x^2}$. La fonction $g$ est dérivable sur $\mathbb{R}^*$ et pour
  tout $x\neq0$, on a
  \[
    g'(x) = -\frac{2}{x^{2+1}}=-\frac{2}{x^3}.
  \]
\end{exemple}

\begin{exemple}
  Soit $f$ la fonction définie sur $\mathbb{R}$ par $f(x)=4x+5$. Alors
  \[
    f'(x) = 4.
  \]
\end{exemple}

\begin{exemple}
  Soit $f$ la fonction définie sur $\mathbb{R}$ par $f(x)=x^3$. Alors
  \[
    f'(x) = 3x^2.
  \]
\end{exemple}

\begin{exemple}
  Soit $f$ la fonction définie sur $\mathbb{R}$ par $f(x)=x^5$. Alors
  \[
    f'(x) = 5x^4.
  \]
\end{exemple}

\begin{rmq}
  Attention, la fonction racine carrée n'est pas dérivable sur tout son ensemble
  de définition ! Elle est définie sur $[0;+\infty[$ mais elle est dérivable sur
  $]0;+\infty[$. Autrement dit, elle n'est pas dérivable en $0$.
\end{rmq}

\subsection{Opérations sur les fonctions dérivées}
\begin{prop}
  Soient $u, v$ et $g$ des fonctions définies et dérivables sur un intervalle
  $I$. Soient $k, a$ et $b$ des réels.
  \begin{center}
  \def\arraystretch{2}
\begin{tabular}{|c|c|c|}
  \hline
  \textbf{Type d'opération} & \textbf{Fonction à dériver} & \textbf{Fonction
  dérivée} \\
  \hline
  Dérivée d'une somme & $u+v$ & $(u+v)' = u'+v'$ \\
  \hline
  Dérivée d'un produit par une constante & $k\times u$ & $(k\times u)' = k\times
  u'$ \\
  \hline
  Dérivée d'un produit & $u\times v$ & $(u\times v)'=u'\times v+u\times v'$ \\
  \hline
  Dérivée d'un inverse & \makecell{$\frac{1}{v}$ avec $v(x)\neq0$\\pour tout
$x\in I$} & $\left( \cfrac{1}{v} \right)'=-\cfrac{v'}{v^2}$\\
\hline
  Dérivée d'un quotient & \makecell{$\frac{u}{v}$ avec $v(x)\neq0$\\pour tout
$x\in I$} & $\left( \cfrac{u}{v} \right)'=\cfrac{u'\times v-u\times v'}{v^2}$\\
\hline
\multicolumn{3}{|c|}{\makecell{Dérivée de $f(x)=g(ax+b)$ : soit $J$ l'intervalle tel que
pour tout $x\in J$, $ax+b\in I$.\\ La fonction $f$ est définie et dérivable sur
$J$ et $f'(x)=a\times g'(ax+b)$.}}\\
\hline
\end{tabular}
  \end{center}
\end{prop}
\begin{exemple}
  Soit $f$ la fonction définie sur $\mathbb{R}$ par $f(x)=4x^3-5x^2+2x-1$. En
  tant que fonction polynôme, $f$ est dérivable sur $\mathbb{R}$, et pour tout
  $x\in\mathbb{R}$, on a
  \[
    f'(x) = 12x^2-10x+2.
  \]
\end{exemple}
\begin{exemple}
  Soit $g$ la fonction définie sur $\mathbb{R}\setminus\left\{ 1 \right\}$ par
  $g(x) = \frac{4+x^2}{x+1}$. La fonction $g$ est dérivable sur
  $\mathbb{R}\setminus\left\{ 1 \right\}$ en tant que fonction rationnelle, et
  pour tout $x\in\mathbb{R}\setminus\left\{ 1 \right\}$, on a
  \[
    g'(x) = \frac{2x(x+1)-(4+x^2)\times 1}{(x+1)^2} = \frac{x^2+2x-4}{(x+1)^2}.
  \]
\end{exemple}

\begin{app}
  Soit $f$ la fonction définie sur $[0;+\infty[$ par $f(x)=x\sqrt x$.\\
    Donner l'ensemble de dérivabilité et la fonction dérivée de la fonction $f$.
\end{app}

\begin{app}
  Soit $f$ la fonction définie sur $I$ par $f(x)=\frac{\sqrt x}{x+1}$.
  \begin{enumerate}
    \item Déterminer l'ensemble $I$.
    \item Justifier que $f$ est dérivable en précisant l'ensemble de
      dérivabilité et déterminer sa fonction dérivée.
  \end{enumerate}
\end{app}

\end{document}
