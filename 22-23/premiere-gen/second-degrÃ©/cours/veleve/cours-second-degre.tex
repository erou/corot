\documentclass[11pt]{article}

\newcommand{\titrechapitre}{Second degré -- Cours}
\newcommand{\titreclasse}{Lycée Jean-Baptiste \textsc{Corot}}
\newcommand{\pagination}{\thepage/\pageref{LastPage}}
\newcommand{\topbotmargins}{2cm}

%%%%%%%%%%%%%%%%%%%%%%%%%%%%%%%%%%%%%%%%%%%%%%%%%%%%%%%%%%%%%%%%%%%%%%%%%%%%%%%%
%
% PACKAGES
% ========
%
%%%%%%%%%%%%%%%%%%%%%%%%%%%%%%%%%%%%%%%%%%%%%%%%%%%%%%%%%%%%%%%%%%%%%%%%%%%%%%%%

\usepackage[english, french]{babel}
\usepackage[utf8]{inputenc}
\usepackage[T1]{fontenc}
\usepackage{graphicx}
\usepackage{amsmath,amssymb,amsthm,amsopn}
\usepackage{hyperref}

% Pour avoir l'écriture \mathscr (math script)
% ============================================

\usepackage{mathrsfs}

% Deal with coma as a decimal separator
% =====================================

\usepackage{icomma}

% Package Geometry
% ================

\usepackage[a4paper, lmargin=2cm, rmargin=2cm, top=\topbotmargins, bottom=\topbotmargins]{geometry}

% Package multicol
% ================

\usepackage{multicol}

% Redefine abstract
% =================

% Note
% ====
%
% Le reste a été commenté pour ne pas charger trop de choses au démarrage. On
% verra si on en a besoin plus tard.
%
% --------
%
%\usepackage{mathrsfs}
%\usepackage{multirow}
%\usepackage{bm}
%\hypersetup{
%    colorlinks=true,
%    linkcolor=blue,
%    citecolor=red,
%}
%\usepackage{diagbox}
%
%\usepackage{algorithm}
%\usepackage{algpseudocode}
%
%\renewcommand{\algorithmicrequire}{\textbf{Input:}}
%\renewcommand{\algorithmicensure}{\textbf{Output:}}


%%%%%%%%%%%%%%%%%%%%%%%%%%%%%%%%%%%%%%%%%%%%%%%%%%%%%%%%%%%%%%%%%%%%%%%%%%%%%%%%
%
% TIKZ
% ====
%
%%%%%%%%%%%%%%%%%%%%%%%%%%%%%%%%%%%%%%%%%%%%%%%%%%%%%%%%%%%%%%%%%%%%%%%%%%%%%%%%

\usepackage{tikz}
\usetikzlibrary{arrows}

\usepackage{tkz-tab} % Variation tables

\usepackage{pgfplots}
%\usepackage{pgf-pie} % Pie charts

\pgfplotsset{
%\newcommand{\settingsgraph}{
x=.5cm,y=.5cm,
xticklabel style = {font=\scriptsize, yshift=.1cm},
yticklabel style = {font=\scriptsize, xshift=.1cm},
axis lines=middle,
ymajorgrids=true,
xmajorgrids=true,
major grid style = {color=white!80!blue},
xmin=-5.5,
xmax=5.5,
ymin=-5.5,
ymax=5.5,
xtick={-5.0,-4.0,...,5.0},
ytick={-5.0,-4.0,...,5.0},
}

% Tikz style

\tikzset{round/.style={circle, draw=black, very thick, scale = 0.7}}
\tikzset{arrow/.style={->, >=latex}}
\tikzset{dashed-arrow/.style={->, >=latex, dashed}}

\newcommand{\point}[3]{\draw[very thick, #3] (#1-.1, #2)--(#1+.1, #2)
(#1, #2-.1)--(#1, #2+.1)}

%%%%%%%%%%%%%%%%%%%%%%%%%%%%%%%%%%%%%%%%%%%%%%%%%%%%%%%%%%%%%%%%%%%%%%%%%%%%%%%%
%
% FANCY HEADER
% ============
%
%%%%%%%%%%%%%%%%%%%%%%%%%%%%%%%%%%%%%%%%%%%%%%%%%%%%%%%%%%%%%%%%%%%%%%%%%%%%%%%%


\usepackage{fancyhdr}
\usepackage{lastpage}

\pagestyle{fancy}
\newcommand{\changefont}{\fontsize{9}{9}\selectfont}
\renewcommand{\headrulewidth}{0mm}
\renewcommand{\footrulewidth}{0mm}

\fancyhead[C]{}
\fancyhead[L]{\titreclasse}
\fancyhead[R]{\titrechapitre}
\fancyfoot[C]{}
\fancyfoot[L]{}
\fancyfoot[R]{\pagination}
\addtolength{\skip\footins}{20pt} % distance between text and footnotes

%%%%%%%%%%%%%%%%%%%%%%%%%%%%%%%%%%%%%%%%%%%%%%%%%%%%%%%%%%%%%%%%%%%%%%%%%%%%%%%%
%
% THEOREM STYLE
% =============
%
%%%%%%%%%%%%%%%%%%%%%%%%%%%%%%%%%%%%%%%%%%%%%%%%%%%%%%%%%%%%%%%%%%%%%%%%%%%%%%%%

\usepackage[tikz]{bclogo}
\usepackage{mdframed}

\usepackage{tcolorbox}
\tcbuselibrary{listings, breakable, theorems, skins}

%\newtheoremstyle{break}%
%{}{}%
%{\itshape}{}%
%{\bfseries}{}%  % Note that final punctuation is omitted.
%{\newline}{}

\newtheoremstyle{scbf}%
{}{}%
{}{}%
%{\scshape}{}%  % Note that final punctuation is omitted.
{\bfseries\scshape}{}%  % Note that final punctuation is omitted.
{\newline}{}

%\theoremstyle{break}
%\theoremstyle{plain}
%\newtheorem{thm}{Theorem}[section]
%\newtheorem{lm}[thm]{Lemma}
%\newtheorem{prop}[thm]{Proposition}
%\newtheorem{cor}[thm]{Corollary}

%\theoremstyle{scbf}
%\newtheorem{exo}{$\star$ Exercice}

%\theoremstyle{definition}
%\newtheorem{defi}[thm]{Definition}
%\newtheorem{ex}[thm]{Example}

%\theoremstyle{remark}
%\newtheorem{rem}[thm]{Remark}

% Defining the Remark environment
% ===============================

\newenvironment{rmq}
  {
    \begin{bclogo}[logo=\bcinfo, noborder=true]{Remarque}
  }
  {
    \end{bclogo}
  }

% Defining the exercise environment
% =================================

\newcounter{exos}
\setcounter{exos}{1}

\newenvironment{exo}
  {
    \begin{bclogo}[logo=\bccrayon, noborder=true]{Exercice \theexos}
  }
  {
    \end{bclogo}
    \addtocounter{exos}{1}
  }


% Redefining the proof environment from amsthm
% ============================================

\tcolorboxenvironment{proof}{
  blanker, breakable, before skip=10pt,after skip=10pt,
  borderline west={1mm}{0pt}{red},
  left=5mm,
}

% Defining the definition environment
% ===================================

\colorlet{coldef}{black!50!green}

\newcounter{defis}
\setcounter{defis}{1}

\newenvironment{defi}[1]
  {
    \begin{defihid}{{#1}}{\thedefis}
  }
  {
    \end{defihid}
    \addtocounter{defis}{1}
  }

\newtcolorbox{defihid}[2]{%
  empty,title={ {\bfseries Définition {#2}} ({#1})},attach boxed title to top left,
boxed title style={empty,size=minimal,toprule=2pt,top=4pt,
overlay={\draw[coldef,line width=2pt]
([yshift=-1pt]frame.north west)--([yshift=-1pt]frame.north east);}},
coltitle=coldef,
before=\par\medskip\noindent,parbox=false,boxsep=0pt,left=0pt,right=3mm,top=4pt,
breakable,pad at break*=0mm,vfill before first,
overlay unbroken={\draw[coldef,line width=1pt]
([yshift=-1pt]title.north east)--([xshift=-0.5pt,yshift=-1pt]title.north-|frame.east)
--([xshift=-0.5pt]frame.south east)--(frame.south west); },
overlay first={\draw[coldef,line width=1pt]
([yshift=-1pt]title.north east)--([xshift=-0.5pt,yshift=-1pt]title.north-|frame.east)
--([xshift=-0.5pt]frame.south east); },
overlay middle={\draw[coldef,line width=1pt] ([xshift=-0.5pt]frame.north east)
--([xshift=-0.5pt]frame.south east); },
overlay last={\draw[coldef,line width=1pt] ([xshift=-0.5pt]frame.north east)
--([xshift=-0.5pt]frame.south east)--(frame.south west);},%
}

\newenvironment{notation}
  {
    \begin{notationhid}{\thedefis}
  }
  {
    \end{notationhid}
    \addtocounter{defis}{1}
  }

\newtcolorbox{notationhid}[1]{%
  empty,title={Notation {#1}},attach boxed title to top left,
boxed title style={empty,size=minimal,toprule=2pt,top=4pt,
overlay={\draw[coldef,line width=2pt]
([yshift=-1pt]frame.north west)--([yshift=-1pt]frame.north east);}},
coltitle=coldef,fonttitle=\bfseries,
before=\par\medskip\noindent,parbox=false,boxsep=0pt,left=0pt,right=3mm,top=4pt,
breakable,pad at break*=0mm,vfill before first,
overlay unbroken={\draw[coldef,line width=1pt]
([yshift=-1pt]title.north east)--([xshift=-0.5pt,yshift=-1pt]title.north-|frame.east)
--([xshift=-0.5pt]frame.south east)--(frame.south west); },
overlay first={\draw[coldef,line width=1pt]
([yshift=-1pt]title.north east)--([xshift=-0.5pt,yshift=-1pt]title.north-|frame.east)
--([xshift=-0.5pt]frame.south east); },
overlay middle={\draw[coldef,line width=1pt] ([xshift=-0.5pt]frame.north east)
--([xshift=-0.5pt]frame.south east); },
overlay last={\draw[coldef,line width=1pt] ([xshift=-0.5pt]frame.north east)
--([xshift=-0.5pt]frame.south east)--(frame.south west);},%
}


% Defining the proposition, theorem, etc. environment
% ===================================================

\colorlet{colprop}{red!75!black}

\newcounter{props}
\setcounter{props}{1}

\newenvironment{prop}
  {
    \begin{prophid}{\theprops}
  }
  {
    \end{prophid}
    \refstepcounter{props}
  }

\newtcolorbox{prophid}[1]{%
empty,title={Propriété {#1}},attach boxed title to top left,
boxed title style={empty,size=minimal,toprule=2pt,top=4pt,
overlay={\draw[colprop,line width=2pt]
([yshift=-1pt]frame.north west)--([yshift=-1pt]frame.north east);}},
coltitle=colprop,fonttitle=\bfseries,
before=\par\medskip\noindent,parbox=false,boxsep=0pt,left=0pt,right=3mm,top=4pt,
breakable,pad at break*=0mm,vfill before first,
overlay unbroken={\draw[colprop,line width=1pt]
([yshift=-1pt]title.north east)--([xshift=-0.5pt,yshift=-1pt]title.north-|frame.east)
--([xshift=-0.5pt]frame.south east)--(frame.south west); },
overlay first={\draw[colprop,line width=1pt]
([yshift=-1pt]title.north east)--([xshift=-0.5pt,yshift=-1pt]title.north-|frame.east)
--([xshift=-0.5pt]frame.south east); },
overlay middle={\draw[colprop,line width=1pt] ([xshift=-0.5pt]frame.north east)
--([xshift=-0.5pt]frame.south east); },
overlay last={\draw[colprop,line width=1pt] ([xshift=-0.5pt]frame.north east)
--([xshift=-0.5pt]frame.south east)--(frame.south west);},%
}

\newenvironment{propadm}
  {
    \begin{propadmhid}{\theprops}
  }
  {
    \end{propadmhid}
    \refstepcounter{props}
  }

  \newtcolorbox{propadmhid}[1]{%
    empty,title={{\bfseries Propriété {#1}} (admise)},attach boxed title to top left,
boxed title style={empty,size=minimal,toprule=2pt,top=4pt,
overlay={\draw[colprop,line width=2pt]
([yshift=-1pt]frame.north west)--([yshift=-1pt]frame.north east);}},
coltitle=colprop,%fonttitle=\bfseries,
before=\par\medskip\noindent,parbox=false,boxsep=0pt,left=0pt,right=3mm,top=4pt,
breakable,pad at break*=0mm,vfill before first,
overlay unbroken={\draw[colprop,line width=1pt]
([yshift=-1pt]title.north east)--([xshift=-0.5pt,yshift=-1pt]title.north-|frame.east)
--([xshift=-0.5pt]frame.south east)--(frame.south west); },
overlay first={\draw[colprop,line width=1pt]
([yshift=-1pt]title.north east)--([xshift=-0.5pt,yshift=-1pt]title.north-|frame.east)
--([xshift=-0.5pt]frame.south east); },
overlay middle={\draw[colprop,line width=1pt] ([xshift=-0.5pt]frame.north east)
--([xshift=-0.5pt]frame.south east); },
overlay last={\draw[colprop,line width=1pt] ([xshift=-0.5pt]frame.north east)
--([xshift=-0.5pt]frame.south east)--(frame.south west);},%
}

\newenvironment{propnom}[1]
  {
    \begin{propnomhid}{#1}{\theprops}
  }
  {
    \end{propnomhid}
    \refstepcounter{props}
  }

\newtcolorbox{propnomhid}[2]{%
empty,title={{\bfseries Propriété {#2}} ({#1})},attach boxed title to top left,
boxed title style={empty,size=minimal,toprule=2pt,top=4pt,
overlay={\draw[colprop,line width=2pt]
([yshift=-1pt]frame.north west)--([yshift=-1pt]frame.north east);}},
coltitle=colprop,
before=\par\medskip\noindent,parbox=false,boxsep=0pt,left=0pt,right=3mm,top=4pt,
breakable,pad at break*=0mm,vfill before first,
overlay unbroken={\draw[colprop,line width=1pt]
([yshift=-1pt]title.north east)--([xshift=-0.5pt,yshift=-1pt]title.north-|frame.east)
--([xshift=-0.5pt]frame.south east)--(frame.south west); },
overlay first={\draw[colprop,line width=1pt]
([yshift=-1pt]title.north east)--([xshift=-0.5pt,yshift=-1pt]title.north-|frame.east)
--([xshift=-0.5pt]frame.south east); },
overlay middle={\draw[colprop,line width=1pt] ([xshift=-0.5pt]frame.north east)
--([xshift=-0.5pt]frame.south east); },
overlay last={\draw[colprop,line width=1pt] ([xshift=-0.5pt]frame.north east)
--([xshift=-0.5pt]frame.south east)--(frame.south west);},%
}




\newenvironment{thm}
  {
    \begin{thmhid}{\theprops}
  }
  {
    \end{thmhid}
    \refstepcounter{props}
  }

\newtcolorbox{thmhid}[1]{%
empty,title={Théorème {#1}},attach boxed title to top left,
boxed title style={empty,size=minimal,toprule=2pt,top=4pt,
overlay={\draw[colprop,line width=2pt]
([yshift=-1pt]frame.north west)--([yshift=-1pt]frame.north east);}},
coltitle=colprop,fonttitle=\bfseries,
before=\par\medskip\noindent,parbox=false,boxsep=0pt,left=0pt,right=3mm,top=4pt,
breakable,pad at break*=0mm,vfill before first,
overlay unbroken={\draw[colprop,line width=1pt]
([yshift=-1pt]title.north east)--([xshift=-0.5pt,yshift=-1pt]title.north-|frame.east)
--([xshift=-0.5pt]frame.south east)--(frame.south west); },
overlay first={\draw[colprop,line width=1pt]
([yshift=-1pt]title.north east)--([xshift=-0.5pt,yshift=-1pt]title.north-|frame.east)
--([xshift=-0.5pt]frame.south east); },
overlay middle={\draw[colprop,line width=1pt] ([xshift=-0.5pt]frame.north east)
--([xshift=-0.5pt]frame.south east); },
overlay last={\draw[colprop,line width=1pt] ([xshift=-0.5pt]frame.north east)
--([xshift=-0.5pt]frame.south east)--(frame.south west);},%
}

\newenvironment{thmadm}
  {
    \begin{thmadmhid}{\theprops}
  }
  {
    \end{thmadmhid}
    \refstepcounter{props}
  }

  \newtcolorbox{thmadmhid}[1]{%
    empty,title={{\bfseries Théorème {#1}} (admis)},attach boxed title to top left,
boxed title style={empty,size=minimal,toprule=2pt,top=4pt,
overlay={\draw[colprop,line width=2pt]
([yshift=-1pt]frame.north west)--([yshift=-1pt]frame.north east);}},
coltitle=colprop,%fonttitle=\bfseries,
before=\par\medskip\noindent,parbox=false,boxsep=0pt,left=0pt,right=3mm,top=4pt,
breakable,pad at break*=0mm,vfill before first,
overlay unbroken={\draw[colprop,line width=1pt]
([yshift=-1pt]title.north east)--([xshift=-0.5pt,yshift=-1pt]title.north-|frame.east)
--([xshift=-0.5pt]frame.south east)--(frame.south west); },
overlay first={\draw[colprop,line width=1pt]
([yshift=-1pt]title.north east)--([xshift=-0.5pt,yshift=-1pt]title.north-|frame.east)
--([xshift=-0.5pt]frame.south east); },
overlay middle={\draw[colprop,line width=1pt] ([xshift=-0.5pt]frame.north east)
--([xshift=-0.5pt]frame.south east); },
overlay last={\draw[colprop,line width=1pt] ([xshift=-0.5pt]frame.north east)
--([xshift=-0.5pt]frame.south east)--(frame.south west);},%
}

\newenvironment{thmnom}[1]
  {
    \begin{thmnomhid}{#1}{\theprops}
  }
  {
    \end{thmnomhid}
    \refstepcounter{props}
  }

\newtcolorbox{thmnomhid}[2]{%
empty,title={{\bfseries Théorème {#2}} ({#1})},attach boxed title to top left,
boxed title style={empty,size=minimal,toprule=2pt,top=4pt,
overlay={\draw[colprop,line width=2pt]
([yshift=-1pt]frame.north west)--([yshift=-1pt]frame.north east);}},
coltitle=colprop,
before=\par\medskip\noindent,parbox=false,boxsep=0pt,left=0pt,right=3mm,top=4pt,
breakable,pad at break*=0mm,vfill before first,
overlay unbroken={\draw[colprop,line width=1pt]
([yshift=-1pt]title.north east)--([xshift=-0.5pt,yshift=-1pt]title.north-|frame.east)
--([xshift=-0.5pt]frame.south east)--(frame.south west); },
overlay first={\draw[colprop,line width=1pt]
([yshift=-1pt]title.north east)--([xshift=-0.5pt,yshift=-1pt]title.north-|frame.east)
--([xshift=-0.5pt]frame.south east); },
overlay middle={\draw[colprop,line width=1pt] ([xshift=-0.5pt]frame.north east)
--([xshift=-0.5pt]frame.south east); },
overlay last={\draw[colprop,line width=1pt] ([xshift=-0.5pt]frame.north east)
--([xshift=-0.5pt]frame.south east)--(frame.south west);},%
}

\newenvironment{coro}
  {
    \begin{corohid}{\theprops}
  }
  {
    \end{corohid}
    \refstepcounter{props}
  }

  \newtcolorbox{corohid}[1]{%
  empty,title={Corollaire {#1}},attach boxed title to top left,
boxed title style={empty,size=minimal,toprule=2pt,top=4pt,
overlay={\draw[colprop,line width=2pt]
([yshift=-1pt]frame.north west)--([yshift=-1pt]frame.north east);}},
coltitle=colprop,fonttitle=\bfseries,
before=\par\medskip\noindent,parbox=false,boxsep=0pt,left=0pt,right=3mm,top=4pt,
breakable,pad at break*=0mm,vfill before first,
overlay unbroken={\draw[colprop,line width=1pt]
([yshift=-1pt]title.north east)--([xshift=-0.5pt,yshift=-1pt]title.north-|frame.east)
--([xshift=-0.5pt]frame.south east)--(frame.south west); },
overlay first={\draw[colprop,line width=1pt]
([yshift=-1pt]title.north east)--([xshift=-0.5pt,yshift=-1pt]title.north-|frame.east)
--([xshift=-0.5pt]frame.south east); },
overlay middle={\draw[colprop,line width=1pt] ([xshift=-0.5pt]frame.north east)
--([xshift=-0.5pt]frame.south east); },
overlay last={\draw[colprop,line width=1pt] ([xshift=-0.5pt]frame.north east)
--([xshift=-0.5pt]frame.south east)--(frame.south west);},%
}

\newenvironment{lemme}
  {
    \begin{lemmehid}{\theprops}
  }
  {
    \end{lemmehid}
    \refstepcounter{props}
  }

  \newtcolorbox{lemmehid}[1]{%
  empty,title={Lemme {#1}},attach boxed title to top left,
boxed title style={empty,size=minimal,toprule=2pt,top=4pt,
overlay={\draw[colprop,line width=2pt]
([yshift=-1pt]frame.north west)--([yshift=-1pt]frame.north east);}},
coltitle=colprop,fonttitle=\bfseries,
before=\par\medskip\noindent,parbox=false,boxsep=0pt,left=0pt,right=3mm,top=4pt,
breakable,pad at break*=0mm,vfill before first,
overlay unbroken={\draw[colprop,line width=1pt]
([yshift=-1pt]title.north east)--([xshift=-0.5pt,yshift=-1pt]title.north-|frame.east)
--([xshift=-0.5pt]frame.south east)--(frame.south west); },
overlay first={\draw[colprop,line width=1pt]
([yshift=-1pt]title.north east)--([xshift=-0.5pt,yshift=-1pt]title.north-|frame.east)
--([xshift=-0.5pt]frame.south east); },
overlay middle={\draw[colprop,line width=1pt] ([xshift=-0.5pt]frame.north east)
--([xshift=-0.5pt]frame.south east); },
overlay last={\draw[colprop,line width=1pt] ([xshift=-0.5pt]frame.north east)
--([xshift=-0.5pt]frame.south east)--(frame.south west);},%
}

\colorlet{colexemple}{blue!50!black}
%\newtcolorbox{exemple}{empty, title=Exemple, attach boxed title to top left,
%  boxed title style={empty, size=minimal, toprule=2pt, top=4pt,
%    overlay={\draw[colexemple,line width=2pt]
%([yshift=-1pt]frame.north west)--([yshift=-1pt]frame.north east);}},
%coltitle=colexemple,fonttitle=\bfseries,%\large\bfseries,
%before=\par\medskip\noindent,parbox=false,boxsep=0pt,left=0pt,right=3mm,top=4pt,
%overlay={\draw[colexemple,line width=1pt]
%([yshift=-1pt]title.north east)--([xshift=-0.5pt,yshift=-1pt]title.north-|frame.east)
%--([xshift=-0.5pt]frame.south east)--(frame.south west); },
%}

\newcounter{exemples}
\setcounter{exemples}{1}

\newenvironment{exemple}
  {
    \begin{exemplehid}{\theexemples}
  }
  {
    \end{exemplehid}
    \addtocounter{exemples}{1}
  }

\newtcolorbox{exemplehid}[1]{%
empty,title={Exemple {#1}},attach boxed title to top left,
boxed title style={empty,size=minimal,toprule=2pt,top=4pt,
overlay={\draw[colexemple,line width=2pt]
([yshift=-1pt]frame.north west)--([yshift=-1pt]frame.north east);}},
coltitle=colexemple,fonttitle=\bfseries,
before=\par\medskip\noindent,parbox=false,boxsep=0pt,left=0pt,right=3mm,top=4pt,
breakable,pad at break*=0mm,vfill before first,
overlay unbroken={\draw[colexemple,line width=1pt]
([yshift=-1pt]title.north east)--([xshift=-0.5pt,yshift=-1pt]title.north-|frame.east)
--([xshift=-0.5pt]frame.south east)--(frame.south west); },
overlay first={\draw[colexemple,line width=1pt]
([yshift=-1pt]title.north east)--([xshift=-0.5pt,yshift=-1pt]title.north-|frame.east)
--([xshift=-0.5pt]frame.south east); },
overlay middle={\draw[colexemple,line width=1pt] ([xshift=-0.5pt]frame.north east)
--([xshift=-0.5pt]frame.south east); },
overlay last={\draw[colexemple,line width=1pt] ([xshift=-0.5pt]frame.north east)
--([xshift=-0.5pt]frame.south east)--(frame.south west);},%
}

\newenvironment{contrex}
  {
    \begin{contrexhid}{\theexemples}
  }
  {
    \end{contrexhid}
    \addtocounter{exemples}{1}
  }

\newtcolorbox{contrexhid}[1]{%
empty,title={Contre-exemple {#1}},attach boxed title to top left,
boxed title style={empty,size=minimal,toprule=2pt,top=4pt,
overlay={\draw[colexemple,line width=2pt]
([yshift=-1pt]frame.north west)--([yshift=-1pt]frame.north east);}},
coltitle=colexemple,fonttitle=\bfseries,
before=\par\medskip\noindent,parbox=false,boxsep=0pt,left=0pt,right=3mm,top=4pt,
breakable,pad at break*=0mm,vfill before first,
overlay unbroken={\draw[colexemple,line width=1pt]
([yshift=-1pt]title.north east)--([xshift=-0.5pt,yshift=-1pt]title.north-|frame.east)
--([xshift=-0.5pt]frame.south east)--(frame.south west); },
overlay first={\draw[colexemple,line width=1pt]
([yshift=-1pt]title.north east)--([xshift=-0.5pt,yshift=-1pt]title.north-|frame.east)
--([xshift=-0.5pt]frame.south east); },
overlay middle={\draw[colexemple,line width=1pt] ([xshift=-0.5pt]frame.north east)
--([xshift=-0.5pt]frame.south east); },
overlay last={\draw[colexemple,line width=1pt] ([xshift=-0.5pt]frame.north east)
--([xshift=-0.5pt]frame.south east)--(frame.south west);},%
}

\newenvironment{app}
  {
    \begin{apphid}{\theexemples}
  }
  {
    \end{apphid}
    \addtocounter{exemples}{1}
  }

\newtcolorbox{apphid}[1]{%
empty,title={Application {#1}},attach boxed title to top left,
boxed title style={empty,size=minimal,toprule=2pt,top=4pt,
overlay={\draw[colexemple,line width=2pt]
([yshift=-1pt]frame.north west)--([yshift=-1pt]frame.north east);}},
coltitle=colexemple,fonttitle=\bfseries,
before=\par\medskip\noindent,parbox=false,boxsep=0pt,left=0pt,right=3mm,top=4pt,
breakable,pad at break*=0mm,vfill before first,
overlay unbroken={\draw[colexemple,line width=1pt]
([yshift=-1pt]title.north east)--([xshift=-0.5pt,yshift=-1pt]title.north-|frame.east)
--([xshift=-0.5pt]frame.south east)--(frame.south west); },
overlay first={\draw[colexemple,line width=1pt]
([yshift=-1pt]title.north east)--([xshift=-0.5pt,yshift=-1pt]title.north-|frame.east)
--([xshift=-0.5pt]frame.south east); },
overlay middle={\draw[colexemple,line width=1pt] ([xshift=-0.5pt]frame.north east)
--([xshift=-0.5pt]frame.south east); },
overlay last={\draw[colexemple,line width=1pt] ([xshift=-0.5pt]frame.north east)
--([xshift=-0.5pt]frame.south east)--(frame.south west);},%
}

%%%%%%%%%%%%%%%%%%%%%%%%%%%%%%%%%%%%%%%%%%%%%%%%%%%%%%%%%%%%%%%%%%%%%%%%%%%%%%%%
%
% ENUMERATE
% =========
%
%%%%%%%%%%%%%%%%%%%%%%%%%%%%%%%%%%%%%%%%%%%%%%%%%%%%%%%%%%%%%%%%%%%%%%%%%%%%%%%%

\usepackage{enumerate}
\usepackage{enumitem}

% To have special enumerate items like
%
% 1/
% 2/
% 3/

%%%%%%%%%%%%%%%%%%%%%%%%%%%%%%%%%%%%%%%%%%%%%%%%%%%%%%%%%%%%%%%%%%%%%%%%%%%%%%%%
%
% ARRAYS
% ======
%
%%%%%%%%%%%%%%%%%%%%%%%%%%%%%%%%%%%%%%%%%%%%%%%%%%%%%%%%%%%%%%%%%%%%%%%%%%%%%%%%


\usepackage{array}
\usepackage{makecell} % Used to break lines within arrays
\usepackage{multirow}
\usepackage{booktabs} % Used to have nice arrays with headrules

%%%%%%%%%%%%%%%%%%%%%%%%%%%%%%%%%%%%%%%%%%%%%%%%%%%%%%%%%%%%%%%%%%%%%%%%%%%%%%%%
%
% WRITE CODE
% ==========
%
%%%%%%%%%%%%%%%%%%%%%%%%%%%%%%%%%%%%%%%%%%%%%%%%%%%%%%%%%%%%%%%%%%%%%%%%%%%%%%%%

\usepackage{listings}
\usepackage{xcolor}

%New colors defined below
\definecolor{codegreen}{rgb}{0,0.6,0}
\definecolor{codegray}{rgb}{0.5,0.5,0.5}
\definecolor{codepurple}{rgb}{0.58,0,0.82}
\definecolor{backcolour}{rgb}{0.95,0.95,0.92}

%Code listing style named "mystyle"
\lstdefinestyle{python}{
  %backgroundcolor=\color{backcolour},
  commentstyle=\color{codegreen},
  keywordstyle=\color{magenta},
  numberstyle=\tiny\color{codegray},
  stringstyle=\color{codepurple},
  basicstyle=\ttfamily\footnotesize,
  breakatwhitespace=false,
  breaklines=true,
  captionpos=b,
  keepspaces=true,
  numbers=left,
  numbersep=5pt,
  showspaces=false,
  showstringspaces=false,
  showtabs=false,
  tabsize=2
}

\lstset{style=python}

%%%%%%%%%%%%%%%%%%%%%%%%%%%%%%%%%%%%%%%%%%%%%%%%%%%%%%%%%%%%%%%%%%%%%%%%%%%%%%%%
%
% Tabular 
% =======
%
%%%%%%%%%%%%%%%%%%%%%%%%%%%%%%%%%%%%%%%%%%%%%%%%%%%%%%%%%%%%%%%%%%%%%%%%%%%%%%%%

% In order to obtain a tabular with given width.

\usepackage{tabularx}
\newcolumntype{Y}{>{\centering\arraybackslash}X}
\newcolumntype{R}{>{\raggedright\arraybackslash}X}
\newcolumntype{L}{>{\raggedleft\arraybackslash}X}
% \usepackage{tabulary} % younger brother

%%%%%%%%%%%%%%%%%%%%%%%%%%%%%%%%%%%%%%%%%%%%%%%%%%%%%%%%%%%%%%%%%%%%%%%%%%%%%%%%
%
% MACROS
% ======
%
%%%%%%%%%%%%%%%%%%%%%%%%%%%%%%%%%%%%%%%%%%%%%%%%%%%%%%%%%%%%%%%%%%%%%%%%%%%%%%%%

% Math Operators

\DeclareMathOperator{\Card}{Card}
\DeclareMathOperator{\Gal}{Gal}
\DeclareMathOperator{\Id}{Id}
\DeclareMathOperator{\Img}{Im}
\DeclareMathOperator{\Ker}{Ker}
\DeclareMathOperator{\Minpoly}{Minpoly}
\DeclareMathOperator{\Mod}{mod}
\DeclareMathOperator{\Ord}{Ord}
\DeclareMathOperator{\ppcm}{ppcm}
\DeclareMathOperator{\pgcd}{pgcd}
\DeclareMathOperator{\tr}{Tr}
\DeclareMathOperator{\Vect}{Vect}
\DeclareMathOperator{\Span}{Span}
\DeclareMathOperator{\rank}{rank}
\DeclareMathOperator{\rg}{rg}
\DeclareMathOperator{\ev}{ev}
\DeclareMathOperator{\Var}{Var}

% Shortcuts

\newcommand{\eg}{\emph{e.g. }}
\newcommand{\ent}[2]{[\![#1,#2]\!]}
\newcommand{\ie}{\emph{i.e. }}
\newcommand{\ps}[2]{\left\langle#1,#2\right\rangle}
\newcommand{\eqdef}{\overset{\text{def}}{=}}
\newcommand{\E}{\mathcal{E}}
\newcommand{\M}{\mathcal{M}}
\newcommand{\A}{\mathcal{A}}
\newcommand{\B}{\mathcal{B}}
\newcommand{\R}{\mathcal{R}}
\newcommand{\D}{\mathcal{D}}
\newcommand{\Pcal}{\mathcal{P}}
\newcommand{\K}{\mathbf{k}}
\newcommand{\vect}[1]{\overrightarrow{#1}}



\title{Chapitre 1 : second degré}
\date{}
\author{}

\begin{document}
\maketitle\thispagestyle{fancy}

\section{Fonctions polynômes du second degré}
\subsection{Forme développée et forme canonique}
\begin{defi}{Fonction polynôme du second degré}
  Une fonction $f$ définie sur $\mathbb{R}$ est appelée \textbf{fonction polynôme
  du second degré} s'il existe $a, b, c\in\mathbb{R}$ des réels avec $a\neq0$ et
  tels que, pour tout réel $x\in\mathbb{R}$, on ait
  \[
    f(x) = ax^2+bx+c.
  \]
\end{defi}

\begin{defi}{Forme développée et coefficients}
  Soit $f$ une fonction polynôme du second degré définie sur $\mathbb{R}$ par
  \[
    f(x) = ax^2+bx+c,
  \]
  où $a,b,c\in\mathbb{R}$ sont des réels et où $a\neq0$. Lorsque la fonction $f$
  est écrite sous cette forme, on parle de \textbf{forme développée}. Les réels
  $a$, $b$, et $c$ sont appelés les \textbf{coefficients} de $f$.
\end{defi}

\begin{exemple}
  Soit $f$ la fonction définie sur $\mathbb{R}$ par
  \[
    f(x) = x^2+2x+1.
  \]
  La fonction $f$ est une fonction polynôme du second degré, donnée sous forme
  développée. Ses coefficients $a$, $b$, et $c$ valent
  \begin{align*}
    a &= &
    b &= &
    c &=
  \end{align*}
  Le graphe de la fonction $f$ est donné ci-dessous.
\begin{center}
\begin{tikzpicture}
\begin{axis}
\addplot[red, very thick, samples=201]{x^2+2*x+1};
\end{axis}
%\draw[color=red] (2,4) node {\LARGE{$y=x^2+2x+1$}};
\end{tikzpicture}
\end{center}
\end{exemple}

\begin{exemple}
  Les fonctions suivantes sont-elles des fonctions polynômes du second degré ?
  Si oui, donner leurs coefficients $a, b, c$ dans l'expression $ax^2+bx+c$.
  \begin{enumerate}
    \item La fonction $f_1$ définie sur $\mathbb{R}$ par $f_1(x) = -x^2-x+10$.
    \item La fonction $f_2$ définie sur $\mathbb{R_+}$ par $f_2(x) = x^2+\sqrt x-1$.
    \item La fonction $f_3$ définie sur $\mathbb{R}$ par $f_3(x) = x^2$.
    \item La fonction $f_4$ définie sur $\mathbb{R}$ par $f_4(x) = x+2$.
    \item La fonction $f_5$ définie sur $\mathbb{R}$ par $f_5(x) = x(x+1)$.
    \item La fonction $f_6$ définie sur $\mathbb{R}$ par $f_6(x) = 0$.
  \end{enumerate}
\end{exemple}

\begin{prop}
  Soit $f$ une fonction polynôme du second degré définie sur $\mathbb{R}$ par
  \[
    f(x) = ax^2+bx+c.
  \]
  Alors $f$ peut s'écrire sous la forme
  \[
    f(x) = a(x-\alpha)^2+\beta,
  \]
  et on a $\alpha=-\cfrac{b}{2a}$ et $\beta=f(\alpha)$. Cette forme s'appelle la
  \textbf{forme canonique} de $f$.
\end{prop}

\begin{proof}
  Soit $f$ la fonction définie sur $\mathbb{R}$ par
  \[
    f(x) = ax^2+bx+c,
  \]
  avec $a,b,c\in\mathbb{R}$ trois réels et $a\neq0$. Comme on a $a\neq0$, on
  peut écrire, pour tout $x\in\mathbb{R}$, que
  \[
    f(x) = a\left( x^2+\frac{b}{a}x+\frac{c}{a} \right).
  \]
  Puis on remarque que les termes $x^2+\frac{b}{a}x$ forment le début du
  développement de
  \[
    \left(x+\frac{b}{2a}\right)^2 = \phantom{x^2 +
      \frac{b}{a}x+\left(\frac{b}{2a}\right)^2.}
  \]
  On peut donc écrire que
  \[
    \left( x^2+\frac{b}{a}x \right) = \phantom{\left( x+\frac{b}{2a} \right)^2-\left(
    \frac{b}{2a} \right)^2}
  \]
  puis que
  \[
    f(x) = a\left(\phantom{ \left( x+\frac{b}{2a} \right)^2-\left( \frac{b}{2a}
    \right)^2+\frac{c}{a}} \right).
  \]
  En mettant sur le même dénominateur, on remarque que
  \begin{align*}
    -\left( \frac{b}{2a} \right)^2+\frac{c}{a} &= \phantom{-\frac{b^2}{4a^2} +
    \frac{c}{a}} \\
    &= \phantom{\frac{-b^2+4ac}{4a^2}}.
 \end{align*}
 On retrouve alors
 \begin{align*}
   f(x) &= a\left( x+\frac{b}{2a} \right)^2+a\times\frac{-b^2+4ac}{4a^2} \\
   &= a\left( x+\frac{b}{2a} \right)^2+\frac{-b^2+4ac}{4a} \\
   &= a\left( x-\alpha \right)^2+\beta,
 \end{align*}
 où $\alpha=\cfrac{-b}{2a}$ et où $\beta=\cfrac{-b^2+4ac}{4a}$. De plus, on a
 aussi 
 \[
   f(\alpha) = a\left( \alpha-\alpha \right)^2+\beta= a\times 0+\beta=\beta.
 \]
\end{proof}

\begin{exemple}
  Soit $f$ la fonction définie sur $\mathbb{R}$ par
  \[
    f(x) = x^2 + 2x + 1.
  \]
  La forme canonique de $f$ est 
  \[
    f(x) = (x+1)^2.
  \]
  En effet, on a dans ce cas
  \begin{align*}
    a &= &
    \alpha &= &
    \beta &=
  \end{align*}
\end{exemple}

\subsection{Variations et représentation graphique}
\begin{prop}
  Soit $f$ la fonction définie sur $\mathbb{R}$ par $f(x)=ax^2+bx+c$, de forme
  canonique $f(x)=a(x-\alpha)^2+\beta$.

  \noindent
  \begin{minipage}[t]{.47\textwidth}
    \begin{center}
      {\bf Si} $\mathbf{a>0}$\vspace{.2cm}

      \begin{tikzpicture}
        \tkzTabInit[lgt=1, espcl=1.5]{$x$ / .7, $f(x)$ / 1.4}{$-\infty$, $\alpha$, $+\infty$}
        \tkzTabVar{+/, -/$\beta$, +/}
      \end{tikzpicture}
    \end{center}
  La fonction $f$ est strictement décroissante sur $]-\infty, \alpha]$,
  strictement croissante sur $[\alpha, +\infty[$, et $f$ admet comme minimum
    $\beta$ en $\alpha$.
    \begin{center}
\begin{tikzpicture}
  \begin{axis}[x=.25cm, y=.25cm, xtick=\empty, ytick=\empty]
    \addplot[red, very thick, samples=201]{x^2-x-3};
  \end{axis}
\end{tikzpicture}
    \end{center}
  \end{minipage}
    \hfill
  \begin{minipage}[t]{.47\textwidth}
    \begin{center}
      {\bf Si} $\mathbf{a<0}$\vspace{.2cm}

      \begin{tikzpicture}
        \tkzTabInit[lgt=1, espcl=1.5]{$x$ / .7, $f(x)$ / 1.4}{$-\infty$, $\alpha$, $+\infty$}
        \tkzTabVar{-/, +/$\beta$, -/}
      \end{tikzpicture}
    \end{center}
  La fonction $f$ est strictement croissante sur $]-\infty, \alpha]$,
  strictement décroissante sur $[\alpha, +\infty[$, et $f$ admet comme maximum
    $\beta$ en $\alpha$.
    \begin{center}
\begin{tikzpicture}
  \begin{axis}[x=.25cm, y=.25cm, xtick=\empty, ytick=\empty]
    \addplot[red, very thick, samples=201]{-x^2-2*x+2};
  \end{axis}
\end{tikzpicture}
    \end{center}
  \end{minipage}
\end{prop}

\begin{defi}{Parabole}
  La courbe représentative d'une fonction polynôme du second degré est appelée
  une \textbf{parabole}.
\end{defi}

\begin{exemple}
  Reprennons l'exemple de la fonction polynôme du second degré définie sur
  $\mathbb{R}$ par
  \[
    f(x) = x^2+2x+1.
  \]
  On a
  \begin{align*}
    a &= 1 &
    b &= 2 &
    c &= 1.
  \end{align*}
  On peut donc calculer
  \[
    \alpha =
  \]
  puis on retrouve $\beta$:
  \[
    \beta =
  \]
  On retombe bien sur la forme canonique
  \[
    f(x) = a(x-\alpha)^2+\beta =
  \]
\begin{itemize}
\item La fonction est strictement décroissante sur
\item La fonction est strictement croissante sur
\item Elle admet comme minimum $\ldots$ en $\ldots$.
\end{itemize}
 \begin{center}
\begin{tikzpicture}
\begin{axis}
\addplot[red, very thick, samples=201]{x^2+2*x+1};
\end{axis}
\end{tikzpicture}
\end{center}
\end{exemple}

\begin{prop}
  Soit $f$ une fonction polynôme du second degré définie sur $\mathbb{R}$ par sa
  forme canonique
  \[
    f(x) = a(x-\alpha)+\beta.
  \]
  Alors $f$ est représentée par une parabole dont le sommet a pour coordonnées
  $(\alpha, \beta)$.
\end{prop}
\begin{exemple}
  La fonction définie par $f(x)=(x+1)^2$ de l'exemple précédent admet une
  parabole dont le sommet est le point $(-1, 0)$.
\end{exemple}

\section{Équations du second degré}

\subsection{Définitions} 
\begin{defi}{Équation du second degré}
  Une \textbf{équation du second degré} à coefficients réels est une équation de
  la forme
  \[
    ax^2+bx+c = 0,
  \]
  avec $a,b,c\in\mathbb{R}$ trois réels et $a\neq0$.
\end{defi}

\begin{defi}{Racines d'une équation}
  Les solutions de l'équation du second degré $ax^2+bx+c=0$ sont appelées les
  \textbf{racines} du trinôme $ax^2+bx+c$. De la même manière, on parle de racine
  pour le polynôme défini par $f(x)=ax^2+bx+c$
\end{defi}

\begin{exemple}
  L'équation $2x^2-x+3=0$ est une équation du second degré avec 
  \begin{align*}
  a &= &
  b &= &
  c &=
  \end{align*}
\end{exemple}

\begin{exemple}
  L'équation $x^2-2=0$ est une équation du second degré avec $a=1, b=0, c=-2$.
  Cette équation admet deux racines :
\end{exemple}

\begin{rmq}
Les racines de l'équation du second degré
  \(
    ax^2+bx+c = 0
  \)
  correspondent aux abcisses des points où la courbe représentative de la
  fonction $f$ définie par $f(x)=ax^2+bx+c$ passe par l'axe des abcisses.
  Ci-dessous l'exemple de $x^2-1=0$.
 \begin{center}
  \begin{tikzpicture}
    \begin{axis}
      \addplot[red, very thick, samples=201]{x^2-1};
    \end{axis}
    \filldraw[blue] (2.25,2.75) circle (1.5pt);
    \filldraw[blue] (3.25,2.75) circle (1.5pt);
  \end{tikzpicture}
\end{center}
\end{rmq}

\begin{defi}{Racine d'une fonction polynôme}
  Soit $f$ la fonction polynôme définie sur $\mathbb{R}$ par
  \[
    f(x) = ax^2+bx+c.
  \]
  On dit que la valeur $x_0$ est une \textbf{racine} de $f$ si
  \[
    f(x_0) = 0.
  \]
\end{defi}

\subsection{Résolution des équations du second degré dans $\mathbb{R}$}

On va maintenant apprendre à résoudre dans $\mathbb{R}$ les équations du second
degré, c'est-à-dire à trouver des solutions réelles à nos équations.

\begin{defi}{Discriminant}
  Le \textbf{discriminant} du trinôme $ax^2+bx+c$, noté $\Delta$ (delta
  majuscule), est le nombre
  \[
    \Delta = b^2-4ac.
  \]
\end{defi}

\begin{prop}
  Soit $\Delta=b^2-4ac$ le discriminant du trinôme $ax^2+bx+c$.
  \begin{itemize}
    \item Si $\Delta<0$, alors l'équation $ax^2+bx+c=0$ n'a pas de solutions
      dans $\mathbb{R}$.
    \item Si $\Delta=0$, alors l'équation $ax^2+bx+c=0$ a une unique solution
      \[x_0 = \frac{-b}{2a}.\] On dit que $x_0$ est \textbf{racine double} du trinôme.
    \item Si $\Delta>0$, alors l'équation $ax^2+bx+c=0$ admet deux solutions
      distinctes : 
      \[
        x_1=\frac{-b-\sqrt\Delta}{2a}
        \text{ et }
        x_2=\frac{-b+\sqrt\Delta}{2a}.
      \]
  \end{itemize}
\end{prop}

\begin{proof}
  Soit $f$ la fonction polynôme du second degré définie sur $\mathbb{R}$ par
  \[
    f(x) = ax^2+bx+c.
  \]
  On a vu au cours de la démonstration concernant la forme canonique que l'on
  pouvait écrire
  \[
    f(x) = a\left(x+\frac{b}{2a}\right)^2-\frac{b^2-4ac}{4a} =
    a\left(x+\frac{b}{2a}\right)^2-\frac{\Delta}{4a} .
  \]
  En raisonnant \textbf{par équivalence}, on a ainsi
  \begin{align*}
    f(x) = 0 &\Longleftrightarrow a\left( x+\frac{b}{2a}
    \right)^2-\frac{\Delta}{4a}=0 \\
    &\Longleftrightarrow a\left( x+\frac{b}{2a}
    \right)^2=\frac{\Delta}{4a} \\
    &\Longleftrightarrow \left( x+\frac{b}{2a}
    \right)^2=\frac{\Delta}{4a^2}.
  \end{align*}
  On peut alors différencier plusieurs cas.
  \paragraph{Premier cas : $\Delta<0$.} Dans ce cas on a
  $\frac{\Delta}{4a^2}<0$. Mais on a aussi $\left( x+\frac{b}{2a} \right)^2\geq0$
  car un carré est toujours positif ou nul. L'équation n'a donc pas de solution.
  \paragraph{Deuxième cas : $\Delta=0$.} Dans ce cas l'équation devient
  \begin{align*}
    f(x) = 0 &\Longleftrightarrow \left( x+\frac{b}{2a} \right)^2 = 0 \\
    &\Longleftrightarrow \left( x+\frac{b}{2a} \right) = 0 \\
    &\Longleftrightarrow x = \frac{-b}{2a}.
  \end{align*}
  L'équation a donc une unique solution, donnée par $x_0=\cfrac{-b}{2a}$.
  \paragraph{Troisième cas : $\Delta>0$.} Cette fois-ci on a
  \begin{align*}
    f(x) = 0 &\Longleftrightarrow \left( x+\frac{b}{2a} \right)^2 =
    \frac{\Delta}{4a^2} \\
    &\Longleftrightarrow \left( x+\frac{b}{2a} \right)^2 =
    \left( \frac{\sqrt\Delta}{2a} \right)^2 \\
    &\Longleftrightarrow x+\frac{b}{2a}=\frac{\sqrt\Delta}{2a}\text{ ou
    }x+\frac{b}{2a}=-\frac{\sqrt\Delta}{2a}\\
    &\Longleftrightarrow x=-\frac{b}{2a}+\frac{\sqrt\Delta}{2a}\text{ ou
    }x=-\frac{b}{2a}+\frac{\sqrt\Delta}{2a}.
  \end{align*}
  Et on obtient finalement deux solutions distinctes, données par
  \[
    x_1 = \frac{-b-\sqrt\Delta}{2a}\text{ et }x_2 =
    \frac{-b+\sqrt\Delta}{2a}.
  \]
\end{proof}

\begin{exemple}
  Considérons l'équation du second degré
  \[
    x^2+2x-3 = 0.
  \]
  On a ici $a=1, b=2, c=-3$. On commence par calculer le discriminant de cette
  équation, on obtient
  \[
    \Delta =
  \]
  On est dans le cas où $\Delta>0$ et on sait qu'on a ainsi deux solutions
  distinctes:
  \[
    x_1 =
  \]
  et
  \[
    x_2 =
  \]
 L'ensemble des solutions de l'équation est donc $\mathscr S =$
\end{exemple}

\section{Propriétés d'un trinôme $ax^2+bx+c$}
\subsection{Factorisation}

\begin{propadm}
  Soit $f$ une fonction polynôme du second degré définie sur $\mathbb{R}$ par
  \[
    f(x) = ax^2+bx+c.
  \]
  \begin{itemize}
    \item Si $\Delta>0$, alors $f(x)=a(x-x_1)(x-x_2)$ où $x_1$ et $x_2$ sont les
      racines de $f$.
    \item Si $\Delta=0$, alors $f(x)=a(x-x_0)^2$ où $x_0$ est la racine double de
      $f$.
    \item Si $\Delta<0$, alors la fonction $f$ ne peut pas s'écrire comme un
      produit de deux polynômes de degré $1$.
  \end{itemize}
\end{propadm}

\begin{exemple}
  On a vu précédemment que la fonction $f$ définie par
  \[
    f(x) = x^2+2x-3
  \]
  avait pour discriminant $\Delta=16$ et pour racines $x_1=-3$ et $x_2=1$. Cela
  signifie que l'on peut aussi écrire $f$ sous la forme factorisée
  \[
    f(x) =\phantom{(x+3)(x-1).}
  \]
\end{exemple}

\subsection{Somme et produit de racines}
\begin{propadm}
  Soit $a,b,c\in\mathbb{R}$ et soit $f$ une fonction polynôme de degré $2$
  définie sur $\mathbb{R}$ par
  \[
    f(x)=ax^2+bx+c,
  \]
  dont le discriminant est strictement positif. La fonction $f$ a alors deux
  racines distinctes $x_1$ et $x_2$ et on a
  \[
    x_1+x_2 = \frac{-b}{a}\text{ et }x_1\times x_2=\frac{c}{a}.
  \]
\end{propadm}

\subsection{Signe d'une fonction polynôme du second degré}
\begin{propadm}
  Soit $f$ une fonction polynôme du second degré définie sur $\mathbb{R}$ par
  \[
    f(x)=ax^2+bx+c,
  \]
  de déterminant $\Delta$.
  \begin{itemize}
    \item Si $\Delta<0$, alors pour tout réel $x\in\mathbb{R}$, $f(x)$ est du
      signe de $a$.
    \item Si $\Delta=0$, alors pour tout réel $x\neq\frac{-b}{2a}$, $f(x)$ est
      du signe de $a$, et $f(\frac{-b}{2a})=0$.
    \item Si $\Delta>0$, alors on a les tableaux de signe suivants.

      \noindent
  \begin{minipage}[t]{.47\textwidth}
    \begin{center}
      {\bf Si} $\mathbf{a>0}$\vspace{.2cm}

      \begin{tikzpicture}
        \tkzTabInit[lgt=1, espcl=1.5]{$x$ / .7, $f(x)$ / 1.4}{$-\infty$, $x_1$,
        $x_2$, $+\infty$}
        \tkzTabLine{, +, z , -, z, +,}
      \end{tikzpicture}
    \end{center}
  \vspace{.2cm}
  \end{minipage}
    \hfill
  \begin{minipage}[t]{.47\textwidth}
    \begin{center}
      {\bf Si} $\mathbf{a<0}$\vspace{.2cm}

      \begin{tikzpicture}
        \tkzTabInit[lgt=1, espcl=1.5]{$x$ / .7, $f(x)$ / 1.4}{$-\infty$, $x_1$,
        $x_2$, $+\infty$}
        \tkzTabLine{, -, z , +, z, -,}
      \end{tikzpicture}
    \end{center}
  \vspace{.2cm}
  \end{minipage}
  On peut retenir que dans ce cas, $f$ est du signe de $a$, sauf entre ses
  racines.
  \end{itemize}
\end{propadm}
\begin{exemple}
  On peut reprendre la fonction $f$ définie sur $\mathbb{R}$ par
  \[
    f(x) = x^2+2x-3.
  \]
  On a vu que le déterminant de $f$ vaut $16$ et que les deux racines de $f$
  sont données par $x_1=-3$ et $x_2=1$. On a ainsi le tableau de signe suivant.
  \begin{center}
      \begin{tikzpicture}
        \tkzTabInit[lgt=1, espcl=1.5]{$x$ / .7, $f(x)$ / 1.4}{$-\infty$, $-3$,
        $1$, $+\infty$}
        \tkzTabLine{, +, z , -, z, +,}
      \end{tikzpicture}
  \end{center}
  Et, en effet, cela concorde avec la représentation graphique de $f$ donnée
  ci-dessous.
  \begin{center}
\begin{tikzpicture}
\begin{axis}
  \addplot[red, very thick, samples=201, domain=-5:-3]{x^2+2*x-3};
  \addplot[blue, very thick, samples=201, domain=-3:1]{x^2+2*x-3};
  \addplot[red, very thick, samples=201, domain=1:5]{x^2+2*x-3};
\end{axis}
\filldraw[black] (1.25, 2.75) circle (2pt);
\filldraw[black] (3.25, 2.75) circle (2pt);
\end{tikzpicture}
  \end{center}
\end{exemple}
\end{document}
