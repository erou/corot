\documentclass[11pt]{article}

\newcommand{\titrechapitre}{Probabilités conditionnelles -- Cours}
\newcommand{\titreclasse}{Lycée Jean-Baptiste \textsc{Corot}}
\newcommand{\pagination}{\thepage/\pageref{LastPage}}
\newcommand{\topbotmargins}{2cm}
\newcommand{\spacebelowexo}{5mm}
%%%%%%%%%%%%%%%%%%%%%%%%%%%%%%%%%%%%%%%%%%%%%%%%%%%%%%%%%%%%%%%%%%%%%%%%%%%%%%%%
%
% PACKAGES
% ========
%
%%%%%%%%%%%%%%%%%%%%%%%%%%%%%%%%%%%%%%%%%%%%%%%%%%%%%%%%%%%%%%%%%%%%%%%%%%%%%%%%

\usepackage[english, french]{babel}
\usepackage[utf8]{inputenc}
\usepackage[T1]{fontenc}
\usepackage{graphicx}
\usepackage{amsmath,amssymb,amsthm,amsopn}
\usepackage{hyperref}

% Pour avoir l'écriture \mathscr (math script)
% ============================================

\usepackage{mathrsfs}

% Deal with coma as a decimal separator
% =====================================

\usepackage{icomma}

% Package Geometry
% ================

\usepackage[a4paper, lmargin=2cm, rmargin=2cm, top=\topbotmargins, bottom=\topbotmargins]{geometry}

% Package multicol
% ================

\usepackage{multicol}

% Redefine abstract
% =================

% Note
% ====
%
% Le reste a été commenté pour ne pas charger trop de choses au démarrage. On
% verra si on en a besoin plus tard.
%
% --------
%
%\usepackage{mathrsfs}
%\usepackage{multirow}
%\usepackage{bm}
%\hypersetup{
%    colorlinks=true,
%    linkcolor=blue,
%    citecolor=red,
%}
%\usepackage{diagbox}
%
%\usepackage{algorithm}
%\usepackage{algpseudocode}
%
%\renewcommand{\algorithmicrequire}{\textbf{Input:}}
%\renewcommand{\algorithmicensure}{\textbf{Output:}}


%%%%%%%%%%%%%%%%%%%%%%%%%%%%%%%%%%%%%%%%%%%%%%%%%%%%%%%%%%%%%%%%%%%%%%%%%%%%%%%%
%
% TIKZ
% ====
%
%%%%%%%%%%%%%%%%%%%%%%%%%%%%%%%%%%%%%%%%%%%%%%%%%%%%%%%%%%%%%%%%%%%%%%%%%%%%%%%%

\usepackage{tikz}
\usetikzlibrary{arrows}

\usepackage{tkz-tab} % Variation tables

\usepackage{pgfplots}
%\usepackage{pgf-pie} % Pie charts

\pgfplotsset{
%\newcommand{\settingsgraph}{
x=.5cm,y=.5cm,
xticklabel style = {font=\scriptsize, yshift=.1cm},
yticklabel style = {font=\scriptsize, xshift=.1cm},
axis lines=middle,
ymajorgrids=true,
xmajorgrids=true,
major grid style = {color=white!80!blue},
xmin=-5.5,
xmax=5.5,
ymin=-5.5,
ymax=5.5,
xtick={-5.0,-4.0,...,5.0},
ytick={-5.0,-4.0,...,5.0},
}

% Tikz style

\tikzset{round/.style={circle, draw=black, very thick, scale = 0.7}}
\tikzset{arrow/.style={->, >=latex}}
\tikzset{dashed-arrow/.style={->, >=latex, dashed}}

\newcommand{\point}[3]{\draw[very thick, #3] (#1-.1, #2)--(#1+.1, #2)
(#1, #2-.1)--(#1, #2+.1)}

%%%%%%%%%%%%%%%%%%%%%%%%%%%%%%%%%%%%%%%%%%%%%%%%%%%%%%%%%%%%%%%%%%%%%%%%%%%%%%%%
%
% FANCY HEADER
% ============
%
%%%%%%%%%%%%%%%%%%%%%%%%%%%%%%%%%%%%%%%%%%%%%%%%%%%%%%%%%%%%%%%%%%%%%%%%%%%%%%%%


\usepackage{fancyhdr}
\usepackage{lastpage}

\pagestyle{fancy}
\newcommand{\changefont}{\fontsize{9}{9}\selectfont}
\renewcommand{\headrulewidth}{0mm}
\renewcommand{\footrulewidth}{0mm}

\fancyhead[C]{}
\fancyhead[L]{\titreclasse}
\fancyhead[R]{\titrechapitre}
\fancyfoot[C]{}
\fancyfoot[L]{}
\fancyfoot[R]{\pagination}
\addtolength{\skip\footins}{20pt} % distance between text and footnotes

%%%%%%%%%%%%%%%%%%%%%%%%%%%%%%%%%%%%%%%%%%%%%%%%%%%%%%%%%%%%%%%%%%%%%%%%%%%%%%%%
%
% THEOREM STYLE
% =============
%
%%%%%%%%%%%%%%%%%%%%%%%%%%%%%%%%%%%%%%%%%%%%%%%%%%%%%%%%%%%%%%%%%%%%%%%%%%%%%%%%

\usepackage[tikz]{bclogo}
\usepackage{mdframed}

\usepackage{tcolorbox}
\tcbuselibrary{listings, breakable, theorems, skins}

%\newtheoremstyle{break}%
%{}{}%
%{\itshape}{}%
%{\bfseries}{}%  % Note that final punctuation is omitted.
%{\newline}{}

\newtheoremstyle{scbf}%
{}{}%
{}{}%
%{\scshape}{}%  % Note that final punctuation is omitted.
{\bfseries\scshape}{}%  % Note that final punctuation is omitted.
{\newline}{}

%\theoremstyle{break}
%\theoremstyle{plain}
%\newtheorem{thm}{Theorem}[section]
%\newtheorem{lm}[thm]{Lemma}
%\newtheorem{prop}[thm]{Proposition}
%\newtheorem{cor}[thm]{Corollary}

%\theoremstyle{scbf}
%\newtheorem{exo}{$\star$ Exercice}

%\theoremstyle{definition}
%\newtheorem{defi}[thm]{Definition}
%\newtheorem{ex}[thm]{Example}

%\theoremstyle{remark}
%\newtheorem{rem}[thm]{Remark}

% Defining the Remark environment
% ===============================

\newenvironment{rmq}
  {
    \begin{bclogo}[logo=\bcinfo, noborder=true]{Remarque}
  }
  {
    \end{bclogo}
  }

% Defining the exercise environment
% =================================

\newcounter{exos}
\setcounter{exos}{1}

\newenvironment{exo}
  {
    \begin{bclogo}[logo=\bccrayon, noborder=true]{Exercice \theexos}
  }
  {
    \end{bclogo}
    \addtocounter{exos}{1}
  }


% Redefining the proof environment from amsthm
% ============================================

\tcolorboxenvironment{proof}{
  blanker, breakable, before skip=10pt,after skip=10pt,
  borderline west={1mm}{0pt}{red},
  left=5mm,
}

% Defining the definition environment
% ===================================

\colorlet{coldef}{black!50!green}

\newcounter{defis}
\setcounter{defis}{1}

\newenvironment{defi}[1]
  {
    \begin{defihid}{{#1}}{\thedefis}
  }
  {
    \end{defihid}
    \addtocounter{defis}{1}
  }

\newtcolorbox{defihid}[2]{%
  empty,title={ {\bfseries Définition {#2}} ({#1})},attach boxed title to top left,
boxed title style={empty,size=minimal,toprule=2pt,top=4pt,
overlay={\draw[coldef,line width=2pt]
([yshift=-1pt]frame.north west)--([yshift=-1pt]frame.north east);}},
coltitle=coldef,
before=\par\medskip\noindent,parbox=false,boxsep=0pt,left=0pt,right=3mm,top=4pt,
breakable,pad at break*=0mm,vfill before first,
overlay unbroken={\draw[coldef,line width=1pt]
([yshift=-1pt]title.north east)--([xshift=-0.5pt,yshift=-1pt]title.north-|frame.east)
--([xshift=-0.5pt]frame.south east)--(frame.south west); },
overlay first={\draw[coldef,line width=1pt]
([yshift=-1pt]title.north east)--([xshift=-0.5pt,yshift=-1pt]title.north-|frame.east)
--([xshift=-0.5pt]frame.south east); },
overlay middle={\draw[coldef,line width=1pt] ([xshift=-0.5pt]frame.north east)
--([xshift=-0.5pt]frame.south east); },
overlay last={\draw[coldef,line width=1pt] ([xshift=-0.5pt]frame.north east)
--([xshift=-0.5pt]frame.south east)--(frame.south west);},%
}

\newenvironment{notation}
  {
    \begin{notationhid}{\thedefis}
  }
  {
    \end{notationhid}
    \addtocounter{defis}{1}
  }

\newtcolorbox{notationhid}[1]{%
  empty,title={Notation {#1}},attach boxed title to top left,
boxed title style={empty,size=minimal,toprule=2pt,top=4pt,
overlay={\draw[coldef,line width=2pt]
([yshift=-1pt]frame.north west)--([yshift=-1pt]frame.north east);}},
coltitle=coldef,fonttitle=\bfseries,
before=\par\medskip\noindent,parbox=false,boxsep=0pt,left=0pt,right=3mm,top=4pt,
breakable,pad at break*=0mm,vfill before first,
overlay unbroken={\draw[coldef,line width=1pt]
([yshift=-1pt]title.north east)--([xshift=-0.5pt,yshift=-1pt]title.north-|frame.east)
--([xshift=-0.5pt]frame.south east)--(frame.south west); },
overlay first={\draw[coldef,line width=1pt]
([yshift=-1pt]title.north east)--([xshift=-0.5pt,yshift=-1pt]title.north-|frame.east)
--([xshift=-0.5pt]frame.south east); },
overlay middle={\draw[coldef,line width=1pt] ([xshift=-0.5pt]frame.north east)
--([xshift=-0.5pt]frame.south east); },
overlay last={\draw[coldef,line width=1pt] ([xshift=-0.5pt]frame.north east)
--([xshift=-0.5pt]frame.south east)--(frame.south west);},%
}


% Defining the proposition, theorem, etc. environment
% ===================================================

\colorlet{colprop}{red!75!black}

\newcounter{props}
\setcounter{props}{1}

\newenvironment{prop}
  {
    \begin{prophid}{\theprops}
  }
  {
    \end{prophid}
    \refstepcounter{props}
  }

\newtcolorbox{prophid}[1]{%
empty,title={Propriété {#1}},attach boxed title to top left,
boxed title style={empty,size=minimal,toprule=2pt,top=4pt,
overlay={\draw[colprop,line width=2pt]
([yshift=-1pt]frame.north west)--([yshift=-1pt]frame.north east);}},
coltitle=colprop,fonttitle=\bfseries,
before=\par\medskip\noindent,parbox=false,boxsep=0pt,left=0pt,right=3mm,top=4pt,
breakable,pad at break*=0mm,vfill before first,
overlay unbroken={\draw[colprop,line width=1pt]
([yshift=-1pt]title.north east)--([xshift=-0.5pt,yshift=-1pt]title.north-|frame.east)
--([xshift=-0.5pt]frame.south east)--(frame.south west); },
overlay first={\draw[colprop,line width=1pt]
([yshift=-1pt]title.north east)--([xshift=-0.5pt,yshift=-1pt]title.north-|frame.east)
--([xshift=-0.5pt]frame.south east); },
overlay middle={\draw[colprop,line width=1pt] ([xshift=-0.5pt]frame.north east)
--([xshift=-0.5pt]frame.south east); },
overlay last={\draw[colprop,line width=1pt] ([xshift=-0.5pt]frame.north east)
--([xshift=-0.5pt]frame.south east)--(frame.south west);},%
}

\newenvironment{propadm}
  {
    \begin{propadmhid}{\theprops}
  }
  {
    \end{propadmhid}
    \refstepcounter{props}
  }

  \newtcolorbox{propadmhid}[1]{%
    empty,title={{\bfseries Propriété {#1}} (admise)},attach boxed title to top left,
boxed title style={empty,size=minimal,toprule=2pt,top=4pt,
overlay={\draw[colprop,line width=2pt]
([yshift=-1pt]frame.north west)--([yshift=-1pt]frame.north east);}},
coltitle=colprop,%fonttitle=\bfseries,
before=\par\medskip\noindent,parbox=false,boxsep=0pt,left=0pt,right=3mm,top=4pt,
breakable,pad at break*=0mm,vfill before first,
overlay unbroken={\draw[colprop,line width=1pt]
([yshift=-1pt]title.north east)--([xshift=-0.5pt,yshift=-1pt]title.north-|frame.east)
--([xshift=-0.5pt]frame.south east)--(frame.south west); },
overlay first={\draw[colprop,line width=1pt]
([yshift=-1pt]title.north east)--([xshift=-0.5pt,yshift=-1pt]title.north-|frame.east)
--([xshift=-0.5pt]frame.south east); },
overlay middle={\draw[colprop,line width=1pt] ([xshift=-0.5pt]frame.north east)
--([xshift=-0.5pt]frame.south east); },
overlay last={\draw[colprop,line width=1pt] ([xshift=-0.5pt]frame.north east)
--([xshift=-0.5pt]frame.south east)--(frame.south west);},%
}

\newenvironment{propnom}[1]
  {
    \begin{propnomhid}{#1}{\theprops}
  }
  {
    \end{propnomhid}
    \refstepcounter{props}
  }

\newtcolorbox{propnomhid}[2]{%
empty,title={{\bfseries Propriété {#2}} ({#1})},attach boxed title to top left,
boxed title style={empty,size=minimal,toprule=2pt,top=4pt,
overlay={\draw[colprop,line width=2pt]
([yshift=-1pt]frame.north west)--([yshift=-1pt]frame.north east);}},
coltitle=colprop,
before=\par\medskip\noindent,parbox=false,boxsep=0pt,left=0pt,right=3mm,top=4pt,
breakable,pad at break*=0mm,vfill before first,
overlay unbroken={\draw[colprop,line width=1pt]
([yshift=-1pt]title.north east)--([xshift=-0.5pt,yshift=-1pt]title.north-|frame.east)
--([xshift=-0.5pt]frame.south east)--(frame.south west); },
overlay first={\draw[colprop,line width=1pt]
([yshift=-1pt]title.north east)--([xshift=-0.5pt,yshift=-1pt]title.north-|frame.east)
--([xshift=-0.5pt]frame.south east); },
overlay middle={\draw[colprop,line width=1pt] ([xshift=-0.5pt]frame.north east)
--([xshift=-0.5pt]frame.south east); },
overlay last={\draw[colprop,line width=1pt] ([xshift=-0.5pt]frame.north east)
--([xshift=-0.5pt]frame.south east)--(frame.south west);},%
}




\newenvironment{thm}
  {
    \begin{thmhid}{\theprops}
  }
  {
    \end{thmhid}
    \refstepcounter{props}
  }

\newtcolorbox{thmhid}[1]{%
empty,title={Théorème {#1}},attach boxed title to top left,
boxed title style={empty,size=minimal,toprule=2pt,top=4pt,
overlay={\draw[colprop,line width=2pt]
([yshift=-1pt]frame.north west)--([yshift=-1pt]frame.north east);}},
coltitle=colprop,fonttitle=\bfseries,
before=\par\medskip\noindent,parbox=false,boxsep=0pt,left=0pt,right=3mm,top=4pt,
breakable,pad at break*=0mm,vfill before first,
overlay unbroken={\draw[colprop,line width=1pt]
([yshift=-1pt]title.north east)--([xshift=-0.5pt,yshift=-1pt]title.north-|frame.east)
--([xshift=-0.5pt]frame.south east)--(frame.south west); },
overlay first={\draw[colprop,line width=1pt]
([yshift=-1pt]title.north east)--([xshift=-0.5pt,yshift=-1pt]title.north-|frame.east)
--([xshift=-0.5pt]frame.south east); },
overlay middle={\draw[colprop,line width=1pt] ([xshift=-0.5pt]frame.north east)
--([xshift=-0.5pt]frame.south east); },
overlay last={\draw[colprop,line width=1pt] ([xshift=-0.5pt]frame.north east)
--([xshift=-0.5pt]frame.south east)--(frame.south west);},%
}

\newenvironment{thmadm}
  {
    \begin{thmadmhid}{\theprops}
  }
  {
    \end{thmadmhid}
    \refstepcounter{props}
  }

  \newtcolorbox{thmadmhid}[1]{%
    empty,title={{\bfseries Théorème {#1}} (admis)},attach boxed title to top left,
boxed title style={empty,size=minimal,toprule=2pt,top=4pt,
overlay={\draw[colprop,line width=2pt]
([yshift=-1pt]frame.north west)--([yshift=-1pt]frame.north east);}},
coltitle=colprop,%fonttitle=\bfseries,
before=\par\medskip\noindent,parbox=false,boxsep=0pt,left=0pt,right=3mm,top=4pt,
breakable,pad at break*=0mm,vfill before first,
overlay unbroken={\draw[colprop,line width=1pt]
([yshift=-1pt]title.north east)--([xshift=-0.5pt,yshift=-1pt]title.north-|frame.east)
--([xshift=-0.5pt]frame.south east)--(frame.south west); },
overlay first={\draw[colprop,line width=1pt]
([yshift=-1pt]title.north east)--([xshift=-0.5pt,yshift=-1pt]title.north-|frame.east)
--([xshift=-0.5pt]frame.south east); },
overlay middle={\draw[colprop,line width=1pt] ([xshift=-0.5pt]frame.north east)
--([xshift=-0.5pt]frame.south east); },
overlay last={\draw[colprop,line width=1pt] ([xshift=-0.5pt]frame.north east)
--([xshift=-0.5pt]frame.south east)--(frame.south west);},%
}

\newenvironment{thmnom}[1]
  {
    \begin{thmnomhid}{#1}{\theprops}
  }
  {
    \end{thmnomhid}
    \refstepcounter{props}
  }

\newtcolorbox{thmnomhid}[2]{%
empty,title={{\bfseries Théorème {#2}} ({#1})},attach boxed title to top left,
boxed title style={empty,size=minimal,toprule=2pt,top=4pt,
overlay={\draw[colprop,line width=2pt]
([yshift=-1pt]frame.north west)--([yshift=-1pt]frame.north east);}},
coltitle=colprop,
before=\par\medskip\noindent,parbox=false,boxsep=0pt,left=0pt,right=3mm,top=4pt,
breakable,pad at break*=0mm,vfill before first,
overlay unbroken={\draw[colprop,line width=1pt]
([yshift=-1pt]title.north east)--([xshift=-0.5pt,yshift=-1pt]title.north-|frame.east)
--([xshift=-0.5pt]frame.south east)--(frame.south west); },
overlay first={\draw[colprop,line width=1pt]
([yshift=-1pt]title.north east)--([xshift=-0.5pt,yshift=-1pt]title.north-|frame.east)
--([xshift=-0.5pt]frame.south east); },
overlay middle={\draw[colprop,line width=1pt] ([xshift=-0.5pt]frame.north east)
--([xshift=-0.5pt]frame.south east); },
overlay last={\draw[colprop,line width=1pt] ([xshift=-0.5pt]frame.north east)
--([xshift=-0.5pt]frame.south east)--(frame.south west);},%
}

\newenvironment{coro}
  {
    \begin{corohid}{\theprops}
  }
  {
    \end{corohid}
    \refstepcounter{props}
  }

  \newtcolorbox{corohid}[1]{%
  empty,title={Corollaire {#1}},attach boxed title to top left,
boxed title style={empty,size=minimal,toprule=2pt,top=4pt,
overlay={\draw[colprop,line width=2pt]
([yshift=-1pt]frame.north west)--([yshift=-1pt]frame.north east);}},
coltitle=colprop,fonttitle=\bfseries,
before=\par\medskip\noindent,parbox=false,boxsep=0pt,left=0pt,right=3mm,top=4pt,
breakable,pad at break*=0mm,vfill before first,
overlay unbroken={\draw[colprop,line width=1pt]
([yshift=-1pt]title.north east)--([xshift=-0.5pt,yshift=-1pt]title.north-|frame.east)
--([xshift=-0.5pt]frame.south east)--(frame.south west); },
overlay first={\draw[colprop,line width=1pt]
([yshift=-1pt]title.north east)--([xshift=-0.5pt,yshift=-1pt]title.north-|frame.east)
--([xshift=-0.5pt]frame.south east); },
overlay middle={\draw[colprop,line width=1pt] ([xshift=-0.5pt]frame.north east)
--([xshift=-0.5pt]frame.south east); },
overlay last={\draw[colprop,line width=1pt] ([xshift=-0.5pt]frame.north east)
--([xshift=-0.5pt]frame.south east)--(frame.south west);},%
}

\newenvironment{lemme}
  {
    \begin{lemmehid}{\theprops}
  }
  {
    \end{lemmehid}
    \refstepcounter{props}
  }

  \newtcolorbox{lemmehid}[1]{%
  empty,title={Lemme {#1}},attach boxed title to top left,
boxed title style={empty,size=minimal,toprule=2pt,top=4pt,
overlay={\draw[colprop,line width=2pt]
([yshift=-1pt]frame.north west)--([yshift=-1pt]frame.north east);}},
coltitle=colprop,fonttitle=\bfseries,
before=\par\medskip\noindent,parbox=false,boxsep=0pt,left=0pt,right=3mm,top=4pt,
breakable,pad at break*=0mm,vfill before first,
overlay unbroken={\draw[colprop,line width=1pt]
([yshift=-1pt]title.north east)--([xshift=-0.5pt,yshift=-1pt]title.north-|frame.east)
--([xshift=-0.5pt]frame.south east)--(frame.south west); },
overlay first={\draw[colprop,line width=1pt]
([yshift=-1pt]title.north east)--([xshift=-0.5pt,yshift=-1pt]title.north-|frame.east)
--([xshift=-0.5pt]frame.south east); },
overlay middle={\draw[colprop,line width=1pt] ([xshift=-0.5pt]frame.north east)
--([xshift=-0.5pt]frame.south east); },
overlay last={\draw[colprop,line width=1pt] ([xshift=-0.5pt]frame.north east)
--([xshift=-0.5pt]frame.south east)--(frame.south west);},%
}

\colorlet{colexemple}{blue!50!black}
%\newtcolorbox{exemple}{empty, title=Exemple, attach boxed title to top left,
%  boxed title style={empty, size=minimal, toprule=2pt, top=4pt,
%    overlay={\draw[colexemple,line width=2pt]
%([yshift=-1pt]frame.north west)--([yshift=-1pt]frame.north east);}},
%coltitle=colexemple,fonttitle=\bfseries,%\large\bfseries,
%before=\par\medskip\noindent,parbox=false,boxsep=0pt,left=0pt,right=3mm,top=4pt,
%overlay={\draw[colexemple,line width=1pt]
%([yshift=-1pt]title.north east)--([xshift=-0.5pt,yshift=-1pt]title.north-|frame.east)
%--([xshift=-0.5pt]frame.south east)--(frame.south west); },
%}

\newcounter{exemples}
\setcounter{exemples}{1}

\newenvironment{exemple}
  {
    \begin{exemplehid}{\theexemples}
  }
  {
    \end{exemplehid}
    \addtocounter{exemples}{1}
  }

\newtcolorbox{exemplehid}[1]{%
empty,title={Exemple {#1}},attach boxed title to top left,
boxed title style={empty,size=minimal,toprule=2pt,top=4pt,
overlay={\draw[colexemple,line width=2pt]
([yshift=-1pt]frame.north west)--([yshift=-1pt]frame.north east);}},
coltitle=colexemple,fonttitle=\bfseries,
before=\par\medskip\noindent,parbox=false,boxsep=0pt,left=0pt,right=3mm,top=4pt,
breakable,pad at break*=0mm,vfill before first,
overlay unbroken={\draw[colexemple,line width=1pt]
([yshift=-1pt]title.north east)--([xshift=-0.5pt,yshift=-1pt]title.north-|frame.east)
--([xshift=-0.5pt]frame.south east)--(frame.south west); },
overlay first={\draw[colexemple,line width=1pt]
([yshift=-1pt]title.north east)--([xshift=-0.5pt,yshift=-1pt]title.north-|frame.east)
--([xshift=-0.5pt]frame.south east); },
overlay middle={\draw[colexemple,line width=1pt] ([xshift=-0.5pt]frame.north east)
--([xshift=-0.5pt]frame.south east); },
overlay last={\draw[colexemple,line width=1pt] ([xshift=-0.5pt]frame.north east)
--([xshift=-0.5pt]frame.south east)--(frame.south west);},%
}

\newenvironment{contrex}
  {
    \begin{contrexhid}{\theexemples}
  }
  {
    \end{contrexhid}
    \addtocounter{exemples}{1}
  }

\newtcolorbox{contrexhid}[1]{%
empty,title={Contre-exemple {#1}},attach boxed title to top left,
boxed title style={empty,size=minimal,toprule=2pt,top=4pt,
overlay={\draw[colexemple,line width=2pt]
([yshift=-1pt]frame.north west)--([yshift=-1pt]frame.north east);}},
coltitle=colexemple,fonttitle=\bfseries,
before=\par\medskip\noindent,parbox=false,boxsep=0pt,left=0pt,right=3mm,top=4pt,
breakable,pad at break*=0mm,vfill before first,
overlay unbroken={\draw[colexemple,line width=1pt]
([yshift=-1pt]title.north east)--([xshift=-0.5pt,yshift=-1pt]title.north-|frame.east)
--([xshift=-0.5pt]frame.south east)--(frame.south west); },
overlay first={\draw[colexemple,line width=1pt]
([yshift=-1pt]title.north east)--([xshift=-0.5pt,yshift=-1pt]title.north-|frame.east)
--([xshift=-0.5pt]frame.south east); },
overlay middle={\draw[colexemple,line width=1pt] ([xshift=-0.5pt]frame.north east)
--([xshift=-0.5pt]frame.south east); },
overlay last={\draw[colexemple,line width=1pt] ([xshift=-0.5pt]frame.north east)
--([xshift=-0.5pt]frame.south east)--(frame.south west);},%
}

\newenvironment{app}
  {
    \begin{apphid}{\theexemples}
  }
  {
    \end{apphid}
    \addtocounter{exemples}{1}
  }

\newtcolorbox{apphid}[1]{%
empty,title={Application {#1}},attach boxed title to top left,
boxed title style={empty,size=minimal,toprule=2pt,top=4pt,
overlay={\draw[colexemple,line width=2pt]
([yshift=-1pt]frame.north west)--([yshift=-1pt]frame.north east);}},
coltitle=colexemple,fonttitle=\bfseries,
before=\par\medskip\noindent,parbox=false,boxsep=0pt,left=0pt,right=3mm,top=4pt,
breakable,pad at break*=0mm,vfill before first,
overlay unbroken={\draw[colexemple,line width=1pt]
([yshift=-1pt]title.north east)--([xshift=-0.5pt,yshift=-1pt]title.north-|frame.east)
--([xshift=-0.5pt]frame.south east)--(frame.south west); },
overlay first={\draw[colexemple,line width=1pt]
([yshift=-1pt]title.north east)--([xshift=-0.5pt,yshift=-1pt]title.north-|frame.east)
--([xshift=-0.5pt]frame.south east); },
overlay middle={\draw[colexemple,line width=1pt] ([xshift=-0.5pt]frame.north east)
--([xshift=-0.5pt]frame.south east); },
overlay last={\draw[colexemple,line width=1pt] ([xshift=-0.5pt]frame.north east)
--([xshift=-0.5pt]frame.south east)--(frame.south west);},%
}

%%%%%%%%%%%%%%%%%%%%%%%%%%%%%%%%%%%%%%%%%%%%%%%%%%%%%%%%%%%%%%%%%%%%%%%%%%%%%%%%
%
% ENUMERATE
% =========
%
%%%%%%%%%%%%%%%%%%%%%%%%%%%%%%%%%%%%%%%%%%%%%%%%%%%%%%%%%%%%%%%%%%%%%%%%%%%%%%%%

\usepackage{enumerate}
\usepackage{enumitem}

% To have special enumerate items like
%
% 1/
% 2/
% 3/

%%%%%%%%%%%%%%%%%%%%%%%%%%%%%%%%%%%%%%%%%%%%%%%%%%%%%%%%%%%%%%%%%%%%%%%%%%%%%%%%
%
% ARRAYS
% ======
%
%%%%%%%%%%%%%%%%%%%%%%%%%%%%%%%%%%%%%%%%%%%%%%%%%%%%%%%%%%%%%%%%%%%%%%%%%%%%%%%%


\usepackage{array}
\usepackage{makecell} % Used to break lines within arrays
\usepackage{multirow}
\usepackage{booktabs} % Used to have nice arrays with headrules

%%%%%%%%%%%%%%%%%%%%%%%%%%%%%%%%%%%%%%%%%%%%%%%%%%%%%%%%%%%%%%%%%%%%%%%%%%%%%%%%
%
% WRITE CODE
% ==========
%
%%%%%%%%%%%%%%%%%%%%%%%%%%%%%%%%%%%%%%%%%%%%%%%%%%%%%%%%%%%%%%%%%%%%%%%%%%%%%%%%

\usepackage{listings}
\usepackage{xcolor}

%New colors defined below
\definecolor{codegreen}{rgb}{0,0.6,0}
\definecolor{codegray}{rgb}{0.5,0.5,0.5}
\definecolor{codepurple}{rgb}{0.58,0,0.82}
\definecolor{backcolour}{rgb}{0.95,0.95,0.92}

%Code listing style named "mystyle"
\lstdefinestyle{python}{
  %backgroundcolor=\color{backcolour},
  commentstyle=\color{codegreen},
  keywordstyle=\color{magenta},
  numberstyle=\tiny\color{codegray},
  stringstyle=\color{codepurple},
  basicstyle=\ttfamily\footnotesize,
  breakatwhitespace=false,
  breaklines=true,
  captionpos=b,
  keepspaces=true,
  numbers=left,
  numbersep=5pt,
  showspaces=false,
  showstringspaces=false,
  showtabs=false,
  tabsize=2
}

\lstset{style=python}

%%%%%%%%%%%%%%%%%%%%%%%%%%%%%%%%%%%%%%%%%%%%%%%%%%%%%%%%%%%%%%%%%%%%%%%%%%%%%%%%
%
% Tabular 
% =======
%
%%%%%%%%%%%%%%%%%%%%%%%%%%%%%%%%%%%%%%%%%%%%%%%%%%%%%%%%%%%%%%%%%%%%%%%%%%%%%%%%

% In order to obtain a tabular with given width.

\usepackage{tabularx}
\newcolumntype{Y}{>{\centering\arraybackslash}X}
\newcolumntype{R}{>{\raggedright\arraybackslash}X}
\newcolumntype{L}{>{\raggedleft\arraybackslash}X}
% \usepackage{tabulary} % younger brother

%%%%%%%%%%%%%%%%%%%%%%%%%%%%%%%%%%%%%%%%%%%%%%%%%%%%%%%%%%%%%%%%%%%%%%%%%%%%%%%%
%
% MACROS
% ======
%
%%%%%%%%%%%%%%%%%%%%%%%%%%%%%%%%%%%%%%%%%%%%%%%%%%%%%%%%%%%%%%%%%%%%%%%%%%%%%%%%

% Math Operators

\DeclareMathOperator{\Card}{Card}
\DeclareMathOperator{\Gal}{Gal}
\DeclareMathOperator{\Id}{Id}
\DeclareMathOperator{\Img}{Im}
\DeclareMathOperator{\Ker}{Ker}
\DeclareMathOperator{\Minpoly}{Minpoly}
\DeclareMathOperator{\Mod}{mod}
\DeclareMathOperator{\Ord}{Ord}
\DeclareMathOperator{\ppcm}{ppcm}
\DeclareMathOperator{\pgcd}{pgcd}
\DeclareMathOperator{\tr}{Tr}
\DeclareMathOperator{\Vect}{Vect}
\DeclareMathOperator{\Span}{Span}
\DeclareMathOperator{\rank}{rank}
\DeclareMathOperator{\rg}{rg}
\DeclareMathOperator{\ev}{ev}

% Shortcuts

\newcommand{\eg}{\emph{e.g. }}
\newcommand{\ent}[2]{[\![#1,#2]\!]}
\newcommand{\ie}{\emph{i.e. }}
\newcommand{\ps}[2]{\left\langle#1,#2\right\rangle}
\newcommand{\eqdef}{\overset{\text{def}}{=}}
\newcommand{\E}{\mathcal{E}}
\newcommand{\M}{\mathcal{M}}
\newcommand{\A}{\mathcal{A}}
\newcommand{\B}{\mathcal{B}}
\newcommand{\R}{\mathcal{R}}
\newcommand{\D}{\mathcal{D}}
\newcommand{\Pcal}{\mathcal{P}}
\newcommand{\K}{\mathbf{k}}


\title{\vspace{-15mm}Chapitre 2 : Probabilités conditionnelles}
\date{\vspace{-12mm}\href{https://erou.forge.aeif.fr/spe-1e/probas-cond.html}
{\includegraphics{qr-probas-cond.png}}\\Utilisez le QR-code
pour retrouver ce cours au format numérique, ainsi que d'autres
ressources.\vspace{-5mm}}
\author{}

\begin{document}
\maketitle\thispagestyle{fancy}

\begin{notation}
  \begin{itemize}
    \item On note $\Omega$ l'univers, c'est-à-dire l'ensemble des issues
      possibles de l'expérience aléatoire.
    \item Un événement correspond à une partie des issues possibles de
      l'expérience, c'est un sous-ensemble de $\Omega$.
    \item On note $P(A)$ la probabilité que l'événement $A$ se réalise.
    \item On note $\overline A$ l'événement complémentaire de $A$.
    \item On note $A\cup B$ l'événement qui se réalise si l'événement $A$
      \textbf{ou} l'événement $B$ se réalise, c'est-à-dire si au moins l'un des
      deux se réalise.
    \item On note $A\cap B$ l'événement qui se réalise si l'événement $A$
      \textbf{et} l'événement $B$ se réalise, c'est-à-dire si les deux
      événements se réalisent simultanément.
  \end{itemize}
\end{notation}

\section{Probabilités conditionnelles}

\noindent Dans tout le chapitre, sauf indication contraire, $A$ et $B$ sont
deux événements d'un univers $\Omega$, tels que
\[
  P(A)\neq0.
\]

\begin{defi}{Probabilité conditionnelle}
  La \textbf{probabilité conditionnelle} que l'événement $B$ se réalise sachant
  que l'événement $A$ est réalisé se note
  \[
    P_A(B)
  \]
  et est définie par
  \[
    P_A(B) = \frac{P(A\cap B)}{P(A)}
  \]
\end{defi}

\begin{exemple}
  Un sac contient quatre boules noires numérotées de $1$ à $4$ et notées $N_1,
  N_2, N_3, N_4$ ainsi que six boules blanches numérotées de $1$ à $6$ et notées
  $B_1, B_2, B_3, B_4, B_5, B_6$. On extrait au hasard une boule dans le sac. On
  a
  \[
    \Omega=\left\{ N_1, N_2, N_3, N_4, B_1, B_2, B_3, B_4, B_5, B_6 \right\}.
  \]
  On note

  \begin{itemize}
    \item $A$ l'événement « la boule tirée porte un numéro $3$ »;
    \item $B$ l'événement « la boule est blanche ».
  \end{itemize}
  On a $A=\left\{ N_3, B_3 \right\}$, $B=\left\{B_1, B_2, B_3, B_4, B_5, B_6
  \right\}$, et $A\cap B=\left\{ B_3 \right\}$. On a ainsi
  $P(A)=\frac{2}{10}\neq0$ et $P(A\cap B)=\frac{1}{10}$. La probabilité
  d'extraire une boule blanche \textbf{sachant qu'}elle porte le numéro $3$ est
  égale à
  \begin{align*}
    P_A(B) &=\frac{\;\frac{1}{10}\;}{\frac{2}{10}} \\
    &= \frac{1}{10}\times\frac{10}{2} \\
    &= \frac{1}{2}.
  \end{align*}
\end{exemple}

\begin{app}
  On jette un dé équilibré à six faces numérotées de $1$ à $6$ et on s'intéresse
  au nombre obtenu. Quelle est la probabilité que le nombre obtenu soit un
  nombre premier sachant que le nombre obtenu est supérieur à $4$ ?
\end{app}

\begin{prop}
  La probabilité $P_A(B)$ vérifie
  \[
    0\leq P_A(B)\leq 1\text{ et }P_A(B)+P_A(\overline{B})=1.
  \]
\end{prop}

\begin{prop}
  Si $A$ et $B$ sont deux événements de probabilité non nulle, alors
  \[
  P(A\cap B) = P_A(B)\times P(A) = P_B(A)\times P(B).
  \]
  \label{prop:1}
\end{prop}

\begin{exemple}
  Si $P(A)=0,7$, $P(B)=0,6$, et $P_A(B)=\frac{4}{7}$, alors
  \[
    P(A\cap B)=P(A)\times P_A(B) = 0,7\times\frac{4}{7}=0,4
  \]
  puis
  \[
    P_B(A) = \frac{P(B\cap A)}{P(B)} = \frac{0,4}{0,6}=\frac{2}{3}.
  \]
\end{exemple}

\begin{rmq}
  La propriété $2$ permet de passer de $P_A(B)$ à $P_B(A)$ (et inversement). On
  voit dans l'exemple $3$ que ce n'est pas la même chose !
\end{rmq}

\begin{app}
  Dans une classe de première, $55$\% des élèves sont des filles et $40$\% des
  élèves sont des filles demi-pensionnaires. On choisit un élève au hasard dans
  cette classe. Quelle est la probabilité qu'un élève soit demi-pensionnaire
  sachant que c'est une fille ?
\end{app}

\subsection{Utilisation de tableaux}

\begin{notation}
Les \textbf{tableaux à double entrée} permettent une présentation claires de certaines
expériences aléatoires et facilitent le calcul des probabilités conditionnelles.
\begin{center}
\renewcommand{\arraystretch}{1.5}
\begin{tabular}{cccc}
  \toprule
  & $\mathbf{B}$ & $\mathbf{\overline B}$ & \textbf{Total} \\ \midrule
  $\mathbf{A}$ & $P(A\cap B)$ & $P(A\cap\overline B)$ & $P(A)$ \\
  $\mathbf{\overline A}$ & $P(\overline A\cap B)$ & $P(\overline A\cap\overline
  B)$ & $P(\overline A)$ \\
  \textbf{Total} & $P(B)$ & $P(\overline B)$ & $1$ \\ \bottomrule
\end{tabular}
\end{center}
\begin{itemize}
  \item $P(A\cap B)$ se lit à l'intersection de la ligne $A$ et de la colonne
    $B$.
  \item $P(A)$ (respectivement $P(B)$) se lit sur la dernière colonne
    (respectivement la dernière ligne).
  \item $P_A(B)$ (ou $P_B(A)$) s'obtient en calculant le quotient des deux
    probabilités adéquates:
    \[
      P_A(B) = \frac{P(A\cap B)}{P(A)}\text{ et }P_B(A) = \frac{P(A\cap
      B)}{P(B)}.
    \]
\end{itemize}
\end{notation}

\begin{exemple}
  Si $P(A)=0,7$, $P(B)=0,6$ et $P(A\cap B)=0,4$, on a alors le tableau
  suivant.\\
  \begin{minipage}[]{.5\textwidth}
 Et on trouve donc
\begin{itemize}
  \item $P(\overline A\cap\overline B)=0,1$;
  \item $P_B(\overline A)=\frac{P(\overline A\cap
    B)}{P(B)}=\frac{0,2}{0,6}=\frac{1}{3}$;
  \item $P_{\overline A}(B)=\frac{P(\overline A\cap
    B)}{P(\overline A)}=\frac{0,2}{0,3}=\frac{2}{3}$;
\end{itemize}
  \end{minipage}
  \begin{minipage}[]{.5\textwidth}
 \begin{center}
\renewcommand{\arraystretch}{1.5}
\begin{tabular}{cccc}
  \toprule
  & $\mathbf{B}$ & $\mathbf{\overline B}$ & \textbf{Total} \\ \midrule
  $\mathbf{A}$ & $0,4$ & $0,3$ & $0,7$ \\
  $\mathbf{\overline A}$ & $0,2$ & $0,1$ & $0,3$ \\
  \textbf{Total} & $0,6$ & $0,4$ & $1$ \\ \bottomrule
\end{tabular}
\end{center}
  \end{minipage}
\end{exemple}

\begin{app}
  Un club sportif rassemble $180$ membres répartis en deux catégories : juniors
  et seniors. On compte $135$ seniors dont $81$ hommes. Il y a $27$ garçons
  parmi les juniors.\\
  En choisissant une femme au hasard, calculer la probabilité d'avoir une
  juniore.
\end{app}

\section{Formule des probabilités totales}
\subsection{Arbre pondéré}

\begin{defi}{Arbre pondéré}
  \begin{minipage}{.4\textwidth}
  Un \textbf{arbre pondéré}, ou \textbf{arbre de probabilité}, est un schéma
  mettant en jeu des probabilités conditionnelles et permettant de calculer
  rapidement des probabilités.
\end{minipage}
  \begin{minipage}{.6\textwidth}
    \begin{center}
\begin{tikzpicture}[scale=.8,level 1/.style={sibling distance=2cm},
    level 2/.style={sibling distance=1cm}]
  \node {} [grow'=right, level distance=3cm]
  child {
    node {$A$}
    child {
      node {$B$}
      edge from parent node[above] {$P_A(B)$}
    }
    child{
      node {$\overline B$}
      edge from parent node[below] {$P_A(\overline B)$}
    }
    edge from parent node[above] {$P(A)$}
  }
  child {
    node {$\overline A$}
    child {
      node {$B$}
      edge from parent node[above] {$P_{\overline A}(B)$}
    }
    child{
      node {$\overline B$}
      edge from parent node[below] {$P_{\overline A}(\overline B)$}
    }
    edge from parent node[below] {$P(\overline A)$}
  };
  \node at (9.2, -.515) {$P(\overline A\cap B)=P(\overline A)\times P_{\overline
  A}(B)$};
\end{tikzpicture}
\end{center}    
  \end{minipage}
\end{defi}
\begin{propadm}
  \begin{enumerate}
    \item La somme des probabilités des branches issues d'un nœud est égale à
      $1$.
    \item La probabilité de l'événement à l'extrémité d'un chemin est égale au
      produit des probabilités des branches composant ce chemin.
    \item La probabilité d'un événement est égale à la somme des probabilités
      des chemins conduisant à cet événement.
  \end{enumerate}
\end{propadm}

\begin{exemple}
  \begin{minipage}{.6\textwidth}
    On considère l'arbre pondéré ci-contre.
    \begin{itemize}
      \item La première propriété nous dit que $0,7+0,1+x=1$, d'où $x=0,2$. De
        même $y=0,6$ et $z=0,6$.
      \item La deuxième propriété nous dit que $P(A\cap D)=P(A)\times
        P_A(D)=0,7\times0,4=0,28$.
      \item La troisième propriété nous dit que $P(D)=P(A\cap D)+P(C\cap
        D)=0,7\times0,4+0,2\times0,5=0,38$.
    \end{itemize}
  \end{minipage}
    \begin{minipage}{.4\textwidth}
    \begin{center}
\begin{tikzpicture}[scale=.8,level 1/.style={sibling distance=3cm},
    level 2/.style={sibling distance=1.5cm}]
  \node {} [grow'=right, level distance=3cm]
  child {
    node {$A$}
    child {
      node {$D$}
      edge from parent node[above] {$0,4$}
    }
    child{
      node {$\overline D$}
    edge from parent node[below] {$y$}
    }
    edge from parent node[above=.1cm] {$0,7$}
  }
  child {
    node {$B$}
    child {
      node {$E$}
      edge from parent node[above] {$0,1$}
    }
    child{
      node {$F$}
      edge from parent node[above=-.1cm] {$0,3$}
    }
    child{
      node {$G$}
    edge from parent node[below] {$z$}
    }
    edge from parent node[above] {$0,1$}
  }
    child {
      node{$C$}
      child{
        node{$D$}
        edge from parent node[above] {$0,5$}
      }
      child{
        node{$\overline D$}
        edge from parent node[below] {$0,5$}
      }
    edge from parent node[below=.1cm] {$x$}
  };
\end{tikzpicture}
\end{center}    
  \end{minipage}
\end{exemple}
\begin{app}
  On considère une expérience aléatoire et deux événements $A$ et $B$ tels que
  $P(A)=0,6$, $P_A(B)=0,7$ et $P_{\overline A}(B)=0,2$.
  \begin{enumerate}
    \item Construire un arbre pondéré complet représentant cette expérience.
    \item Déterminer la probabilité de l'événement $A\cap B$.
  \end{enumerate}
\end{app}

\subsection{Probabilités totales}
\noindent\begin{minipage}{.6\textwidth}
\begin{defi}{Partition de l'univers}
  Soit $k\in\mathbb{N}^*$ un entier naturel non nul et $A_1, A_2, \dots, A_k$
  des événements non vides de $\Omega$. Les événements $A_1, A_2,
  \dots, A_k$ forment une \textbf{partition de l'univers} $\Omega$
  si et seulement si
  \begin{itemize}
    \item ils sont deux à deux \textbf{incompatibles} : pour tous entiers
      distincts $i$ et $j$ entre $1$ et $k$, on a $A_i\cap A_j=\emptyset$ ;
    \item leur réunion forme tout l'univers : $A_1\cup A_2\cup\cdots\cup
      A_k=\Omega$.
  \end{itemize}
\end{defi}

\begin{propnom}{Formule des probabilités totales}
  On considère une expérience aléatoire d'univers $\Omega$ et un événement $B$.
  On note $A_1, \dots, A_k$ $k$ événements formant une partition de l'univers.
  Alors on a
  \[
    P(B) = P(A_1\cap B)+P(A_2\cap B)+\cdots+P(A_k\cap B)
  \]
\end{propnom}
  \end{minipage}
  \begin{minipage}{.4\textwidth}
    \begin{center}
      \begin{tikzpicture}[scale=.97]
  \draw[thick] (0,0) -- (5, 0) -- (5, 8) -- (0, 8) -- (0,0);
  \fill[blue, opacity=.15] (0,0) -- (3, 0) -- (3, 2) -- (0, 2);
  \draw[blue, opacity=.3] (3, 0) -- (3, 2) -- (0, 2);
  \fill[red, opacity=.15] (3,0) -- (3, 5) -- (5, 5) -- (5, 0);
  \draw[red, opacity=.3] (3,0) -- (3, 5) -- (5, 5);
  \fill[cyan, opacity=.15] (5,5) -- (1, 5) -- (1, 7) -- (5, 7);
  \draw[cyan, opacity=.3] (5,5) -- (1, 5) -- (1, 7) -- (5, 7);
  \fill[orange, opacity=.15] (0,2) -- (3, 2) -- (3, 5) -- (1, 5) -- (1, 7) --
  (5, 7) -- (5, 8) -- (0, 8);
  \node[blue] at (1, 1) {\Large $A_1$};
  \node[red] at (4.5, 1) {\Large $A_2$};
  \node[cyan] at (1.7, 6.2) {\Large $A_3$};
  \node[orange] at (1, 3.5) {\Large $A_4$};
  \node at (5.5, 4) {\Huge $\Omega$};
  \node[green!50!black] at (4, 6.5) {\LARGE $B$};
  \draw[thick, green!50!black, opacity=.6] (3, 3.5) ellipse (1.5 and 3);
  \fill[green!50!black, opacity=.15] (3, 3.5) ellipse (1.5 and 3);
\end{tikzpicture}
    \end{center}
\end{minipage}

\begin{rmq}
  Un événement $A$ et son complémentaire $\overline A$ forment toujours une
  partition de l'univers. On a donc
  \[
    P(B) = P(A\cap B)+P(\overline A\cap B).
  \]
\end{rmq}

\begin{exemple}
  \begin{minipage}{.6\textwidth}
    On considère l'arbre pondéré ci-contre. Les événements $A$, $B$ et $C$
    forment une partition de l'univers $\Omega$, ainsi
    \begin{align*}
      P(D) &= P(A\cap D)+P(B\cap D)+P(C\cap D) \\
      &= P(A)\times P_A(D)+P(B)\times P_B(D)+P(C)\times P_C(D) \\
      &= 0,1\times 0,2 + 0,5\times0,7+0,4\times0,1\\
      &=0,41
    \end{align*}
  \end{minipage}
    \begin{minipage}{.4\textwidth}
    \begin{center}
\begin{tikzpicture}[scale=.8,level 1/.style={sibling distance=2cm},
    level 2/.style={sibling distance=1cm}]
  \node {} [grow'=right, level distance=3cm]
  child {
    node {$A$}
    child {
      node {$D$}
      edge from parent node[above] {$0,2$}
    }
    child{
      node {$\overline D$}
    edge from parent node[below] {$0,8$}
    }
    edge from parent node[above=.1cm] {$0,1$}
  }
  child {
    node {$B$}
    child {
      node {$D$}
      edge from parent node[above] {$0,7$}
    }
    child{
      node {$\overline D$}
      edge from parent node[below] {$0,3$}
    }
    edge from parent node[above] {$0,5$}
  }
    child {
      node{$C$}
      child{
        node{$D$}
        edge from parent node[above] {$0,1$}
      }
      child{
        node{$\overline D$}
        edge from parent node[below] {$0,9$}
      }
    edge from parent node[below=.1cm] {$0,4$}
  };
\end{tikzpicture}
\end{center}    
  \end{minipage}
\end{exemple}

\begin{app}
  \begin{minipage}{.6\textwidth}
    On considère les événements $A$ et $B$ vérifiant l'arbre pondéré
    ci-contre.\\
     Déterminer $P(B)$.
  \end{minipage}
    \begin{minipage}{.4\textwidth}
    \begin{center}
\begin{tikzpicture}[scale=.8,level 1/.style={sibling distance=2cm},
    level 2/.style={sibling distance=1cm}]
  \node {} [grow'=right, level distance=3cm]
  child {
    node {$A$}
    child {
      node {$B$}
      edge from parent node[above] {$0,7$}
    }
    child{
      node {$\overline B$}
    edge from parent node[below] {$0,3$}
    }
    edge from parent node[above] {$0,6$}
  }
  child {
    node {$\overline A$}
    child {
      node {$B$}
      edge from parent node[above] {$0,2$}
    }
    child{
      node {$\overline B$}
      edge from parent node[below] {$0,8$}
    }
    edge from parent node[below] {$0,4$}
  };
\end{tikzpicture}
\end{center}    
  \end{minipage}
\end{app}

\section{Indépendance}
\begin{defi}{Indépendance}
  Soient $A$ et $B$ deux événements d'un univers $\Omega$. On dit que $A$ et $B$
  sont \textbf{indépendants} lorsque
  \[
    P(A\cap B) = P(A)\times P(B).
  \]
\end{defi}
\begin{prop}
  Soient $A$ et $B$ deux événements d'un univers $\Omega$, tels que $P(A)\neq0$.
  Alors $A$ et $B$ sont indépendants si et seulement si $P_A(B)=P(B)$.
\end{prop}
\begin{proof}
  C'est un résultat d'équivalence (``si et seulement si''). Commençons par le
  sens direct : si $A$ et $B$ sont indépendants alors $P_A(B)=P(B)$. Cela vient de la définition de $P_A(B)$. En effet $P_A(B)=\frac{P(A\cap
  B)}{P(A)}$. Mais dans le cas de deux événements indépendants on a
  $P(A\cap B)=P(A)\times P(B)$, donc
  \[
    P_A(B)=\frac{P(A)P(B)}{P(A)}=P(B).
  \]
  L'autre sens (si $P_A(B)=P(B)$ alors $A$ et $B$ sont indépendants) vient aussi
  de la définition. On a $P_A(B)=P(B)$ donc $\frac{P(A\cap B)}{P(A)}=P(B)$, d'où
  finalement $P(A\cap B)=P(A)P(B)$, et donc les événements $A$ et $B$ sont
  indépendants.
\end{proof}
\begin{rmq}
  L'intuition derrière la notion d'indépendance est que si deux événements sont
  indépendants, la réalisation de l'un n'influence pas la réalisation
  de l'autre.
\end{rmq}
\begin{exemple}
  Soient $A$ et $B$ deux événements indépendants tels que $P(A)=0,8$ et
  $P(B)=0,35$. Danc ce cas, on a
  \[
    P(A\cap B) = 0,8\times0,35=0,28.
  \]
\end{exemple}
\begin{propadm}
  Si $A$ et $B$ sont deux événements indépendants, alors $\overline A$ et $B$
  sont aussi deux événements indépendants.
\end{propadm}
\begin{app}
  On dispose d'une urne qui contient trois boules rouges numérotées $1, 2, 3$
  ainsi que six boules noires numérotées $1, 1, 1, 2, 2$ et $3$. Les boules sont
  indiscernables au toucher. On tire une boule au hasard dans cette urne et on
  s'intéresse aux événements suivants :
  \begin{itemize}
    \item $R$ : « Tirer une boule rouge. »
    \item $P$ : « Tirer une boule dont le numéro est pair. »
    \item $U$ : « Tirer une boule dont le numéro est $1$. »
  \end{itemize}
  \begin{enumerate}
    \item Montrer que les événements $P$ et $R$ sont indépendants.
    \item Les événements $R$ et $U$ sont-ils indépendants ? Justifier.
  \end{enumerate}
\end{app}

\begin{app}
  Exercice plus difficile avec des inconnues sur les branches.
\end{app}

\end{document}
