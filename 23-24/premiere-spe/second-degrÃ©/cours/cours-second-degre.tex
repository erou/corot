\documentclass[11pt]{article}

\newcommand{\titrechapitre}{Second degré -- Cours}
\newcommand{\titreclasse}{Lycée Jean-Baptiste \textsc{Corot}}
\newcommand{\pagination}{\thepage/\pageref{LastPage}}
\newcommand{\topbotmargins}{2cm}

%%%%%%%%%%%%%%%%%%%%%%%%%%%%%%%%%%%%%%%%%%%%%%%%%%%%%%%%%%%%%%%%%%%%%%%%%%%%%%%%
%
% PACKAGES
% ========
%
%%%%%%%%%%%%%%%%%%%%%%%%%%%%%%%%%%%%%%%%%%%%%%%%%%%%%%%%%%%%%%%%%%%%%%%%%%%%%%%%

\usepackage[english, french]{babel}
\usepackage[utf8]{inputenc}
\usepackage[T1]{fontenc}
\usepackage{graphicx}
\usepackage{amsmath,amssymb,amsthm,amsopn}
\usepackage{hyperref}

% Pour avoir l'écriture \mathscr (math script)
% ============================================

\usepackage{mathrsfs}

% Deal with coma as a decimal separator
% =====================================

\usepackage{icomma}

% Package Geometry
% ================

\usepackage[a4paper, lmargin=2cm, rmargin=2cm, top=\topbotmargins, bottom=\topbotmargins]{geometry}

% Package multicol
% ================

\usepackage{multicol}

% Redefine abstract
% =================

% Note
% ====
%
% Le reste a été commenté pour ne pas charger trop de choses au démarrage. On
% verra si on en a besoin plus tard.
%
% --------
%
%\usepackage{mathrsfs}
%\usepackage{multirow}
%\usepackage{bm}
%\hypersetup{
%    colorlinks=true,
%    linkcolor=blue,
%    citecolor=red,
%}
%\usepackage{diagbox}
%
%\usepackage{algorithm}
%\usepackage{algpseudocode}
%
%\renewcommand{\algorithmicrequire}{\textbf{Input:}}
%\renewcommand{\algorithmicensure}{\textbf{Output:}}


%%%%%%%%%%%%%%%%%%%%%%%%%%%%%%%%%%%%%%%%%%%%%%%%%%%%%%%%%%%%%%%%%%%%%%%%%%%%%%%%
%
% TIKZ
% ====
%
%%%%%%%%%%%%%%%%%%%%%%%%%%%%%%%%%%%%%%%%%%%%%%%%%%%%%%%%%%%%%%%%%%%%%%%%%%%%%%%%

\usepackage{tikz}
\usetikzlibrary{arrows}

\usepackage{tkz-tab} % Variation tables

\usepackage{pgfplots}
%\usepackage{pgf-pie} % Pie charts

\pgfplotsset{
%\newcommand{\settingsgraph}{
x=.5cm,y=.5cm,
xticklabel style = {font=\scriptsize, yshift=.1cm},
yticklabel style = {font=\scriptsize, xshift=.1cm},
axis lines=middle,
ymajorgrids=true,
xmajorgrids=true,
major grid style = {color=white!80!blue},
xmin=-5.5,
xmax=5.5,
ymin=-5.5,
ymax=5.5,
xtick={-5.0,-4.0,...,5.0},
ytick={-5.0,-4.0,...,5.0},
}

% Tikz style

\tikzset{round/.style={circle, draw=black, very thick, scale = 0.7}}
\tikzset{arrow/.style={->, >=latex}}
\tikzset{dashed-arrow/.style={->, >=latex, dashed}}

\newcommand{\point}[3]{\draw[very thick, #3] (#1-.1, #2)--(#1+.1, #2)
(#1, #2-.1)--(#1, #2+.1)}

%%%%%%%%%%%%%%%%%%%%%%%%%%%%%%%%%%%%%%%%%%%%%%%%%%%%%%%%%%%%%%%%%%%%%%%%%%%%%%%%
%
% FANCY HEADER
% ============
%
%%%%%%%%%%%%%%%%%%%%%%%%%%%%%%%%%%%%%%%%%%%%%%%%%%%%%%%%%%%%%%%%%%%%%%%%%%%%%%%%


\usepackage{fancyhdr}
\usepackage{lastpage}

\pagestyle{fancy}
\newcommand{\changefont}{\fontsize{9}{9}\selectfont}
\renewcommand{\headrulewidth}{0mm}
\renewcommand{\footrulewidth}{0mm}

\fancyhead[C]{}
\fancyhead[L]{\titreclasse}
\fancyhead[R]{\titrechapitre}
\fancyfoot[C]{}
\fancyfoot[L]{}
\fancyfoot[R]{\pagination}
\addtolength{\skip\footins}{20pt} % distance between text and footnotes

%%%%%%%%%%%%%%%%%%%%%%%%%%%%%%%%%%%%%%%%%%%%%%%%%%%%%%%%%%%%%%%%%%%%%%%%%%%%%%%%
%
% THEOREM STYLE
% =============
%
%%%%%%%%%%%%%%%%%%%%%%%%%%%%%%%%%%%%%%%%%%%%%%%%%%%%%%%%%%%%%%%%%%%%%%%%%%%%%%%%

\usepackage[tikz]{bclogo}
\usepackage{mdframed}

\usepackage{tcolorbox}
\tcbuselibrary{listings, breakable, theorems, skins}

%\newtheoremstyle{break}%
%{}{}%
%{\itshape}{}%
%{\bfseries}{}%  % Note that final punctuation is omitted.
%{\newline}{}

\newtheoremstyle{scbf}%
{}{}%
{}{}%
%{\scshape}{}%  % Note that final punctuation is omitted.
{\bfseries\scshape}{}%  % Note that final punctuation is omitted.
{\newline}{}

%\theoremstyle{break}
%\theoremstyle{plain}
%\newtheorem{thm}{Theorem}[section]
%\newtheorem{lm}[thm]{Lemma}
%\newtheorem{prop}[thm]{Proposition}
%\newtheorem{cor}[thm]{Corollary}

%\theoremstyle{scbf}
%\newtheorem{exo}{$\star$ Exercice}

%\theoremstyle{definition}
%\newtheorem{defi}[thm]{Definition}
%\newtheorem{ex}[thm]{Example}

%\theoremstyle{remark}
%\newtheorem{rem}[thm]{Remark}

% Defining the Remark environment
% ===============================

\newenvironment{rmq}
  {
    \begin{bclogo}[logo=\bcinfo, noborder=true]{Remarque}
  }
  {
    \end{bclogo}
  }

% Defining the exercise environment
% =================================

\newcounter{exos}
\setcounter{exos}{1}

\newenvironment{exo}
  {
    \begin{bclogo}[logo=\bccrayon, noborder=true]{Exercice \theexos}
  }
  {
    \end{bclogo}
    \addtocounter{exos}{1}
  }


% Redefining the proof environment from amsthm
% ============================================

\tcolorboxenvironment{proof}{
  blanker, breakable, before skip=10pt,after skip=10pt,
  borderline west={1mm}{0pt}{red},
  left=5mm,
}

% Defining the definition environment
% ===================================

\colorlet{coldef}{black!50!green}

\newcounter{defis}
\setcounter{defis}{1}

\newenvironment{defi}[1]
  {
    \begin{defihid}{{#1}}{\thedefis}
  }
  {
    \end{defihid}
    \addtocounter{defis}{1}
  }

\newtcolorbox{defihid}[2]{%
  empty,title={ {\bfseries Définition {#2}} ({#1})},attach boxed title to top left,
boxed title style={empty,size=minimal,toprule=2pt,top=4pt,
overlay={\draw[coldef,line width=2pt]
([yshift=-1pt]frame.north west)--([yshift=-1pt]frame.north east);}},
coltitle=coldef,
before=\par\medskip\noindent,parbox=false,boxsep=0pt,left=0pt,right=3mm,top=4pt,
breakable,pad at break*=0mm,vfill before first,
overlay unbroken={\draw[coldef,line width=1pt]
([yshift=-1pt]title.north east)--([xshift=-0.5pt,yshift=-1pt]title.north-|frame.east)
--([xshift=-0.5pt]frame.south east)--(frame.south west); },
overlay first={\draw[coldef,line width=1pt]
([yshift=-1pt]title.north east)--([xshift=-0.5pt,yshift=-1pt]title.north-|frame.east)
--([xshift=-0.5pt]frame.south east); },
overlay middle={\draw[coldef,line width=1pt] ([xshift=-0.5pt]frame.north east)
--([xshift=-0.5pt]frame.south east); },
overlay last={\draw[coldef,line width=1pt] ([xshift=-0.5pt]frame.north east)
--([xshift=-0.5pt]frame.south east)--(frame.south west);},%
}

\newenvironment{notation}
  {
    \begin{notationhid}{\thedefis}
  }
  {
    \end{notationhid}
    \addtocounter{defis}{1}
  }

\newtcolorbox{notationhid}[1]{%
  empty,title={Notation {#1}},attach boxed title to top left,
boxed title style={empty,size=minimal,toprule=2pt,top=4pt,
overlay={\draw[coldef,line width=2pt]
([yshift=-1pt]frame.north west)--([yshift=-1pt]frame.north east);}},
coltitle=coldef,fonttitle=\bfseries,
before=\par\medskip\noindent,parbox=false,boxsep=0pt,left=0pt,right=3mm,top=4pt,
breakable,pad at break*=0mm,vfill before first,
overlay unbroken={\draw[coldef,line width=1pt]
([yshift=-1pt]title.north east)--([xshift=-0.5pt,yshift=-1pt]title.north-|frame.east)
--([xshift=-0.5pt]frame.south east)--(frame.south west); },
overlay first={\draw[coldef,line width=1pt]
([yshift=-1pt]title.north east)--([xshift=-0.5pt,yshift=-1pt]title.north-|frame.east)
--([xshift=-0.5pt]frame.south east); },
overlay middle={\draw[coldef,line width=1pt] ([xshift=-0.5pt]frame.north east)
--([xshift=-0.5pt]frame.south east); },
overlay last={\draw[coldef,line width=1pt] ([xshift=-0.5pt]frame.north east)
--([xshift=-0.5pt]frame.south east)--(frame.south west);},%
}


% Defining the proposition, theorem, etc. environment
% ===================================================

\colorlet{colprop}{red!75!black}

\newcounter{props}
\setcounter{props}{1}

\newenvironment{prop}
  {
    \begin{prophid}{\theprops}
  }
  {
    \end{prophid}
    \refstepcounter{props}
  }

\newtcolorbox{prophid}[1]{%
empty,title={Propriété {#1}},attach boxed title to top left,
boxed title style={empty,size=minimal,toprule=2pt,top=4pt,
overlay={\draw[colprop,line width=2pt]
([yshift=-1pt]frame.north west)--([yshift=-1pt]frame.north east);}},
coltitle=colprop,fonttitle=\bfseries,
before=\par\medskip\noindent,parbox=false,boxsep=0pt,left=0pt,right=3mm,top=4pt,
breakable,pad at break*=0mm,vfill before first,
overlay unbroken={\draw[colprop,line width=1pt]
([yshift=-1pt]title.north east)--([xshift=-0.5pt,yshift=-1pt]title.north-|frame.east)
--([xshift=-0.5pt]frame.south east)--(frame.south west); },
overlay first={\draw[colprop,line width=1pt]
([yshift=-1pt]title.north east)--([xshift=-0.5pt,yshift=-1pt]title.north-|frame.east)
--([xshift=-0.5pt]frame.south east); },
overlay middle={\draw[colprop,line width=1pt] ([xshift=-0.5pt]frame.north east)
--([xshift=-0.5pt]frame.south east); },
overlay last={\draw[colprop,line width=1pt] ([xshift=-0.5pt]frame.north east)
--([xshift=-0.5pt]frame.south east)--(frame.south west);},%
}

\newenvironment{propadm}
  {
    \begin{propadmhid}{\theprops}
  }
  {
    \end{propadmhid}
    \refstepcounter{props}
  }

  \newtcolorbox{propadmhid}[1]{%
    empty,title={{\bfseries Propriété {#1}} (admise)},attach boxed title to top left,
boxed title style={empty,size=minimal,toprule=2pt,top=4pt,
overlay={\draw[colprop,line width=2pt]
([yshift=-1pt]frame.north west)--([yshift=-1pt]frame.north east);}},
coltitle=colprop,%fonttitle=\bfseries,
before=\par\medskip\noindent,parbox=false,boxsep=0pt,left=0pt,right=3mm,top=4pt,
breakable,pad at break*=0mm,vfill before first,
overlay unbroken={\draw[colprop,line width=1pt]
([yshift=-1pt]title.north east)--([xshift=-0.5pt,yshift=-1pt]title.north-|frame.east)
--([xshift=-0.5pt]frame.south east)--(frame.south west); },
overlay first={\draw[colprop,line width=1pt]
([yshift=-1pt]title.north east)--([xshift=-0.5pt,yshift=-1pt]title.north-|frame.east)
--([xshift=-0.5pt]frame.south east); },
overlay middle={\draw[colprop,line width=1pt] ([xshift=-0.5pt]frame.north east)
--([xshift=-0.5pt]frame.south east); },
overlay last={\draw[colprop,line width=1pt] ([xshift=-0.5pt]frame.north east)
--([xshift=-0.5pt]frame.south east)--(frame.south west);},%
}

\newenvironment{propnom}[1]
  {
    \begin{propnomhid}{#1}{\theprops}
  }
  {
    \end{propnomhid}
    \refstepcounter{props}
  }

\newtcolorbox{propnomhid}[2]{%
empty,title={{\bfseries Propriété {#2}} ({#1})},attach boxed title to top left,
boxed title style={empty,size=minimal,toprule=2pt,top=4pt,
overlay={\draw[colprop,line width=2pt]
([yshift=-1pt]frame.north west)--([yshift=-1pt]frame.north east);}},
coltitle=colprop,
before=\par\medskip\noindent,parbox=false,boxsep=0pt,left=0pt,right=3mm,top=4pt,
breakable,pad at break*=0mm,vfill before first,
overlay unbroken={\draw[colprop,line width=1pt]
([yshift=-1pt]title.north east)--([xshift=-0.5pt,yshift=-1pt]title.north-|frame.east)
--([xshift=-0.5pt]frame.south east)--(frame.south west); },
overlay first={\draw[colprop,line width=1pt]
([yshift=-1pt]title.north east)--([xshift=-0.5pt,yshift=-1pt]title.north-|frame.east)
--([xshift=-0.5pt]frame.south east); },
overlay middle={\draw[colprop,line width=1pt] ([xshift=-0.5pt]frame.north east)
--([xshift=-0.5pt]frame.south east); },
overlay last={\draw[colprop,line width=1pt] ([xshift=-0.5pt]frame.north east)
--([xshift=-0.5pt]frame.south east)--(frame.south west);},%
}




\newenvironment{thm}
  {
    \begin{thmhid}{\theprops}
  }
  {
    \end{thmhid}
    \refstepcounter{props}
  }

\newtcolorbox{thmhid}[1]{%
empty,title={Théorème {#1}},attach boxed title to top left,
boxed title style={empty,size=minimal,toprule=2pt,top=4pt,
overlay={\draw[colprop,line width=2pt]
([yshift=-1pt]frame.north west)--([yshift=-1pt]frame.north east);}},
coltitle=colprop,fonttitle=\bfseries,
before=\par\medskip\noindent,parbox=false,boxsep=0pt,left=0pt,right=3mm,top=4pt,
breakable,pad at break*=0mm,vfill before first,
overlay unbroken={\draw[colprop,line width=1pt]
([yshift=-1pt]title.north east)--([xshift=-0.5pt,yshift=-1pt]title.north-|frame.east)
--([xshift=-0.5pt]frame.south east)--(frame.south west); },
overlay first={\draw[colprop,line width=1pt]
([yshift=-1pt]title.north east)--([xshift=-0.5pt,yshift=-1pt]title.north-|frame.east)
--([xshift=-0.5pt]frame.south east); },
overlay middle={\draw[colprop,line width=1pt] ([xshift=-0.5pt]frame.north east)
--([xshift=-0.5pt]frame.south east); },
overlay last={\draw[colprop,line width=1pt] ([xshift=-0.5pt]frame.north east)
--([xshift=-0.5pt]frame.south east)--(frame.south west);},%
}

\newenvironment{thmadm}
  {
    \begin{thmadmhid}{\theprops}
  }
  {
    \end{thmadmhid}
    \refstepcounter{props}
  }

  \newtcolorbox{thmadmhid}[1]{%
    empty,title={{\bfseries Théorème {#1}} (admis)},attach boxed title to top left,
boxed title style={empty,size=minimal,toprule=2pt,top=4pt,
overlay={\draw[colprop,line width=2pt]
([yshift=-1pt]frame.north west)--([yshift=-1pt]frame.north east);}},
coltitle=colprop,%fonttitle=\bfseries,
before=\par\medskip\noindent,parbox=false,boxsep=0pt,left=0pt,right=3mm,top=4pt,
breakable,pad at break*=0mm,vfill before first,
overlay unbroken={\draw[colprop,line width=1pt]
([yshift=-1pt]title.north east)--([xshift=-0.5pt,yshift=-1pt]title.north-|frame.east)
--([xshift=-0.5pt]frame.south east)--(frame.south west); },
overlay first={\draw[colprop,line width=1pt]
([yshift=-1pt]title.north east)--([xshift=-0.5pt,yshift=-1pt]title.north-|frame.east)
--([xshift=-0.5pt]frame.south east); },
overlay middle={\draw[colprop,line width=1pt] ([xshift=-0.5pt]frame.north east)
--([xshift=-0.5pt]frame.south east); },
overlay last={\draw[colprop,line width=1pt] ([xshift=-0.5pt]frame.north east)
--([xshift=-0.5pt]frame.south east)--(frame.south west);},%
}

\newenvironment{thmnom}[1]
  {
    \begin{thmnomhid}{#1}{\theprops}
  }
  {
    \end{thmnomhid}
    \refstepcounter{props}
  }

\newtcolorbox{thmnomhid}[2]{%
empty,title={{\bfseries Théorème {#2}} ({#1})},attach boxed title to top left,
boxed title style={empty,size=minimal,toprule=2pt,top=4pt,
overlay={\draw[colprop,line width=2pt]
([yshift=-1pt]frame.north west)--([yshift=-1pt]frame.north east);}},
coltitle=colprop,
before=\par\medskip\noindent,parbox=false,boxsep=0pt,left=0pt,right=3mm,top=4pt,
breakable,pad at break*=0mm,vfill before first,
overlay unbroken={\draw[colprop,line width=1pt]
([yshift=-1pt]title.north east)--([xshift=-0.5pt,yshift=-1pt]title.north-|frame.east)
--([xshift=-0.5pt]frame.south east)--(frame.south west); },
overlay first={\draw[colprop,line width=1pt]
([yshift=-1pt]title.north east)--([xshift=-0.5pt,yshift=-1pt]title.north-|frame.east)
--([xshift=-0.5pt]frame.south east); },
overlay middle={\draw[colprop,line width=1pt] ([xshift=-0.5pt]frame.north east)
--([xshift=-0.5pt]frame.south east); },
overlay last={\draw[colprop,line width=1pt] ([xshift=-0.5pt]frame.north east)
--([xshift=-0.5pt]frame.south east)--(frame.south west);},%
}

\newenvironment{coro}
  {
    \begin{corohid}{\theprops}
  }
  {
    \end{corohid}
    \refstepcounter{props}
  }

  \newtcolorbox{corohid}[1]{%
  empty,title={Corollaire {#1}},attach boxed title to top left,
boxed title style={empty,size=minimal,toprule=2pt,top=4pt,
overlay={\draw[colprop,line width=2pt]
([yshift=-1pt]frame.north west)--([yshift=-1pt]frame.north east);}},
coltitle=colprop,fonttitle=\bfseries,
before=\par\medskip\noindent,parbox=false,boxsep=0pt,left=0pt,right=3mm,top=4pt,
breakable,pad at break*=0mm,vfill before first,
overlay unbroken={\draw[colprop,line width=1pt]
([yshift=-1pt]title.north east)--([xshift=-0.5pt,yshift=-1pt]title.north-|frame.east)
--([xshift=-0.5pt]frame.south east)--(frame.south west); },
overlay first={\draw[colprop,line width=1pt]
([yshift=-1pt]title.north east)--([xshift=-0.5pt,yshift=-1pt]title.north-|frame.east)
--([xshift=-0.5pt]frame.south east); },
overlay middle={\draw[colprop,line width=1pt] ([xshift=-0.5pt]frame.north east)
--([xshift=-0.5pt]frame.south east); },
overlay last={\draw[colprop,line width=1pt] ([xshift=-0.5pt]frame.north east)
--([xshift=-0.5pt]frame.south east)--(frame.south west);},%
}

\newenvironment{lemme}
  {
    \begin{lemmehid}{\theprops}
  }
  {
    \end{lemmehid}
    \refstepcounter{props}
  }

  \newtcolorbox{lemmehid}[1]{%
  empty,title={Lemme {#1}},attach boxed title to top left,
boxed title style={empty,size=minimal,toprule=2pt,top=4pt,
overlay={\draw[colprop,line width=2pt]
([yshift=-1pt]frame.north west)--([yshift=-1pt]frame.north east);}},
coltitle=colprop,fonttitle=\bfseries,
before=\par\medskip\noindent,parbox=false,boxsep=0pt,left=0pt,right=3mm,top=4pt,
breakable,pad at break*=0mm,vfill before first,
overlay unbroken={\draw[colprop,line width=1pt]
([yshift=-1pt]title.north east)--([xshift=-0.5pt,yshift=-1pt]title.north-|frame.east)
--([xshift=-0.5pt]frame.south east)--(frame.south west); },
overlay first={\draw[colprop,line width=1pt]
([yshift=-1pt]title.north east)--([xshift=-0.5pt,yshift=-1pt]title.north-|frame.east)
--([xshift=-0.5pt]frame.south east); },
overlay middle={\draw[colprop,line width=1pt] ([xshift=-0.5pt]frame.north east)
--([xshift=-0.5pt]frame.south east); },
overlay last={\draw[colprop,line width=1pt] ([xshift=-0.5pt]frame.north east)
--([xshift=-0.5pt]frame.south east)--(frame.south west);},%
}

\colorlet{colexemple}{blue!50!black}
%\newtcolorbox{exemple}{empty, title=Exemple, attach boxed title to top left,
%  boxed title style={empty, size=minimal, toprule=2pt, top=4pt,
%    overlay={\draw[colexemple,line width=2pt]
%([yshift=-1pt]frame.north west)--([yshift=-1pt]frame.north east);}},
%coltitle=colexemple,fonttitle=\bfseries,%\large\bfseries,
%before=\par\medskip\noindent,parbox=false,boxsep=0pt,left=0pt,right=3mm,top=4pt,
%overlay={\draw[colexemple,line width=1pt]
%([yshift=-1pt]title.north east)--([xshift=-0.5pt,yshift=-1pt]title.north-|frame.east)
%--([xshift=-0.5pt]frame.south east)--(frame.south west); },
%}

\newcounter{exemples}
\setcounter{exemples}{1}

\newenvironment{exemple}
  {
    \begin{exemplehid}{\theexemples}
  }
  {
    \end{exemplehid}
    \addtocounter{exemples}{1}
  }

\newtcolorbox{exemplehid}[1]{%
empty,title={Exemple {#1}},attach boxed title to top left,
boxed title style={empty,size=minimal,toprule=2pt,top=4pt,
overlay={\draw[colexemple,line width=2pt]
([yshift=-1pt]frame.north west)--([yshift=-1pt]frame.north east);}},
coltitle=colexemple,fonttitle=\bfseries,
before=\par\medskip\noindent,parbox=false,boxsep=0pt,left=0pt,right=3mm,top=4pt,
breakable,pad at break*=0mm,vfill before first,
overlay unbroken={\draw[colexemple,line width=1pt]
([yshift=-1pt]title.north east)--([xshift=-0.5pt,yshift=-1pt]title.north-|frame.east)
--([xshift=-0.5pt]frame.south east)--(frame.south west); },
overlay first={\draw[colexemple,line width=1pt]
([yshift=-1pt]title.north east)--([xshift=-0.5pt,yshift=-1pt]title.north-|frame.east)
--([xshift=-0.5pt]frame.south east); },
overlay middle={\draw[colexemple,line width=1pt] ([xshift=-0.5pt]frame.north east)
--([xshift=-0.5pt]frame.south east); },
overlay last={\draw[colexemple,line width=1pt] ([xshift=-0.5pt]frame.north east)
--([xshift=-0.5pt]frame.south east)--(frame.south west);},%
}

\newenvironment{contrex}
  {
    \begin{contrexhid}{\theexemples}
  }
  {
    \end{contrexhid}
    \addtocounter{exemples}{1}
  }

\newtcolorbox{contrexhid}[1]{%
empty,title={Contre-exemple {#1}},attach boxed title to top left,
boxed title style={empty,size=minimal,toprule=2pt,top=4pt,
overlay={\draw[colexemple,line width=2pt]
([yshift=-1pt]frame.north west)--([yshift=-1pt]frame.north east);}},
coltitle=colexemple,fonttitle=\bfseries,
before=\par\medskip\noindent,parbox=false,boxsep=0pt,left=0pt,right=3mm,top=4pt,
breakable,pad at break*=0mm,vfill before first,
overlay unbroken={\draw[colexemple,line width=1pt]
([yshift=-1pt]title.north east)--([xshift=-0.5pt,yshift=-1pt]title.north-|frame.east)
--([xshift=-0.5pt]frame.south east)--(frame.south west); },
overlay first={\draw[colexemple,line width=1pt]
([yshift=-1pt]title.north east)--([xshift=-0.5pt,yshift=-1pt]title.north-|frame.east)
--([xshift=-0.5pt]frame.south east); },
overlay middle={\draw[colexemple,line width=1pt] ([xshift=-0.5pt]frame.north east)
--([xshift=-0.5pt]frame.south east); },
overlay last={\draw[colexemple,line width=1pt] ([xshift=-0.5pt]frame.north east)
--([xshift=-0.5pt]frame.south east)--(frame.south west);},%
}

\newenvironment{app}
  {
    \begin{apphid}{\theexemples}
  }
  {
    \end{apphid}
    \addtocounter{exemples}{1}
  }

\newtcolorbox{apphid}[1]{%
empty,title={Application {#1}},attach boxed title to top left,
boxed title style={empty,size=minimal,toprule=2pt,top=4pt,
overlay={\draw[colexemple,line width=2pt]
([yshift=-1pt]frame.north west)--([yshift=-1pt]frame.north east);}},
coltitle=colexemple,fonttitle=\bfseries,
before=\par\medskip\noindent,parbox=false,boxsep=0pt,left=0pt,right=3mm,top=4pt,
breakable,pad at break*=0mm,vfill before first,
overlay unbroken={\draw[colexemple,line width=1pt]
([yshift=-1pt]title.north east)--([xshift=-0.5pt,yshift=-1pt]title.north-|frame.east)
--([xshift=-0.5pt]frame.south east)--(frame.south west); },
overlay first={\draw[colexemple,line width=1pt]
([yshift=-1pt]title.north east)--([xshift=-0.5pt,yshift=-1pt]title.north-|frame.east)
--([xshift=-0.5pt]frame.south east); },
overlay middle={\draw[colexemple,line width=1pt] ([xshift=-0.5pt]frame.north east)
--([xshift=-0.5pt]frame.south east); },
overlay last={\draw[colexemple,line width=1pt] ([xshift=-0.5pt]frame.north east)
--([xshift=-0.5pt]frame.south east)--(frame.south west);},%
}

%%%%%%%%%%%%%%%%%%%%%%%%%%%%%%%%%%%%%%%%%%%%%%%%%%%%%%%%%%%%%%%%%%%%%%%%%%%%%%%%
%
% ENUMERATE
% =========
%
%%%%%%%%%%%%%%%%%%%%%%%%%%%%%%%%%%%%%%%%%%%%%%%%%%%%%%%%%%%%%%%%%%%%%%%%%%%%%%%%

\usepackage{enumerate}
\usepackage{enumitem}

% To have special enumerate items like
%
% 1/
% 2/
% 3/

%%%%%%%%%%%%%%%%%%%%%%%%%%%%%%%%%%%%%%%%%%%%%%%%%%%%%%%%%%%%%%%%%%%%%%%%%%%%%%%%
%
% ARRAYS
% ======
%
%%%%%%%%%%%%%%%%%%%%%%%%%%%%%%%%%%%%%%%%%%%%%%%%%%%%%%%%%%%%%%%%%%%%%%%%%%%%%%%%


\usepackage{array}
\usepackage{makecell} % Used to break lines within arrays
\usepackage{multirow}
\usepackage{booktabs} % Used to have nice arrays with headrules

%%%%%%%%%%%%%%%%%%%%%%%%%%%%%%%%%%%%%%%%%%%%%%%%%%%%%%%%%%%%%%%%%%%%%%%%%%%%%%%%
%
% WRITE CODE
% ==========
%
%%%%%%%%%%%%%%%%%%%%%%%%%%%%%%%%%%%%%%%%%%%%%%%%%%%%%%%%%%%%%%%%%%%%%%%%%%%%%%%%

\usepackage{listings}
\usepackage{xcolor}

%New colors defined below
\definecolor{codegreen}{rgb}{0,0.6,0}
\definecolor{codegray}{rgb}{0.5,0.5,0.5}
\definecolor{codepurple}{rgb}{0.58,0,0.82}
\definecolor{backcolour}{rgb}{0.95,0.95,0.92}

%Code listing style named "mystyle"
\lstdefinestyle{python}{
  %backgroundcolor=\color{backcolour},
  commentstyle=\color{codegreen},
  keywordstyle=\color{magenta},
  numberstyle=\tiny\color{codegray},
  stringstyle=\color{codepurple},
  basicstyle=\ttfamily\footnotesize,
  breakatwhitespace=false,
  breaklines=true,
  captionpos=b,
  keepspaces=true,
  numbers=left,
  numbersep=5pt,
  showspaces=false,
  showstringspaces=false,
  showtabs=false,
  tabsize=2
}

\lstset{style=python}

%%%%%%%%%%%%%%%%%%%%%%%%%%%%%%%%%%%%%%%%%%%%%%%%%%%%%%%%%%%%%%%%%%%%%%%%%%%%%%%%
%
% Tabular 
% =======
%
%%%%%%%%%%%%%%%%%%%%%%%%%%%%%%%%%%%%%%%%%%%%%%%%%%%%%%%%%%%%%%%%%%%%%%%%%%%%%%%%

% In order to obtain a tabular with given width.

\usepackage{tabularx}
\newcolumntype{Y}{>{\centering\arraybackslash}X}
\newcolumntype{R}{>{\raggedright\arraybackslash}X}
\newcolumntype{L}{>{\raggedleft\arraybackslash}X}
% \usepackage{tabulary} % younger brother

%%%%%%%%%%%%%%%%%%%%%%%%%%%%%%%%%%%%%%%%%%%%%%%%%%%%%%%%%%%%%%%%%%%%%%%%%%%%%%%%
%
% MACROS
% ======
%
%%%%%%%%%%%%%%%%%%%%%%%%%%%%%%%%%%%%%%%%%%%%%%%%%%%%%%%%%%%%%%%%%%%%%%%%%%%%%%%%

% Math Operators

\DeclareMathOperator{\Card}{Card}
\DeclareMathOperator{\Gal}{Gal}
\DeclareMathOperator{\Id}{Id}
\DeclareMathOperator{\Img}{Im}
\DeclareMathOperator{\Ker}{Ker}
\DeclareMathOperator{\Minpoly}{Minpoly}
\DeclareMathOperator{\Mod}{mod}
\DeclareMathOperator{\Ord}{Ord}
\DeclareMathOperator{\ppcm}{ppcm}
\DeclareMathOperator{\pgcd}{pgcd}
\DeclareMathOperator{\tr}{Tr}
\DeclareMathOperator{\Vect}{Vect}
\DeclareMathOperator{\Span}{Span}
\DeclareMathOperator{\rank}{rank}
\DeclareMathOperator{\rg}{rg}
\DeclareMathOperator{\ev}{ev}
\DeclareMathOperator{\Var}{Var}

% Shortcuts

\newcommand{\eg}{\emph{e.g. }}
\newcommand{\ent}[2]{[\![#1,#2]\!]}
\newcommand{\ie}{\emph{i.e. }}
\newcommand{\ps}[2]{\left\langle#1,#2\right\rangle}
\newcommand{\eqdef}{\overset{\text{def}}{=}}
\newcommand{\E}{\mathcal{E}}
\newcommand{\M}{\mathcal{M}}
\newcommand{\A}{\mathcal{A}}
\newcommand{\B}{\mathcal{B}}
\newcommand{\R}{\mathcal{R}}
\newcommand{\D}{\mathcal{D}}
\newcommand{\Pcal}{\mathcal{P}}
\newcommand{\K}{\mathbf{k}}
\newcommand{\vect}[1]{\overrightarrow{#1}}



\title{Chapitre 1 : second degré}
\date{}
\author{}

\begin{document}
\maketitle\thispagestyle{fancy}

\section{Fonctions polynômes du second degré}
\subsection{Forme développée et forme canonique}
\begin{defi}{Fonction polynôme du second degré}
  Une fonction $f$ définie sur $\mathbb{R}$ est appelée \textbf{fonction polynôme
  du second degré} s'il existe $a, b, c\in\mathbb{R}$ des réels avec $a\neq0$ et
  tels que, pour tout réel $x\in\mathbb{R}$, on ait
  \[
    f(x) = ax^2+bx+c.
  \]
\end{defi}

\begin{defi}{Forme développée et coefficients}
  Soit $f$ une fonction polynôme du second degré définie sur $\mathbb{R}$ par
  \[
    f(x) = ax^2+bx+c,
  \]
  où $a,b,c\in\mathbb{R}$ sont des réels et où $a\neq0$. Lorsque la fonction $f$
  est écrite sous cette forme, on parle de \textbf{forme développée}. Les réels
  $a$, $b$, et $c$ sont appelés les \textbf{coefficients} de $f$.
\end{defi}

\begin{exemple}
  Soit $f$ la fonction définie sur $\mathbb{R}$ par
  \[
    f(x) = x^2+2x+1.
  \]
  La fonction $f$ est une fonction polynôme du second degré, donnée sous forme
  développée. Ses coefficients $a$, $b$, et $c$ valent
  \begin{align*}
    a &= 1 &
    b &= 2 &
    c &= 1.
  \end{align*}
  On a par exemple $f(3) = 3^2+2\times3+1=9+6+1=16$.
\end{exemple}

\begin{app}
  Soit $f$ la fonction définie sur $\mathbb{R}$ par $f(x) = 3x^2-x+2$.
  \begin{enumerate}
    \item La fonction $f$ est-elle une fonction polynôme du second degré ?
    \item Si oui, donner les coefficients $a$, $b$, et $c$ de $f$.
    \item Calculer $f(5)$.
  \end{enumerate}
\end{app}

\begin{app}
  Soit $f$ la fonction définie sur $\mathbb{R}$ par $f(x) = (x-1)(x+1)$.
  \begin{enumerate}
    \item La fonction $f$ est-elle une fonction polynôme du second degré ?
    \item Si oui, donner les coefficients $a$, $b$, et $c$ de $f$.
    \item Calculer $f(-1)$.
  \end{enumerate}
\end{app}

\begin{prop}
  Soit $f$ une fonction polynôme du second degré définie sur $\mathbb{R}$ par
  \(
    f(x) = ax^2+bx+c.
  \)
  Alors $f$ peut s'écrire sous la forme
  \[
    f(x) = a(x-\alpha)^2+\beta,
  \]
  où on a $\alpha=-\cfrac{b}{2a}$ et $\beta=f(\alpha)$. Cette forme s'appelle la
  \textbf{forme canonique} de $f$.
\end{prop}

\begin{proof}[Démonstration à connaître.]
\end{proof}

\begin{exemple}
  Soit $f$ la fonction définie sur $\mathbb{R}$ par $f(x) = x^2 + 2x + 1$.
  La forme canonique de $f$ est 
  \[
    f(x) = (x+1)^2.
  \]
  En effet, on a dans ce cas $a=1$, $b=2$ et $c=1$, on obtient donc
  \begin{align*}
    \alpha &= \frac{-2}{2\times1} = -1 &
    \beta &= f(-1) = (-1+1)^2 = 0.
  \end{align*}
\end{exemple}

\begin{app}
  Donner la forme canonique de $f$, la fonction définie sur $\mathbb{R}$ par
  $f(x) = 2x^2-12x+10$.
\end{app}

\subsection{Variations et représentation graphique}
\begin{prop}
  Soit $f$ la fonction définie sur $\mathbb{R}$ par $f(x)=ax^2+bx+c$, de forme
  canonique $f(x)=a(x-\alpha)^2+\beta$.

  \noindent
  \begin{minipage}[t]{.47\textwidth}
    \begin{center}
      {\bf Si} $\mathbf{a>0}$\vspace{.2cm}

      \begin{tikzpicture}
        \tkzTabInit[lgt=1, espcl=1.5]{$x$ / .7, $f(x)$ / 1.4}{$-\infty$, $\alpha$, $+\infty$}
        \tkzTabVar{+/, -/$\beta$, +/}
      \end{tikzpicture}
    \end{center}
  La fonction $f$ est strictement décroissante sur $]-\infty, \alpha]$,
  strictement croissante sur $[\alpha, +\infty[$, et $f$ admet comme minimum
    $\beta$ en $\alpha$.
    \begin{center}
\begin{tikzpicture}
  \begin{axis}[x=.25cm, y=.25cm, xtick=\empty, ytick=\empty]
    \addplot[red, very thick, samples=201]{x^2-x-3};
  \end{axis}
\end{tikzpicture}
    \end{center}
  \end{minipage}
    \hfill
  \begin{minipage}[t]{.47\textwidth}
    \begin{center}
      {\bf Si} $\mathbf{a<0}$\vspace{.2cm}

      \begin{tikzpicture}
        \tkzTabInit[lgt=1, espcl=1.5]{$x$ / .7, $f(x)$ / 1.4}{$-\infty$, $\alpha$, $+\infty$}
        \tkzTabVar{-/, +/$\beta$, -/}
      \end{tikzpicture}
    \end{center}
  La fonction $f$ est strictement croissante sur $]-\infty, \alpha]$,
  strictement décroissante sur $[\alpha, +\infty[$, et $f$ admet comme maximum
    $\beta$ en $\alpha$.
    \begin{center}
\begin{tikzpicture}
  \begin{axis}[x=.25cm, y=.25cm, xtick=\empty, ytick=\empty]
    \addplot[red, very thick, samples=201]{-x^2-2*x+2};
  \end{axis}
\end{tikzpicture}
    \end{center}
  \end{minipage}
\end{prop}

\begin{defi}{Parabole}
  La courbe représentative d'une fonction polynôme du second degré est appelée
  une \textbf{parabole}.
\end{defi}

\begin{exemple}~\\[-4mm]
  \begin{minipage}[]{.65\textwidth}
  On reprend la fonction $f$ définie sur $\mathbb{R}$ par $f(x) = x^2+2x+1$.
  On a vu que sa forme canonique était $f(x) = (x+1)^2$, c'est-à-dire que $a=1$
  donc $a>0$, $\alpha=-1$ et $\beta=0$. On en déduit que
\begin{itemize}
\item la fonction est strictement décroissante sur $]-\infty, -1]$;
\item la fonction est strictement croissante sur $[-1; +\infty[$;
\item elle admet comme minimum $0$ en $-1$.
\end{itemize}
\end{minipage}
\begin{minipage}[]{.35\textwidth}
 \begin{center}
\begin{tikzpicture}
\begin{axis}
\addplot[red, very thick, samples=201]{x^2+2*x+1};
\end{axis}
\end{tikzpicture}
\end{center}
  \end{minipage}
\end{exemple}

\begin{app}
  Soit $f$ la fonction définie sur $\mathbb{R}$ par $f(x)=-x^2+4x+7$.
  \begin{enumerate}
    \item Donner la forme canonique de $f$.
    \item En déduire le tableau de variations de $f$.
  \end{enumerate}
\end{app}

\begin{prop}
  Soit $f$ une fonction polynôme du second degré définie sur $\mathbb{R}$ par sa
  forme canonique
  \[
    f(x) = a(x-\alpha)+\beta.
  \]
  Alors $f$ est représentée par une parabole dont le sommet a pour coordonnées
  $(\alpha, \beta)$.
\end{prop}
\begin{exemple}
  La fonction définie par $f(x)=(x+1)^2$ de l'exemple précédent admet une
  parabole dont le sommet est le point $(-1, 0)$.
\end{exemple}

\begin{app}
  Soit $f$ la fonction définie sur $\mathbb{R}$ par $f(x)=(x+2)^2-3$. On note
  $\mathscr C_f$ la courbe représentative de $f$. Donner le sommet de la
  parabole $\mathscr C_f$.
\end{app}

\section{Équations du second degré}

\subsection{Définitions} 
\begin{defi}{Équation du second degré}
  Une \textbf{équation du second degré} à coefficients réels est une équation de
  la forme
  \[
    ax^2+bx+c = 0,
  \]
  avec $a, b, c\in\mathbb{R}$ trois réels et $a\neq0$.
\end{defi}

\begin{defi}{Racines d'une équation}
  Les solutions de l'équation du second degré $ax^2+bx+c=0$ sont appelées les
  \textbf{racines} du trinôme $ax^2+bx+c$.
\end{defi}

\begin{exemple}
  L'équation $2x^2-x+3=0$ est une équation du second degré avec $a=2$, $b=-1$ et
  $c=3$.
\end{exemple}

\begin{app}
  Déterminer les racines de l'équation $x^2-2=0$.
\end{app}~\\[-13mm]
\begin{rmq}
  \begin{minipage}[]{.65\textwidth}
Les racines de l'équation du second degré
  \[
    ax^2+bx+c = 0
  \]
  correspondent aux abcisses des points où la courbe représentative de la
  fonction $f$ définie par $f(x)=ax^2+bx+c$ passe par l'axe des abcisses.
  Ci-contre l'exemple de $x^2-1=0$.
  \end{minipage}
  \begin{minipage}[]{.35\textwidth}
 \begin{center}
  \begin{tikzpicture}
    \begin{axis}
      \addplot[red, very thick, samples=201]{x^2-1};
    \end{axis}
    \filldraw[blue] (2.25,2.75) circle (1.5pt);
    \filldraw[blue] (3.25,2.75) circle (1.5pt);
  \end{tikzpicture}
\end{center}
  \end{minipage}
\end{rmq}

\begin{defi}{Racine d'une fonction polynôme}
  Soit $f$ la fonction polynôme définie sur $\mathbb{R}$ par $f(x) = ax^2+bx+c$.
  On dit que la valeur $x_0$ est une \textbf{racine} de $f$ si
  \[
    f(x_0) = 0.
  \]
\end{defi}

\begin{exemple}
  Soit $f$ la fonction definie sur $\mathbb{R}$ par $f(x)=x^2-4x+4$. On a $f(2)
  =2^2-4\times2+4=4-8+4=0$. Donc $2$ est une racine de $f$.
\end{exemple}

\begin{app}
  Soit $g$ la fonction définie sur $\mathbb{R}$ par $g(x)=x^2+6x+9$.
  \begin{enumerate}
    \item Calculer $g(-3)$.
    \item Donner une racine de $g$.
  \end{enumerate}
\end{app}

\subsection{Résolution des équations du second degré dans $\mathbb{R}$}

\noindent On va maintenant apprendre à résoudre dans $\mathbb{R}$ les équations du second
degré, c'est-à-dire à trouver des solutions réelles à nos équations.

\begin{defi}{Discriminant}
  Le \textbf{discriminant} du trinôme $ax^2+bx+c$, noté $\Delta$ (delta
  majuscule), est le nombre
  \[
    \Delta = b^2-4ac.
  \]
\end{defi}

\begin{prop}
  Soit $\Delta=b^2-4ac$ le discriminant du trinôme $ax^2+bx+c$.
  \begin{itemize}
    \item Si $\Delta<0$, alors l'équation $ax^2+bx+c=0$ n'a pas de solutions
      dans $\mathbb{R}$.
    \item Si $\Delta=0$, alors l'équation $ax^2+bx+c=0$ a une unique solution
      \[x_0 = \frac{-b}{2a}.\] On dit que $x_0$ est \textbf{racine double} du trinôme.
    \item Si $\Delta>0$, alors l'équation $ax^2+bx+c=0$ admet deux solutions
      distinctes : 
      \[
        x_1=\frac{-b-\sqrt\Delta}{2a}
        \text{ et }
        x_2=\frac{-b+\sqrt\Delta}{2a}.
      \]
  \end{itemize}
\end{prop}

\begin{proof}[Démonstration à connaître]
\end{proof}

\begin{exemple}
  Considérons l'équation du second degré $x^2+2x-3 = 0$.
  On a ici $a=1, b=2, c=-3$. On commence par calculer le discriminant de cette
  équation, on obtient $\Delta = 2^2 - 4\times 1\times(-3) = 16$.
  On est dans le cas où $\Delta>0$ et on sait qu'on a ainsi deux solutions
  distinctes:
  \begin{align*}
    x_1 &= \frac{-2+\sqrt{16}}{2\times1} = \frac{-2+4}{2} = 1 &
    &\text{ et } &
    x_2 &= \frac{-2-\sqrt{16}}{2\times1} = \frac{-2-4}{2} = -3.
  \end{align*}
 L'ensemble des solutions de l'équation est donc $\mathscr S =\left\{ -3; 1
 \right\}$
\end{exemple}

\begin{app}
  On considère l'équation $2x^2+3x-1=0$.
  \begin{enumerate}
    \item Calculer le discriminant $\Delta$ associé à cette équation.
    \item En déduire l'ensemble des solutions de l'équation.
  \end{enumerate}
\end{app}

\section{Propriétés d'un trinôme $ax^2+bx+c$}
\subsection{Factorisation}

\begin{propadm}
  Soit $f$ une fonction polynôme du second degré définie sur $\mathbb{R}$ par
  \[
    f(x) = ax^2+bx+c.
  \]
  \begin{itemize}
    \item Si $\Delta>0$, alors $f(x)=a(x-x_1)(x-x_2)$ où $x_1$ et $x_2$ sont les
      racines de $f$.
    \item Si $\Delta=0$, alors $f(x)=a(x-x_0)^2$ où $x_0$ est la racine double de
      $f$.
    \item Si $\Delta<0$, alors la fonction $f$ ne peut pas s'écrire comme un
      produit de deux polynômes de degré $1$.
  \end{itemize}
\end{propadm}

\begin{exemple}
  On a vu précédemment que la fonction $f$ définie par $f(x) = x^2+2x-3$
  avait pour discriminant $\Delta=16$ et pour racines $x_1=-3$ et $x_2=1$. Cela
  signifie que l'on peut aussi écrire $f$ sous la forme factorisée
  \[
    f(x) =(x+3)(x-1).
  \]
\end{exemple}
\begin{app}
  On considère l'expression $f(x)=x^2+x-6$.
  \begin{enumerate}
    \item Calculer le discriminant $\Delta$ associé à $f$.
    \item En déduire les racines de $f$.
    \item Écrire $f$ sous sa forme factorisée.
  \end{enumerate}
\end{app}

\subsection{Somme et produit de racines}
\begin{propadm}
  Soit $a,b,c\in\mathbb{R}$ et soit $f$ une fonction polynôme de degré $2$
  définie sur $\mathbb{R}$ par $f(x)=ax^2+bx+c$,
  dont le discriminant est strictement positif. La fonction $f$ a alors deux
  racines distinctes $x_1$ et $x_2$ et on a
  \begin{align*}
    x_1+x_2 &= \frac{-b}{a} &
    &\text{ et }&
    x_1\times x_2 &=\frac{c}{a}.
  \end{align*}
\end{propadm}

\subsection{Signe d'une fonction polynôme du second degré}
\begin{propadm}
  Soit $f$ une fonction du second degré définie sur $\mathbb{R}$ par
    $f(x)=ax^2+bx+c$,
  de discriminant $\Delta$.
  \begin{itemize}
    \item Si $\Delta<0$, alors pour tout réel $x\in\mathbb{R}$, $f(x)$ est du
      signe de $a$.
    \item Si $\Delta=0$, alors pour tout réel $x\neq\frac{-b}{2a}$, $f(x)$ est
      du signe de $a$, et $f(\frac{-b}{2a})=0$.
    \item Si $\Delta>0$, alors on a les tableaux de signe suivants.

      \noindent
  \begin{minipage}[t]{.47\textwidth}
    \begin{center}
      {\bf Si} $\mathbf{a>0}$\vspace{.2cm}

      \begin{tikzpicture}
        \tkzTabInit[lgt=1, espcl=1.5]{$x$ / .7, $f(x)$ / 1.4}{$-\infty$, $x_1$,
        $x_2$, $+\infty$}
        \tkzTabLine{, +, z , -, z, +,}
      \end{tikzpicture}
    \end{center}
  \vspace{.2cm}
  \end{minipage}
    \hfill
  \begin{minipage}[t]{.47\textwidth}
    \begin{center}
      {\bf Si} $\mathbf{a<0}$\vspace{.2cm}

      \begin{tikzpicture}
        \tkzTabInit[lgt=1, espcl=1.5]{$x$ / .7, $f(x)$ / 1.4}{$-\infty$, $x_1$,
        $x_2$, $+\infty$}
        \tkzTabLine{, -, z , +, z, -,}
      \end{tikzpicture}
    \end{center}
  \vspace{.2cm}
  \end{minipage}
  On peut retenir que dans ce cas, $f$ est du signe de $a$, sauf entre ses
  racines.
  \end{itemize}
\end{propadm}
\begin{exemple}
  On peut reprendre la fonction $f$ définie sur $\mathbb{R}$ par
    $f(x) = x^2+2x-3$.\\
  \begin{minipage}[]{.65\textwidth}
  On a vu que le déterminant de $f$ vaut $16$ et que les deux racines de $f$
  sont données par $x_1=-3$ et $x_2=1$. On a ainsi le tableau de signe suivant.
  \begin{center}
      \begin{tikzpicture}
        \tkzTabInit[lgt=1, espcl=1.5]{$x$ / .7, $f(x)$ / 1.4}{$-\infty$, $-3$,
        $1$, $+\infty$}
        \tkzTabLine{, +, z , -, z, +,}
      \end{tikzpicture}
  \end{center}
  Et, en effet, cela concorde avec la représentation graphique de $f$ donnée
  ci-dessous.
  \end{minipage}
  \begin{minipage}[]{.35\textwidth}
  \begin{center}
\begin{tikzpicture}
\begin{axis}
  \addplot[red, very thick, samples=201, domain=-5:-3]{x^2+2*x-3};
  \addplot[blue, very thick, samples=201, domain=-3:1]{x^2+2*x-3};
  \addplot[red, very thick, samples=201, domain=1:5]{x^2+2*x-3};
\end{axis}
\filldraw[black] (1.25, 2.75) circle (2pt);
\filldraw[black] (3.25, 2.75) circle (2pt);
\end{tikzpicture}
  \end{center}
  \end{minipage}
\end{exemple}
\end{document}
