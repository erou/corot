\documentclass[11pt]{article}

\newcommand{\titrechapitre}{Information chiffrée -- Cours}
\newcommand{\titreclasse}{Lycée Jean-Baptiste \textsc{Corot}}
\newcommand{\pagination}{\thepage/\pageref{LastPage}}
\newcommand{\topbotmargins}{2cm}
%%%%%%%%%%%%%%%%%%%%%%%%%%%%%%%%%%%%%%%%%%%%%%%%%%%%%%%%%%%%%%%%%%%%%%%%%%%%%%%%
%
% PACKAGES
% ========
%
%%%%%%%%%%%%%%%%%%%%%%%%%%%%%%%%%%%%%%%%%%%%%%%%%%%%%%%%%%%%%%%%%%%%%%%%%%%%%%%%

\usepackage[english, french]{babel}
\usepackage[utf8]{inputenc}
\usepackage[T1]{fontenc}
\usepackage{graphicx}
\usepackage{amsmath,amssymb,amsthm,amsopn}
\usepackage{hyperref}

% Pour avoir l'écriture \mathscr (math script)
% ============================================

\usepackage{mathrsfs}

% Deal with coma as a decimal separator
% =====================================

\usepackage{icomma}

% Package Geometry
% ================

\usepackage[a4paper, lmargin=2cm, rmargin=2cm, top=\topbotmargins, bottom=\topbotmargins]{geometry}

% Package multicol
% ================

\usepackage{multicol}

% Redefine abstract
% =================

% Note
% ====
%
% Le reste a été commenté pour ne pas charger trop de choses au démarrage. On
% verra si on en a besoin plus tard.
%
% --------
%
%\usepackage{mathrsfs}
%\usepackage{multirow}
%\usepackage{bm}
%\hypersetup{
%    colorlinks=true,
%    linkcolor=blue,
%    citecolor=red,
%}
%\usepackage{diagbox}
%
%\usepackage{algorithm}
%\usepackage{algpseudocode}
%
%\renewcommand{\algorithmicrequire}{\textbf{Input:}}
%\renewcommand{\algorithmicensure}{\textbf{Output:}}


%%%%%%%%%%%%%%%%%%%%%%%%%%%%%%%%%%%%%%%%%%%%%%%%%%%%%%%%%%%%%%%%%%%%%%%%%%%%%%%%
%
% TIKZ
% ====
%
%%%%%%%%%%%%%%%%%%%%%%%%%%%%%%%%%%%%%%%%%%%%%%%%%%%%%%%%%%%%%%%%%%%%%%%%%%%%%%%%

\usepackage{tikz}
\usetikzlibrary{arrows}

\usepackage{tkz-tab} % Variation tables

\usepackage{pgfplots}
%\usepackage{pgf-pie} % Pie charts

\pgfplotsset{
%\newcommand{\settingsgraph}{
x=.5cm,y=.5cm,
xticklabel style = {font=\scriptsize, yshift=.1cm},
yticklabel style = {font=\scriptsize, xshift=.1cm},
axis lines=middle,
ymajorgrids=true,
xmajorgrids=true,
major grid style = {color=white!80!blue},
xmin=-5.5,
xmax=5.5,
ymin=-5.5,
ymax=5.5,
xtick={-5.0,-4.0,...,5.0},
ytick={-5.0,-4.0,...,5.0},
}

% Tikz style

\tikzset{round/.style={circle, draw=black, very thick, scale = 0.7}}
\tikzset{arrow/.style={->, >=latex}}
\tikzset{dashed-arrow/.style={->, >=latex, dashed}}

\newcommand{\point}[3]{\draw[very thick, #3] (#1-.1, #2)--(#1+.1, #2)
(#1, #2-.1)--(#1, #2+.1)}

%%%%%%%%%%%%%%%%%%%%%%%%%%%%%%%%%%%%%%%%%%%%%%%%%%%%%%%%%%%%%%%%%%%%%%%%%%%%%%%%
%
% FANCY HEADER
% ============
%
%%%%%%%%%%%%%%%%%%%%%%%%%%%%%%%%%%%%%%%%%%%%%%%%%%%%%%%%%%%%%%%%%%%%%%%%%%%%%%%%


\usepackage{fancyhdr}
\usepackage{lastpage}

\pagestyle{fancy}
\newcommand{\changefont}{\fontsize{9}{9}\selectfont}
\renewcommand{\headrulewidth}{0mm}
\renewcommand{\footrulewidth}{0mm}

\fancyhead[C]{}
\fancyhead[L]{\titreclasse}
\fancyhead[R]{\titrechapitre}
\fancyfoot[C]{}
\fancyfoot[L]{}
\fancyfoot[R]{\pagination}
\addtolength{\skip\footins}{20pt} % distance between text and footnotes

%%%%%%%%%%%%%%%%%%%%%%%%%%%%%%%%%%%%%%%%%%%%%%%%%%%%%%%%%%%%%%%%%%%%%%%%%%%%%%%%
%
% THEOREM STYLE
% =============
%
%%%%%%%%%%%%%%%%%%%%%%%%%%%%%%%%%%%%%%%%%%%%%%%%%%%%%%%%%%%%%%%%%%%%%%%%%%%%%%%%

\usepackage[tikz]{bclogo}
\usepackage{mdframed}

\usepackage{tcolorbox}
\tcbuselibrary{listings, breakable, theorems, skins}

%\newtheoremstyle{break}%
%{}{}%
%{\itshape}{}%
%{\bfseries}{}%  % Note that final punctuation is omitted.
%{\newline}{}

\newtheoremstyle{scbf}%
{}{}%
{}{}%
%{\scshape}{}%  % Note that final punctuation is omitted.
{\bfseries\scshape}{}%  % Note that final punctuation is omitted.
{\newline}{}

%\theoremstyle{break}
%\theoremstyle{plain}
%\newtheorem{thm}{Theorem}[section]
%\newtheorem{lm}[thm]{Lemma}
%\newtheorem{prop}[thm]{Proposition}
%\newtheorem{cor}[thm]{Corollary}

%\theoremstyle{scbf}
%\newtheorem{exo}{$\star$ Exercice}

%\theoremstyle{definition}
%\newtheorem{defi}[thm]{Definition}
%\newtheorem{ex}[thm]{Example}

%\theoremstyle{remark}
%\newtheorem{rem}[thm]{Remark}

% Defining the Remark environment
% ===============================

\newenvironment{rmq}
  {
    \begin{bclogo}[logo=\bcinfo, noborder=true]{Remarque}
  }
  {
    \end{bclogo}
  }

% Defining the exercise environment
% =================================

\newcounter{exos}
\setcounter{exos}{1}

\newenvironment{exo}
  {
    \begin{bclogo}[logo=\bccrayon, noborder=true]{Exercice \theexos}
  }
  {
    \end{bclogo}
    \addtocounter{exos}{1}
  }


% Redefining the proof environment from amsthm
% ============================================

\tcolorboxenvironment{proof}{
  blanker, breakable, before skip=10pt,after skip=10pt,
  borderline west={1mm}{0pt}{red},
  left=5mm,
}

% Defining the definition environment
% ===================================

\colorlet{coldef}{black!50!green}

\newcounter{defis}
\setcounter{defis}{1}

\newenvironment{defi}[1]
  {
    \begin{defihid}{{#1}}{\thedefis}
  }
  {
    \end{defihid}
    \addtocounter{defis}{1}
  }

\newtcolorbox{defihid}[2]{%
  empty,title={ {\bfseries Définition {#2}} ({#1})},attach boxed title to top left,
boxed title style={empty,size=minimal,toprule=2pt,top=4pt,
overlay={\draw[coldef,line width=2pt]
([yshift=-1pt]frame.north west)--([yshift=-1pt]frame.north east);}},
coltitle=coldef,
before=\par\medskip\noindent,parbox=false,boxsep=0pt,left=0pt,right=3mm,top=4pt,
breakable,pad at break*=0mm,vfill before first,
overlay unbroken={\draw[coldef,line width=1pt]
([yshift=-1pt]title.north east)--([xshift=-0.5pt,yshift=-1pt]title.north-|frame.east)
--([xshift=-0.5pt]frame.south east)--(frame.south west); },
overlay first={\draw[coldef,line width=1pt]
([yshift=-1pt]title.north east)--([xshift=-0.5pt,yshift=-1pt]title.north-|frame.east)
--([xshift=-0.5pt]frame.south east); },
overlay middle={\draw[coldef,line width=1pt] ([xshift=-0.5pt]frame.north east)
--([xshift=-0.5pt]frame.south east); },
overlay last={\draw[coldef,line width=1pt] ([xshift=-0.5pt]frame.north east)
--([xshift=-0.5pt]frame.south east)--(frame.south west);},%
}

\newenvironment{notation}
  {
    \begin{notationhid}{\thedefis}
  }
  {
    \end{notationhid}
    \addtocounter{defis}{1}
  }

\newtcolorbox{notationhid}[1]{%
  empty,title={Notation {#1}},attach boxed title to top left,
boxed title style={empty,size=minimal,toprule=2pt,top=4pt,
overlay={\draw[coldef,line width=2pt]
([yshift=-1pt]frame.north west)--([yshift=-1pt]frame.north east);}},
coltitle=coldef,fonttitle=\bfseries,
before=\par\medskip\noindent,parbox=false,boxsep=0pt,left=0pt,right=3mm,top=4pt,
breakable,pad at break*=0mm,vfill before first,
overlay unbroken={\draw[coldef,line width=1pt]
([yshift=-1pt]title.north east)--([xshift=-0.5pt,yshift=-1pt]title.north-|frame.east)
--([xshift=-0.5pt]frame.south east)--(frame.south west); },
overlay first={\draw[coldef,line width=1pt]
([yshift=-1pt]title.north east)--([xshift=-0.5pt,yshift=-1pt]title.north-|frame.east)
--([xshift=-0.5pt]frame.south east); },
overlay middle={\draw[coldef,line width=1pt] ([xshift=-0.5pt]frame.north east)
--([xshift=-0.5pt]frame.south east); },
overlay last={\draw[coldef,line width=1pt] ([xshift=-0.5pt]frame.north east)
--([xshift=-0.5pt]frame.south east)--(frame.south west);},%
}


% Defining the proposition, theorem, etc. environment
% ===================================================

\colorlet{colprop}{red!75!black}

\newcounter{props}
\setcounter{props}{1}

\newenvironment{prop}
  {
    \begin{prophid}{\theprops}
  }
  {
    \end{prophid}
    \refstepcounter{props}
  }

\newtcolorbox{prophid}[1]{%
empty,title={Propriété {#1}},attach boxed title to top left,
boxed title style={empty,size=minimal,toprule=2pt,top=4pt,
overlay={\draw[colprop,line width=2pt]
([yshift=-1pt]frame.north west)--([yshift=-1pt]frame.north east);}},
coltitle=colprop,fonttitle=\bfseries,
before=\par\medskip\noindent,parbox=false,boxsep=0pt,left=0pt,right=3mm,top=4pt,
breakable,pad at break*=0mm,vfill before first,
overlay unbroken={\draw[colprop,line width=1pt]
([yshift=-1pt]title.north east)--([xshift=-0.5pt,yshift=-1pt]title.north-|frame.east)
--([xshift=-0.5pt]frame.south east)--(frame.south west); },
overlay first={\draw[colprop,line width=1pt]
([yshift=-1pt]title.north east)--([xshift=-0.5pt,yshift=-1pt]title.north-|frame.east)
--([xshift=-0.5pt]frame.south east); },
overlay middle={\draw[colprop,line width=1pt] ([xshift=-0.5pt]frame.north east)
--([xshift=-0.5pt]frame.south east); },
overlay last={\draw[colprop,line width=1pt] ([xshift=-0.5pt]frame.north east)
--([xshift=-0.5pt]frame.south east)--(frame.south west);},%
}

\newenvironment{propadm}
  {
    \begin{propadmhid}{\theprops}
  }
  {
    \end{propadmhid}
    \refstepcounter{props}
  }

  \newtcolorbox{propadmhid}[1]{%
    empty,title={{\bfseries Propriété {#1}} (admise)},attach boxed title to top left,
boxed title style={empty,size=minimal,toprule=2pt,top=4pt,
overlay={\draw[colprop,line width=2pt]
([yshift=-1pt]frame.north west)--([yshift=-1pt]frame.north east);}},
coltitle=colprop,%fonttitle=\bfseries,
before=\par\medskip\noindent,parbox=false,boxsep=0pt,left=0pt,right=3mm,top=4pt,
breakable,pad at break*=0mm,vfill before first,
overlay unbroken={\draw[colprop,line width=1pt]
([yshift=-1pt]title.north east)--([xshift=-0.5pt,yshift=-1pt]title.north-|frame.east)
--([xshift=-0.5pt]frame.south east)--(frame.south west); },
overlay first={\draw[colprop,line width=1pt]
([yshift=-1pt]title.north east)--([xshift=-0.5pt,yshift=-1pt]title.north-|frame.east)
--([xshift=-0.5pt]frame.south east); },
overlay middle={\draw[colprop,line width=1pt] ([xshift=-0.5pt]frame.north east)
--([xshift=-0.5pt]frame.south east); },
overlay last={\draw[colprop,line width=1pt] ([xshift=-0.5pt]frame.north east)
--([xshift=-0.5pt]frame.south east)--(frame.south west);},%
}

\newenvironment{propnom}[1]
  {
    \begin{propnomhid}{#1}{\theprops}
  }
  {
    \end{propnomhid}
    \refstepcounter{props}
  }

\newtcolorbox{propnomhid}[2]{%
empty,title={{\bfseries Propriété {#2}} ({#1})},attach boxed title to top left,
boxed title style={empty,size=minimal,toprule=2pt,top=4pt,
overlay={\draw[colprop,line width=2pt]
([yshift=-1pt]frame.north west)--([yshift=-1pt]frame.north east);}},
coltitle=colprop,
before=\par\medskip\noindent,parbox=false,boxsep=0pt,left=0pt,right=3mm,top=4pt,
breakable,pad at break*=0mm,vfill before first,
overlay unbroken={\draw[colprop,line width=1pt]
([yshift=-1pt]title.north east)--([xshift=-0.5pt,yshift=-1pt]title.north-|frame.east)
--([xshift=-0.5pt]frame.south east)--(frame.south west); },
overlay first={\draw[colprop,line width=1pt]
([yshift=-1pt]title.north east)--([xshift=-0.5pt,yshift=-1pt]title.north-|frame.east)
--([xshift=-0.5pt]frame.south east); },
overlay middle={\draw[colprop,line width=1pt] ([xshift=-0.5pt]frame.north east)
--([xshift=-0.5pt]frame.south east); },
overlay last={\draw[colprop,line width=1pt] ([xshift=-0.5pt]frame.north east)
--([xshift=-0.5pt]frame.south east)--(frame.south west);},%
}




\newenvironment{thm}
  {
    \begin{thmhid}{\theprops}
  }
  {
    \end{thmhid}
    \refstepcounter{props}
  }

\newtcolorbox{thmhid}[1]{%
empty,title={Théorème {#1}},attach boxed title to top left,
boxed title style={empty,size=minimal,toprule=2pt,top=4pt,
overlay={\draw[colprop,line width=2pt]
([yshift=-1pt]frame.north west)--([yshift=-1pt]frame.north east);}},
coltitle=colprop,fonttitle=\bfseries,
before=\par\medskip\noindent,parbox=false,boxsep=0pt,left=0pt,right=3mm,top=4pt,
breakable,pad at break*=0mm,vfill before first,
overlay unbroken={\draw[colprop,line width=1pt]
([yshift=-1pt]title.north east)--([xshift=-0.5pt,yshift=-1pt]title.north-|frame.east)
--([xshift=-0.5pt]frame.south east)--(frame.south west); },
overlay first={\draw[colprop,line width=1pt]
([yshift=-1pt]title.north east)--([xshift=-0.5pt,yshift=-1pt]title.north-|frame.east)
--([xshift=-0.5pt]frame.south east); },
overlay middle={\draw[colprop,line width=1pt] ([xshift=-0.5pt]frame.north east)
--([xshift=-0.5pt]frame.south east); },
overlay last={\draw[colprop,line width=1pt] ([xshift=-0.5pt]frame.north east)
--([xshift=-0.5pt]frame.south east)--(frame.south west);},%
}

\newenvironment{thmadm}
  {
    \begin{thmadmhid}{\theprops}
  }
  {
    \end{thmadmhid}
    \refstepcounter{props}
  }

  \newtcolorbox{thmadmhid}[1]{%
    empty,title={{\bfseries Théorème {#1}} (admis)},attach boxed title to top left,
boxed title style={empty,size=minimal,toprule=2pt,top=4pt,
overlay={\draw[colprop,line width=2pt]
([yshift=-1pt]frame.north west)--([yshift=-1pt]frame.north east);}},
coltitle=colprop,%fonttitle=\bfseries,
before=\par\medskip\noindent,parbox=false,boxsep=0pt,left=0pt,right=3mm,top=4pt,
breakable,pad at break*=0mm,vfill before first,
overlay unbroken={\draw[colprop,line width=1pt]
([yshift=-1pt]title.north east)--([xshift=-0.5pt,yshift=-1pt]title.north-|frame.east)
--([xshift=-0.5pt]frame.south east)--(frame.south west); },
overlay first={\draw[colprop,line width=1pt]
([yshift=-1pt]title.north east)--([xshift=-0.5pt,yshift=-1pt]title.north-|frame.east)
--([xshift=-0.5pt]frame.south east); },
overlay middle={\draw[colprop,line width=1pt] ([xshift=-0.5pt]frame.north east)
--([xshift=-0.5pt]frame.south east); },
overlay last={\draw[colprop,line width=1pt] ([xshift=-0.5pt]frame.north east)
--([xshift=-0.5pt]frame.south east)--(frame.south west);},%
}

\newenvironment{thmnom}[1]
  {
    \begin{thmnomhid}{#1}{\theprops}
  }
  {
    \end{thmnomhid}
    \refstepcounter{props}
  }

\newtcolorbox{thmnomhid}[2]{%
empty,title={{\bfseries Théorème {#2}} ({#1})},attach boxed title to top left,
boxed title style={empty,size=minimal,toprule=2pt,top=4pt,
overlay={\draw[colprop,line width=2pt]
([yshift=-1pt]frame.north west)--([yshift=-1pt]frame.north east);}},
coltitle=colprop,
before=\par\medskip\noindent,parbox=false,boxsep=0pt,left=0pt,right=3mm,top=4pt,
breakable,pad at break*=0mm,vfill before first,
overlay unbroken={\draw[colprop,line width=1pt]
([yshift=-1pt]title.north east)--([xshift=-0.5pt,yshift=-1pt]title.north-|frame.east)
--([xshift=-0.5pt]frame.south east)--(frame.south west); },
overlay first={\draw[colprop,line width=1pt]
([yshift=-1pt]title.north east)--([xshift=-0.5pt,yshift=-1pt]title.north-|frame.east)
--([xshift=-0.5pt]frame.south east); },
overlay middle={\draw[colprop,line width=1pt] ([xshift=-0.5pt]frame.north east)
--([xshift=-0.5pt]frame.south east); },
overlay last={\draw[colprop,line width=1pt] ([xshift=-0.5pt]frame.north east)
--([xshift=-0.5pt]frame.south east)--(frame.south west);},%
}

\newenvironment{coro}
  {
    \begin{corohid}{\theprops}
  }
  {
    \end{corohid}
    \refstepcounter{props}
  }

  \newtcolorbox{corohid}[1]{%
  empty,title={Corollaire {#1}},attach boxed title to top left,
boxed title style={empty,size=minimal,toprule=2pt,top=4pt,
overlay={\draw[colprop,line width=2pt]
([yshift=-1pt]frame.north west)--([yshift=-1pt]frame.north east);}},
coltitle=colprop,fonttitle=\bfseries,
before=\par\medskip\noindent,parbox=false,boxsep=0pt,left=0pt,right=3mm,top=4pt,
breakable,pad at break*=0mm,vfill before first,
overlay unbroken={\draw[colprop,line width=1pt]
([yshift=-1pt]title.north east)--([xshift=-0.5pt,yshift=-1pt]title.north-|frame.east)
--([xshift=-0.5pt]frame.south east)--(frame.south west); },
overlay first={\draw[colprop,line width=1pt]
([yshift=-1pt]title.north east)--([xshift=-0.5pt,yshift=-1pt]title.north-|frame.east)
--([xshift=-0.5pt]frame.south east); },
overlay middle={\draw[colprop,line width=1pt] ([xshift=-0.5pt]frame.north east)
--([xshift=-0.5pt]frame.south east); },
overlay last={\draw[colprop,line width=1pt] ([xshift=-0.5pt]frame.north east)
--([xshift=-0.5pt]frame.south east)--(frame.south west);},%
}

\newenvironment{lemme}
  {
    \begin{lemmehid}{\theprops}
  }
  {
    \end{lemmehid}
    \refstepcounter{props}
  }

  \newtcolorbox{lemmehid}[1]{%
  empty,title={Lemme {#1}},attach boxed title to top left,
boxed title style={empty,size=minimal,toprule=2pt,top=4pt,
overlay={\draw[colprop,line width=2pt]
([yshift=-1pt]frame.north west)--([yshift=-1pt]frame.north east);}},
coltitle=colprop,fonttitle=\bfseries,
before=\par\medskip\noindent,parbox=false,boxsep=0pt,left=0pt,right=3mm,top=4pt,
breakable,pad at break*=0mm,vfill before first,
overlay unbroken={\draw[colprop,line width=1pt]
([yshift=-1pt]title.north east)--([xshift=-0.5pt,yshift=-1pt]title.north-|frame.east)
--([xshift=-0.5pt]frame.south east)--(frame.south west); },
overlay first={\draw[colprop,line width=1pt]
([yshift=-1pt]title.north east)--([xshift=-0.5pt,yshift=-1pt]title.north-|frame.east)
--([xshift=-0.5pt]frame.south east); },
overlay middle={\draw[colprop,line width=1pt] ([xshift=-0.5pt]frame.north east)
--([xshift=-0.5pt]frame.south east); },
overlay last={\draw[colprop,line width=1pt] ([xshift=-0.5pt]frame.north east)
--([xshift=-0.5pt]frame.south east)--(frame.south west);},%
}

\colorlet{colexemple}{blue!50!black}
%\newtcolorbox{exemple}{empty, title=Exemple, attach boxed title to top left,
%  boxed title style={empty, size=minimal, toprule=2pt, top=4pt,
%    overlay={\draw[colexemple,line width=2pt]
%([yshift=-1pt]frame.north west)--([yshift=-1pt]frame.north east);}},
%coltitle=colexemple,fonttitle=\bfseries,%\large\bfseries,
%before=\par\medskip\noindent,parbox=false,boxsep=0pt,left=0pt,right=3mm,top=4pt,
%overlay={\draw[colexemple,line width=1pt]
%([yshift=-1pt]title.north east)--([xshift=-0.5pt,yshift=-1pt]title.north-|frame.east)
%--([xshift=-0.5pt]frame.south east)--(frame.south west); },
%}

\newcounter{exemples}
\setcounter{exemples}{1}

\newenvironment{exemple}
  {
    \begin{exemplehid}{\theexemples}
  }
  {
    \end{exemplehid}
    \addtocounter{exemples}{1}
  }

\newtcolorbox{exemplehid}[1]{%
empty,title={Exemple {#1}},attach boxed title to top left,
boxed title style={empty,size=minimal,toprule=2pt,top=4pt,
overlay={\draw[colexemple,line width=2pt]
([yshift=-1pt]frame.north west)--([yshift=-1pt]frame.north east);}},
coltitle=colexemple,fonttitle=\bfseries,
before=\par\medskip\noindent,parbox=false,boxsep=0pt,left=0pt,right=3mm,top=4pt,
breakable,pad at break*=0mm,vfill before first,
overlay unbroken={\draw[colexemple,line width=1pt]
([yshift=-1pt]title.north east)--([xshift=-0.5pt,yshift=-1pt]title.north-|frame.east)
--([xshift=-0.5pt]frame.south east)--(frame.south west); },
overlay first={\draw[colexemple,line width=1pt]
([yshift=-1pt]title.north east)--([xshift=-0.5pt,yshift=-1pt]title.north-|frame.east)
--([xshift=-0.5pt]frame.south east); },
overlay middle={\draw[colexemple,line width=1pt] ([xshift=-0.5pt]frame.north east)
--([xshift=-0.5pt]frame.south east); },
overlay last={\draw[colexemple,line width=1pt] ([xshift=-0.5pt]frame.north east)
--([xshift=-0.5pt]frame.south east)--(frame.south west);},%
}

\newenvironment{contrex}
  {
    \begin{contrexhid}{\theexemples}
  }
  {
    \end{contrexhid}
    \addtocounter{exemples}{1}
  }

\newtcolorbox{contrexhid}[1]{%
empty,title={Contre-exemple {#1}},attach boxed title to top left,
boxed title style={empty,size=minimal,toprule=2pt,top=4pt,
overlay={\draw[colexemple,line width=2pt]
([yshift=-1pt]frame.north west)--([yshift=-1pt]frame.north east);}},
coltitle=colexemple,fonttitle=\bfseries,
before=\par\medskip\noindent,parbox=false,boxsep=0pt,left=0pt,right=3mm,top=4pt,
breakable,pad at break*=0mm,vfill before first,
overlay unbroken={\draw[colexemple,line width=1pt]
([yshift=-1pt]title.north east)--([xshift=-0.5pt,yshift=-1pt]title.north-|frame.east)
--([xshift=-0.5pt]frame.south east)--(frame.south west); },
overlay first={\draw[colexemple,line width=1pt]
([yshift=-1pt]title.north east)--([xshift=-0.5pt,yshift=-1pt]title.north-|frame.east)
--([xshift=-0.5pt]frame.south east); },
overlay middle={\draw[colexemple,line width=1pt] ([xshift=-0.5pt]frame.north east)
--([xshift=-0.5pt]frame.south east); },
overlay last={\draw[colexemple,line width=1pt] ([xshift=-0.5pt]frame.north east)
--([xshift=-0.5pt]frame.south east)--(frame.south west);},%
}

\newenvironment{app}
  {
    \begin{apphid}{\theexemples}
  }
  {
    \end{apphid}
    \addtocounter{exemples}{1}
  }

\newtcolorbox{apphid}[1]{%
empty,title={Application {#1}},attach boxed title to top left,
boxed title style={empty,size=minimal,toprule=2pt,top=4pt,
overlay={\draw[colexemple,line width=2pt]
([yshift=-1pt]frame.north west)--([yshift=-1pt]frame.north east);}},
coltitle=colexemple,fonttitle=\bfseries,
before=\par\medskip\noindent,parbox=false,boxsep=0pt,left=0pt,right=3mm,top=4pt,
breakable,pad at break*=0mm,vfill before first,
overlay unbroken={\draw[colexemple,line width=1pt]
([yshift=-1pt]title.north east)--([xshift=-0.5pt,yshift=-1pt]title.north-|frame.east)
--([xshift=-0.5pt]frame.south east)--(frame.south west); },
overlay first={\draw[colexemple,line width=1pt]
([yshift=-1pt]title.north east)--([xshift=-0.5pt,yshift=-1pt]title.north-|frame.east)
--([xshift=-0.5pt]frame.south east); },
overlay middle={\draw[colexemple,line width=1pt] ([xshift=-0.5pt]frame.north east)
--([xshift=-0.5pt]frame.south east); },
overlay last={\draw[colexemple,line width=1pt] ([xshift=-0.5pt]frame.north east)
--([xshift=-0.5pt]frame.south east)--(frame.south west);},%
}

%%%%%%%%%%%%%%%%%%%%%%%%%%%%%%%%%%%%%%%%%%%%%%%%%%%%%%%%%%%%%%%%%%%%%%%%%%%%%%%%
%
% ENUMERATE
% =========
%
%%%%%%%%%%%%%%%%%%%%%%%%%%%%%%%%%%%%%%%%%%%%%%%%%%%%%%%%%%%%%%%%%%%%%%%%%%%%%%%%

\usepackage{enumerate}
\usepackage{enumitem}

% To have special enumerate items like
%
% 1/
% 2/
% 3/

%%%%%%%%%%%%%%%%%%%%%%%%%%%%%%%%%%%%%%%%%%%%%%%%%%%%%%%%%%%%%%%%%%%%%%%%%%%%%%%%
%
% ARRAYS
% ======
%
%%%%%%%%%%%%%%%%%%%%%%%%%%%%%%%%%%%%%%%%%%%%%%%%%%%%%%%%%%%%%%%%%%%%%%%%%%%%%%%%


\usepackage{array}
\usepackage{makecell} % Used to break lines within arrays
\usepackage{multirow}
\usepackage{booktabs} % Used to have nice arrays with headrules

%%%%%%%%%%%%%%%%%%%%%%%%%%%%%%%%%%%%%%%%%%%%%%%%%%%%%%%%%%%%%%%%%%%%%%%%%%%%%%%%
%
% WRITE CODE
% ==========
%
%%%%%%%%%%%%%%%%%%%%%%%%%%%%%%%%%%%%%%%%%%%%%%%%%%%%%%%%%%%%%%%%%%%%%%%%%%%%%%%%

\usepackage{listings}
\usepackage{xcolor}

%New colors defined below
\definecolor{codegreen}{rgb}{0,0.6,0}
\definecolor{codegray}{rgb}{0.5,0.5,0.5}
\definecolor{codepurple}{rgb}{0.58,0,0.82}
\definecolor{backcolour}{rgb}{0.95,0.95,0.92}

%Code listing style named "mystyle"
\lstdefinestyle{python}{
  %backgroundcolor=\color{backcolour},
  commentstyle=\color{codegreen},
  keywordstyle=\color{magenta},
  numberstyle=\tiny\color{codegray},
  stringstyle=\color{codepurple},
  basicstyle=\ttfamily\footnotesize,
  breakatwhitespace=false,
  breaklines=true,
  captionpos=b,
  keepspaces=true,
  numbers=left,
  numbersep=5pt,
  showspaces=false,
  showstringspaces=false,
  showtabs=false,
  tabsize=2
}

\lstset{style=python}

%%%%%%%%%%%%%%%%%%%%%%%%%%%%%%%%%%%%%%%%%%%%%%%%%%%%%%%%%%%%%%%%%%%%%%%%%%%%%%%%
%
% Tabular 
% =======
%
%%%%%%%%%%%%%%%%%%%%%%%%%%%%%%%%%%%%%%%%%%%%%%%%%%%%%%%%%%%%%%%%%%%%%%%%%%%%%%%%

% In order to obtain a tabular with given width.

\usepackage{tabularx}
\newcolumntype{Y}{>{\centering\arraybackslash}X}
\newcolumntype{R}{>{\raggedright\arraybackslash}X}
\newcolumntype{L}{>{\raggedleft\arraybackslash}X}
% \usepackage{tabulary} % younger brother

%%%%%%%%%%%%%%%%%%%%%%%%%%%%%%%%%%%%%%%%%%%%%%%%%%%%%%%%%%%%%%%%%%%%%%%%%%%%%%%%
%
% MACROS
% ======
%
%%%%%%%%%%%%%%%%%%%%%%%%%%%%%%%%%%%%%%%%%%%%%%%%%%%%%%%%%%%%%%%%%%%%%%%%%%%%%%%%

% Math Operators

\DeclareMathOperator{\Card}{Card}
\DeclareMathOperator{\Gal}{Gal}
\DeclareMathOperator{\Id}{Id}
\DeclareMathOperator{\Img}{Im}
\DeclareMathOperator{\Ker}{Ker}
\DeclareMathOperator{\Minpoly}{Minpoly}
\DeclareMathOperator{\Mod}{mod}
\DeclareMathOperator{\Ord}{Ord}
\DeclareMathOperator{\ppcm}{ppcm}
\DeclareMathOperator{\pgcd}{pgcd}
\DeclareMathOperator{\tr}{Tr}
\DeclareMathOperator{\Vect}{Vect}
\DeclareMathOperator{\Span}{Span}
\DeclareMathOperator{\rank}{rank}
\DeclareMathOperator{\rg}{rg}
\DeclareMathOperator{\ev}{ev}
\DeclareMathOperator{\Var}{Var}

% Shortcuts

\newcommand{\eg}{\emph{e.g. }}
\newcommand{\ent}[2]{[\![#1,#2]\!]}
\newcommand{\ie}{\emph{i.e. }}
\newcommand{\ps}[2]{\left\langle#1,#2\right\rangle}
\newcommand{\eqdef}{\overset{\text{def}}{=}}
\newcommand{\E}{\mathcal{E}}
\newcommand{\M}{\mathcal{M}}
\newcommand{\A}{\mathcal{A}}
\newcommand{\B}{\mathcal{B}}
\newcommand{\R}{\mathcal{R}}
\newcommand{\D}{\mathcal{D}}
\newcommand{\Pcal}{\mathcal{P}}
\newcommand{\K}{\mathbf{k}}
\newcommand{\vect}[1]{\overrightarrow{#1}}


%\input{layout-nb.tex}

\title{Chapitre 6 : Information chiffrée}
\date{}
\author{}

\begin{document}
\maketitle\thispagestyle{fancy}

\section{Proportion et pourcentage}
\subsection{Calculer un pourcentage}
\begin{prop}
  Soit $t>0$ un nombre positif. Prendre $t\%$ d'une quantité, c'est la
  multiplier par
  \(
    \frac{t}{100}.
  \)
\end{prop}
\begin{exemple}
  Prendre $4\%$ de $120$ euros correspond à
  \[
    \frac{4}{100}\times 120 = 4,8\text{ euros.}
  \]
\end{exemple}
\begin{app}
  \begin{enumerate}
    \item Prendre $18\%$ de $500$.
    \item Calculer $33\%$ de $921$.
  \end{enumerate}
\end{app}

\subsection{Exprimer une proportion}
\begin{defi}{Population et sous-population}
  Soient $E$ un ensemble de référence non vide et $n_E$ le nombre d'éléments de
  $E$. Soient $A$ une partie de l'ensemble $E$ et $n_A$ le nombre d'éléments de
  $A$.
  \begin{itemize}
    \item L'ensemble $E$ est appelé la \textbf{population}, les éléments de $E$
      sont appelés les \textbf{individus}, et le nombre d'individus $n_E$ est
      appelé l'\textbf{effectif} de $E$.
    \item L'ensemble $A$ est appelé \textbf{sous-population}, et $n_A$ est
      appelé l'effectif de $A$.
  \end{itemize}
\end{defi}
\begin{exemple}
  On considère une classe de $35$ élèves, dont $20$ sont des filles. On pose
  alors $E$ l'ensemble des élèves de la classe et $A$ l'ensemble des filles de
  la classe.
  \begin{itemize}
    \item L'ensemble $E$ est la population, son effectif est $n_E=35$.
    \item L'ensemble $A$ est la sous-population, son effectif est $n_A=20$.
  \end{itemize}
\end{exemple}

\begin{defi}{Proportion}
  Soit $E$ une population et $A$ une sous-population. La \textbf{proportion} $p$
  de $A$ dans $E$ est le réel défini par
      \[
        p = \frac{n_A}{n_E}.
      \]
\end{defi}
\begin{exemple}
  Lors d'une élection, sur $864$ inscrits, $648$ personnes ont voté. La
  proportion des votants est $p=\frac{648}{864}=0,75.$ Il y a donc $75\%$ de
  votants.
\end{exemple}
\begin{rmq}
  Une proportion peut être exprimée sous forme décimale, sous forme de fraction,
  ou de pourcentage
  \[
    0,3 = \frac{30}{100} = 30\%.
  \]
\end{rmq}
\begin{app}
  Au sein du lycée Toroc, il y a $2400$ élèves, dont $1800$ sont
  demi-pensionnaires.
  \begin{enumerate}
    \item Donner, sous forme décimale, la proportion d'élèves demi-pensionnaires
      dans le lycée Toroc.
    \item Donner cette proportion sous forme fractionnaire et sous forme de
      pourcentage.
  \end{enumerate}
\end{app}

\begin{prop}
  Connaissant la proportion $p$ de $A$ dans $E$, on peut retrouver l'effectif
  manquant $n_A$ ou $n_E$ :
  \[
    n_A = p\times n_E\text{ et, pour }p\neq0,\;n_E=\frac{n_A}{p}.
  \]
\end{prop}
\begin{prop}
  Pour tout ensemble $A$ contenu dans un ensemble non vide $E$, on a $0\leq
  p\leq1$.
\end{prop}
\begin{exemple}
  Sachant que $n_A=200$ et que $p=0,25$, on retrouve l'effectif $n_E$ de la
  population avec le calcul
  \[
    n_E = \frac{n_A}{p} = \frac{200}{0,25}=800.
  \]
\end{exemple}
\begin{app}
  On s'intéresse à la composition d'une tablette de chocolat de $180$ g.
  \begin{enumerate}
    \item Elle comporte $72$ g de sucre : quelle est la proporion (en
      pourcentage) cela représente-t-il ?
    \item Le cacao constitue $55\%$ de la tablette : quelle masse cela
      représente-t-il ?
  \end{enumerate}
\end{app}

\subsection{Pourcentage de pourcentage}
\begin{prop}
  Soient $E$ une population et $A$ une sous-population de $E$ de proportion
  $p_A$. Soit $B$ une sous-population de $A$ de proportion $p_B$ par rapport à
  $A$. Alors $B$ est une sous-population de $E$, et la proportion de $B$ par
  rapport à $E$ est $p=p_A\times p_B$.
\end{prop}
\begin{exemple}
  Le Syndicat des Éditeurs de Logiciels de Loisirs déclare que $53$\% des
  français jouent régulièrement aux jeux vidéos. Parmi eux, $47$\% sont des
  femmes. En notant $p$ la proportion de femmes jouant aux jeux vidéos parmi
  tous les français, on a
  \[
    p = \frac{53}{100}\times\frac{47}{100} = 0,2491 = 24,91\%.
  \]
  Parmi l'ensemble des français, la proportion de femmes jouant aux jeux vidéos
  est de $24,91\%$.
\end{exemple}
\begin{app}
  Dans une boulangerie, le rayon p\^atisserie représente $30$\% du montant des
  ventes par jour et on sait que $3$ p\^atisseries sur $5$ sont des éclairs au
  chocolat.\\
  Le propriétaire ne fait un bénéfice que si le montant des ventes d'éclairs au
  chocolat représente plus de $20$\% de la recette par jour.
  \begin{enumerate}
    \item Quelle proportion de la recette (en pourcentage) représente le montant
      des ventes d'éclairs au chocolat ?
    \item Le propriétaire doit-il continuer à proposer des éclairs au chocolat ?
      Justifier.
  \end{enumerate}
\end{app}

\section{Taux d'évolution}
On considère une quantité qui varie. On note
\begin{itemize}
  \item $V_D$ la \textbf{valeur de départ} de cette quantité ;
  \item $V_A$ la \textbf{valeur d'arrivée} de cette quantité.
\end{itemize}
\subsection{Variation absolue et variation relative}
\begin{defi}{Variation absolue}
  La \textbf{variation absolue} $\Delta V$ est donnée par $\Delta V = V_A-V_D$.
\end{defi}
\begin{defi}{Variation relative}
  La \textbf{variation relative} (ou \textbf{taux d'évolution}) $t$ est le
  quotient de la différence $V_A$ et $V_D$ par $V_D$. On a donc
  \[
    t = \frac{V_A-V_D}{V_D}.
  \]
\end{defi}
\begin{rmq}
  Si la quantité \textbf{augmente}, les variations sont \textbf{positives}. Si
  la quantité \textbf{diminue}, alors les variations sont \textbf{négatives}.
\end{rmq}
\begin{exemple}
  Dans la ville de Trifouilli-Les-Oies, il a fait $10^\circ$C en moyenne le
  lundi, et $15^\circ$C en moyenne le mardi.
  \begin{itemize}
    \item On identifie $V_D=10$ et $V_A=15$.
    \item La variation absolue est de $\Delta V=15-10=5^\circ$C.
    \item Le taux d'évolution est de $\frac{15-10}{10}=0,5$. En pourcentage, le
      taux d'évolution est de $50\%$.
  \end{itemize}
\end{exemple}
\begin{app}
  Adam place $110$ euros en Bourse. Il se rend compte $15$ jours plus tard que
  ses actions valent $132$ euros.
  \begin{enumerate}
    \item Calculer la variation absolue de la somme placée.
    \item Calculer la variation relative de cette somme placée (en pourcentage).
  \end{enumerate}
\end{app}

\subsection{Coefficient multiplicateur}
\begin{defi}{Coefficient multiplicateur}
  On considère une quantité qui passe de la valeur $V_D$ à $V_A$. Le
  \textbf{coefficient multiplicateur}, noté $CM$, associé à cette évolution est
  \[
    CM = \frac{V_A}{V_D}.
  \]
\end{defi}
\begin{prop}
  Le coefficient multiplicateur permet de passer de la valeur de départ à la
  valeur d'arrivée (et inversement).
  \begin{align*}
    V_A &= CM\times V_D &
    V_D &= \frac{V_A}{CM}
  \end{align*}
\end{prop}
\begin{prop}
  \begin{minipage}{0.6\textwidth}
  Soit $t$ le taux d'évolution qui permet à une quantité de passer de $V_D$ à
  $V_A$. On a alors
  \[
    V_A=(1+t)\times V_D.
  \]
  On a donc $CM=1+t$.
\end{minipage}
  \begin{minipage}{0.4\textwidth}
    \begin{center}
  \begin{tikzpicture}
    \node[circle, draw] (VD) at (0,0) {$V_D$};
    \node[circle, draw] (VA) at (3,0) {$V_A$};
    \draw[-latex] (VD.north) to[out=60, in=120] (VA.north);
    \node at (1.5, .8) {$\times CM$};
    \node at (1.5, 1.7) {taux d'évolution $t$};
  \end{tikzpicture}
\end{center}

  \end{minipage}
\end{prop}
\begin{prop}
  \begin{itemize}
    \item Dans le cas d'une baisse, $t$ est négatif et $CM$ est un réel compris entre
      $0$ et $1$.
    \item Dans le cas d'une augmentation, $t$ est positif et $CM$ est un réel supérieur
      $1$.
  \end{itemize}
\end{prop}
\begin{exemple}
  Un article à $85$ euros est soldé à $-25\%$. On a $V_D=85$ et $t=-25\%=-0,25$.
  On en déduit $$CM=1+(-0,25)=0,75$$ et $$ V_A=V_D\times CM=85\times
  0,75=63,75.$$
  Le nouveau prix de cet article est $63,75$ euros.
\end{exemple}
\begin{app}
  Adam effectue un autre placement de $110$ euros. Ses actions risquent de subir
  une des deux modifications suivantes : soit elels augmentent de $10\%$, soit
  elles baissent de $15\%$.
  \begin{enumerate}
    \item Donner les cofficients multiplicateurs liés à chacune de ces
      évolutions.
    \item Dans chacun des cas, calculer la nouvelle valeur de ses actions.
  \end{enumerate}
\end{app}

\section{Évolutions successives et réciproques}
\subsection{Évolutions successives}
\begin{defi}{Évolutions successives et globale}
  Lorsqu'une quantité subit des \textbf{évolutions successives} de taux $t_1,
  t_2, \dots, t_n$ de sa valeur, elle subit alors une \textbf{evolution globale}
  de taux $t$.
\end{defi}
\begin{prop}
  Le coefficient multiplicateur global $CM$ associé à l'évolution $t$ est le
  produit des coefficients multiplicateurs $CM_1, CM_2, \dots, CM_n$, associés
  respectivement aux évolutions $t_1, t_2, \dots, t_n$. On a
  \[
    CM = CM_1\times CM_2\times\dots\times CM_n.
  \]
  \begin{center}
  \begin{tikzpicture}
    \node[circle, draw] (VD) at (0,0) {$V_D$};
    \node[circle, draw] (V1) at (3,0) {\phantom{$V_A$}};
    \node[circle, draw] (V2) at (6,0) {\phantom{$V_A$}};
    \node[circle, draw] (VA) at (9,0) {$V_A$};
    \draw[-latex] (VD.north) to[out=60, in=120] (V1.north);
    \draw[-latex] (V1.north) to[out=60, in=120] (V2.north);
    \node at (1.5, .8) {$\times CM_1$};
    \node at (1.5, 1.7) {taux $t_1$};
    \node at (4.5, .8) {$\times CM_2$};
    \node at (4.5, 1.7) {taux $t_2$};
    \node at (7.5,0) {$\dots$};
    \draw[-latex] (VD.south) to[out=-15, in=-165] (VA.south);
    \node at (4.5, -.7) {$CM_1\times CM_2\times\dots\times CM_n$};
    \node at (4.5, -1.4) {taux d'évolution $t$};
  \end{tikzpicture}
\end{center}

\end{prop}
\begin{exemple}
  Une valeur subit une hausse de $6\%$ puis une hausse de $14$\%. Le coefficient
  multiplicateur associé au taux d'évolution global $t$ est alors
  \[
    CM = 1,06\times1,14=1,2084.
  \]
  On en déduit $t=1,2084-1=0,2084$ soit une augmentation globale de $20,84\%$.
\end{exemple}
\begin{rmq}
  \textbf{Attention !} Le taux d'évolution global n'est pas égal à somme des
  taux d'évolution successifs.
\end{rmq}
\begin{app}
  Déterminer le taux d'évolution global d'une valeur suite à une augmentation de
  $50\%$ puis à une diminution de $50\%$.
\end{app}

\subsection{Évolutions réciproques}
\begin{defi}{Taux réciproque}
  Une quantité non nulle $V_D$ subit une évolution de taux $t$ et devient égale
  à une quantité $V_A$. Le \textbf{taux réciproque} de $t$ est le taux $t'$
  permettant de passer de $V_A$ à $V_D$.
\end{defi}
\begin{exemple}
  Un article coûte $50$ euros. Une baisse de $20\%$ fait passer le prix à $40$
  euros. Il faut une augmentation de $25\%$ pour revenir au prix initial de $50$
  euros.
  Ici $t=-20\%$ et $t'=+25\%$.

\end{exemple}
\begin{prop}~\\[-5mm]
  \begin{minipage}{0.6\textwidth}
  Le coefficient multiplicateur réciproque $CM'$ associé à l'évolution
  réciproque $t'$ est l'inverse du coefficient multiplicateur non nul $CM$
  associé à l'évolution de départ $t$. On a
  \[
    CM' = \frac{1}{CM}.
  \]
\end{minipage}
  \begin{minipage}{0.4\textwidth}
    \begin{center}
  \begin{tikzpicture}
    \node[circle, draw] (VD) at (0,0) {$V_D$};
    \node[circle, draw] (VA) at (3,0) {$V_A$};
    \draw[-latex] (VD.north) to[out=30, in=150] (VA.north);
    \node at (1.5, .6) {$\times CM$};
    \node at (1.5, 1.3) {taux d'évolution $t$};
    \draw[-latex] (VA.south) to[out=-150, in=-30] (VD.south);
    \node at (1.5, -.35) {$\times \cfrac{1}{CM}$};
    \node at (1.5, -1.3) {taux d'évolution $t'$};
  \end{tikzpicture}
\end{center}

  \end{minipage}

\end{prop}
\begin{app}
  Déterminer l'évolution réciproque d'une augmentation de $60$\%.
\end{app}

\end{document}
