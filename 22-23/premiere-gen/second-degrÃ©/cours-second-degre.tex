\documentclass[11pt]{article}

%%%%%%%%%%%%%%%%%%%%%%%%%%%%%%%%%%%%%%%%%%%%%%%%%%%%%%%%%%%%%%%%%%%%%%%%%%%%%%%%
%
% PACKAGES
% ========
%
%%%%%%%%%%%%%%%%%%%%%%%%%%%%%%%%%%%%%%%%%%%%%%%%%%%%%%%%%%%%%%%%%%%%%%%%%%%%%%%%

\usepackage[english, french]{babel}
\usepackage[utf8]{inputenc}
\usepackage[T1]{fontenc}
\usepackage{graphicx}
\usepackage{amsmath,amssymb,amsthm,amsopn}
\usepackage{hyperref}

% Pour avoir l'écriture \mathscr (math script)
% ============================================

\usepackage{mathrsfs}

% Deal with coma as a decimal separator
% =====================================

\usepackage{icomma}

% Package Geometry
% ================

\usepackage[a4paper, lmargin=2cm, rmargin=2cm, top=\topbotmargins, bottom=\topbotmargins]{geometry}

% Package multicol
% ================

\usepackage{multicol}

% Redefine abstract
% =================

% Note
% ====
%
% Le reste a été commenté pour ne pas charger trop de choses au démarrage. On
% verra si on en a besoin plus tard.
%
% --------
%
%\usepackage{mathrsfs}
%\usepackage{multirow}
%\usepackage{bm}
%\hypersetup{
%    colorlinks=true,
%    linkcolor=blue,
%    citecolor=red,
%}
%\usepackage{diagbox}
%
%\usepackage{algorithm}
%\usepackage{algpseudocode}
%
%\renewcommand{\algorithmicrequire}{\textbf{Input:}}
%\renewcommand{\algorithmicensure}{\textbf{Output:}}


%%%%%%%%%%%%%%%%%%%%%%%%%%%%%%%%%%%%%%%%%%%%%%%%%%%%%%%%%%%%%%%%%%%%%%%%%%%%%%%%
%
% TIKZ
% ====
%
%%%%%%%%%%%%%%%%%%%%%%%%%%%%%%%%%%%%%%%%%%%%%%%%%%%%%%%%%%%%%%%%%%%%%%%%%%%%%%%%

\usepackage{tikz}
\usetikzlibrary{arrows}

\usepackage{tkz-tab} % Variation tables

\usepackage{pgfplots}
%\usepackage{pgf-pie} % Pie charts

\pgfplotsset{
%\newcommand{\settingsgraph}{
x=.5cm,y=.5cm,
xticklabel style = {font=\scriptsize, yshift=.1cm},
yticklabel style = {font=\scriptsize, xshift=.1cm},
axis lines=middle,
ymajorgrids=true,
xmajorgrids=true,
major grid style = {color=white!80!blue},
xmin=-5.5,
xmax=5.5,
ymin=-5.5,
ymax=5.5,
xtick={-5.0,-4.0,...,5.0},
ytick={-5.0,-4.0,...,5.0},
}

% Tikz style

\tikzset{round/.style={circle, draw=black, very thick, scale = 0.7}}
\tikzset{arrow/.style={->, >=latex}}
\tikzset{dashed-arrow/.style={->, >=latex, dashed}}

\newcommand{\point}[3]{\draw[very thick, #3] (#1-.1, #2)--(#1+.1, #2)
(#1, #2-.1)--(#1, #2+.1)}

%%%%%%%%%%%%%%%%%%%%%%%%%%%%%%%%%%%%%%%%%%%%%%%%%%%%%%%%%%%%%%%%%%%%%%%%%%%%%%%%
%
% FANCY HEADER
% ============
%
%%%%%%%%%%%%%%%%%%%%%%%%%%%%%%%%%%%%%%%%%%%%%%%%%%%%%%%%%%%%%%%%%%%%%%%%%%%%%%%%


\usepackage{fancyhdr}
\usepackage{lastpage}

\pagestyle{fancy}
\newcommand{\changefont}{\fontsize{9}{9}\selectfont}
\renewcommand{\headrulewidth}{0mm}
\renewcommand{\footrulewidth}{0mm}

\fancyhead[C]{}
\fancyhead[L]{\titreclasse}
\fancyhead[R]{\titrechapitre}
\fancyfoot[C]{}
\fancyfoot[L]{}
\fancyfoot[R]{\pagination}
\addtolength{\skip\footins}{20pt} % distance between text and footnotes

%%%%%%%%%%%%%%%%%%%%%%%%%%%%%%%%%%%%%%%%%%%%%%%%%%%%%%%%%%%%%%%%%%%%%%%%%%%%%%%%
%
% THEOREM STYLE
% =============
%
%%%%%%%%%%%%%%%%%%%%%%%%%%%%%%%%%%%%%%%%%%%%%%%%%%%%%%%%%%%%%%%%%%%%%%%%%%%%%%%%

\usepackage[tikz]{bclogo}
\usepackage{mdframed}

\usepackage{tcolorbox}
\tcbuselibrary{listings, breakable, theorems, skins}

%\newtheoremstyle{break}%
%{}{}%
%{\itshape}{}%
%{\bfseries}{}%  % Note that final punctuation is omitted.
%{\newline}{}

\newtheoremstyle{scbf}%
{}{}%
{}{}%
%{\scshape}{}%  % Note that final punctuation is omitted.
{\bfseries\scshape}{}%  % Note that final punctuation is omitted.
{\newline}{}

%\theoremstyle{break}
%\theoremstyle{plain}
%\newtheorem{thm}{Theorem}[section]
%\newtheorem{lm}[thm]{Lemma}
%\newtheorem{prop}[thm]{Proposition}
%\newtheorem{cor}[thm]{Corollary}

%\theoremstyle{scbf}
%\newtheorem{exo}{$\star$ Exercice}

%\theoremstyle{definition}
%\newtheorem{defi}[thm]{Definition}
%\newtheorem{ex}[thm]{Example}

%\theoremstyle{remark}
%\newtheorem{rem}[thm]{Remark}

% Defining the Remark environment
% ===============================

\newenvironment{rmq}
  {
    \begin{bclogo}[logo=\bcinfo, noborder=true]{Remarque}
  }
  {
    \end{bclogo}
  }

% Defining the exercise environment
% =================================

\newcounter{exos}
\setcounter{exos}{1}

\newenvironment{exo}
  {
    \begin{bclogo}[logo=\bccrayon, noborder=true]{Exercice \theexos}
  }
  {
    \end{bclogo}
    \addtocounter{exos}{1}
  }


% Redefining the proof environment from amsthm
% ============================================

\tcolorboxenvironment{proof}{
  blanker, breakable, before skip=10pt,after skip=10pt,
  borderline west={1mm}{0pt}{red},
  left=5mm,
}

% Defining the definition environment
% ===================================

\colorlet{coldef}{black!50!green}

\newcounter{defis}
\setcounter{defis}{1}

\newenvironment{defi}[1]
  {
    \begin{defihid}{{#1}}{\thedefis}
  }
  {
    \end{defihid}
    \addtocounter{defis}{1}
  }

\newtcolorbox{defihid}[2]{%
  empty,title={ {\bfseries Définition {#2}} ({#1})},attach boxed title to top left,
boxed title style={empty,size=minimal,toprule=2pt,top=4pt,
overlay={\draw[coldef,line width=2pt]
([yshift=-1pt]frame.north west)--([yshift=-1pt]frame.north east);}},
coltitle=coldef,
before=\par\medskip\noindent,parbox=false,boxsep=0pt,left=0pt,right=3mm,top=4pt,
breakable,pad at break*=0mm,vfill before first,
overlay unbroken={\draw[coldef,line width=1pt]
([yshift=-1pt]title.north east)--([xshift=-0.5pt,yshift=-1pt]title.north-|frame.east)
--([xshift=-0.5pt]frame.south east)--(frame.south west); },
overlay first={\draw[coldef,line width=1pt]
([yshift=-1pt]title.north east)--([xshift=-0.5pt,yshift=-1pt]title.north-|frame.east)
--([xshift=-0.5pt]frame.south east); },
overlay middle={\draw[coldef,line width=1pt] ([xshift=-0.5pt]frame.north east)
--([xshift=-0.5pt]frame.south east); },
overlay last={\draw[coldef,line width=1pt] ([xshift=-0.5pt]frame.north east)
--([xshift=-0.5pt]frame.south east)--(frame.south west);},%
}

\newenvironment{notation}
  {
    \begin{notationhid}{\thedefis}
  }
  {
    \end{notationhid}
    \addtocounter{defis}{1}
  }

\newtcolorbox{notationhid}[1]{%
  empty,title={Notation {#1}},attach boxed title to top left,
boxed title style={empty,size=minimal,toprule=2pt,top=4pt,
overlay={\draw[coldef,line width=2pt]
([yshift=-1pt]frame.north west)--([yshift=-1pt]frame.north east);}},
coltitle=coldef,fonttitle=\bfseries,
before=\par\medskip\noindent,parbox=false,boxsep=0pt,left=0pt,right=3mm,top=4pt,
breakable,pad at break*=0mm,vfill before first,
overlay unbroken={\draw[coldef,line width=1pt]
([yshift=-1pt]title.north east)--([xshift=-0.5pt,yshift=-1pt]title.north-|frame.east)
--([xshift=-0.5pt]frame.south east)--(frame.south west); },
overlay first={\draw[coldef,line width=1pt]
([yshift=-1pt]title.north east)--([xshift=-0.5pt,yshift=-1pt]title.north-|frame.east)
--([xshift=-0.5pt]frame.south east); },
overlay middle={\draw[coldef,line width=1pt] ([xshift=-0.5pt]frame.north east)
--([xshift=-0.5pt]frame.south east); },
overlay last={\draw[coldef,line width=1pt] ([xshift=-0.5pt]frame.north east)
--([xshift=-0.5pt]frame.south east)--(frame.south west);},%
}


% Defining the proposition, theorem, etc. environment
% ===================================================

\colorlet{colprop}{red!75!black}

\newcounter{props}
\setcounter{props}{1}

\newenvironment{prop}
  {
    \begin{prophid}{\theprops}
  }
  {
    \end{prophid}
    \refstepcounter{props}
  }

\newtcolorbox{prophid}[1]{%
empty,title={Propriété {#1}},attach boxed title to top left,
boxed title style={empty,size=minimal,toprule=2pt,top=4pt,
overlay={\draw[colprop,line width=2pt]
([yshift=-1pt]frame.north west)--([yshift=-1pt]frame.north east);}},
coltitle=colprop,fonttitle=\bfseries,
before=\par\medskip\noindent,parbox=false,boxsep=0pt,left=0pt,right=3mm,top=4pt,
breakable,pad at break*=0mm,vfill before first,
overlay unbroken={\draw[colprop,line width=1pt]
([yshift=-1pt]title.north east)--([xshift=-0.5pt,yshift=-1pt]title.north-|frame.east)
--([xshift=-0.5pt]frame.south east)--(frame.south west); },
overlay first={\draw[colprop,line width=1pt]
([yshift=-1pt]title.north east)--([xshift=-0.5pt,yshift=-1pt]title.north-|frame.east)
--([xshift=-0.5pt]frame.south east); },
overlay middle={\draw[colprop,line width=1pt] ([xshift=-0.5pt]frame.north east)
--([xshift=-0.5pt]frame.south east); },
overlay last={\draw[colprop,line width=1pt] ([xshift=-0.5pt]frame.north east)
--([xshift=-0.5pt]frame.south east)--(frame.south west);},%
}

\newenvironment{propadm}
  {
    \begin{propadmhid}{\theprops}
  }
  {
    \end{propadmhid}
    \refstepcounter{props}
  }

  \newtcolorbox{propadmhid}[1]{%
    empty,title={{\bfseries Propriété {#1}} (admise)},attach boxed title to top left,
boxed title style={empty,size=minimal,toprule=2pt,top=4pt,
overlay={\draw[colprop,line width=2pt]
([yshift=-1pt]frame.north west)--([yshift=-1pt]frame.north east);}},
coltitle=colprop,%fonttitle=\bfseries,
before=\par\medskip\noindent,parbox=false,boxsep=0pt,left=0pt,right=3mm,top=4pt,
breakable,pad at break*=0mm,vfill before first,
overlay unbroken={\draw[colprop,line width=1pt]
([yshift=-1pt]title.north east)--([xshift=-0.5pt,yshift=-1pt]title.north-|frame.east)
--([xshift=-0.5pt]frame.south east)--(frame.south west); },
overlay first={\draw[colprop,line width=1pt]
([yshift=-1pt]title.north east)--([xshift=-0.5pt,yshift=-1pt]title.north-|frame.east)
--([xshift=-0.5pt]frame.south east); },
overlay middle={\draw[colprop,line width=1pt] ([xshift=-0.5pt]frame.north east)
--([xshift=-0.5pt]frame.south east); },
overlay last={\draw[colprop,line width=1pt] ([xshift=-0.5pt]frame.north east)
--([xshift=-0.5pt]frame.south east)--(frame.south west);},%
}

\newenvironment{propnom}[1]
  {
    \begin{propnomhid}{#1}{\theprops}
  }
  {
    \end{propnomhid}
    \refstepcounter{props}
  }

\newtcolorbox{propnomhid}[2]{%
empty,title={{\bfseries Propriété {#2}} ({#1})},attach boxed title to top left,
boxed title style={empty,size=minimal,toprule=2pt,top=4pt,
overlay={\draw[colprop,line width=2pt]
([yshift=-1pt]frame.north west)--([yshift=-1pt]frame.north east);}},
coltitle=colprop,
before=\par\medskip\noindent,parbox=false,boxsep=0pt,left=0pt,right=3mm,top=4pt,
breakable,pad at break*=0mm,vfill before first,
overlay unbroken={\draw[colprop,line width=1pt]
([yshift=-1pt]title.north east)--([xshift=-0.5pt,yshift=-1pt]title.north-|frame.east)
--([xshift=-0.5pt]frame.south east)--(frame.south west); },
overlay first={\draw[colprop,line width=1pt]
([yshift=-1pt]title.north east)--([xshift=-0.5pt,yshift=-1pt]title.north-|frame.east)
--([xshift=-0.5pt]frame.south east); },
overlay middle={\draw[colprop,line width=1pt] ([xshift=-0.5pt]frame.north east)
--([xshift=-0.5pt]frame.south east); },
overlay last={\draw[colprop,line width=1pt] ([xshift=-0.5pt]frame.north east)
--([xshift=-0.5pt]frame.south east)--(frame.south west);},%
}




\newenvironment{thm}
  {
    \begin{thmhid}{\theprops}
  }
  {
    \end{thmhid}
    \refstepcounter{props}
  }

\newtcolorbox{thmhid}[1]{%
empty,title={Théorème {#1}},attach boxed title to top left,
boxed title style={empty,size=minimal,toprule=2pt,top=4pt,
overlay={\draw[colprop,line width=2pt]
([yshift=-1pt]frame.north west)--([yshift=-1pt]frame.north east);}},
coltitle=colprop,fonttitle=\bfseries,
before=\par\medskip\noindent,parbox=false,boxsep=0pt,left=0pt,right=3mm,top=4pt,
breakable,pad at break*=0mm,vfill before first,
overlay unbroken={\draw[colprop,line width=1pt]
([yshift=-1pt]title.north east)--([xshift=-0.5pt,yshift=-1pt]title.north-|frame.east)
--([xshift=-0.5pt]frame.south east)--(frame.south west); },
overlay first={\draw[colprop,line width=1pt]
([yshift=-1pt]title.north east)--([xshift=-0.5pt,yshift=-1pt]title.north-|frame.east)
--([xshift=-0.5pt]frame.south east); },
overlay middle={\draw[colprop,line width=1pt] ([xshift=-0.5pt]frame.north east)
--([xshift=-0.5pt]frame.south east); },
overlay last={\draw[colprop,line width=1pt] ([xshift=-0.5pt]frame.north east)
--([xshift=-0.5pt]frame.south east)--(frame.south west);},%
}

\newenvironment{thmadm}
  {
    \begin{thmadmhid}{\theprops}
  }
  {
    \end{thmadmhid}
    \refstepcounter{props}
  }

  \newtcolorbox{thmadmhid}[1]{%
    empty,title={{\bfseries Théorème {#1}} (admis)},attach boxed title to top left,
boxed title style={empty,size=minimal,toprule=2pt,top=4pt,
overlay={\draw[colprop,line width=2pt]
([yshift=-1pt]frame.north west)--([yshift=-1pt]frame.north east);}},
coltitle=colprop,%fonttitle=\bfseries,
before=\par\medskip\noindent,parbox=false,boxsep=0pt,left=0pt,right=3mm,top=4pt,
breakable,pad at break*=0mm,vfill before first,
overlay unbroken={\draw[colprop,line width=1pt]
([yshift=-1pt]title.north east)--([xshift=-0.5pt,yshift=-1pt]title.north-|frame.east)
--([xshift=-0.5pt]frame.south east)--(frame.south west); },
overlay first={\draw[colprop,line width=1pt]
([yshift=-1pt]title.north east)--([xshift=-0.5pt,yshift=-1pt]title.north-|frame.east)
--([xshift=-0.5pt]frame.south east); },
overlay middle={\draw[colprop,line width=1pt] ([xshift=-0.5pt]frame.north east)
--([xshift=-0.5pt]frame.south east); },
overlay last={\draw[colprop,line width=1pt] ([xshift=-0.5pt]frame.north east)
--([xshift=-0.5pt]frame.south east)--(frame.south west);},%
}

\newenvironment{thmnom}[1]
  {
    \begin{thmnomhid}{#1}{\theprops}
  }
  {
    \end{thmnomhid}
    \refstepcounter{props}
  }

\newtcolorbox{thmnomhid}[2]{%
empty,title={{\bfseries Théorème {#2}} ({#1})},attach boxed title to top left,
boxed title style={empty,size=minimal,toprule=2pt,top=4pt,
overlay={\draw[colprop,line width=2pt]
([yshift=-1pt]frame.north west)--([yshift=-1pt]frame.north east);}},
coltitle=colprop,
before=\par\medskip\noindent,parbox=false,boxsep=0pt,left=0pt,right=3mm,top=4pt,
breakable,pad at break*=0mm,vfill before first,
overlay unbroken={\draw[colprop,line width=1pt]
([yshift=-1pt]title.north east)--([xshift=-0.5pt,yshift=-1pt]title.north-|frame.east)
--([xshift=-0.5pt]frame.south east)--(frame.south west); },
overlay first={\draw[colprop,line width=1pt]
([yshift=-1pt]title.north east)--([xshift=-0.5pt,yshift=-1pt]title.north-|frame.east)
--([xshift=-0.5pt]frame.south east); },
overlay middle={\draw[colprop,line width=1pt] ([xshift=-0.5pt]frame.north east)
--([xshift=-0.5pt]frame.south east); },
overlay last={\draw[colprop,line width=1pt] ([xshift=-0.5pt]frame.north east)
--([xshift=-0.5pt]frame.south east)--(frame.south west);},%
}

\newenvironment{coro}
  {
    \begin{corohid}{\theprops}
  }
  {
    \end{corohid}
    \refstepcounter{props}
  }

  \newtcolorbox{corohid}[1]{%
  empty,title={Corollaire {#1}},attach boxed title to top left,
boxed title style={empty,size=minimal,toprule=2pt,top=4pt,
overlay={\draw[colprop,line width=2pt]
([yshift=-1pt]frame.north west)--([yshift=-1pt]frame.north east);}},
coltitle=colprop,fonttitle=\bfseries,
before=\par\medskip\noindent,parbox=false,boxsep=0pt,left=0pt,right=3mm,top=4pt,
breakable,pad at break*=0mm,vfill before first,
overlay unbroken={\draw[colprop,line width=1pt]
([yshift=-1pt]title.north east)--([xshift=-0.5pt,yshift=-1pt]title.north-|frame.east)
--([xshift=-0.5pt]frame.south east)--(frame.south west); },
overlay first={\draw[colprop,line width=1pt]
([yshift=-1pt]title.north east)--([xshift=-0.5pt,yshift=-1pt]title.north-|frame.east)
--([xshift=-0.5pt]frame.south east); },
overlay middle={\draw[colprop,line width=1pt] ([xshift=-0.5pt]frame.north east)
--([xshift=-0.5pt]frame.south east); },
overlay last={\draw[colprop,line width=1pt] ([xshift=-0.5pt]frame.north east)
--([xshift=-0.5pt]frame.south east)--(frame.south west);},%
}

\newenvironment{lemme}
  {
    \begin{lemmehid}{\theprops}
  }
  {
    \end{lemmehid}
    \refstepcounter{props}
  }

  \newtcolorbox{lemmehid}[1]{%
  empty,title={Lemme {#1}},attach boxed title to top left,
boxed title style={empty,size=minimal,toprule=2pt,top=4pt,
overlay={\draw[colprop,line width=2pt]
([yshift=-1pt]frame.north west)--([yshift=-1pt]frame.north east);}},
coltitle=colprop,fonttitle=\bfseries,
before=\par\medskip\noindent,parbox=false,boxsep=0pt,left=0pt,right=3mm,top=4pt,
breakable,pad at break*=0mm,vfill before first,
overlay unbroken={\draw[colprop,line width=1pt]
([yshift=-1pt]title.north east)--([xshift=-0.5pt,yshift=-1pt]title.north-|frame.east)
--([xshift=-0.5pt]frame.south east)--(frame.south west); },
overlay first={\draw[colprop,line width=1pt]
([yshift=-1pt]title.north east)--([xshift=-0.5pt,yshift=-1pt]title.north-|frame.east)
--([xshift=-0.5pt]frame.south east); },
overlay middle={\draw[colprop,line width=1pt] ([xshift=-0.5pt]frame.north east)
--([xshift=-0.5pt]frame.south east); },
overlay last={\draw[colprop,line width=1pt] ([xshift=-0.5pt]frame.north east)
--([xshift=-0.5pt]frame.south east)--(frame.south west);},%
}

\colorlet{colexemple}{blue!50!black}
%\newtcolorbox{exemple}{empty, title=Exemple, attach boxed title to top left,
%  boxed title style={empty, size=minimal, toprule=2pt, top=4pt,
%    overlay={\draw[colexemple,line width=2pt]
%([yshift=-1pt]frame.north west)--([yshift=-1pt]frame.north east);}},
%coltitle=colexemple,fonttitle=\bfseries,%\large\bfseries,
%before=\par\medskip\noindent,parbox=false,boxsep=0pt,left=0pt,right=3mm,top=4pt,
%overlay={\draw[colexemple,line width=1pt]
%([yshift=-1pt]title.north east)--([xshift=-0.5pt,yshift=-1pt]title.north-|frame.east)
%--([xshift=-0.5pt]frame.south east)--(frame.south west); },
%}

\newcounter{exemples}
\setcounter{exemples}{1}

\newenvironment{exemple}
  {
    \begin{exemplehid}{\theexemples}
  }
  {
    \end{exemplehid}
    \addtocounter{exemples}{1}
  }

\newtcolorbox{exemplehid}[1]{%
empty,title={Exemple {#1}},attach boxed title to top left,
boxed title style={empty,size=minimal,toprule=2pt,top=4pt,
overlay={\draw[colexemple,line width=2pt]
([yshift=-1pt]frame.north west)--([yshift=-1pt]frame.north east);}},
coltitle=colexemple,fonttitle=\bfseries,
before=\par\medskip\noindent,parbox=false,boxsep=0pt,left=0pt,right=3mm,top=4pt,
breakable,pad at break*=0mm,vfill before first,
overlay unbroken={\draw[colexemple,line width=1pt]
([yshift=-1pt]title.north east)--([xshift=-0.5pt,yshift=-1pt]title.north-|frame.east)
--([xshift=-0.5pt]frame.south east)--(frame.south west); },
overlay first={\draw[colexemple,line width=1pt]
([yshift=-1pt]title.north east)--([xshift=-0.5pt,yshift=-1pt]title.north-|frame.east)
--([xshift=-0.5pt]frame.south east); },
overlay middle={\draw[colexemple,line width=1pt] ([xshift=-0.5pt]frame.north east)
--([xshift=-0.5pt]frame.south east); },
overlay last={\draw[colexemple,line width=1pt] ([xshift=-0.5pt]frame.north east)
--([xshift=-0.5pt]frame.south east)--(frame.south west);},%
}

\newenvironment{contrex}
  {
    \begin{contrexhid}{\theexemples}
  }
  {
    \end{contrexhid}
    \addtocounter{exemples}{1}
  }

\newtcolorbox{contrexhid}[1]{%
empty,title={Contre-exemple {#1}},attach boxed title to top left,
boxed title style={empty,size=minimal,toprule=2pt,top=4pt,
overlay={\draw[colexemple,line width=2pt]
([yshift=-1pt]frame.north west)--([yshift=-1pt]frame.north east);}},
coltitle=colexemple,fonttitle=\bfseries,
before=\par\medskip\noindent,parbox=false,boxsep=0pt,left=0pt,right=3mm,top=4pt,
breakable,pad at break*=0mm,vfill before first,
overlay unbroken={\draw[colexemple,line width=1pt]
([yshift=-1pt]title.north east)--([xshift=-0.5pt,yshift=-1pt]title.north-|frame.east)
--([xshift=-0.5pt]frame.south east)--(frame.south west); },
overlay first={\draw[colexemple,line width=1pt]
([yshift=-1pt]title.north east)--([xshift=-0.5pt,yshift=-1pt]title.north-|frame.east)
--([xshift=-0.5pt]frame.south east); },
overlay middle={\draw[colexemple,line width=1pt] ([xshift=-0.5pt]frame.north east)
--([xshift=-0.5pt]frame.south east); },
overlay last={\draw[colexemple,line width=1pt] ([xshift=-0.5pt]frame.north east)
--([xshift=-0.5pt]frame.south east)--(frame.south west);},%
}

\newenvironment{app}
  {
    \begin{apphid}{\theexemples}
  }
  {
    \end{apphid}
    \addtocounter{exemples}{1}
  }

\newtcolorbox{apphid}[1]{%
empty,title={Application {#1}},attach boxed title to top left,
boxed title style={empty,size=minimal,toprule=2pt,top=4pt,
overlay={\draw[colexemple,line width=2pt]
([yshift=-1pt]frame.north west)--([yshift=-1pt]frame.north east);}},
coltitle=colexemple,fonttitle=\bfseries,
before=\par\medskip\noindent,parbox=false,boxsep=0pt,left=0pt,right=3mm,top=4pt,
breakable,pad at break*=0mm,vfill before first,
overlay unbroken={\draw[colexemple,line width=1pt]
([yshift=-1pt]title.north east)--([xshift=-0.5pt,yshift=-1pt]title.north-|frame.east)
--([xshift=-0.5pt]frame.south east)--(frame.south west); },
overlay first={\draw[colexemple,line width=1pt]
([yshift=-1pt]title.north east)--([xshift=-0.5pt,yshift=-1pt]title.north-|frame.east)
--([xshift=-0.5pt]frame.south east); },
overlay middle={\draw[colexemple,line width=1pt] ([xshift=-0.5pt]frame.north east)
--([xshift=-0.5pt]frame.south east); },
overlay last={\draw[colexemple,line width=1pt] ([xshift=-0.5pt]frame.north east)
--([xshift=-0.5pt]frame.south east)--(frame.south west);},%
}

%%%%%%%%%%%%%%%%%%%%%%%%%%%%%%%%%%%%%%%%%%%%%%%%%%%%%%%%%%%%%%%%%%%%%%%%%%%%%%%%
%
% ENUMERATE
% =========
%
%%%%%%%%%%%%%%%%%%%%%%%%%%%%%%%%%%%%%%%%%%%%%%%%%%%%%%%%%%%%%%%%%%%%%%%%%%%%%%%%

\usepackage{enumerate}
\usepackage{enumitem}

% To have special enumerate items like
%
% 1/
% 2/
% 3/

%%%%%%%%%%%%%%%%%%%%%%%%%%%%%%%%%%%%%%%%%%%%%%%%%%%%%%%%%%%%%%%%%%%%%%%%%%%%%%%%
%
% ARRAYS
% ======
%
%%%%%%%%%%%%%%%%%%%%%%%%%%%%%%%%%%%%%%%%%%%%%%%%%%%%%%%%%%%%%%%%%%%%%%%%%%%%%%%%


\usepackage{array}
\usepackage{makecell} % Used to break lines within arrays
\usepackage{multirow}
\usepackage{booktabs} % Used to have nice arrays with headrules

%%%%%%%%%%%%%%%%%%%%%%%%%%%%%%%%%%%%%%%%%%%%%%%%%%%%%%%%%%%%%%%%%%%%%%%%%%%%%%%%
%
% WRITE CODE
% ==========
%
%%%%%%%%%%%%%%%%%%%%%%%%%%%%%%%%%%%%%%%%%%%%%%%%%%%%%%%%%%%%%%%%%%%%%%%%%%%%%%%%

\usepackage{listings}
\usepackage{xcolor}

%New colors defined below
\definecolor{codegreen}{rgb}{0,0.6,0}
\definecolor{codegray}{rgb}{0.5,0.5,0.5}
\definecolor{codepurple}{rgb}{0.58,0,0.82}
\definecolor{backcolour}{rgb}{0.95,0.95,0.92}

%Code listing style named "mystyle"
\lstdefinestyle{python}{
  %backgroundcolor=\color{backcolour},
  commentstyle=\color{codegreen},
  keywordstyle=\color{magenta},
  numberstyle=\tiny\color{codegray},
  stringstyle=\color{codepurple},
  basicstyle=\ttfamily\footnotesize,
  breakatwhitespace=false,
  breaklines=true,
  captionpos=b,
  keepspaces=true,
  numbers=left,
  numbersep=5pt,
  showspaces=false,
  showstringspaces=false,
  showtabs=false,
  tabsize=2
}

\lstset{style=python}

%%%%%%%%%%%%%%%%%%%%%%%%%%%%%%%%%%%%%%%%%%%%%%%%%%%%%%%%%%%%%%%%%%%%%%%%%%%%%%%%
%
% Tabular 
% =======
%
%%%%%%%%%%%%%%%%%%%%%%%%%%%%%%%%%%%%%%%%%%%%%%%%%%%%%%%%%%%%%%%%%%%%%%%%%%%%%%%%

% In order to obtain a tabular with given width.

\usepackage{tabularx}
\newcolumntype{Y}{>{\centering\arraybackslash}X}
\newcolumntype{R}{>{\raggedright\arraybackslash}X}
\newcolumntype{L}{>{\raggedleft\arraybackslash}X}
% \usepackage{tabulary} % younger brother

%%%%%%%%%%%%%%%%%%%%%%%%%%%%%%%%%%%%%%%%%%%%%%%%%%%%%%%%%%%%%%%%%%%%%%%%%%%%%%%%
%
% MACROS
% ======
%
%%%%%%%%%%%%%%%%%%%%%%%%%%%%%%%%%%%%%%%%%%%%%%%%%%%%%%%%%%%%%%%%%%%%%%%%%%%%%%%%

% Math Operators

\DeclareMathOperator{\Card}{Card}
\DeclareMathOperator{\Gal}{Gal}
\DeclareMathOperator{\Id}{Id}
\DeclareMathOperator{\Img}{Im}
\DeclareMathOperator{\Ker}{Ker}
\DeclareMathOperator{\Minpoly}{Minpoly}
\DeclareMathOperator{\Mod}{mod}
\DeclareMathOperator{\Ord}{Ord}
\DeclareMathOperator{\ppcm}{ppcm}
\DeclareMathOperator{\pgcd}{pgcd}
\DeclareMathOperator{\tr}{Tr}
\DeclareMathOperator{\Vect}{Vect}
\DeclareMathOperator{\Span}{Span}
\DeclareMathOperator{\rank}{rank}
\DeclareMathOperator{\rg}{rg}
\DeclareMathOperator{\ev}{ev}
\DeclareMathOperator{\Var}{Var}

% Shortcuts

\newcommand{\eg}{\emph{e.g. }}
\newcommand{\ent}[2]{[\![#1,#2]\!]}
\newcommand{\ie}{\emph{i.e. }}
\newcommand{\ps}[2]{\left\langle#1,#2\right\rangle}
\newcommand{\eqdef}{\overset{\text{def}}{=}}
\newcommand{\E}{\mathcal{E}}
\newcommand{\M}{\mathcal{M}}
\newcommand{\A}{\mathcal{A}}
\newcommand{\B}{\mathcal{B}}
\newcommand{\R}{\mathcal{R}}
\newcommand{\D}{\mathcal{D}}
\newcommand{\Pcal}{\mathcal{P}}
\newcommand{\K}{\mathbf{k}}
\newcommand{\vect}[1]{\overrightarrow{#1}}


%\input{layout-nb.tex}

\title{Chapitre 1 : second degré}
\date{}
\author{}

\begin{document}
\maketitle\thispagestyle{fancy}

\abstract{Dans ce premier chapitre, nous allons étudier les fonctions et les
  équations du \emph{second degré}, c'est-à-dire les expressions de la forme
\[
  ax^2 + bx + c,
\]
où $a$, $b$ et $c$ sont des nombres fixés, et où on considérera que $x$ est une
variable. Le nom « second degré » vient du fait que la variable $x$ intervient
avec un degré $2$ dans l'expression. Nous allons alterner entre deux visions
différentes : les fonctions du second degré et les équations du second degré.
Dans les deux cas, cela viendra compléter ce que nous savons déjà sur les
fonctions et les équations, en allant \emph{un peu plus loin}. Les expressions
du second degré sont très présentes dans les sciences, et permettent de
modéliser de nombreux problèmes, comme nous le verrons par la suite.}


\section{Fonctions polynômes du second degré}
\subsection{Forme développée et forme canonique}

On connaît déjà de nombreux types de fonctions, dont les fonctions affines, qui
sont des fonctions de la forme
\[
  f:x\mapsto bx+c,
\]
où $b$ et $c$ sont des constantes. Les représentations graphiques de ces
fonctions sont des droites.
\begin{center}
\begin{tikzpicture}
\begin{axis}
  \addplot[red, very thick, samples=201]{2*x-3};
\end{axis}
\end{tikzpicture}
\end{center}
On va maintenant généraliser les fonctions affines
en ajoutant un terme en $x^2$.

\begin{defi}{Fonction polynôme du second degré}
  Une fonction $f$ définie sur $\mathbb{R}$ est appelée \emph{fonction polynôme
  du second degré} s'il existe $a, b, c\in\mathbb{R}$ des réels avec $a\neq0$ et
  tels que, pour tout réel $x\in\mathbb{R}$, on ait
  \[
    f(x) = ax^2+bx+c.
  \]
\end{defi}

\begin{defi}{Forme développée et coefficients}
  Soit $f$ une fonction polynôme du second degré définie sur $\mathbb{R}$ par
  \[
    f(x) = ax^2+bx+c,
  \]
  où $a,b,c\in\mathbb{R}$ sont des réels et où $a\neq0$. Lorsque la fonction $f$
  est écrite sous cette forme, on parle de \emph{forme développée}. Les réels
  $a$, $b$, et $c$ sont appelés les \emph{coefficients} de $f$.
\end{defi}

\begin{exemple}
  Soit $f$ la fonction définie sur $\mathbb{R}$ par
  \[
    f(x) = x^2+2x+1.
  \]
  La fonction $f$ est une fonction polynôme du second degré, donnée sous forme
  développée. Ses coefficients $a$, $b$, et $c$ valent respectivement $1$, $2$,
  et $1$.
\begin{center}
\begin{tikzpicture}
\begin{axis}
\addplot[red, very thick, samples=201]{x^2+2*x+1};
\end{axis}
%\draw[color=red] (2,4) node {\LARGE{$y=x^2+2x+1$}};
\end{tikzpicture}
\end{center}

\end{exemple}

\begin{exemple}
  La fonction $g$ définie sur $\mathbb{R}$ par
  \[
    g(x) = -3x^2+\sqrt 2
  \]
  est un autre exemple de fonction polynôme du second degré. Cette fois-ci le
  coefficient devant $x$ est nul, c'est-à-dire que $b=0$, et on a $a=-3$ et
  $c=\sqrt2$.

\begin{center}
\begin{tikzpicture}
\begin{axis}
\addplot[red, very thick, samples=201]{-3*x^2+sqrt(2)};
\end{axis}
%\draw[color=red] (2,4) node {\LARGE{$y=x^2+2x+1$}};
\end{tikzpicture}
\end{center}

\end{exemple}

\begin{contrex}
  La fonction $h$ définie sur $\mathbb{R}_+$ par
  \[
    h(x) = \sqrt x + 1
  \]
  n'est pas une fonction polynôme du second degré.
\begin{center}
\begin{tikzpicture}
\begin{axis}
\addplot[red, very thick, samples=201,domain=0:5.5]{sqrt(x)+1};
\end{axis}
%\draw[color=red] (2,4) node {\LARGE{$y=x^2+2x+1$}};
\end{tikzpicture}
\end{center}

\end{contrex}

\begin{exo}
  Les fonctions suivantes sont-elles des fonctions polynômes du second degré ?
  \begin{enumerate}
    \item La fonction $f_1$ définie sur $\mathbb{R}$ par $f_1(x) = -x^2-x+10$.
    \item La fonction $f_2$ définie sur $\mathbb{R_+}$ par $f_2(x) = x^2+\sqrt x-1$.
    \item La fonction $f_3$ définie sur $\mathbb{R}$ par $f_3(x) = x^2$.
    \item La fonction $f_4$ définie sur $\mathbb{R}$ par $f_4(x) = x+2$.
    \item La fonction $f_5$ définie sur $\mathbb{R}$ par $f_5(x) = x(x+1)$.
    \item La fonction $f_6$ définie sur $\mathbb{R}$ par $f_6(x) = 0$.
  \end{enumerate}
\end{exo}

\begin{exo}
  Ex. $56$ p. $90$.
\end{exo}

Si on parle de forme développée, c'est parce qu'il ne s'agit pas de la seule
forme possible. Par exemple, la fonction $f$ définie sur $\mathbb{R}$ par
\[
  f(x) = x^2+2x+1
\]
peut aussi s'écrire, en utilisant une identité remarquable :
\[
  f(x) = (x+1)^2.
\]
Cette forme porte elle aussi un nom particulier.


\begin{prop}
  Soit $f$ une fonction polynôme du second degré définie sur $\mathbb{R}$ par
  \[
    f(x) = ax^2+bx+c.
  \]
  Alors $f$ peut s'écrire sous la forme
  \[
    f(x) = a(x-\alpha)^2+\beta,
  \]
  et on a $\alpha=-\cfrac{b}{2a}$ et $\beta=f(\alpha)$.
\end{prop}

\begin{proof}
  Soit $f$ la fonction définie sur $\mathbb{R}$ par
  \[
    f(x) = ax^2+bx+c,
  \]
  avec $a,b,c\in\mathbb{R}$ trois réels et $a\neq0$. Comme on a $a\neq0$, on
  peut écrire, pour tout $x\in\mathbb{R}$, que
  \[
    f(x) = a\left( x^2+\frac{b}{a}x+\frac{c}{a} \right).
  \]
  Puis on remarque que les termes $x^2+\frac{b}{a}x$ forment le début du
  développement de
  \[
    \left(x+\frac{b}{2a}\right)^2 = x^2 +
    \frac{b}{a}x+\left(\frac{b}{2a}\right)^2.
  \]
  On peut donc écrire que
  \[
    \left( x^2+\frac{b}{a}x \right) = \left( x+\frac{b}{2a} \right)^2-\left(
    \frac{b}{2a} \right)^2
  \]
  puis que
  \[
    f(x) = a\left( \left( x+\frac{b}{2a} \right)^2-\left( \frac{b}{2a}
    \right)^2+\frac{c}{a} \right).
  \]
  En mettant sur le même dénominateur, on remarque que
  \begin{align*}
    -\left( \frac{b}{2a} \right)^2+\frac{c}{a} &= -\frac{b^2}{4a^2} +
    \frac{c}{a} \\
    &= \frac{-b^2+4ac}{4a^2}.
 \end{align*}
 On retrouve alors
 \begin{align*}
   f(x) &= a\left( x+\frac{b}{2a} \right)^2+a\times\frac{-b^2+4ac}{4a^2} \\
   &= a\left( x+\frac{b}{2a} \right)^2+\frac{-b^2+4ac}{4a} \\
   &= a\left( x-\alpha \right)^2+\beta,
 \end{align*}
 où $\alpha=\cfrac{-b}{2a}$ et où $\beta=\cfrac{-b^2+4ac}{4a}$. De plus, on a
 aussi 
 \[
   f(\alpha) = a\left( \alpha-\alpha \right)^2+\beta= a\times 0+\beta=\beta.
 \]
\end{proof}

\begin{defi}{Forme canonique}
  La forme $f(x) = a(x-\alpha)^2+\beta$ s'appelle la \emph{forme canonique}.
\end{defi}

\begin{rmq}
  Les quantités $\cfrac{-b}{2a}$ et $b^2-4ac$, que l'on a rencontré dans la
  démonstration de la propriété précédente, vont jouer un rôle important dans
  la suite.
\end{rmq}

\begin{exemple}
  Soit $f$ la fonction définie sur $\mathbb{R}$ par
  \[
    f(x) = x^2 + 2x + 1.
  \]
  La forme canonique de $f$ est 
  \[
    f(x) = (x+1)^2.
  \]
  En effet, on a dans ce cas $a=1$, $\alpha=-1$ et $\beta=0$.
\end{exemple}

\subsection{Variations et représentation graphique}

Comme souvent lorsqu'on s'intéresse aux fonctions, on va essayer de comprendre
comment elles se comportent, c'est-à-dire à quoi ressemble leur représentation
graphique, comment elles varient, etc.

\begin{prop}
  Soit $f$ la fonction définie sur $\mathbb{R}$ par $f(x)=ax^2+bx+c$, de forme
  canonique $f(x)=a(x-\alpha)^2+\beta$.

  \noindent
  \begin{minipage}[t]{.47\textwidth}
    \begin{center}
      {\bf Si} $\mathbf{a>0}$\vspace{.2cm}

      \begin{tikzpicture}
        \tkzTabInit[lgt=1, espcl=1.5]{$x$ / .7, $f(x)$ / 1.4}{$-\infty$, $\alpha$, $+\infty$}
        \tkzTabVar{+/, -/$\beta$, +/}
      \end{tikzpicture}
    \end{center}
  La fonction $f$ est strictement décroissante sur $]-\infty, \alpha]$,
  strictement croissante sur $[\alpha, +\infty[$, et $f$ admet comme minimum
    $\beta$ en $\alpha$.
    \begin{center}
\begin{tikzpicture}
  \begin{axis}[x=.25cm, y=.25cm, xtick=\empty, ytick=\empty]
    \addplot[red, very thick, samples=201]{x^2-x-3};
  \end{axis}
\end{tikzpicture}
    \end{center}
  \end{minipage}
    \hfill
  \begin{minipage}[t]{.47\textwidth}
    \begin{center}
      {\bf Si} $\mathbf{a<0}$\vspace{.2cm}

      \begin{tikzpicture}
        \tkzTabInit[lgt=1, espcl=1.5]{$x$ / .7, $f(x)$ / 1.4}{$-\infty$, $\alpha$, $+\infty$}
        \tkzTabVar{-/, +/$\beta$, -/}
      \end{tikzpicture}
    \end{center}
  La fonction $f$ est strictement croissante sur $]-\infty, \alpha]$,
  strictement décroissante sur $[\alpha, +\infty[$, et $f$ admet comme maximum
    $\beta$ en $\alpha$.
    \begin{center}
\begin{tikzpicture}
  \begin{axis}[x=.25cm, y=.25cm, xtick=\empty, ytick=\empty]
    \addplot[red, very thick, samples=201]{-x^2-2*x+2};
  \end{axis}
\end{tikzpicture}
    \end{center}
  \end{minipage}
\end{prop}

\begin{proof}
  Soit $a, b, c\in\mathbb{R}$ trois réels, avec $a\neq0$, et soit $f$ la
  fonction définie sur $\mathbb{R}$ par
  \[
    f(x) = ax^2 + bx + c.
  \]
  On rappelle que $f$ admet alors une forme canonique
  \[
    f(x) = a(x-\alpha)^2+\beta,
  \]
  avec $\alpha=\frac{-b}{2a}$ et $\beta=f(\alpha)$.

  Commençons par le cas où $a>0$. Nous allons d'abord démontrer que la fonction $f$ est
  strictement décroissante sur $]-\infty, \alpha]$. Soit $x_1, x_2\in\mathbb{R}$
  deux réels tels que 
  \[
    x_1 < x_2 \leq \alpha.
  \]
  En retranchant $\alpha$ dans cette inégalité, on obtient que
  $x_1-\alpha<x_2-\alpha$. Le fonction carré étant strictement décroissante sur
  $]-\infty, 0]$, on a donc
  \[
    (x_1-\alpha)^2 > (x_2-\alpha)^2.
  \]
  Puis, en multipliant par $a>0$, on obtient $a(x_1-\alpha)^2>a(x_2-\alpha)^2$,
  et en ajoutant $\beta$ dans les deux membres de cette inéquation, on a
  finalement
  \[
    a(x_1-\alpha)^2+\beta > a(x_2-\alpha)^2+\beta.
  \]
  On a bien prouvé que si $x_1<x_2\leq \alpha$, alors $f(x_1)>f(x_2)$,
  c'est-à-dire que $f$ est strictement décroissante sur $]-\infty, \alpha]$.

  Démontrons désormais que $f$ est strictement croissante sur $[\alpha,
  +\infty[$. On va procédér de façon similaire à la première partie de la
    preuve. Soit $x_1, x_2\in\mathbb{R}$ deux réels tels que
  \[
    \alpha\leq x_1 < x_2.
  \]
  En retranchant $\alpha$, il vient $0\leq x_1-\alpha < x_2-\alpha$. Puis, la
  fonction carré est étant strictement croissante sur $[0, +\infty[$, on obtient
  \[
    (x_1-\alpha)^2 > (x_2-\alpha)^2.
  \]
  Ensuite, en multipliant par $a>0$ et en ajoutant $\beta$, ce qui ne change pas
  le sens de l'inégalité, on a
  \[
    a(x_1-\alpha)^2+\beta > a(x_2-\alpha)^2+\beta,
  \]
  c'est-à-dire $f(x_1)>f(x_2)$. On a donc prouvé que la fonction $f$ est
  strictement croissante sur $[\alpha, +\infty[$.

  Le cas où $a<0$ est très similaire, la seule chose qui change est que lorsqu'on
  multiplie par $a<0$, on change le sens des inégalités, ce qui intervertit 
  finalement les zones où la fonction $f$ est croissante ou décroissante.
\end{proof}

\begin{exo}
  Dans la preuve précédente, rédigez en détails le cas où $a<0$.
\end{exo}

\begin{defi}{Parabole}
  La courbe représentative d'une fonction polynôme du second degré est appelée
  une \emph{parabole}.
\end{defi}

\begin{exemple}
  Reprennons l'exemple de la fonction polynôme du second degré définie sur
  $\mathbb{R}$ par
  \[
    f(x) = x^2+2x+1.
  \]
  On peut retrouver la forme canonique gr\^ace à une identité remarquable, ou
  bien retrouver la forme canonique en se servant des formules. Pour cette
  fonction, on a les coefficients $a, b, c$ qui valent respectivement $1, 2$ et
  $1$. On peut donc calculer
  \[
    \alpha =\frac{-b}{2a}=\frac{-2}{2\times 1} = \frac{-2}{2} = -1,
  \]
  puis on retrouve $\beta$:
  \[
    \beta = f(\alpha) = f(-1) = (-1)^2+2\times(-1)+1 = 1 - 2 + 1 = 0.
  \]
  On retombe bien sur la forme canonique
  \[
    f(x) = a(x-\alpha)^2+\beta = 1\times(x-(-1))^2+0=(x+1)^2.
  \]
  La proposition précédent nous donne alors les variations de la fonction $f$.
  Tout d'abord il faut regarder le signe du coefficient $a$. Ici $a=1$ est
  strictement positif, on est donc dans le cas de gauche. La proposition nous
enseigne donc que la fonction est strictement décroissante sur $]-\infty, -1[$,
puis strictement croissante sur $]-1, +\infty[$. Elle admet comme minimum $0$
    en $-1$. Et, effectivement, on peut vérifier que tout cela est vrai en
    regardant la représentation graphique de la fonction.
 \begin{center}
\begin{tikzpicture}
\begin{axis}
\addplot[red, very thick, samples=201]{x^2+2*x+1};
\end{axis}
%\draw[color=red] (2,4) node {\LARGE{$y=x^2+2x+1$}};
\end{tikzpicture}
\end{center}
\end{exemple}

\begin{exo}
  Pour les fonctions suivantes, déterminez la forme canonique, puis les
  variations et le maximum ou minimum. Vérifiez vos résultats en regardant la
  représentation graphique de la fonction.
  \begin{enumerate}
    \item La fonction $f_1$ définie sur $\mathbb{R}$ par $f_1(x)=x^2-2x+1$.
    \item La fonction $f_2$ définie sur $\mathbb{R}$ par $f_2(x)=-x^2+4x-5$.
  \end{enumerate}
\end{exo}

\section{Équations du second degré}

Jusqu'à maintenant, on s'est intéressé aux expressions du second degré
\[
  ax^2+bx+c
\]
comme des fonctions, et on a donc cherché à comprendre le sens de variation de
ces expressions lorsque la variable $x$ varie. On va maintenant changer un peu
de point de vue et considérer la variable $x$ comme une inconnue: l'expression
sera alors considérée comme une équation
\[
  ax^2+bx+c = 0.
\]

\subsection{Définitions} 
\begin{defi}{Équation du second degré}
  Une \emph{équation du second degré} à coefficients réels est une équation de
  la forme
  \[
    ax^2+bx+c = 0,
  \]
  avec $a,b,c\in\mathbb{R}$ trois réels et $a\neq0$.
\end{defi}

\begin{defi}{Racines d'une équation}
  Les solutions de l'équation du second degré $ax^2+bx+c=0$ sont appelées les
  \emph{racines} du trinôme $ax^2+bx+c$. De la même manière, on parle de racine
  pour le polynôme défini par $f(x)=ax^2+bx+c$
\end{defi}


\begin{exemple}
  L'équation $2x^2-x+3=0$ est une équation du second degré avec $a=2, b=-1, c=3$.
\end{exemple}

\begin{exemple}
  L'équation $x^2-2=0$ est une équation du second degré avec $a=1, b=0, c=-2$.
  Cette équation admet deux racines : $\sqrt 2$ et $-\sqrt 2$.
\end{exemple}

\begin{rmq}
  Il y a (bien entendu) un lien entre les deux points de vue que nous avons
  développé : si $f$ est une fonction polynôme du second degré définie sur
  $\mathbb{R}$ par
  \[
    f(x) = ax^2+bx+c,
  \]
  alors les racines de l'équation du second degré
  \[
    ax^2+bx+c = 0
  \]
  correspondent aux abcisses des points où la courbe représentative de la
  fonction $f$ passe par l'axe des abcisses. Ces points ont été mis en
  évidence dans le graphique ci-dessous, qui correspond à la fonction définie
  par $f(x) = x^2-1$.
 \begin{center}
  \begin{tikzpicture}
    \begin{axis}
      \addplot[red, very thick, samples=201]{x^2-1};
    \end{axis}
    \filldraw[blue] (2.25,2.75) circle (1.5pt);
    \filldraw[blue] (3.25,2.75) circle (1.5pt);
  \end{tikzpicture}
\end{center}
La parabole coupe l'axe des abcisses en les points de coordonnées $(-1, 0)$ et
$(1, 0)$, et on remarque que $-1$ et $1$ sont des racines de l'équation du second
degré
\[
  x^2-1 = 0.
\]
\end{rmq}
La remarque ci-dessus induit aussi la définition suivante.
\begin{defi}{Racine d'une fonction polynôme}
  Soit $f$ la fonction polynôme définie sur $\mathbb{R}$ par
  \[
    f(x) = ax^2+bx+c.
  \]
  On dit que la valeur $x_0$ est une \emph{racine} de $f$ si
  \[
    f(x_0) = 0.
  \]
\end{defi}

\begin{rmq}
  Les deux définitions de racines sont équivalentes. Ainsi, quand on parle de racine d'un
  trinôme $$ax^2+bx+c,$$ on peut parler indifféremment de racine de l'équation
  associée au trinôme, ou de racine de la fonction polynôme associée, cela ne
  change rien.
\end{rmq}

\subsection{Résolution des équations du second degré dans $\mathbb{R}$}

On va maintenant apprendre à résoudre dans $\mathbb{R}$ les équations du second
degré, c'est-à-dire à trouver des solutions réelles à nos équations.

\begin{defi}{Discriminant}
  Le \emph{discriminant} du trinôme $ax^2+bx+c$, noté $\Delta$ (delta
  majuscule), est le nombre
  \[
    \Delta = b^2-4ac.
  \]
\end{defi}

\begin{prop}
  Soit $\Delta=b^2-4ac$ le discriminant du trinôme $ax^2+bx+c$.
  \begin{itemize}
    \item Si $\Delta<0$, alors l'équation $ax^2+bx+c=0$ n'a pas de solutions
      dans $\mathbb{R}$.
    \item Si $\Delta=0$, alors l'équation $ax^2+bx+c=0$ a une unique solution
      \[x_0 = \frac{-b}{2a}.\] On dit que $x_0$ est \emph{racine double} du trinôme.
    \item Si $\Delta>0$, alors l'équation $ax^2+bx+c=0$ admet deux solutions
      distinctes : 
      \[
        x_1=\frac{-b-\sqrt\Delta}{2a}
        \text{ et }
        x_2=\frac{-b+\sqrt\Delta}{2a}.
      \]
  \end{itemize}
\end{prop}

\begin{proof}
  Soit $f$ la fonction polynôme du second degré définie sur $\mathbb{R}$ par
  \[
    f(x) = ax^2+bx+c.
  \]
  On a vu au cours de la démonstration concernant la forme canonique que l'on
  pouvait écrire
  \[
    f(x) = a\left(x+\frac{b}{2a}\right)^2-\frac{b^2-4ac}{4a} =
    a\left(x+\frac{b}{2a}\right)^2-\frac{\Delta}{4a} .
  \]
  En raisonnant \emph{par équivalence}, on a ainsi
  \begin{align*}
    f(x) = 0 &\Longleftrightarrow a\left( x+\frac{b}{2a}
    \right)^2-\frac{\Delta}{4a}=0 \\
    &\Longleftrightarrow a\left( x+\frac{b}{2a}
    \right)^2=\frac{\Delta}{4a} \\
    &\Longleftrightarrow \left( x+\frac{b}{2a}
    \right)^2=\frac{\Delta}{4a^2}.
  \end{align*}
  On peut alors différencier plusieurs cas.
  \paragraph{Premier cas : $\Delta<0$.} Dans ce cas on a
  $\frac{\Delta}{4a^2}<0$. Mais on a aussi $\left( x+\frac{b}{2a} \right)^2\geq0$
  car un carré est toujours positif ou nul. L'équation n'a donc pas de solution.
  \paragraph{Deuxième cas : $\Delta=0$.} Dans ce cas l'équation devient
  \begin{align*}
    f(x) = 0 &\Longleftrightarrow \left( x+\frac{b}{2a} \right)^2 = 0 \\
    &\Longleftrightarrow \left( x+\frac{b}{2a} \right) = 0 \\
    &\Longleftrightarrow x = \frac{-b}{2a}.
  \end{align*}
  L'équation a donc une unique solution, donnée par $x_0=\cfrac{-b}{2a}$.
  \paragraph{Troisième cas : $\Delta>0$.} Cette fois-ci on a
  \begin{align*}
    f(x) = 0 &\Longleftrightarrow \left( x+\frac{b}{2a} \right)^2 =
    \frac{\Delta}{4a^2} \\
    &\Longleftrightarrow \left( x+\frac{b}{2a} \right)^2 =
    \left( \frac{\sqrt\Delta}{2a} \right)^2 \\
    &\Longleftrightarrow x+\frac{b}{2a}=\frac{\sqrt\Delta}{2a}\text{ ou
    }x+\frac{b}{2a}=-\frac{\sqrt\Delta}{2a}\\
    &\Longleftrightarrow x=-\frac{b}{2a}+\frac{\sqrt\Delta}{2a}\text{ ou
    }x=-\frac{b}{2a}+\frac{\sqrt\Delta}{2a}.
  \end{align*}
  Et on obtient finalement deux solutions distinctes, données par
  \[
    x_1 = \frac{-b-\sqrt\Delta}{2a}\text{ et }x_2 =
    \frac{-b+\sqrt\Delta}{2a}.
  \]
\end{proof}

\begin{exemple}
  Considérons l'équation du second degré
  \[
    x^2+2x-3 = 0.
  \]
  On a ici $a=1, b=2, c=-3$. On commence par calculer le discriminant de cette
  équation, on obtient
  \[
    \Delta = b^2-4ac = 2^2 - 4\times1\times(-3)=16.
  \]
  On est dans le cas où $\Delta>0$ et on sait qu'on a ainsi deux solutions
  distinctes:
  \begin{align*}
    x_1 &= \frac{-b-\sqrt\Delta}{2a} \\
    &= \frac{-2-4}{2}\\
    &= \frac{-6}{2}\\
    &= -3
  \end{align*}
  et
  \begin{align*}
    x_2 &= \frac{-b+\sqrt\Delta}{2a} \\
    &= \frac{-2+4}{2}\\
    &= \frac{2}{2}\\
    &= 1.
  \end{align*}
  L'ensemble des solutions de l'équation est donc $\mathscr S = \left\{ -3, 1
  \right\}$.
\end{exemple}

\section{Propriétés d'un trinôme $ax^2+bx+c$}

Maintenant que nous savons le comportement d'un trinôme vu comme un fonction et
que nous savons résoudre une équation du second degré, nous allons donner
quelques propriétés supplémentaires concernant les trinômes.

\subsection{Factorisation}

Nous avons pour
l'instant étudié la forme développée et la forme canonique d'un trinôme, mais il
en existe encore une autre : la forme \emph{factorisée}.

\begin{prop}
  Soit $f$ une fonction polynôme du second degré définie sur $\mathbb{R}$ par
  \[
    f(x) = ax^2+bx+c.
  \]
  \begin{itemize}
    \item Si $\Delta>0$, alors $f(x)=a(x-x_1)(x-x_2)$ où $x_1$ et $x_2$ sont les
      racines de $f$.
    \item Si $\Delta=0$, alors $f(x)=a(x-x_0)^2$ où $x_0$ est la racine double de
      $f$.
    \item Si $\Delta<0$, alors la fonction $f$ ne peut pas s'écrire comme un
      produit de deux polynômes de degré $1$.
  \end{itemize}
\end{prop}

\begin{proof}
  On a vu que l'on pouvait toujours écrire $f$ sous la forme
\begin{align*}
  f(x) &= a\left( x+\frac{b}{2a} \right)^2-\frac{\Delta}{4a} \\
  &= a\left[ \left( x+\frac{b}{2a} \right)^2-\frac{\Delta}{4a^2} \right].
\end{align*}
\paragraph{Si $\Delta>0$,} on peut alors utiliser l'identité remarquable
$y^2-z^2=(y+z)(y-z)$, et on a
\begin{align*}
  f(x) &= a\left[ \left( x+\frac{b}{2a} \right)^2-\frac{\Delta}{4a^2} \right] \\
  &= a\left( x+\frac{b}{2a}+\frac{\sqrt\Delta}{2a} \right)\left(
  x+\frac{b}{2a}-\frac{\sqrt\Delta}{2a}\right) \\
  &= a\left( x-\frac{-b-\sqrt\Delta}{2a}\right)\left(
  x-\frac{-b+\sqrt\Delta}{2a}\right) \\
  &= a(x-x_1)(x-x_2).
\end{align*}
\paragraph{Si $\Delta=0$,} on a alors
\begin{align*}
  f(x) &= a\left(x+\frac{b}{2a}  \right)^2 \\
  &= a\left(x-\frac{-b}{2a}  \right)^2 \\
  &= a(x-x_0)^2.
\end{align*}
\end{proof}

\begin{exemple}
  On a vu précédemment que la fonction $f$ définie par
  \[
    f(x) = x^2+2x-3
  \]
  avait pour discriminant $\Delta=16$ et pour racines $x_1=-3$ et $x_2=1$. Cela
  signifie que l'on peut aussi écrire $f$ sous la forme factorisée
  \[
    f(x) = (x+3)(x-1).
  \]
\end{exemple}

\subsection{Somme et produit de racines}
\begin{prop}
  Soit $a,b,c\in\mathbb{R}$ et soit $f$ une fonction polynôme de degré $2$
  définie sur $\mathbb{R}$ par
  \[
    f(x)=ax^2+bx+c,
  \]
  dont le discriminant est strictement positif. La fonction $f$ a alors deux
  racines distinctes $x_1$ et $x_2$ et on a
  \[
    x_1+x_2 = \frac{-b}{a}\text{ et }x_1\times x_2=\frac{c}{a}.
  \]
\end{prop}
\begin{proof}
  D'après la proposition sur la factorisation, on sait que l'on peut écrire
  \[
    f(x) = a(x-x_1)(x-x_2).
  \]
  En développant, on obtient ainsi
  \begin{align*}
    f(x) &= ax^2 - axx_1 - axx_2+ax_1x_2 \\
    &= ax^2 - a(x_1+x_2)x + ax_1x_2
  \end{align*}
  D'autre part, on a $f(x) = ax^2+bx+c$, ainsi en identifiant les coefficients
  des deux écritures, on obtient
  \[
    -a(x_1+x_2) = b\text{ et }ax_1x_2=c,
  \]
  puis on a bien
  \[
    x_1+x_2 = \frac{-b}{a}\text{ et }x_1\times x_2=\frac{c}{a}.
  \]
\end{proof}

\subsection{Signe d'une fonction polynôme du second degré}
\begin{prop}
  Soit $f$ une fonction polynôme du second degré définie sur $\mathbb{R}$ par
  \[
    f(x)=ax^2+bx+c,
  \]
  de déterminant $\Delta$.
  \begin{itemize}
    \item Si $\Delta<0$, alors pour tout réel $x\in\mathbb{R}$, $f(x)$ est du
      signe de $a$.
    \item Si $\Delta=0$, alors pour tout réel $x\neq\frac{-b}{2a}$, $f(x)$ est
      du signe de $a$, et $f(\frac{-b}{2a})=0$.
    \item Si $\Delta>0$, alors on a les tableaux de signe suivants.

      \noindent
  \begin{minipage}[t]{.47\textwidth}
    \begin{center}
      {\bf Si} $\mathbf{a>0}$\vspace{.2cm}

      \begin{tikzpicture}
        \tkzTabInit[lgt=1, espcl=1.5]{$x$ / .7, $f(x)$ / 1.4}{$-\infty$, $x_1$,
        $x_2$, $+\infty$}
        \tkzTabLine{, +, z , -, z, +,}
      \end{tikzpicture}
    \end{center}
  \vspace{.2cm}
  \end{minipage}
    \hfill
  \begin{minipage}[t]{.47\textwidth}
    \begin{center}
      {\bf Si} $\mathbf{a<0}$\vspace{.2cm}

      \begin{tikzpicture}
        \tkzTabInit[lgt=1, espcl=1.5]{$x$ / .7, $f(x)$ / 1.4}{$-\infty$, $x_1$,
        $x_2$, $+\infty$}
        \tkzTabLine{, -, z , +, z, -,}
      \end{tikzpicture}
    \end{center}
  \vspace{.2cm}
  \end{minipage}
  On peut retenir que dans ce cas, $f$ est du signe de $a$, sauf entre ses
  racines.
  \end{itemize}
\end{prop}
\begin{exemple}
  On peut reprendre la fonction $f$ définie sur $\mathbb{R}$ par
  \[
    f(x) = x^2+2x-3.
  \]
  On a vu que le déterminant de $f$ vaut $16$ et que les deux racines de $f$
  sont données par $x_1=-3$ et $x_2=1$. On a ainsi le tableau de signe suivant.
  \begin{center}
      \begin{tikzpicture}
        \tkzTabInit[lgt=1, espcl=1.5]{$x$ / .7, $f(x)$ / 1.4}{$-\infty$, $-3$,
        $1$, $+\infty$}
        \tkzTabLine{, +, z , -, z, +,}
      \end{tikzpicture}
  \end{center}
  Et, en effet, cela concorde avec la représentation graphique de $f$ donnée
  ci-dessous.
  \begin{center}
\begin{tikzpicture}
\begin{axis}
  \addplot[red, very thick, samples=201, domain=-5:-3]{x^2+2*x-3};
  \addplot[blue, very thick, samples=201, domain=-3:1]{x^2+2*x-3};
  \addplot[red, very thick, samples=201, domain=1:5]{x^2+2*x-3};
\end{axis}
\filldraw[black] (1.25, 2.75) circle (2pt);
\filldraw[black] (3.25, 2.75) circle (2pt);
\end{tikzpicture}
  \end{center}
\end{exemple}

\end{document}
