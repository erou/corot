\documentclass[11pt]{article}

\newcommand{\titrechapitre}{Généralités sur les fonctions -- Cours}
\newcommand{\titreclasse}{Lycée Jean-Baptiste \textsc{Corot}}
\newcommand{\pagination}{\thepage/\pageref{LastPage}}
\newcommand{\topbotmargins}{2cm}
\newcommand{\spacebelowexo}{5mm}

%%%%%%%%%%%%%%%%%%%%%%%%%%%%%%%%%%%%%%%%%%%%%%%%%%%%%%%%%%%%%%%%%%%%%%%%%%%%%%%%
%
% PACKAGES
% ========
%
%%%%%%%%%%%%%%%%%%%%%%%%%%%%%%%%%%%%%%%%%%%%%%%%%%%%%%%%%%%%%%%%%%%%%%%%%%%%%%%%

\usepackage[english, french]{babel}
\usepackage[utf8]{inputenc}
\usepackage[T1]{fontenc}
\usepackage{graphicx}
\usepackage{amsmath,amssymb,amsthm,amsopn}
\usepackage{hyperref}

% Pour avoir l'écriture \mathscr (math script)
% ============================================

\usepackage{mathrsfs}

% Deal with coma as a decimal separator
% =====================================

\usepackage{icomma}

% Package Geometry
% ================

\usepackage[a4paper, lmargin=2cm, rmargin=2cm, top=\topbotmargins, bottom=\topbotmargins]{geometry}

% Package multicol
% ================

\usepackage{multicol}

% Redefine abstract
% =================

% Note
% ====
%
% Le reste a été commenté pour ne pas charger trop de choses au démarrage. On
% verra si on en a besoin plus tard.
%
% --------
%
%\usepackage{mathrsfs}
%\usepackage{multirow}
%\usepackage{bm}
%\hypersetup{
%    colorlinks=true,
%    linkcolor=blue,
%    citecolor=red,
%}
%\usepackage{diagbox}
%
%\usepackage{algorithm}
%\usepackage{algpseudocode}
%
%\renewcommand{\algorithmicrequire}{\textbf{Input:}}
%\renewcommand{\algorithmicensure}{\textbf{Output:}}


%%%%%%%%%%%%%%%%%%%%%%%%%%%%%%%%%%%%%%%%%%%%%%%%%%%%%%%%%%%%%%%%%%%%%%%%%%%%%%%%
%
% TIKZ
% ====
%
%%%%%%%%%%%%%%%%%%%%%%%%%%%%%%%%%%%%%%%%%%%%%%%%%%%%%%%%%%%%%%%%%%%%%%%%%%%%%%%%

\usepackage{tikz}
\usetikzlibrary{arrows}

\usepackage{tkz-tab} % Variation tables

\usepackage{pgfplots}
%\usepackage{pgf-pie} % Pie charts

\pgfplotsset{
%\newcommand{\settingsgraph}{
x=.5cm,y=.5cm,
xticklabel style = {font=\scriptsize, yshift=.1cm},
yticklabel style = {font=\scriptsize, xshift=.1cm},
axis lines=middle,
ymajorgrids=true,
xmajorgrids=true,
major grid style = {color=white!80!blue},
xmin=-5.5,
xmax=5.5,
ymin=-5.5,
ymax=5.5,
xtick={-5.0,-4.0,...,5.0},
ytick={-5.0,-4.0,...,5.0},
}

% Tikz style

\tikzset{round/.style={circle, draw=black, very thick, scale = 0.7}}
\tikzset{arrow/.style={->, >=latex}}
\tikzset{dashed-arrow/.style={->, >=latex, dashed}}

\newcommand{\point}[3]{\draw[very thick, #3] (#1-.1, #2)--(#1+.1, #2)
(#1, #2-.1)--(#1, #2+.1)}

%%%%%%%%%%%%%%%%%%%%%%%%%%%%%%%%%%%%%%%%%%%%%%%%%%%%%%%%%%%%%%%%%%%%%%%%%%%%%%%%
%
% FANCY HEADER
% ============
%
%%%%%%%%%%%%%%%%%%%%%%%%%%%%%%%%%%%%%%%%%%%%%%%%%%%%%%%%%%%%%%%%%%%%%%%%%%%%%%%%


\usepackage{fancyhdr}
\usepackage{lastpage}

\pagestyle{fancy}
\newcommand{\changefont}{\fontsize{9}{9}\selectfont}
\renewcommand{\headrulewidth}{0mm}
\renewcommand{\footrulewidth}{0mm}

\fancyhead[C]{}
\fancyhead[L]{\titreclasse}
\fancyhead[R]{\titrechapitre}
\fancyfoot[C]{}
\fancyfoot[L]{}
\fancyfoot[R]{\pagination}
\addtolength{\skip\footins}{20pt} % distance between text and footnotes

%%%%%%%%%%%%%%%%%%%%%%%%%%%%%%%%%%%%%%%%%%%%%%%%%%%%%%%%%%%%%%%%%%%%%%%%%%%%%%%%
%
% THEOREM STYLE
% =============
%
%%%%%%%%%%%%%%%%%%%%%%%%%%%%%%%%%%%%%%%%%%%%%%%%%%%%%%%%%%%%%%%%%%%%%%%%%%%%%%%%

\usepackage[tikz]{bclogo}
\usepackage{mdframed}

\usepackage{tcolorbox}
\tcbuselibrary{listings, breakable, theorems, skins}

%\newtheoremstyle{break}%
%{}{}%
%{\itshape}{}%
%{\bfseries}{}%  % Note that final punctuation is omitted.
%{\newline}{}

\newtheoremstyle{scbf}%
{}{}%
{}{}%
%{\scshape}{}%  % Note that final punctuation is omitted.
{\bfseries\scshape}{}%  % Note that final punctuation is omitted.
{\newline}{}

%\theoremstyle{break}
%\theoremstyle{plain}
%\newtheorem{thm}{Theorem}[section]
%\newtheorem{lm}[thm]{Lemma}
%\newtheorem{prop}[thm]{Proposition}
%\newtheorem{cor}[thm]{Corollary}

%\theoremstyle{scbf}
%\newtheorem{exo}{$\star$ Exercice}

%\theoremstyle{definition}
%\newtheorem{defi}[thm]{Definition}
%\newtheorem{ex}[thm]{Example}

%\theoremstyle{remark}
%\newtheorem{rem}[thm]{Remark}

% Defining the Remark environment
% ===============================

\newenvironment{rmq}
  {
    \begin{bclogo}[logo=\bcinfo, noborder=true]{Remarque}
  }
  {
    \end{bclogo}
  }

% Defining the exercise environment
% =================================

\newcounter{exos}
\setcounter{exos}{1}

\newenvironment{exo}
  {
    \begin{bclogo}[logo=\bccrayon, noborder=true]{Exercice \theexos}
  }
  {
    \end{bclogo}
    \addtocounter{exos}{1}
  }


% Redefining the proof environment from amsthm
% ============================================

\tcolorboxenvironment{proof}{
  blanker, breakable, before skip=10pt,after skip=10pt,
  borderline west={1mm}{0pt}{red},
  left=5mm,
}

% Defining the definition environment
% ===================================

\colorlet{coldef}{black!50!green}

\newcounter{defis}
\setcounter{defis}{1}

\newenvironment{defi}[1]
  {
    \begin{defihid}{{#1}}{\thedefis}
  }
  {
    \end{defihid}
    \addtocounter{defis}{1}
  }

\newtcolorbox{defihid}[2]{%
  empty,title={ {\bfseries Définition {#2}} ({#1})},attach boxed title to top left,
boxed title style={empty,size=minimal,toprule=2pt,top=4pt,
overlay={\draw[coldef,line width=2pt]
([yshift=-1pt]frame.north west)--([yshift=-1pt]frame.north east);}},
coltitle=coldef,
before=\par\medskip\noindent,parbox=false,boxsep=0pt,left=0pt,right=3mm,top=4pt,
breakable,pad at break*=0mm,vfill before first,
overlay unbroken={\draw[coldef,line width=1pt]
([yshift=-1pt]title.north east)--([xshift=-0.5pt,yshift=-1pt]title.north-|frame.east)
--([xshift=-0.5pt]frame.south east)--(frame.south west); },
overlay first={\draw[coldef,line width=1pt]
([yshift=-1pt]title.north east)--([xshift=-0.5pt,yshift=-1pt]title.north-|frame.east)
--([xshift=-0.5pt]frame.south east); },
overlay middle={\draw[coldef,line width=1pt] ([xshift=-0.5pt]frame.north east)
--([xshift=-0.5pt]frame.south east); },
overlay last={\draw[coldef,line width=1pt] ([xshift=-0.5pt]frame.north east)
--([xshift=-0.5pt]frame.south east)--(frame.south west);},%
}

\newenvironment{notation}
  {
    \begin{notationhid}{\thedefis}
  }
  {
    \end{notationhid}
    \addtocounter{defis}{1}
  }

\newtcolorbox{notationhid}[1]{%
  empty,title={Notation {#1}},attach boxed title to top left,
boxed title style={empty,size=minimal,toprule=2pt,top=4pt,
overlay={\draw[coldef,line width=2pt]
([yshift=-1pt]frame.north west)--([yshift=-1pt]frame.north east);}},
coltitle=coldef,fonttitle=\bfseries,
before=\par\medskip\noindent,parbox=false,boxsep=0pt,left=0pt,right=3mm,top=4pt,
breakable,pad at break*=0mm,vfill before first,
overlay unbroken={\draw[coldef,line width=1pt]
([yshift=-1pt]title.north east)--([xshift=-0.5pt,yshift=-1pt]title.north-|frame.east)
--([xshift=-0.5pt]frame.south east)--(frame.south west); },
overlay first={\draw[coldef,line width=1pt]
([yshift=-1pt]title.north east)--([xshift=-0.5pt,yshift=-1pt]title.north-|frame.east)
--([xshift=-0.5pt]frame.south east); },
overlay middle={\draw[coldef,line width=1pt] ([xshift=-0.5pt]frame.north east)
--([xshift=-0.5pt]frame.south east); },
overlay last={\draw[coldef,line width=1pt] ([xshift=-0.5pt]frame.north east)
--([xshift=-0.5pt]frame.south east)--(frame.south west);},%
}


% Defining the proposition, theorem, etc. environment
% ===================================================

\colorlet{colprop}{red!75!black}

\newcounter{props}
\setcounter{props}{1}

\newenvironment{prop}
  {
    \begin{prophid}{\theprops}
  }
  {
    \end{prophid}
    \refstepcounter{props}
  }

\newtcolorbox{prophid}[1]{%
empty,title={Propriété {#1}},attach boxed title to top left,
boxed title style={empty,size=minimal,toprule=2pt,top=4pt,
overlay={\draw[colprop,line width=2pt]
([yshift=-1pt]frame.north west)--([yshift=-1pt]frame.north east);}},
coltitle=colprop,fonttitle=\bfseries,
before=\par\medskip\noindent,parbox=false,boxsep=0pt,left=0pt,right=3mm,top=4pt,
breakable,pad at break*=0mm,vfill before first,
overlay unbroken={\draw[colprop,line width=1pt]
([yshift=-1pt]title.north east)--([xshift=-0.5pt,yshift=-1pt]title.north-|frame.east)
--([xshift=-0.5pt]frame.south east)--(frame.south west); },
overlay first={\draw[colprop,line width=1pt]
([yshift=-1pt]title.north east)--([xshift=-0.5pt,yshift=-1pt]title.north-|frame.east)
--([xshift=-0.5pt]frame.south east); },
overlay middle={\draw[colprop,line width=1pt] ([xshift=-0.5pt]frame.north east)
--([xshift=-0.5pt]frame.south east); },
overlay last={\draw[colprop,line width=1pt] ([xshift=-0.5pt]frame.north east)
--([xshift=-0.5pt]frame.south east)--(frame.south west);},%
}

\newenvironment{propadm}
  {
    \begin{propadmhid}{\theprops}
  }
  {
    \end{propadmhid}
    \refstepcounter{props}
  }

  \newtcolorbox{propadmhid}[1]{%
    empty,title={{\bfseries Propriété {#1}} (admise)},attach boxed title to top left,
boxed title style={empty,size=minimal,toprule=2pt,top=4pt,
overlay={\draw[colprop,line width=2pt]
([yshift=-1pt]frame.north west)--([yshift=-1pt]frame.north east);}},
coltitle=colprop,%fonttitle=\bfseries,
before=\par\medskip\noindent,parbox=false,boxsep=0pt,left=0pt,right=3mm,top=4pt,
breakable,pad at break*=0mm,vfill before first,
overlay unbroken={\draw[colprop,line width=1pt]
([yshift=-1pt]title.north east)--([xshift=-0.5pt,yshift=-1pt]title.north-|frame.east)
--([xshift=-0.5pt]frame.south east)--(frame.south west); },
overlay first={\draw[colprop,line width=1pt]
([yshift=-1pt]title.north east)--([xshift=-0.5pt,yshift=-1pt]title.north-|frame.east)
--([xshift=-0.5pt]frame.south east); },
overlay middle={\draw[colprop,line width=1pt] ([xshift=-0.5pt]frame.north east)
--([xshift=-0.5pt]frame.south east); },
overlay last={\draw[colprop,line width=1pt] ([xshift=-0.5pt]frame.north east)
--([xshift=-0.5pt]frame.south east)--(frame.south west);},%
}

\newenvironment{propnom}[1]
  {
    \begin{propnomhid}{#1}{\theprops}
  }
  {
    \end{propnomhid}
    \refstepcounter{props}
  }

\newtcolorbox{propnomhid}[2]{%
empty,title={{\bfseries Propriété {#2}} ({#1})},attach boxed title to top left,
boxed title style={empty,size=minimal,toprule=2pt,top=4pt,
overlay={\draw[colprop,line width=2pt]
([yshift=-1pt]frame.north west)--([yshift=-1pt]frame.north east);}},
coltitle=colprop,
before=\par\medskip\noindent,parbox=false,boxsep=0pt,left=0pt,right=3mm,top=4pt,
breakable,pad at break*=0mm,vfill before first,
overlay unbroken={\draw[colprop,line width=1pt]
([yshift=-1pt]title.north east)--([xshift=-0.5pt,yshift=-1pt]title.north-|frame.east)
--([xshift=-0.5pt]frame.south east)--(frame.south west); },
overlay first={\draw[colprop,line width=1pt]
([yshift=-1pt]title.north east)--([xshift=-0.5pt,yshift=-1pt]title.north-|frame.east)
--([xshift=-0.5pt]frame.south east); },
overlay middle={\draw[colprop,line width=1pt] ([xshift=-0.5pt]frame.north east)
--([xshift=-0.5pt]frame.south east); },
overlay last={\draw[colprop,line width=1pt] ([xshift=-0.5pt]frame.north east)
--([xshift=-0.5pt]frame.south east)--(frame.south west);},%
}




\newenvironment{thm}
  {
    \begin{thmhid}{\theprops}
  }
  {
    \end{thmhid}
    \refstepcounter{props}
  }

\newtcolorbox{thmhid}[1]{%
empty,title={Théorème {#1}},attach boxed title to top left,
boxed title style={empty,size=minimal,toprule=2pt,top=4pt,
overlay={\draw[colprop,line width=2pt]
([yshift=-1pt]frame.north west)--([yshift=-1pt]frame.north east);}},
coltitle=colprop,fonttitle=\bfseries,
before=\par\medskip\noindent,parbox=false,boxsep=0pt,left=0pt,right=3mm,top=4pt,
breakable,pad at break*=0mm,vfill before first,
overlay unbroken={\draw[colprop,line width=1pt]
([yshift=-1pt]title.north east)--([xshift=-0.5pt,yshift=-1pt]title.north-|frame.east)
--([xshift=-0.5pt]frame.south east)--(frame.south west); },
overlay first={\draw[colprop,line width=1pt]
([yshift=-1pt]title.north east)--([xshift=-0.5pt,yshift=-1pt]title.north-|frame.east)
--([xshift=-0.5pt]frame.south east); },
overlay middle={\draw[colprop,line width=1pt] ([xshift=-0.5pt]frame.north east)
--([xshift=-0.5pt]frame.south east); },
overlay last={\draw[colprop,line width=1pt] ([xshift=-0.5pt]frame.north east)
--([xshift=-0.5pt]frame.south east)--(frame.south west);},%
}

\newenvironment{thmadm}
  {
    \begin{thmadmhid}{\theprops}
  }
  {
    \end{thmadmhid}
    \refstepcounter{props}
  }

  \newtcolorbox{thmadmhid}[1]{%
    empty,title={{\bfseries Théorème {#1}} (admis)},attach boxed title to top left,
boxed title style={empty,size=minimal,toprule=2pt,top=4pt,
overlay={\draw[colprop,line width=2pt]
([yshift=-1pt]frame.north west)--([yshift=-1pt]frame.north east);}},
coltitle=colprop,%fonttitle=\bfseries,
before=\par\medskip\noindent,parbox=false,boxsep=0pt,left=0pt,right=3mm,top=4pt,
breakable,pad at break*=0mm,vfill before first,
overlay unbroken={\draw[colprop,line width=1pt]
([yshift=-1pt]title.north east)--([xshift=-0.5pt,yshift=-1pt]title.north-|frame.east)
--([xshift=-0.5pt]frame.south east)--(frame.south west); },
overlay first={\draw[colprop,line width=1pt]
([yshift=-1pt]title.north east)--([xshift=-0.5pt,yshift=-1pt]title.north-|frame.east)
--([xshift=-0.5pt]frame.south east); },
overlay middle={\draw[colprop,line width=1pt] ([xshift=-0.5pt]frame.north east)
--([xshift=-0.5pt]frame.south east); },
overlay last={\draw[colprop,line width=1pt] ([xshift=-0.5pt]frame.north east)
--([xshift=-0.5pt]frame.south east)--(frame.south west);},%
}

\newenvironment{thmnom}[1]
  {
    \begin{thmnomhid}{#1}{\theprops}
  }
  {
    \end{thmnomhid}
    \refstepcounter{props}
  }

\newtcolorbox{thmnomhid}[2]{%
empty,title={{\bfseries Théorème {#2}} ({#1})},attach boxed title to top left,
boxed title style={empty,size=minimal,toprule=2pt,top=4pt,
overlay={\draw[colprop,line width=2pt]
([yshift=-1pt]frame.north west)--([yshift=-1pt]frame.north east);}},
coltitle=colprop,
before=\par\medskip\noindent,parbox=false,boxsep=0pt,left=0pt,right=3mm,top=4pt,
breakable,pad at break*=0mm,vfill before first,
overlay unbroken={\draw[colprop,line width=1pt]
([yshift=-1pt]title.north east)--([xshift=-0.5pt,yshift=-1pt]title.north-|frame.east)
--([xshift=-0.5pt]frame.south east)--(frame.south west); },
overlay first={\draw[colprop,line width=1pt]
([yshift=-1pt]title.north east)--([xshift=-0.5pt,yshift=-1pt]title.north-|frame.east)
--([xshift=-0.5pt]frame.south east); },
overlay middle={\draw[colprop,line width=1pt] ([xshift=-0.5pt]frame.north east)
--([xshift=-0.5pt]frame.south east); },
overlay last={\draw[colprop,line width=1pt] ([xshift=-0.5pt]frame.north east)
--([xshift=-0.5pt]frame.south east)--(frame.south west);},%
}

\newenvironment{coro}
  {
    \begin{corohid}{\theprops}
  }
  {
    \end{corohid}
    \refstepcounter{props}
  }

  \newtcolorbox{corohid}[1]{%
  empty,title={Corollaire {#1}},attach boxed title to top left,
boxed title style={empty,size=minimal,toprule=2pt,top=4pt,
overlay={\draw[colprop,line width=2pt]
([yshift=-1pt]frame.north west)--([yshift=-1pt]frame.north east);}},
coltitle=colprop,fonttitle=\bfseries,
before=\par\medskip\noindent,parbox=false,boxsep=0pt,left=0pt,right=3mm,top=4pt,
breakable,pad at break*=0mm,vfill before first,
overlay unbroken={\draw[colprop,line width=1pt]
([yshift=-1pt]title.north east)--([xshift=-0.5pt,yshift=-1pt]title.north-|frame.east)
--([xshift=-0.5pt]frame.south east)--(frame.south west); },
overlay first={\draw[colprop,line width=1pt]
([yshift=-1pt]title.north east)--([xshift=-0.5pt,yshift=-1pt]title.north-|frame.east)
--([xshift=-0.5pt]frame.south east); },
overlay middle={\draw[colprop,line width=1pt] ([xshift=-0.5pt]frame.north east)
--([xshift=-0.5pt]frame.south east); },
overlay last={\draw[colprop,line width=1pt] ([xshift=-0.5pt]frame.north east)
--([xshift=-0.5pt]frame.south east)--(frame.south west);},%
}

\newenvironment{lemme}
  {
    \begin{lemmehid}{\theprops}
  }
  {
    \end{lemmehid}
    \refstepcounter{props}
  }

  \newtcolorbox{lemmehid}[1]{%
  empty,title={Lemme {#1}},attach boxed title to top left,
boxed title style={empty,size=minimal,toprule=2pt,top=4pt,
overlay={\draw[colprop,line width=2pt]
([yshift=-1pt]frame.north west)--([yshift=-1pt]frame.north east);}},
coltitle=colprop,fonttitle=\bfseries,
before=\par\medskip\noindent,parbox=false,boxsep=0pt,left=0pt,right=3mm,top=4pt,
breakable,pad at break*=0mm,vfill before first,
overlay unbroken={\draw[colprop,line width=1pt]
([yshift=-1pt]title.north east)--([xshift=-0.5pt,yshift=-1pt]title.north-|frame.east)
--([xshift=-0.5pt]frame.south east)--(frame.south west); },
overlay first={\draw[colprop,line width=1pt]
([yshift=-1pt]title.north east)--([xshift=-0.5pt,yshift=-1pt]title.north-|frame.east)
--([xshift=-0.5pt]frame.south east); },
overlay middle={\draw[colprop,line width=1pt] ([xshift=-0.5pt]frame.north east)
--([xshift=-0.5pt]frame.south east); },
overlay last={\draw[colprop,line width=1pt] ([xshift=-0.5pt]frame.north east)
--([xshift=-0.5pt]frame.south east)--(frame.south west);},%
}

\colorlet{colexemple}{blue!50!black}
%\newtcolorbox{exemple}{empty, title=Exemple, attach boxed title to top left,
%  boxed title style={empty, size=minimal, toprule=2pt, top=4pt,
%    overlay={\draw[colexemple,line width=2pt]
%([yshift=-1pt]frame.north west)--([yshift=-1pt]frame.north east);}},
%coltitle=colexemple,fonttitle=\bfseries,%\large\bfseries,
%before=\par\medskip\noindent,parbox=false,boxsep=0pt,left=0pt,right=3mm,top=4pt,
%overlay={\draw[colexemple,line width=1pt]
%([yshift=-1pt]title.north east)--([xshift=-0.5pt,yshift=-1pt]title.north-|frame.east)
%--([xshift=-0.5pt]frame.south east)--(frame.south west); },
%}

\newcounter{exemples}
\setcounter{exemples}{1}

\newenvironment{exemple}
  {
    \begin{exemplehid}{\theexemples}
  }
  {
    \end{exemplehid}
    \addtocounter{exemples}{1}
  }

\newtcolorbox{exemplehid}[1]{%
empty,title={Exemple {#1}},attach boxed title to top left,
boxed title style={empty,size=minimal,toprule=2pt,top=4pt,
overlay={\draw[colexemple,line width=2pt]
([yshift=-1pt]frame.north west)--([yshift=-1pt]frame.north east);}},
coltitle=colexemple,fonttitle=\bfseries,
before=\par\medskip\noindent,parbox=false,boxsep=0pt,left=0pt,right=3mm,top=4pt,
breakable,pad at break*=0mm,vfill before first,
overlay unbroken={\draw[colexemple,line width=1pt]
([yshift=-1pt]title.north east)--([xshift=-0.5pt,yshift=-1pt]title.north-|frame.east)
--([xshift=-0.5pt]frame.south east)--(frame.south west); },
overlay first={\draw[colexemple,line width=1pt]
([yshift=-1pt]title.north east)--([xshift=-0.5pt,yshift=-1pt]title.north-|frame.east)
--([xshift=-0.5pt]frame.south east); },
overlay middle={\draw[colexemple,line width=1pt] ([xshift=-0.5pt]frame.north east)
--([xshift=-0.5pt]frame.south east); },
overlay last={\draw[colexemple,line width=1pt] ([xshift=-0.5pt]frame.north east)
--([xshift=-0.5pt]frame.south east)--(frame.south west);},%
}

\newenvironment{contrex}
  {
    \begin{contrexhid}{\theexemples}
  }
  {
    \end{contrexhid}
    \addtocounter{exemples}{1}
  }

\newtcolorbox{contrexhid}[1]{%
empty,title={Contre-exemple {#1}},attach boxed title to top left,
boxed title style={empty,size=minimal,toprule=2pt,top=4pt,
overlay={\draw[colexemple,line width=2pt]
([yshift=-1pt]frame.north west)--([yshift=-1pt]frame.north east);}},
coltitle=colexemple,fonttitle=\bfseries,
before=\par\medskip\noindent,parbox=false,boxsep=0pt,left=0pt,right=3mm,top=4pt,
breakable,pad at break*=0mm,vfill before first,
overlay unbroken={\draw[colexemple,line width=1pt]
([yshift=-1pt]title.north east)--([xshift=-0.5pt,yshift=-1pt]title.north-|frame.east)
--([xshift=-0.5pt]frame.south east)--(frame.south west); },
overlay first={\draw[colexemple,line width=1pt]
([yshift=-1pt]title.north east)--([xshift=-0.5pt,yshift=-1pt]title.north-|frame.east)
--([xshift=-0.5pt]frame.south east); },
overlay middle={\draw[colexemple,line width=1pt] ([xshift=-0.5pt]frame.north east)
--([xshift=-0.5pt]frame.south east); },
overlay last={\draw[colexemple,line width=1pt] ([xshift=-0.5pt]frame.north east)
--([xshift=-0.5pt]frame.south east)--(frame.south west);},%
}

\newenvironment{app}
  {
    \begin{apphid}{\theexemples}
  }
  {
    \end{apphid}
    \addtocounter{exemples}{1}
  }

\newtcolorbox{apphid}[1]{%
empty,title={Application {#1}},attach boxed title to top left,
boxed title style={empty,size=minimal,toprule=2pt,top=4pt,
overlay={\draw[colexemple,line width=2pt]
([yshift=-1pt]frame.north west)--([yshift=-1pt]frame.north east);}},
coltitle=colexemple,fonttitle=\bfseries,
before=\par\medskip\noindent,parbox=false,boxsep=0pt,left=0pt,right=3mm,top=4pt,
breakable,pad at break*=0mm,vfill before first,
overlay unbroken={\draw[colexemple,line width=1pt]
([yshift=-1pt]title.north east)--([xshift=-0.5pt,yshift=-1pt]title.north-|frame.east)
--([xshift=-0.5pt]frame.south east)--(frame.south west); },
overlay first={\draw[colexemple,line width=1pt]
([yshift=-1pt]title.north east)--([xshift=-0.5pt,yshift=-1pt]title.north-|frame.east)
--([xshift=-0.5pt]frame.south east); },
overlay middle={\draw[colexemple,line width=1pt] ([xshift=-0.5pt]frame.north east)
--([xshift=-0.5pt]frame.south east); },
overlay last={\draw[colexemple,line width=1pt] ([xshift=-0.5pt]frame.north east)
--([xshift=-0.5pt]frame.south east)--(frame.south west);},%
}

%%%%%%%%%%%%%%%%%%%%%%%%%%%%%%%%%%%%%%%%%%%%%%%%%%%%%%%%%%%%%%%%%%%%%%%%%%%%%%%%
%
% ENUMERATE
% =========
%
%%%%%%%%%%%%%%%%%%%%%%%%%%%%%%%%%%%%%%%%%%%%%%%%%%%%%%%%%%%%%%%%%%%%%%%%%%%%%%%%

\usepackage{enumerate}
\usepackage{enumitem}

% To have special enumerate items like
%
% 1/
% 2/
% 3/

%%%%%%%%%%%%%%%%%%%%%%%%%%%%%%%%%%%%%%%%%%%%%%%%%%%%%%%%%%%%%%%%%%%%%%%%%%%%%%%%
%
% ARRAYS
% ======
%
%%%%%%%%%%%%%%%%%%%%%%%%%%%%%%%%%%%%%%%%%%%%%%%%%%%%%%%%%%%%%%%%%%%%%%%%%%%%%%%%


\usepackage{array}
\usepackage{makecell} % Used to break lines within arrays
\usepackage{multirow}
\usepackage{booktabs} % Used to have nice arrays with headrules

%%%%%%%%%%%%%%%%%%%%%%%%%%%%%%%%%%%%%%%%%%%%%%%%%%%%%%%%%%%%%%%%%%%%%%%%%%%%%%%%
%
% WRITE CODE
% ==========
%
%%%%%%%%%%%%%%%%%%%%%%%%%%%%%%%%%%%%%%%%%%%%%%%%%%%%%%%%%%%%%%%%%%%%%%%%%%%%%%%%

\usepackage{listings}
\usepackage{xcolor}

%New colors defined below
\definecolor{codegreen}{rgb}{0,0.6,0}
\definecolor{codegray}{rgb}{0.5,0.5,0.5}
\definecolor{codepurple}{rgb}{0.58,0,0.82}
\definecolor{backcolour}{rgb}{0.95,0.95,0.92}

%Code listing style named "mystyle"
\lstdefinestyle{python}{
  %backgroundcolor=\color{backcolour},
  commentstyle=\color{codegreen},
  keywordstyle=\color{magenta},
  numberstyle=\tiny\color{codegray},
  stringstyle=\color{codepurple},
  basicstyle=\ttfamily\footnotesize,
  breakatwhitespace=false,
  breaklines=true,
  captionpos=b,
  keepspaces=true,
  numbers=left,
  numbersep=5pt,
  showspaces=false,
  showstringspaces=false,
  showtabs=false,
  tabsize=2
}

\lstset{style=python}

%%%%%%%%%%%%%%%%%%%%%%%%%%%%%%%%%%%%%%%%%%%%%%%%%%%%%%%%%%%%%%%%%%%%%%%%%%%%%%%%
%
% Tabular 
% =======
%
%%%%%%%%%%%%%%%%%%%%%%%%%%%%%%%%%%%%%%%%%%%%%%%%%%%%%%%%%%%%%%%%%%%%%%%%%%%%%%%%

% In order to obtain a tabular with given width.

\usepackage{tabularx}
\newcolumntype{Y}{>{\centering\arraybackslash}X}
\newcolumntype{R}{>{\raggedright\arraybackslash}X}
\newcolumntype{L}{>{\raggedleft\arraybackslash}X}
% \usepackage{tabulary} % younger brother

%%%%%%%%%%%%%%%%%%%%%%%%%%%%%%%%%%%%%%%%%%%%%%%%%%%%%%%%%%%%%%%%%%%%%%%%%%%%%%%%
%
% MACROS
% ======
%
%%%%%%%%%%%%%%%%%%%%%%%%%%%%%%%%%%%%%%%%%%%%%%%%%%%%%%%%%%%%%%%%%%%%%%%%%%%%%%%%

% Math Operators

\DeclareMathOperator{\Card}{Card}
\DeclareMathOperator{\Gal}{Gal}
\DeclareMathOperator{\Id}{Id}
\DeclareMathOperator{\Img}{Im}
\DeclareMathOperator{\Ker}{Ker}
\DeclareMathOperator{\Minpoly}{Minpoly}
\DeclareMathOperator{\Mod}{mod}
\DeclareMathOperator{\Ord}{Ord}
\DeclareMathOperator{\ppcm}{ppcm}
\DeclareMathOperator{\pgcd}{pgcd}
\DeclareMathOperator{\tr}{Tr}
\DeclareMathOperator{\Vect}{Vect}
\DeclareMathOperator{\Span}{Span}
\DeclareMathOperator{\rank}{rank}
\DeclareMathOperator{\rg}{rg}
\DeclareMathOperator{\ev}{ev}

% Shortcuts

\newcommand{\eg}{\emph{e.g. }}
\newcommand{\ent}[2]{[\![#1,#2]\!]}
\newcommand{\ie}{\emph{i.e. }}
\newcommand{\ps}[2]{\left\langle#1,#2\right\rangle}
\newcommand{\eqdef}{\overset{\text{def}}{=}}
\newcommand{\E}{\mathcal{E}}
\newcommand{\M}{\mathcal{M}}
\newcommand{\A}{\mathcal{A}}
\newcommand{\B}{\mathcal{B}}
\newcommand{\R}{\mathcal{R}}
\newcommand{\D}{\mathcal{D}}
\newcommand{\Pcal}{\mathcal{P}}
\newcommand{\K}{\mathbf{k}}


\newcommand{\Cf}{\mathscr{C}_f}
\newcommand{\Cg}{\mathscr{C}_g}

\title{\vspace{-16mm}Chapitre 7 : Généralités sur les fonctions}
\date{\vspace{-14mm}
\href{https://erou.forge.aeif.fr/s11/fonctions.html}{
  \includegraphics[scale=.6]{qrcode.png}}
\vspace{-12mm}}
\author{}

\begin{document}
\maketitle\thispagestyle{fancy}

% TODO
% ====
%
% Dans les exercices : ajouter un exercice avec une fonction et demander son
% tableau de variation, car ce n'était pas bien compris en 2022-2023. En outre,
% insister sur le fait que +inf et -inf ne sont pas des nombres et ne devraient
% jamais être utilisés dans des phrases du type ``Le maximum est +inf,
% d'antécédent 2.''.

\section{Définitions}
\subsection{Fonctions, images, antécédents}
\begin{defi}{Fonction}
  Définir une \textbf{fonction} $f$ sur un ensemble de réels $D$ consiste à
  associer à chaque réel $x\in D$ un unique réel $y$.\\
  Pour signifier que $y$ est le réel associé à $x$ par la fonction $f$, on note
  $y=f(x)$. On note cette correspondance comme suit.
  \[
    \begin{array}{cccc}
      f: & D & \to & \mathbb{R} \\
   & x & \mapsto & f(x)
 \end{array}
 \]
\end{defi}
\begin{defi}{Ensemble de définition, image et antécédent}
  \begin{itemize}
    \item L'ensemble $D$ est appelé l'\textbf{ensemble de définition} de $f$ : il s'agit
      donc de l'ensemble des réels $x$ ayant un réel $y$ associé par $f$.
    \item On dit que $y$ est l'\textbf{image} de $x$ par $f$.
    \item On dit que $x$ est un \textbf{antécédent} de $y$ par $f$.
  \end{itemize}
\end{defi}
\begin{rmq}
  On peut nommer la fonction par une autre lettre que $f$ ($g$, $u$, $v$, etc.). De même, on peut
  remplacer la variable $x$ par une autre ($t$, $\ell$, etc.)
\end{rmq}
\begin{exemple}
  % TODO: changer cet exemple, il est un peu trop abstrait sans doute
  On observe la température d'une pièce pendant $24$ heures. À chaque instant
  $t$ de la journée correspond donc une température unique, notée $f(t)$. La
  fonction $f$ est donc définie sur $D=\left[ 0;24 \right]$.
  \begin{itemize}
    \item S'il fait $12^\circ$C au bout d'une heure, on notera $f(1)=12$.
    \item S'il fait $18,7^\circ$C au bout de six heures, on
      notera $f(6)=18,7$.
    \item Écrire $f(4)=f(18)$ veut dire que la température était identique au
      bout de $4$ heures et au bout de $18$ heures.
  \end{itemize}
  Les températures atteintes plusieurs fois dans la journée ont plusieurs
  antécédents : la température $20^\circ$C peut par exemple être atteinte
  plusieurs fois dans la journée).
\end{exemple}
\begin{app}
  Un élève entre un nombre à la calculatrice puis il appuie sur la touche $\left[ x^2
  \right]$.
  \begin{enumerate}
    \item Justifier que cela revient à définir une fonction $f$ que l'on
      précisera.
    \item Par cette fonction $f$, déterminer
      \begin{enumerate}
        \item l'image de $0$ et ses éventuels antécédents ;
        \item l'image de $2$ et ses éventuels antécédents ;
        \item l'image de $-1$ et ses éventuels antécédents.
      \end{enumerate}
  \end{enumerate}
\end{app}

\subsection{Définir une fonction}

\begin{defi}{Définition d'une fonction}
  \begin{minipage}{.6\textwidth}
  Il y a trois principaux modes de définition d'une fonction $f$ permettant
  d'associer à un réel $x\in D$, où $D$ est l'ensemble de définition, son image
  $y$.
  \begin{enumerate}
    \item Avec une \textbf{relation algébrique} : on connaît directement
      l'expression de $f(x)$ en fonction de $x$. Par exemple, la fonction $f$
      définie sur $\mathbb{R}$ par
      \[
        f(x)=x+1.
      \]
    \item Avec un \textbf{tableau de valeurs} : on donne explicitement les
      images associées à différentes valeurs de $x$. Par exemple, ici,
      $f(2)=3$, $f(-1)=0$.
    \item Avec une \textbf{courbe} : la courbe représentative d'une fonction $f$ est
  l'ensemble des points de coordonnées $(x;y)$ tels que $y=f(x)$.
  \end{enumerate}
  \end{minipage}
    \begin{minipage}{.4\textwidth}
      \[
      \begin{array}[]{ccccc}
        \toprule
        \mathbf{x} & -1 & 2 & 5 & 10 \\
        \midrule
        \mathbf{f(x)} & 0 & 3 & 6 & 11\\
        \bottomrule
      \end{array}
    \]
    \begin{center}
      \begin{tikzpicture}%[scale=.5]
        \begin{axis}%[x=.25cm, y=.25cm]
          \addplot[red, very thick, samples=101]{x+1};
        \end{axis}
        \draw[very thick, dashed, blue] (3.75, 2.75) -- (3.75, 4.25) -- (2.75,
        4.25);
        \node[thick, blue] at (4, 3) {\Large $x$};
        \node[thick, blue] at (1.4, 4.25) {\Large $y=f(x)$};
      \end{tikzpicture}
    \end{center}
  \end{minipage}
\end{defi}
\begin{app}
  On considère la fonction $f$ définie sur $D=\left[ -2;7 \right]$ par
  $f(x)=6x-x^2$.
  \begin{enumerate}
    \item Recopier et compléter le tableau de valeurs suivant.
    \[
      \arraycolsep=15pt
      \begin{array}[]{cccccccccc}
        \toprule
        \mathbf{x} & -1 & 0 & 1 & 2 & 3 & 4 & 5 & 6 & 7 \\
        \midrule
        \mathbf{f(x)} & & & & & & & & & \\
        \bottomrule
      \end{array}
    \]
    \item Utiliser ce tableau pour tracer la courbe représentative de $f$.
  \end{enumerate}
\end{app}

%\subsection{Notion de parité}
%
%Soit $f$ une fonction définie sur un intervlle $I$ centré en $0$. On note
%$\mathscr C_f$ sa courbe représentative dans un repère orthogonal.
%\begin{defi}{Parité}
%  On dit que $f$ est
%  \begin{itemize}
%    \item \textbf{paire} lorsque, pour tout $x\in I$, $f(-x)=f(x)$ ;
%    \item \textbf{impaire} lorsque, pour tout $x\in I$, $f(-x)=-f(x)$.
%  \end{itemize}
%\end{defi}
%\begin{prop}
%  \begin{itemize}
%    \item La fonction $f$ est paire si, et seulement si, $\mathscr C_f$ est
%      symétrique par rapport à l'axe des ordonnées.
%    \item La fonction $f$ est impaire si, et seulement si, $\mathscr C_f$ est
%      symétrique par rapport à l'origine du repère.
%  \end{itemize}
%\end{prop}

\section{Résolution graphique d'équations}
\noindent On note respectivement $\Cf$ et $\Cg$ les courbes de $f$ et $g$ dans
un repère orthogonal.

\subsection{Résolution graphique du type $f(x)=k$}
\begin{defi}{Résolution d'équation du type $f(x)=k$}
  Soient $f$ une fonction définie sur un ensemble $D$ et $k$ un réel fixé.
  Résoudre l'équation $f(x)=k$:
  \begin{itemize}
    \item consiste à déterminer tous les réels $x$ de $D$ qui ont pour image $k$;
    \item revient donc à déterminer l'ensemble des antécédents de $k$ par $f$.
  \end{itemize}
\end{defi}
\begin{prop}
  Graphiquement, les solutions de $f(x)=k$ sont les abscisses de tous les points
  de $\Cf$ ayant pour ordonnée $k$.
\end{prop}
\begin{exemple}
  \begin{minipage}{.6\textwidth}
    On considère la représentation graphique d'une fonction $f$ définie sur
    $\left[ -2;1 \right]$.
    \begin{itemize}
      \item L'équation $f(x)=-2$ admet une solution ($x=-1$).
      \item L'équation $f(x)=-1$ admet deux solutions ($x=-2$ et $x=0$).
      \item L'équation $f(x)=2$ admet une solution $(x=1)$.
    \end{itemize}
  \end{minipage}
  \begin{minipage}{.4\textwidth}
    \begin{center}
      \begin{tikzpicture}
        \begin{axis}[x=1cm, y=1cm, xmin=-2.5, xmax=1.5, ymin=-2.5, ymax=2.5]
          \addplot[red, very thick, samples=101, domain=-2:1]{x^2+2*x-1};
        \end{axis}
        \draw[very thick, green!50!black, loosely dashed] (0, 0.5) -- (4, .5);
        \draw[very thick, green!50!black, loosely dashed] (1.5, .5) -- (1.5, 2.5);

        \draw[very thick, green!50!black, loosely dashed] (0, 4.5) -- (4, 4.5);
        \draw[very thick, green!50!black, loosely dashed] (3.5, 2.5) -- (3.5, 4.5);

        \draw[very thick, blue!50!black, loosely dashed] (0, 1.5) -- (4, 1.5);
        \draw[very thick, blue!50!black, loosely dashed] (.5, 1.5) -- (.5, 2.5);
        \draw[very thick, blue!50!black, loosely dashed] (2.5, 1.5) -- (2.5, 2.5);
      \end{tikzpicture}
    \end{center}
  \end{minipage}
\end{exemple}

\begin{app}
  \begin{minipage}{.6\textwidth}
    On considère la représentation graphique ci-contre d'une fonction $f$ définie sur
    $\left[ -1;2 \right]$.
    \begin{enumerate}
      \item Résoudre graphiquement (approximativement) l'équation $f(x)=1$.
      \item Résoudre graphiquement l'équation $f(x)=0$.
      \item Résoudre graphiquement l'équation $f(x)=-1$.
    \end{enumerate}
  \end{minipage}
  \begin{minipage}{.4\textwidth}
    \begin{center}
      \begin{tikzpicture}
        \begin{axis}[x=1cm, y=1cm, xmin=-1.5, xmax=2.5, ymin=-2.5, ymax=2.5]
          \addplot[red, very thick, samples=101, domain=-1:2]{x^3-2*x^2-x+2};
        \end{axis}
      \end{tikzpicture}
    \end{center}
  \end{minipage}
\end{app}

\subsection{Résolution graphique du type $f(x)=g(x)$}
\begin{defi}{Résolution d'équation du type $f(x)=g(x)$}
  Soient $f$ et $g$ deux fonctions définies sur un ensemble $D$. Résoudre
  l'équation
  \[
    f(x)=g(x)
  \]
  consiste à déterminer tous les réels $x$ de $D$ qui ont la même image par $f$
  et par $g$.
\end{defi}

\begin{prop}
  Graphiquement, les solutions de
  \[
    f(x)=g(x)
  \]
  sont les abscisses des points d'intersection des courbes représentatives de
  $f$ et de $g$.
\end{prop}

\begin{exemple}
    \begin{minipage}{.5\textwidth}
    On considère les deux représentations graphiques dans le repère orthogonal
    ci-contre.\\
    Ces courbes ont exactement trois intersections : $A$, $B$ et $C$ d'abscisses
    respectives $-1,5$; $0$ et $2$.\\
    L'ensemble des solutions de l'équation $f(x)=g(x)$ est donc $\mathscr
    S=\left\{ 1,5;0;2 \right\}$.
  \end{minipage}
  \begin{minipage}{.5\textwidth}
    \begin{center}
      \begin{tikzpicture}
        \begin{axis}[x=1cm, y=.5cm, xmin=-2.5, xmax=2.5, ymin=-6.5, ymax=3.5,
          xtick={-2.5, -2, ..., 3.5}, ytick={-7, -6, ..., 4}]
          \addplot[red, very thick, samples=101]{2*x^3-6*x-2};
          \addplot[green!50!black, very thick, samples=101]{x^2-2};
        \end{axis}
        \draw[very thick] (1-.1, 3.375) -- (1+.1, 3.375) (1, 3.375-.1) -- (1,
        3.375+.1);
        \draw[very thick] (2.5-.1, 2.25) -- (2.5+.1, 2.25) (2.5, 2.25-.1) --
        (2.5, 2.25+.1);
        \draw[very thick] (4.5-.1, 4.25) -- (4.5+.1, 4.25) (4.5, 4.25-.1) --
        (4.5,4.25+.1);
        \node at (1.3, 3.5) {$A$};
        \node at (2.7, 2.5) {$B$};
        \node at (4.2, 4.5) {$C$};
      \end{tikzpicture}
    \end{center}
  \end{minipage}
\end{exemple}

\begin{app}
    \begin{minipage}{.5\textwidth}
    On considère les deux représentations graphiques de deux fonctions $f$ et
    $g$ dans le repère orthogonal ci-contre.\\
    Résoudre graphiquement l'équation
    \[
      f(x)=g(x).
    \]
  \end{minipage}
  \begin{minipage}{.5\textwidth}
    \begin{center}
      \begin{tikzpicture}
        \begin{axis}[x=1cm, y=1cm, xmin=-2.5, xmax=4.5, ymin=-.5, ymax=4.5]
          %xtick={-2.5, -2, ..., 3.5}, ytick={-7, -6, ..., 4}]
          \addplot[red, very thick, samples=101]{0.2*(x^3-x-x^2+1)};
          \addplot[green!50!black, very thick, samples=101]{0.2*(2*x^2-2)};
        \end{axis}
      \end{tikzpicture}
    \end{center}
  \end{minipage}
\end{app}

\section{Sens de variation}
\subsection{Extremum}
  \noindent\begin{minipage}{.67\textwidth}
    \begin{defi}{Maximum}
      Le \textbf{maximum} d'une fonction $f$ définie sur un intervalle $I$ est,
      s'il existe, la plus grande valeur possible des images, atteinte pour un
      réel $a$ de $I$. Ainsi, pour tout réel $x$ de $I$, on a
      \[
        f(x)\leq f(a)
      \]
    \end{defi}
    \begin{defi}{Minimum}
      Le \textbf{minimum} d'une fonction $f$ définie sur un intervalle $I$ est,
      s'il existe, la plus petite valeur possible des images, atteinte pour un
      réel $b$ de $I$. Ainsi, pour tout réel $x$ de $I$, on a
      \[
        f(x)\geq f(b)
      \]
    \end{defi}
  \end{minipage}
  \begin{minipage}{.33\textwidth}
    \begin{center}
      \begin{tikzpicture}
        \begin{axis}[x=1cm, y=1cm, xmin=-1.5, xmax=2.5, ymin=-2.5, ymax=2.5]
          \addplot[green!50!black, very thick, samples=101, domain=-1:2]{x^3-2*x^2-x+2};
          \addplot[red, very thick, samples=101, dashed]{2.112};
          \addplot[blue, very thick, samples=101, dashed]{-0.631};
       \end{axis}
       \node[blue] at (2.5, 1.65) {Minimum};
       \node[red] at (2.5, 4.85) {Maximum};
       \draw[red, dashed, very thick] (1.285, 2.5) -- (1.285, 4.613);
       \node[red] at (1.285, 2.3) {$a$};
       \node[red] at (4.4, 4.613) {$f(a)$};
       \draw[blue, dashed, very thick] (3.049, 2.5) -- (3.049, 1.869);
       \node[blue] at (3.049, 2.75) {$b$};
       \node[blue] at (4.4, 1.869) {$f(b)$};
      \end{tikzpicture}
    \end{center}
   \end{minipage}
    \begin{app}
  \begin{minipage}{.6\textwidth}
    On représente ci-contre une fonction $f$. Donner le maximum et le minimum de
    $f$.
  \end{minipage}
  \begin{minipage}{.4\textwidth}
    \begin{center}
      \begin{tikzpicture}[scale=.7]
        \begin{axis}[x=1cm, y=1cm, xmin=-2.5, xmax=2.5, ymin=-1.5, ymax=1.5]
          \addplot[green!50!black, very thick, samples=101,
          domain=-1.4:1.6]{0.5*x^3-1.5*x};
       \end{axis}
      \end{tikzpicture}
    \end{center}
  \end{minipage}
    \end{app}

\subsection{Sens de variation}
\begin{defi}{Fonction croissante}
  \begin{minipage}{.5\textwidth}
    La fonction $f$ est dite \textbf{croissante} sur l'intervalle $I$ lorsque,
    pour tous réels $x_1$ et $x_2$ tels que $x_1\leq x_2$, alors
    \[
      f(x_1)\leq f(x_2).
    \]
  \end{minipage}
  \begin{minipage}{.5\textwidth}
    \begin{center}
      \begin{tikzpicture}[scale=.75]
        \begin{axis}
          \addplot[red, very thick, samples=101, domain=-5.5:5.5]{ln(x+6)-2+1.4^x};
        \end{axis}
      \end{tikzpicture}
    \end{center}
  \end{minipage}
\end{defi}

\begin{defi}{Fonction décroissante}
  \begin{minipage}{.5\textwidth}
    La fonction $f$ est dite \textbf{décroissante} sur l'intervalle $I$ lorsque,
    pour tous réels $x_1$ et $x_2$ tels que $x_1\leq x_2$, alors
    \[
      f(x_1)\geq f(x_2).
    \]
  \end{minipage}
  \begin{minipage}{.5\textwidth}
    \begin{center}
      \begin{tikzpicture}
        \begin{axis}
          \addplot[red, very thick, samples=101,
          domain=-5.5:5.5]{1.4^(-x)-.01*x^3};
        \end{axis}
      \end{tikzpicture}
    \end{center}
  \end{minipage}
\end{defi}
\begin{defi}{Fonction constante}
  La fonction $f$ est dite \textbf{constante} sur l'intervalle $I$ lorsque, pour
  tous réels $x_1$ et $x_2$ de $I$, on a 
  \[
    f(x_1)=f(x_2).
  \]
\end{defi}

\begin{defi}{Tableau de variation}
  Pour représenter les variations d'une fonction $f$, on utilise un tableau avec
  des flèches représentant les variations sur des intervalles les plus grands
  possibles.

  \begin{itemize}
    \item On utilise une flèche vers le haut pour représenter une fonction
  croissante.
\item On utilise une flèche vers le bas pour représenter une fonction
  décroissante.
\item Si on les connaît, on écrit les images au bout des flèches.
  \end{itemize}
  L'ensemble forme le \textbf{tableau de variations} de $f$.
\end{defi}

\begin{app}
  On donne ci-dessous le tableau de variation d'une fonction $f$.
  \begin{center}
  \begin{tikzpicture}
   \tkzTabInit{$x$ / 1 , $f(x)$ / 2}{$-10$, $-5$, $2$, $4$}
   \tkzTabVar{-/ 0, +/ 2, -/ -1, +/ 1}
\end{tikzpicture}
  \end{center}
  \begin{enumerate}
    \item Donner l'ensemble de définition de $f$.
    \item Donner $f(2)$.
    \item \begin{enumerate}
        \item Le maximum de $f$ sur $[-10;4]$ est $\dots\dots\dots$. Il est
          atteint en $x=\dots\dots$.
        \item Le maximum de $f$ sur $[2;4]$ est $\dots\dots\dots$. Il est
          atteint en $x=\dots\dots$.
        \item Le minimum de $f$ sur $[-10;4]$ est $\dots\dots\dots$. Il est
          atteint en $x=\dots\dots$.
      \end{enumerate}
    \item Comparer les nombres suivants (dire lequel est le plus grand). Justifier.
      \begin{enumerate}
        \item Les nombres $f(-7)$ et $f(-5)$.
        \item Les nombres $f(-4)$ et $f(0)$.
        \item Les nombres $f(-1)$ et $f(3)$.
      \end{enumerate}
    \item 
      \begin{enumerate}
        \item Donner un encadrement de $f(x)$ pour $x\in[-5;2]$.
        \item Donner un encadrement de $f(x)$ pour $x\in[-10;2]$.
      \end{enumerate}
    \item Dessiner la courbe représentative d'une fonction $f$ qui pourrait
      avoir ce tableau de variations.
  \end{enumerate}
\end{app}



% Sens de variation (lecture graphique)
% Tableau de variation

\end{document}
