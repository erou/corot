%%%%%%%%%%%%%%%%%%%%%%%%%%%%%%%%%%%%%%%%%%%%%%%%%%%%%%%%%%%%
%%  This Beamer template was created by Cameron Bracken.
%%  Anyone can freely use or modify it for any purpose
%%  without attribution.
%%
%%  Last Modified: January 9, 2009
%%

%\documentclass[20pt,xcolor=x11names,compress, aspectratio=43]{beamer}
\documentclass[17pt,xcolor=x11names]{beamer}
% change to aspectratio=169 to obtain 16:9 ratio (standard on many computers)

\usepackage[utf8]{inputenc}
\usepackage[T1]{fontenc}
\usepackage[french]{babel}
\usepackage{xcolor}

\useoutertheme[subsection=false,shadow]{miniframes}
\useinnertheme{default}
\usefonttheme{serif}
\usepackage{palatino}

\setbeamerfont{title like}{shape=\scshape}
\setbeamerfont{frametitle}{shape=\scshape}
\setbeamertemplate{navigation symbols}{}
\setbeamertemplate{footline}[frame number]

\setbeamercolor*{lower separation line head}{bg=DeepSkyBlue4} 
\setbeamercolor*{page number in head/foot}{fg=gray} 

% \setbeamercolor*{normal text}{fg=black,bg=white} 
% \setbeamercolor*{alerted text}{fg=red} 
% \setbeamercolor*{example text}{fg=black} 
% \setbeamercolor*{structure}{fg=black} 
  
\setbeamercolor*{palette tertiary}{fg=black,bg=black!10} 
%\setbeamercolor*{palette quaternary}{fg=black,bg=black!10} 

\definecolor{mygreen}{rgb}{0.20,0.43,0.09}
\newcommand{\bib}[2]{\textcolor{blue}{[#1 '#2]}}
\newcommand{\fvb}[1]{\textcolor{violet}{\textbf{#1}}}
\newcommand{\frb}[1]{\textcolor{red}{\mathbf{#1}}}
\newcommand{\comp}[1]{\textcolor{purple}{$O(#1)$}}
\newcommand{\softcomp}[1]{\textcolor{purple}{$\tilde O(#1)$}}
\newcommand{\good}{\textcolor{mygreen}{\smiley{}}}
\newcommand{\bad}{\textcolor{red}{\frownie{}}}
\newcommand{\openquestion}{\includegraphics[scale=.03]{../logos/open-question.png}\fvb{Open
question: }}

% In order to have a slide dedicated to announcing the section
%\AtBeginSection[]
%{
%    \begin{frame}
%        \frametitle{Table of Contents}
%        \tableofcontents[currentsection]
%    \end{frame}
%}

\AtBeginSection[]{
  \begin{frame}
  \vfill
  \centering
  \begin{beamercolorbox}[sep=8pt,center,shadow=false,rounded=true]{title}
    \usebeamerfont{title}\secname\par%
  \end{beamercolorbox}
  \vfill
  \end{frame}
}

% So that alerted text is bold
% \setbeamerfont{alerted text}{series=\bfseries}


\begin{document}
\begin{frame}
  \title{Réunion parents d'élèves\\S11}
  \author{Édouard \textsc{Rousseau}}
\date{23 Septembre}
\titlepage
\end{frame}

\section*{Numérique}
\begin{frame}{ENT}
  \begin{itemize}
    \item \fvb{Très important :} l'Espace Numérique de Travail (ENT)
      \begin{itemize}
        \item \url{https://ent.iledefrance.fr/}
        \item identifiants à conserver
        \item mails
        \item \fvb{Pronote} : notes, cahier de texte
      \end{itemize}
  \end{itemize}
\end{frame}

\begin{frame}{Site de classe}
  \begin{center}
    \small
    \url{https://erou.forge.aeif.fr/s11/}
  \end{center}
  \begin{itemize}
    \item Informations administratives
      \begin{itemize}
        \item Ce diaporama
      \end{itemize}
    \item Ressources pour l'orientation
    \item Ressources pédagogiques
  \end{itemize}
\end{frame}

\section*{Équipe pédagogique}
\begin{frame}{Équipe pédagogique 1/3}
  \begin{itemize}
    \item M. \textsc{Rousseau} : \fvb{professeur principal}, mathématiques, SNT
    \item Mme \textsc{Elédut} : Conseillère Principale d'Éducation (CPE)
    \item Mme \textsc{Molinier} : français
  \end{itemize}
\end{frame}
\begin{frame}{Équipe pédagogique 2/3}
  \begin{itemize}
    \item M. \textsc{Lannelongue} : physique-chimie
    \item Mme \textsc{Barbier} : Sciences de la Vie \& de la Terre (SVT)
    \item Mme \textsc{Le Guel} : Sciences Économiques et Sociales (SES)
  \end{itemize}
\end{frame}
\begin{frame}{Équipe pédagogique 3/3}
  \begin{itemize}
    \item M. \textsc{Giovanni} : histoire-géographie
    \item M. \textsc{Zarforoushan} : anglais
    \item Mme \textsc{Gadaud} : espagnol
    \item M. \textsc{Planes} : Éducation Physique et Sportive (EPS)
  \end{itemize}
\end{frame}

\section*{Orientation}
\begin{frame}{Année importante}
  \begin{itemize}
    \item Trois voies possibles
      \begin{itemize}
        \item Voie générale
        \item Voie technologique
        \item Voie professionnelle
      \end{itemize}
    \item Vœux au $2^e$ conseil de classe (Fév.)
    \item À réfléchir avant
    \item Pas de redoublement
  \end{itemize}
\end{frame}

\begin{frame}{CIO}
  \begin{itemize}
    \item Centre d'Information et d'Orientation
    \item 01 69 44 53 21
    \item
      \href{mailto:cio-savigny@ac-versailles.fr}{cio-savigny@ac-versailles.fr}
  \end{itemize}
\end{frame}

\begin{frame}{Voie générale}
  \begin{itemize}
    \item Choix de $3$ spécialités
      \begin{itemize}
        \item $13$ spécialités possibles à Corot
        \item Disciplines favorites / bonnes notes
        \item Cohérence avec le projet pro.
      \end{itemize}
    \item Un seul diplôme (baccalauréat général)
    \item Études longues (DUT, master, médecine, ...)
  \end{itemize}
\end{frame}

\begin{frame}{Voie technologique}
  \begin{itemize}
    \item Domaines d'activité
    \item Sélection à l'entrée
      \begin{itemize}
        \item Même pour Corot (STMG ou S2TMD)
        \item Faire plusieurs vœux
      \end{itemize}
    \item Études (un peu moins) longues (BTS, DUT, ...)
    \item $8$ séries (STI2D, STD2A, STMG, ST2S, STL, S2TMD, STHR, STAV)
  \end{itemize}
\end{frame}

\begin{frame}{Voie professionnelle}
  \begin{itemize}
    \item Encore plus précis
    \item Plus rare après une Seconde GT
    \item Études plus courtes
  \end{itemize}
\end{frame}

\section{Mathématiques}
\begin{frame}{Matériel}
  \begin{itemize}
    \item Un cahier de cours $24\times32$
    \item Un cahier d'exercices $24\times32$
    \item Manuel : format numérique
    \item Calculatrice TI $82$ Edition Python
      \begin{itemize}
        \item Commande groupée en ligne (à passer sous peu)
        \item $51,90$ €
      \end{itemize}
  \end{itemize}
\end{frame}

\begin{frame}{Organisation du travail}
  \begin{itemize}
    \item $4$ heures par semaines
      \begin{itemize}
        \item $1$ heure en demi-groupe
      \end{itemize}
    \item Devoirs écrits sur Pronote
  \end{itemize}
\end{frame}

\begin{frame}{Évaluations}
  \begin{itemize}
    \item Interrogations régulières
      \begin{itemize}
        \item $5$ à $30$ minutes
      \end{itemize}
    \item Devoir sur un ou deux chapitres
      \begin{itemize}
        \item $1$ heure
      \end{itemize}
    \item Test de positionnement 25 Sept.
      \begin{itemize}
        \item National, non noté
      \end{itemize}
  \end{itemize}
\end{frame}

\begin{frame}{Programme}
$5$ parties :
  \begin{itemize}
    \item Nombres et calculs
    \item Géométrie
    \item Fonctions
    \item Statistiques et probabilités
    \item Algorithmique et programmation
      \begin{itemize}
        \item SNT
      \end{itemize}
  \end{itemize}
\end{frame}

\section{SNT}
\begin{frame}
« Science Numérique et Technologie »
  \begin{itemize}
    \item Culture numérique
      \begin{itemize}
        \item Internet
        \item Web
        \item Réseaux sociaux
        \item Données
        \item Localisation (GPS) / Cartographie
        \item Objets connectés
        \item Photographie numérique
      \end{itemize}
    \item Algorithmique et programmation
  \end{itemize}
\end{frame}

\end{document}
