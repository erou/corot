\documentclass[11pt]{article}

\newcommand{\titrechapitre}{Information chiffrée -- Exercices}
\newcommand{\titreclasse}{Lycée Jean-Baptiste \textsc{Corot}}
\newcommand{\pagination}{\thepage/\pageref{LastPage}}
\newcommand{\topbotmargins}{2cm}
\newcommand{\spacebelowexo}{4mm}
%%%%%%%%%%%%%%%%%%%%%%%%%%%%%%%%%%%%%%%%%%%%%%%%%%%%%%%%%%%%%%%%%%%%%%%%%%%%%%%%
%
% PACKAGES
% ========
%
%%%%%%%%%%%%%%%%%%%%%%%%%%%%%%%%%%%%%%%%%%%%%%%%%%%%%%%%%%%%%%%%%%%%%%%%%%%%%%%%

\usepackage[english, french]{babel}
\usepackage[utf8]{inputenc}
\usepackage[T1]{fontenc}
\usepackage{graphicx}
\usepackage{amsmath,amssymb,amsthm,amsopn}
\usepackage{hyperref}

% Pour avoir l'écriture \mathscr (math script)
% ============================================

\usepackage{mathrsfs}

% Deal with coma as a decimal separator
% =====================================

\usepackage{icomma}

% Package Geometry
% ================

\usepackage[a4paper, lmargin=2cm, rmargin=2cm, top=\topbotmargins, bottom=\topbotmargins]{geometry}

% Package multicol
% ================

\usepackage{multicol}

% Redefine abstract
% =================

% Note
% ====
%
% Le reste a été commenté pour ne pas charger trop de choses au démarrage. On
% verra si on en a besoin plus tard.
%
% --------
%
%\usepackage{mathrsfs}
%\usepackage{multirow}
%\usepackage{bm}
%\hypersetup{
%    colorlinks=true,
%    linkcolor=blue,
%    citecolor=red,
%}
%\usepackage{diagbox}
%
%\usepackage{algorithm}
%\usepackage{algpseudocode}
%
%\renewcommand{\algorithmicrequire}{\textbf{Input:}}
%\renewcommand{\algorithmicensure}{\textbf{Output:}}


%%%%%%%%%%%%%%%%%%%%%%%%%%%%%%%%%%%%%%%%%%%%%%%%%%%%%%%%%%%%%%%%%%%%%%%%%%%%%%%%
%
% TIKZ
% ====
%
%%%%%%%%%%%%%%%%%%%%%%%%%%%%%%%%%%%%%%%%%%%%%%%%%%%%%%%%%%%%%%%%%%%%%%%%%%%%%%%%

\usepackage{tikz}
\usetikzlibrary{arrows}

\usepackage{tkz-tab} % Variation tables

\usepackage{pgfplots}
%\usepackage{pgf-pie} % Pie charts

\pgfplotsset{
%\newcommand{\settingsgraph}{
x=.5cm,y=.5cm,
xticklabel style = {font=\scriptsize, yshift=.1cm},
yticklabel style = {font=\scriptsize, xshift=.1cm},
axis lines=middle,
ymajorgrids=true,
xmajorgrids=true,
major grid style = {color=white!80!blue},
xmin=-5.5,
xmax=5.5,
ymin=-5.5,
ymax=5.5,
xtick={-5.0,-4.0,...,5.0},
ytick={-5.0,-4.0,...,5.0},
}

% Tikz style

\tikzset{round/.style={circle, draw=black, very thick, scale = 0.7}}
\tikzset{arrow/.style={->, >=latex}}
\tikzset{dashed-arrow/.style={->, >=latex, dashed}}

\newcommand{\point}[3]{\draw[very thick, #3] (#1-.1, #2)--(#1+.1, #2)
(#1, #2-.1)--(#1, #2+.1)}

%%%%%%%%%%%%%%%%%%%%%%%%%%%%%%%%%%%%%%%%%%%%%%%%%%%%%%%%%%%%%%%%%%%%%%%%%%%%%%%%
%
% FANCY HEADER
% ============
%
%%%%%%%%%%%%%%%%%%%%%%%%%%%%%%%%%%%%%%%%%%%%%%%%%%%%%%%%%%%%%%%%%%%%%%%%%%%%%%%%


\usepackage{fancyhdr}
\usepackage{lastpage}

\pagestyle{fancy}
\newcommand{\changefont}{\fontsize{9}{9}\selectfont}
\renewcommand{\headrulewidth}{0mm}
\renewcommand{\footrulewidth}{0mm}

\fancyhead[C]{}
\fancyhead[L]{\titreclasse}
\fancyhead[R]{\titrechapitre}
\fancyfoot[C]{}
\fancyfoot[L]{}
\fancyfoot[R]{\pagination}
\addtolength{\skip\footins}{20pt} % distance between text and footnotes

%%%%%%%%%%%%%%%%%%%%%%%%%%%%%%%%%%%%%%%%%%%%%%%%%%%%%%%%%%%%%%%%%%%%%%%%%%%%%%%%
%
% THEOREM STYLE
% =============
%
%%%%%%%%%%%%%%%%%%%%%%%%%%%%%%%%%%%%%%%%%%%%%%%%%%%%%%%%%%%%%%%%%%%%%%%%%%%%%%%%

\usepackage[tikz]{bclogo}
\usepackage{mdframed}

\usepackage{tcolorbox}
\tcbuselibrary{listings, breakable, theorems, skins}

%\newtheoremstyle{break}%
%{}{}%
%{\itshape}{}%
%{\bfseries}{}%  % Note that final punctuation is omitted.
%{\newline}{}

\newtheoremstyle{scbf}%
{}{}%
{}{}%
%{\scshape}{}%  % Note that final punctuation is omitted.
{\bfseries\scshape}{}%  % Note that final punctuation is omitted.
{\newline}{}

%\theoremstyle{break}
%\theoremstyle{plain}
%\newtheorem{thm}{Theorem}[section]
%\newtheorem{lm}[thm]{Lemma}
%\newtheorem{prop}[thm]{Proposition}
%\newtheorem{cor}[thm]{Corollary}

%\theoremstyle{scbf}
%\newtheorem{exo}{$\star$ Exercice}

%\theoremstyle{definition}
%\newtheorem{defi}[thm]{Definition}
%\newtheorem{ex}[thm]{Example}

%\theoremstyle{remark}
%\newtheorem{rem}[thm]{Remark}

% Defining the Remark environment
% ===============================

\newenvironment{rmq}
  {
    \begin{bclogo}[logo=\bcinfo, noborder=true]{Remarque}
  }
  {
    \end{bclogo}
  }

% Defining the exercise environment
% =================================

\newcounter{exos}
\setcounter{exos}{1}

\newenvironment{exo}
  {
    \begin{bclogo}[logo=\bccrayon, noborder=true]{Exercice \theexos}
  }
  {
    \end{bclogo}
    \addtocounter{exos}{1}
  }


% Redefining the proof environment from amsthm
% ============================================

\tcolorboxenvironment{proof}{
  blanker, breakable, before skip=10pt,after skip=10pt,
  borderline west={1mm}{0pt}{red},
  left=5mm,
}

% Defining the definition environment
% ===================================

\colorlet{coldef}{black!50!green}

\newcounter{defis}
\setcounter{defis}{1}

\newenvironment{defi}[1]
  {
    \begin{defihid}{{#1}}{\thedefis}
  }
  {
    \end{defihid}
    \addtocounter{defis}{1}
  }

\newtcolorbox{defihid}[2]{%
  empty,title={ {\bfseries Définition {#2}} ({#1})},attach boxed title to top left,
boxed title style={empty,size=minimal,toprule=2pt,top=4pt,
overlay={\draw[coldef,line width=2pt]
([yshift=-1pt]frame.north west)--([yshift=-1pt]frame.north east);}},
coltitle=coldef,
before=\par\medskip\noindent,parbox=false,boxsep=0pt,left=0pt,right=3mm,top=4pt,
breakable,pad at break*=0mm,vfill before first,
overlay unbroken={\draw[coldef,line width=1pt]
([yshift=-1pt]title.north east)--([xshift=-0.5pt,yshift=-1pt]title.north-|frame.east)
--([xshift=-0.5pt]frame.south east)--(frame.south west); },
overlay first={\draw[coldef,line width=1pt]
([yshift=-1pt]title.north east)--([xshift=-0.5pt,yshift=-1pt]title.north-|frame.east)
--([xshift=-0.5pt]frame.south east); },
overlay middle={\draw[coldef,line width=1pt] ([xshift=-0.5pt]frame.north east)
--([xshift=-0.5pt]frame.south east); },
overlay last={\draw[coldef,line width=1pt] ([xshift=-0.5pt]frame.north east)
--([xshift=-0.5pt]frame.south east)--(frame.south west);},%
}

\newenvironment{notation}
  {
    \begin{notationhid}{\thedefis}
  }
  {
    \end{notationhid}
    \addtocounter{defis}{1}
  }

\newtcolorbox{notationhid}[1]{%
  empty,title={Notation {#1}},attach boxed title to top left,
boxed title style={empty,size=minimal,toprule=2pt,top=4pt,
overlay={\draw[coldef,line width=2pt]
([yshift=-1pt]frame.north west)--([yshift=-1pt]frame.north east);}},
coltitle=coldef,fonttitle=\bfseries,
before=\par\medskip\noindent,parbox=false,boxsep=0pt,left=0pt,right=3mm,top=4pt,
breakable,pad at break*=0mm,vfill before first,
overlay unbroken={\draw[coldef,line width=1pt]
([yshift=-1pt]title.north east)--([xshift=-0.5pt,yshift=-1pt]title.north-|frame.east)
--([xshift=-0.5pt]frame.south east)--(frame.south west); },
overlay first={\draw[coldef,line width=1pt]
([yshift=-1pt]title.north east)--([xshift=-0.5pt,yshift=-1pt]title.north-|frame.east)
--([xshift=-0.5pt]frame.south east); },
overlay middle={\draw[coldef,line width=1pt] ([xshift=-0.5pt]frame.north east)
--([xshift=-0.5pt]frame.south east); },
overlay last={\draw[coldef,line width=1pt] ([xshift=-0.5pt]frame.north east)
--([xshift=-0.5pt]frame.south east)--(frame.south west);},%
}


% Defining the proposition, theorem, etc. environment
% ===================================================

\colorlet{colprop}{red!75!black}

\newcounter{props}
\setcounter{props}{1}

\newenvironment{prop}
  {
    \begin{prophid}{\theprops}
  }
  {
    \end{prophid}
    \refstepcounter{props}
  }

\newtcolorbox{prophid}[1]{%
empty,title={Propriété {#1}},attach boxed title to top left,
boxed title style={empty,size=minimal,toprule=2pt,top=4pt,
overlay={\draw[colprop,line width=2pt]
([yshift=-1pt]frame.north west)--([yshift=-1pt]frame.north east);}},
coltitle=colprop,fonttitle=\bfseries,
before=\par\medskip\noindent,parbox=false,boxsep=0pt,left=0pt,right=3mm,top=4pt,
breakable,pad at break*=0mm,vfill before first,
overlay unbroken={\draw[colprop,line width=1pt]
([yshift=-1pt]title.north east)--([xshift=-0.5pt,yshift=-1pt]title.north-|frame.east)
--([xshift=-0.5pt]frame.south east)--(frame.south west); },
overlay first={\draw[colprop,line width=1pt]
([yshift=-1pt]title.north east)--([xshift=-0.5pt,yshift=-1pt]title.north-|frame.east)
--([xshift=-0.5pt]frame.south east); },
overlay middle={\draw[colprop,line width=1pt] ([xshift=-0.5pt]frame.north east)
--([xshift=-0.5pt]frame.south east); },
overlay last={\draw[colprop,line width=1pt] ([xshift=-0.5pt]frame.north east)
--([xshift=-0.5pt]frame.south east)--(frame.south west);},%
}

\newenvironment{propadm}
  {
    \begin{propadmhid}{\theprops}
  }
  {
    \end{propadmhid}
    \refstepcounter{props}
  }

  \newtcolorbox{propadmhid}[1]{%
    empty,title={{\bfseries Propriété {#1}} (admise)},attach boxed title to top left,
boxed title style={empty,size=minimal,toprule=2pt,top=4pt,
overlay={\draw[colprop,line width=2pt]
([yshift=-1pt]frame.north west)--([yshift=-1pt]frame.north east);}},
coltitle=colprop,%fonttitle=\bfseries,
before=\par\medskip\noindent,parbox=false,boxsep=0pt,left=0pt,right=3mm,top=4pt,
breakable,pad at break*=0mm,vfill before first,
overlay unbroken={\draw[colprop,line width=1pt]
([yshift=-1pt]title.north east)--([xshift=-0.5pt,yshift=-1pt]title.north-|frame.east)
--([xshift=-0.5pt]frame.south east)--(frame.south west); },
overlay first={\draw[colprop,line width=1pt]
([yshift=-1pt]title.north east)--([xshift=-0.5pt,yshift=-1pt]title.north-|frame.east)
--([xshift=-0.5pt]frame.south east); },
overlay middle={\draw[colprop,line width=1pt] ([xshift=-0.5pt]frame.north east)
--([xshift=-0.5pt]frame.south east); },
overlay last={\draw[colprop,line width=1pt] ([xshift=-0.5pt]frame.north east)
--([xshift=-0.5pt]frame.south east)--(frame.south west);},%
}

\newenvironment{propnom}[1]
  {
    \begin{propnomhid}{#1}{\theprops}
  }
  {
    \end{propnomhid}
    \refstepcounter{props}
  }

\newtcolorbox{propnomhid}[2]{%
empty,title={{\bfseries Propriété {#2}} ({#1})},attach boxed title to top left,
boxed title style={empty,size=minimal,toprule=2pt,top=4pt,
overlay={\draw[colprop,line width=2pt]
([yshift=-1pt]frame.north west)--([yshift=-1pt]frame.north east);}},
coltitle=colprop,
before=\par\medskip\noindent,parbox=false,boxsep=0pt,left=0pt,right=3mm,top=4pt,
breakable,pad at break*=0mm,vfill before first,
overlay unbroken={\draw[colprop,line width=1pt]
([yshift=-1pt]title.north east)--([xshift=-0.5pt,yshift=-1pt]title.north-|frame.east)
--([xshift=-0.5pt]frame.south east)--(frame.south west); },
overlay first={\draw[colprop,line width=1pt]
([yshift=-1pt]title.north east)--([xshift=-0.5pt,yshift=-1pt]title.north-|frame.east)
--([xshift=-0.5pt]frame.south east); },
overlay middle={\draw[colprop,line width=1pt] ([xshift=-0.5pt]frame.north east)
--([xshift=-0.5pt]frame.south east); },
overlay last={\draw[colprop,line width=1pt] ([xshift=-0.5pt]frame.north east)
--([xshift=-0.5pt]frame.south east)--(frame.south west);},%
}




\newenvironment{thm}
  {
    \begin{thmhid}{\theprops}
  }
  {
    \end{thmhid}
    \refstepcounter{props}
  }

\newtcolorbox{thmhid}[1]{%
empty,title={Théorème {#1}},attach boxed title to top left,
boxed title style={empty,size=minimal,toprule=2pt,top=4pt,
overlay={\draw[colprop,line width=2pt]
([yshift=-1pt]frame.north west)--([yshift=-1pt]frame.north east);}},
coltitle=colprop,fonttitle=\bfseries,
before=\par\medskip\noindent,parbox=false,boxsep=0pt,left=0pt,right=3mm,top=4pt,
breakable,pad at break*=0mm,vfill before first,
overlay unbroken={\draw[colprop,line width=1pt]
([yshift=-1pt]title.north east)--([xshift=-0.5pt,yshift=-1pt]title.north-|frame.east)
--([xshift=-0.5pt]frame.south east)--(frame.south west); },
overlay first={\draw[colprop,line width=1pt]
([yshift=-1pt]title.north east)--([xshift=-0.5pt,yshift=-1pt]title.north-|frame.east)
--([xshift=-0.5pt]frame.south east); },
overlay middle={\draw[colprop,line width=1pt] ([xshift=-0.5pt]frame.north east)
--([xshift=-0.5pt]frame.south east); },
overlay last={\draw[colprop,line width=1pt] ([xshift=-0.5pt]frame.north east)
--([xshift=-0.5pt]frame.south east)--(frame.south west);},%
}

\newenvironment{thmadm}
  {
    \begin{thmadmhid}{\theprops}
  }
  {
    \end{thmadmhid}
    \refstepcounter{props}
  }

  \newtcolorbox{thmadmhid}[1]{%
    empty,title={{\bfseries Théorème {#1}} (admis)},attach boxed title to top left,
boxed title style={empty,size=minimal,toprule=2pt,top=4pt,
overlay={\draw[colprop,line width=2pt]
([yshift=-1pt]frame.north west)--([yshift=-1pt]frame.north east);}},
coltitle=colprop,%fonttitle=\bfseries,
before=\par\medskip\noindent,parbox=false,boxsep=0pt,left=0pt,right=3mm,top=4pt,
breakable,pad at break*=0mm,vfill before first,
overlay unbroken={\draw[colprop,line width=1pt]
([yshift=-1pt]title.north east)--([xshift=-0.5pt,yshift=-1pt]title.north-|frame.east)
--([xshift=-0.5pt]frame.south east)--(frame.south west); },
overlay first={\draw[colprop,line width=1pt]
([yshift=-1pt]title.north east)--([xshift=-0.5pt,yshift=-1pt]title.north-|frame.east)
--([xshift=-0.5pt]frame.south east); },
overlay middle={\draw[colprop,line width=1pt] ([xshift=-0.5pt]frame.north east)
--([xshift=-0.5pt]frame.south east); },
overlay last={\draw[colprop,line width=1pt] ([xshift=-0.5pt]frame.north east)
--([xshift=-0.5pt]frame.south east)--(frame.south west);},%
}

\newenvironment{thmnom}[1]
  {
    \begin{thmnomhid}{#1}{\theprops}
  }
  {
    \end{thmnomhid}
    \refstepcounter{props}
  }

\newtcolorbox{thmnomhid}[2]{%
empty,title={{\bfseries Théorème {#2}} ({#1})},attach boxed title to top left,
boxed title style={empty,size=minimal,toprule=2pt,top=4pt,
overlay={\draw[colprop,line width=2pt]
([yshift=-1pt]frame.north west)--([yshift=-1pt]frame.north east);}},
coltitle=colprop,
before=\par\medskip\noindent,parbox=false,boxsep=0pt,left=0pt,right=3mm,top=4pt,
breakable,pad at break*=0mm,vfill before first,
overlay unbroken={\draw[colprop,line width=1pt]
([yshift=-1pt]title.north east)--([xshift=-0.5pt,yshift=-1pt]title.north-|frame.east)
--([xshift=-0.5pt]frame.south east)--(frame.south west); },
overlay first={\draw[colprop,line width=1pt]
([yshift=-1pt]title.north east)--([xshift=-0.5pt,yshift=-1pt]title.north-|frame.east)
--([xshift=-0.5pt]frame.south east); },
overlay middle={\draw[colprop,line width=1pt] ([xshift=-0.5pt]frame.north east)
--([xshift=-0.5pt]frame.south east); },
overlay last={\draw[colprop,line width=1pt] ([xshift=-0.5pt]frame.north east)
--([xshift=-0.5pt]frame.south east)--(frame.south west);},%
}

\newenvironment{coro}
  {
    \begin{corohid}{\theprops}
  }
  {
    \end{corohid}
    \refstepcounter{props}
  }

  \newtcolorbox{corohid}[1]{%
  empty,title={Corollaire {#1}},attach boxed title to top left,
boxed title style={empty,size=minimal,toprule=2pt,top=4pt,
overlay={\draw[colprop,line width=2pt]
([yshift=-1pt]frame.north west)--([yshift=-1pt]frame.north east);}},
coltitle=colprop,fonttitle=\bfseries,
before=\par\medskip\noindent,parbox=false,boxsep=0pt,left=0pt,right=3mm,top=4pt,
breakable,pad at break*=0mm,vfill before first,
overlay unbroken={\draw[colprop,line width=1pt]
([yshift=-1pt]title.north east)--([xshift=-0.5pt,yshift=-1pt]title.north-|frame.east)
--([xshift=-0.5pt]frame.south east)--(frame.south west); },
overlay first={\draw[colprop,line width=1pt]
([yshift=-1pt]title.north east)--([xshift=-0.5pt,yshift=-1pt]title.north-|frame.east)
--([xshift=-0.5pt]frame.south east); },
overlay middle={\draw[colprop,line width=1pt] ([xshift=-0.5pt]frame.north east)
--([xshift=-0.5pt]frame.south east); },
overlay last={\draw[colprop,line width=1pt] ([xshift=-0.5pt]frame.north east)
--([xshift=-0.5pt]frame.south east)--(frame.south west);},%
}

\newenvironment{lemme}
  {
    \begin{lemmehid}{\theprops}
  }
  {
    \end{lemmehid}
    \refstepcounter{props}
  }

  \newtcolorbox{lemmehid}[1]{%
  empty,title={Lemme {#1}},attach boxed title to top left,
boxed title style={empty,size=minimal,toprule=2pt,top=4pt,
overlay={\draw[colprop,line width=2pt]
([yshift=-1pt]frame.north west)--([yshift=-1pt]frame.north east);}},
coltitle=colprop,fonttitle=\bfseries,
before=\par\medskip\noindent,parbox=false,boxsep=0pt,left=0pt,right=3mm,top=4pt,
breakable,pad at break*=0mm,vfill before first,
overlay unbroken={\draw[colprop,line width=1pt]
([yshift=-1pt]title.north east)--([xshift=-0.5pt,yshift=-1pt]title.north-|frame.east)
--([xshift=-0.5pt]frame.south east)--(frame.south west); },
overlay first={\draw[colprop,line width=1pt]
([yshift=-1pt]title.north east)--([xshift=-0.5pt,yshift=-1pt]title.north-|frame.east)
--([xshift=-0.5pt]frame.south east); },
overlay middle={\draw[colprop,line width=1pt] ([xshift=-0.5pt]frame.north east)
--([xshift=-0.5pt]frame.south east); },
overlay last={\draw[colprop,line width=1pt] ([xshift=-0.5pt]frame.north east)
--([xshift=-0.5pt]frame.south east)--(frame.south west);},%
}

\colorlet{colexemple}{blue!50!black}
%\newtcolorbox{exemple}{empty, title=Exemple, attach boxed title to top left,
%  boxed title style={empty, size=minimal, toprule=2pt, top=4pt,
%    overlay={\draw[colexemple,line width=2pt]
%([yshift=-1pt]frame.north west)--([yshift=-1pt]frame.north east);}},
%coltitle=colexemple,fonttitle=\bfseries,%\large\bfseries,
%before=\par\medskip\noindent,parbox=false,boxsep=0pt,left=0pt,right=3mm,top=4pt,
%overlay={\draw[colexemple,line width=1pt]
%([yshift=-1pt]title.north east)--([xshift=-0.5pt,yshift=-1pt]title.north-|frame.east)
%--([xshift=-0.5pt]frame.south east)--(frame.south west); },
%}

\newcounter{exemples}
\setcounter{exemples}{1}

\newenvironment{exemple}
  {
    \begin{exemplehid}{\theexemples}
  }
  {
    \end{exemplehid}
    \addtocounter{exemples}{1}
  }

\newtcolorbox{exemplehid}[1]{%
empty,title={Exemple {#1}},attach boxed title to top left,
boxed title style={empty,size=minimal,toprule=2pt,top=4pt,
overlay={\draw[colexemple,line width=2pt]
([yshift=-1pt]frame.north west)--([yshift=-1pt]frame.north east);}},
coltitle=colexemple,fonttitle=\bfseries,
before=\par\medskip\noindent,parbox=false,boxsep=0pt,left=0pt,right=3mm,top=4pt,
breakable,pad at break*=0mm,vfill before first,
overlay unbroken={\draw[colexemple,line width=1pt]
([yshift=-1pt]title.north east)--([xshift=-0.5pt,yshift=-1pt]title.north-|frame.east)
--([xshift=-0.5pt]frame.south east)--(frame.south west); },
overlay first={\draw[colexemple,line width=1pt]
([yshift=-1pt]title.north east)--([xshift=-0.5pt,yshift=-1pt]title.north-|frame.east)
--([xshift=-0.5pt]frame.south east); },
overlay middle={\draw[colexemple,line width=1pt] ([xshift=-0.5pt]frame.north east)
--([xshift=-0.5pt]frame.south east); },
overlay last={\draw[colexemple,line width=1pt] ([xshift=-0.5pt]frame.north east)
--([xshift=-0.5pt]frame.south east)--(frame.south west);},%
}

\newenvironment{contrex}
  {
    \begin{contrexhid}{\theexemples}
  }
  {
    \end{contrexhid}
    \addtocounter{exemples}{1}
  }

\newtcolorbox{contrexhid}[1]{%
empty,title={Contre-exemple {#1}},attach boxed title to top left,
boxed title style={empty,size=minimal,toprule=2pt,top=4pt,
overlay={\draw[colexemple,line width=2pt]
([yshift=-1pt]frame.north west)--([yshift=-1pt]frame.north east);}},
coltitle=colexemple,fonttitle=\bfseries,
before=\par\medskip\noindent,parbox=false,boxsep=0pt,left=0pt,right=3mm,top=4pt,
breakable,pad at break*=0mm,vfill before first,
overlay unbroken={\draw[colexemple,line width=1pt]
([yshift=-1pt]title.north east)--([xshift=-0.5pt,yshift=-1pt]title.north-|frame.east)
--([xshift=-0.5pt]frame.south east)--(frame.south west); },
overlay first={\draw[colexemple,line width=1pt]
([yshift=-1pt]title.north east)--([xshift=-0.5pt,yshift=-1pt]title.north-|frame.east)
--([xshift=-0.5pt]frame.south east); },
overlay middle={\draw[colexemple,line width=1pt] ([xshift=-0.5pt]frame.north east)
--([xshift=-0.5pt]frame.south east); },
overlay last={\draw[colexemple,line width=1pt] ([xshift=-0.5pt]frame.north east)
--([xshift=-0.5pt]frame.south east)--(frame.south west);},%
}

\newenvironment{app}
  {
    \begin{apphid}{\theexemples}
  }
  {
    \end{apphid}
    \addtocounter{exemples}{1}
  }

\newtcolorbox{apphid}[1]{%
empty,title={Application {#1}},attach boxed title to top left,
boxed title style={empty,size=minimal,toprule=2pt,top=4pt,
overlay={\draw[colexemple,line width=2pt]
([yshift=-1pt]frame.north west)--([yshift=-1pt]frame.north east);}},
coltitle=colexemple,fonttitle=\bfseries,
before=\par\medskip\noindent,parbox=false,boxsep=0pt,left=0pt,right=3mm,top=4pt,
breakable,pad at break*=0mm,vfill before first,
overlay unbroken={\draw[colexemple,line width=1pt]
([yshift=-1pt]title.north east)--([xshift=-0.5pt,yshift=-1pt]title.north-|frame.east)
--([xshift=-0.5pt]frame.south east)--(frame.south west); },
overlay first={\draw[colexemple,line width=1pt]
([yshift=-1pt]title.north east)--([xshift=-0.5pt,yshift=-1pt]title.north-|frame.east)
--([xshift=-0.5pt]frame.south east); },
overlay middle={\draw[colexemple,line width=1pt] ([xshift=-0.5pt]frame.north east)
--([xshift=-0.5pt]frame.south east); },
overlay last={\draw[colexemple,line width=1pt] ([xshift=-0.5pt]frame.north east)
--([xshift=-0.5pt]frame.south east)--(frame.south west);},%
}

%%%%%%%%%%%%%%%%%%%%%%%%%%%%%%%%%%%%%%%%%%%%%%%%%%%%%%%%%%%%%%%%%%%%%%%%%%%%%%%%
%
% ENUMERATE
% =========
%
%%%%%%%%%%%%%%%%%%%%%%%%%%%%%%%%%%%%%%%%%%%%%%%%%%%%%%%%%%%%%%%%%%%%%%%%%%%%%%%%

\usepackage{enumerate}
\usepackage{enumitem}

% To have special enumerate items like
%
% 1/
% 2/
% 3/

%%%%%%%%%%%%%%%%%%%%%%%%%%%%%%%%%%%%%%%%%%%%%%%%%%%%%%%%%%%%%%%%%%%%%%%%%%%%%%%%
%
% ARRAYS
% ======
%
%%%%%%%%%%%%%%%%%%%%%%%%%%%%%%%%%%%%%%%%%%%%%%%%%%%%%%%%%%%%%%%%%%%%%%%%%%%%%%%%


\usepackage{array}
\usepackage{makecell} % Used to break lines within arrays
\usepackage{multirow}
\usepackage{booktabs} % Used to have nice arrays with headrules

%%%%%%%%%%%%%%%%%%%%%%%%%%%%%%%%%%%%%%%%%%%%%%%%%%%%%%%%%%%%%%%%%%%%%%%%%%%%%%%%
%
% WRITE CODE
% ==========
%
%%%%%%%%%%%%%%%%%%%%%%%%%%%%%%%%%%%%%%%%%%%%%%%%%%%%%%%%%%%%%%%%%%%%%%%%%%%%%%%%

\usepackage{listings}
\usepackage{xcolor}

%New colors defined below
\definecolor{codegreen}{rgb}{0,0.6,0}
\definecolor{codegray}{rgb}{0.5,0.5,0.5}
\definecolor{codepurple}{rgb}{0.58,0,0.82}
\definecolor{backcolour}{rgb}{0.95,0.95,0.92}

%Code listing style named "mystyle"
\lstdefinestyle{python}{
  %backgroundcolor=\color{backcolour},
  commentstyle=\color{codegreen},
  keywordstyle=\color{magenta},
  numberstyle=\tiny\color{codegray},
  stringstyle=\color{codepurple},
  basicstyle=\ttfamily\footnotesize,
  breakatwhitespace=false,
  breaklines=true,
  captionpos=b,
  keepspaces=true,
  numbers=left,
  numbersep=5pt,
  showspaces=false,
  showstringspaces=false,
  showtabs=false,
  tabsize=2
}

\lstset{style=python}

%%%%%%%%%%%%%%%%%%%%%%%%%%%%%%%%%%%%%%%%%%%%%%%%%%%%%%%%%%%%%%%%%%%%%%%%%%%%%%%%
%
% Tabular 
% =======
%
%%%%%%%%%%%%%%%%%%%%%%%%%%%%%%%%%%%%%%%%%%%%%%%%%%%%%%%%%%%%%%%%%%%%%%%%%%%%%%%%

% In order to obtain a tabular with given width.

\usepackage{tabularx}
\newcolumntype{Y}{>{\centering\arraybackslash}X}
\newcolumntype{R}{>{\raggedright\arraybackslash}X}
\newcolumntype{L}{>{\raggedleft\arraybackslash}X}
% \usepackage{tabulary} % younger brother

%%%%%%%%%%%%%%%%%%%%%%%%%%%%%%%%%%%%%%%%%%%%%%%%%%%%%%%%%%%%%%%%%%%%%%%%%%%%%%%%
%
% MACROS
% ======
%
%%%%%%%%%%%%%%%%%%%%%%%%%%%%%%%%%%%%%%%%%%%%%%%%%%%%%%%%%%%%%%%%%%%%%%%%%%%%%%%%

% Math Operators

\DeclareMathOperator{\Card}{Card}
\DeclareMathOperator{\Gal}{Gal}
\DeclareMathOperator{\Id}{Id}
\DeclareMathOperator{\Img}{Im}
\DeclareMathOperator{\Ker}{Ker}
\DeclareMathOperator{\Minpoly}{Minpoly}
\DeclareMathOperator{\Mod}{mod}
\DeclareMathOperator{\Ord}{Ord}
\DeclareMathOperator{\ppcm}{ppcm}
\DeclareMathOperator{\pgcd}{pgcd}
\DeclareMathOperator{\tr}{Tr}
\DeclareMathOperator{\Vect}{Vect}
\DeclareMathOperator{\Span}{Span}
\DeclareMathOperator{\rank}{rank}
\DeclareMathOperator{\rg}{rg}
\DeclareMathOperator{\ev}{ev}
\DeclareMathOperator{\Var}{Var}

% Shortcuts

\newcommand{\eg}{\emph{e.g. }}
\newcommand{\ent}[2]{[\![#1,#2]\!]}
\newcommand{\ie}{\emph{i.e. }}
\newcommand{\ps}[2]{\left\langle#1,#2\right\rangle}
\newcommand{\eqdef}{\overset{\text{def}}{=}}
\newcommand{\E}{\mathcal{E}}
\newcommand{\M}{\mathcal{M}}
\newcommand{\A}{\mathcal{A}}
\newcommand{\B}{\mathcal{B}}
\newcommand{\R}{\mathcal{R}}
\newcommand{\D}{\mathcal{D}}
\newcommand{\Pcal}{\mathcal{P}}
\newcommand{\K}{\mathbf{k}}
\newcommand{\vect}[1]{\overrightarrow{#1}}



\begin{document}

\begin{exo}~
 \begin{enumerate}
  \item Le directeur d’un conservatoire étudie le profil des 500 étudiants
    inscrits. Il a remarqué que $320$ pratiquaient le piano. Déterminer la
    proportion d’étudiants pratiquant le piano parmi tous les étudiants de ce
    conservatoire.
  \item Un maraicher finit de remplir sa camionnette de fruits et de légumes. Il
    a remarqué que sur les $70$ cageots chargés, $49$ comportaient des fruits.
    Déterminer la proportion de cageots de fruits parmi l’ensemble des cageots.
  \item L’Assemblée nationale élue en juin 2017 comportait $224$ femmes députées
    sur les $577$ élus. Déterminer la proportion de femmes députées, sous forme
    de fraction, puis sous forme de pourcentage, en arrondissant à $0,01\%$ près.
  \item Un paquet de pâtes de $500$ g contient $60\%$ de pâtes de
    couleur. Déterminer la masse de pâtes de couleur.
\end{enumerate}
\end{exo}

\begin{exo}
En 2018, le projet de budget de la France prévoyait
$42,55$ milliards de dépense concernant le ministère de la Défense, ce qui
représentait $6\%$ des dépenses du budget. Quel est le montant des dépenses
total prévu par le projet de budget 2018 ?\hfill\emph{(Source : Le Monde.)}
\end{exo}

\begin{exo}
En 2015, en France $3,7$ millions de personnes prenaient
un traitement médicamenteux pour leur diabète (soit $5,4\%$ de la population).
Il existe 2 types de diabète; l’insulinorésistance (diabète de type 2) touche
plus de $90\%$ des diabétiques.
\begin{enumerate}
  \item À combien de millions de personnes pouvait-on estimer la population
    française en 2015 ?
  \item Quelle proportion de la population les diabétiques insulinorésistants
    (diabète de type 2) représentaient-ils ?
\end{enumerate}
\end{exo}

\begin{exo}~
\begin{enumerate}
  \item La carte d’un restaurant est composée pour moitié de plats. Parmi eux,
    $20\%$ sont végétariens. Déterminer la proportion de plats végétariens dans la carte de ce restaurant
  \item $80\%$ des ventes d’un concessionnaire sont des utilitaires. Parmi
    ceux-ci, $35\%$ sont de couleur blanche. Déterminer la proportion d’utilitaires blancs parmi les ventes de ce concessionnaire.
  \item Dans une classe, $45\%$ des élèves sont des garçons. Parmi eux, $10\%$
    portent des chemises régulièrement. Déterminer la proportion de garçons
    portant des chemises régulièrement dans l’ensemble de la classe.
\end{enumerate}
\end{exo}

\begin{exo}
Dans un club de sport, il y a $750$ adhérents, $150$
sont inscrits dans un cours de tennis. Les autres font un sport nautique. Parmi
ces derniers, $15\%$ font du ski nautique, $45\%$ font de la planche à voile et le reste fait du catamaran.
\begin{enumerate}
  \item Parmi les adhérents, quel est le pourcentage de ceux qui pratiquent le tennis ?
  \item Combien d’adhérents font du ski nautique ? Du catamaran ?
  \item Quel pourcentage des adhérents du club représentent-ils ?
\end{enumerate}
\end{exo}

\begin{exo}~\\
\begin{minipage}{.5\textwidth}
  Lors d'une enquête portant sur les $2000$ salariés d'une entreprise, on a
  obtenu les informations suivantes :
\begin{itemize}
  \item $30$\% des salariés ont $40$ ans ou plus;
  \item $40$\% des salariés de plus de $40$ ans sont des cadres;
  \item $25$\% des salariés de moins de $40$ ans sont des cadres;
\end{itemize}
\end{minipage}
\begin{minipage}{.5\textwidth}
  \begin{center}
    \def\arraystretch{1.75}
 \begin{tabular}{|c|c|c|c|}
   \hline
  & \textbf{$<40$ ans} & \textbf{$\geq40$ ans} & \textbf{Total} \\
   \hline
  \textbf{Cadres} &  &  &  \\
   \hline
  \textbf{Non cadres} &  &  &  \\
   \hline
  \textbf{Total} &  & & $2000$ \\
   \hline
\end{tabular}
  \end{center}
\end{minipage}
Compléter le tableau ci-contre.
\end{exo}

\begin{exo}
À l’issue du conseil de classe du troisième trimestre, le professeur principal
d’une classe de Seconde, qui compte $32$ élèves, fait le bilan des orientations de ses élèves pour l’année de première. Il a noté que :
\begin{itemize}
  \item parmi les $20$ filles de la classe, une se dirige vers la voie
    professionnelle alors qu’un quart d’entre elles poursuivront leurs études en
    classe de première technologique ;
  \item la moitié des garçons iront en première générale ;
  \item il y a deux fois plus de garçons que de filles qui iront en première
    professionnelle.
\end{itemize}
  \begin{center}
    \def\arraystretch{1.75}
 \begin{tabular}{|c|c|c|c|c|}
   \hline
   & \textbf{$1^\text{ière}$ générale} & \textbf{$1^\text{ière}$ techno} &
   \textbf{$1^\text{ière}$ pro} &\textbf{Total} \\
   \hline
  \textbf{Filles} &  &  &  & \\
   \hline
   \textbf{Garçons} &  &  & &  \\
   \hline
  \textbf{Total} &  & & & \\
   \hline
\end{tabular}
\end{center}
\begin{enumerate}
  \item Recopier et compléter le tableau ci-dessus, qui indique l’orientation des
    élèves de cette classe selon les voies avec les informations de l’énoncé.
  \item Déterminer la proportion de filles dans la classe.
  \item Déterminer la proportion d’élèves se dirigeant vers la voie
    technologique dans la classe.
  \item Déterminer la proportion de filles se dirigeant vers la voie
    technologique dans la classe.
  \item Les affirmations sont-elles vraies ou fausses ? Argumenter.
    \begin{enumerate}
      \item Plus de trois quarts des filles iront dans la voie générale.
      \item Un tiers des garçons se dirigent vers la voie technologique.
      \item Il y a plus de garçons que de filles qui se dirigent vers la voie
        technologique.
    \end{enumerate}
\end{enumerate}
\end{exo}

\begin{exo}
Le taux horaire brut du SMIC (en euros) est passé de $9,76$ en $2017$ à $9,88$
en 2018 \emph{(source : Insee)}. Déterminer l'évolution en pourcentage du SMIC
entre 2017 et 2018. Arrondir le résultat à $0,1\%$ près.
\end{exo}

\begin{exo}
Un élève a eu $12,3$ de moyenne générale au premier
trimestre, puis $13,5$ au deuxième et $10,4$ au troisième.
\begin{enumerate}
  \item \begin{enumerate}
      \item Déterminer la variation absolue de sa moyenne générale entre le
        premier et le deuxième trimestre.
      \item Déterminer la variation relative (évolution en pourcentage) de sa
        moyenne générale entre le premier et le deuxième trimestre.
    \end{enumerate}
      \item Déterminer la variation relative (évolution en pourcentage) de sa
        moyenne générale entre le deuxième et le troisième trimestre.
\end{enumerate}
\end{exo}

\begin{exo}
Le tableau suivant le PIB du Brésil et des États-Unis en 2000 et en 2010 (en
milliards de dollars).\hfill\emph{(Source : banque mondiale.)}\\
\begin{minipage}{.7\textwidth}
  \begin{enumerate}
    \item Déterminer la variation absolue du PIB entre 2000 et 2010 pour chacun
      de ces pays.
    \item Déterminer les taux d'évolution (en pourcentage) correspondant.
    \item Quel PIB a progressé le plus rapidement entre ces deux pays en
      pourcentage ?
  \end{enumerate}
\end{minipage}
\begin{minipage}{.3\textwidth}
  \begin{tabular}{ccc}
    \toprule
    & \textbf{2000} & \textbf{2010} \\
    \midrule
    \textbf{Brésil} & $ 655$ & $2\;209$ \\
    \textbf{États-Unis} & $10\;285$ & $14\;964$\\
    \bottomrule
  \end{tabular}
\end{minipage}
\end{exo}

\begin{exo}
Déterminer les coeffcients multiplicateurs associés aux évolutions suivantes.
\begin{align*}
  \textbf{a)}\;& \text{hausse de } 30\% &
  \textbf{b)}\;& \text{baisse de } 10\% &
  \textbf{c)}\;& \text{hausse de } 45\% \\
  \textbf{d)}\;& \text{hausse de } 2,3\% &
  \textbf{e)}\;& \text{baisse de } 0,7\% &
  \textbf{f)}\;& \text{hausse de } 100\%
\end{align*}
\end{exo}

\begin{exo}
Déterminer les taux d'évolutions en pourcentage associés aux coefficients
multiplicateurs suivants.
\begin{align*}
  \textbf{a)}\;& CM=1,2 &
  \textbf{b)}\;& CM = 0,89 &
  \textbf{c)}\;& CM = 1,03 \\
  \textbf{d)}\;& CM = 2,5 &
  \textbf{e)}\;& CM = 0,81 &
  \textbf{f)}\;& CM = 1,33
\end{align*}
\end{exo}

\begin{exo}
On considère la fonction Python suivante.
\begin{lstlisting}[language=Python]
def mafonction(t):
    c = 1+t
    return(c)
\end{lstlisting}
\begin{enumerate}
  \item Que renvoie cet algorithme si on donne $\frac{35}{100}$ en entrée ?
  \item À quoi cet algorithme peut-il servir ?
  \item Que doit-on donner en entrée pour qu'il renvoie $1,071$ ?
  \item \begin{enumerate}
      \item Que renvoie l'algorithme si on donne $-1,2$ en entrée ?
      \item Modifier l'algorithme pour qu'il affiche un message d'erreur si on
        donne une proportion inférieure à $-1$ en entrée.
    \end{enumerate}
\end{enumerate}
\end{exo}

\begin{exo}
Le cours d'une action s'écroule en bourse. Après avoir baissé de $20\%$ le
lundi, voilà que son action baisse à nouveau de $30\%$ le mardi. Déterminer
l'évolution globale subie par le cours de cette action.
\end{exo}

\begin{exo}
Compléter les phrases suivantes.
\begin{enumerate}
  \item Une hausse de $6\%$ suivie d'une hausse de $\dots\dots\%$ correspondent
    à une hausse de $21,9\%$.
  \item Trois baisses de $30\%$ correspondent à une baisse de $\dots\dots\%$.
  \item Une baisse de $\dots\dots\%$ suivie d'une baisse de $12\%$ correspondent
    à une baisse de $34\%$.
\end{enumerate}
\end{exo}

\begin{exo}
Un élève fournit un travail acharné pour améliorer ses
résultats. Quand il reçoit sa copie de SVT avec la note de $18$ il s'exclame «
Tout ce travail pour une hausse de seulement $12,5\%$ !~» Déterminer sa note
précédente.
\end{exo}

\begin{exo}
Déterminer l'évolution réciproque de chacun de ces taux
d'évolutions. Arrondir à $0,01\%$.
\begin{align*}
  \textbf{a)}\;& t = 24\% &
  \textbf{b)}\;& t = -7\% &
  \textbf{c)}\;& t = 0,056\% &
  \textbf{d)}\;& t = -45\%
\end{align*}
\end{exo}

\begin{exo}
Une femme vit seule dans un appartement.
\begin{enumerate}
  \item \begin{enumerate}
      \item En 2018, le loyer de son appartement s'élevait à $500$ euros. Il
        représente $40\%$ de son salaire. Déterminer le montant de son salaire.
      \item Le reste des charges représente $8\%$ de son salaire. Déterminer le
        montant du reste des charges.
    \end{enumerate}
  \item Sa cheffe lui accorde une augmentation de $100$ euros. Déterminer
    l'évolution en pourcentage que cela représente.
  \item Son loyer augmente de $2\%$ chaque année.
    \begin{enumerate}
      \item Déterminer le montant de son loyer en $2019$.
      \item Déterminer en quelle année l'augmenttion de son loyer absorbera son
        augmentation de salaire.
    \end{enumerate}
\end{enumerate}
\end{exo}

\begin{exo}
Deux magasins situés l'un en face de l'autre se
livrent une concurrence acharnée. Chacun propose le kilo de pommes à $2$ euros,
mais avec des offres différentes.
\begin{itemize}
  \item Le premier annonce : « Pour deux kilos minimum de pommes achetés, $10\%$
    de réduction immédiate !~»
  \item Le second propose : « Pour deux kilos minimum de pommes achetés, $10\%$
    de produit en plus !~»
\end{itemize}
Les offres sont-elles équivalentes ?
\end{exo}

\end{document}
