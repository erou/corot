%%%%%%%%%%%%%%%%%%%%%%%%%%%%%%%%%%%%%%%%%%%%%%%%%%%%%%%%%%%%
%%  This Beamer template was created by Cameron Bracken.
%%  Anyone can freely use or modify it for any purpose
%%  without attribution.
%%
%%  Last Modified: January 9, 2009
%%

%\documentclass[20pt,xcolor=x11names,compress, aspectratio=43]{beamer}
\documentclass[17pt,xcolor=x11names]{beamer}
% change to aspectratio=169 to obtain 16:9 ratio (standard on many computers)

\usepackage[utf8]{inputenc}
\usepackage[T1]{fontenc}
\usepackage[french]{babel}

%\usepackage[english]{babel}
\usepackage[utf8]{inputenc}
\usepackage[T1]{fontenc}
\usepackage{graphicx}
\usepackage{amsmath,amssymb,amsthm,amsopn}
\usepackage{hyperref}

% Pour avoir l'écriture \mathscr (math script)
\usepackage{mathrsfs}

% Pour avoir le 1 des indicatrices
%\usepackage{bbold}

% Note
% ====
%
% Le reste a été commenté pour ne pas charger trop de choses au démarrage. On
% verra si on en a besoin plus tard.
%
% --------
%
%\usepackage{mathrsfs}
%\usepackage{graphicx}
%\usepackage{array}
%\usepackage{multirow}
%\usepackage{makecell}
%\usepackage{bm}
%\hypersetup{
%    colorlinks=true,
%    linkcolor=blue,
%    citecolor=red,
%}
%\usepackage{diagbox}
%
%\usepackage{algorithm}
%\usepackage{algpseudocode}
%
%\renewcommand{\algorithmicrequire}{\textbf{Input:}}
%\renewcommand{\algorithmicensure}{\textbf{Output:}}

%\usepackage[top=1cm,bottom=1cm]{geometry}
%\usepackage{listings}
%\usepackage{xcolor}

%\input{tikz.tex}
%\input{thmstyle.tex}
%% Math Operators

\DeclareMathOperator{\Card}{Card}
\DeclareMathOperator{\Gal}{Gal}
\DeclareMathOperator{\Id}{Id}
\DeclareMathOperator{\Img}{Im}
\DeclareMathOperator{\Ker}{Ker}
\DeclareMathOperator{\minpoly}{minpoly}
\DeclareMathOperator{\Mod}{mod}
\DeclareMathOperator{\Ord}{Ord}
\DeclareMathOperator{\ppcm}{ppcm}
\DeclareMathOperator{\lcm}{lcm}
\DeclareMathOperator{\Tr}{Tr}
\DeclareMathOperator{\Vect}{Vect}

% Shortcuts

\newcommand{\dE}{\partial(E)}
\newcommand{\dF}{\partial(F)}
\newcommand{\dG}{\partial(G)}
\newcommand{\diff}{\mathop{}\!\mathrm{d}}
\newcommand{\eg}{\emph{e.g. }}
\newcommand{\emb}{\hookrightarrow}
\newcommand{\embed}[2]{\phi_{#1\hookrightarrow#2}}
\newcommand{\ent}[2]{[\![#1,#2]\!]}
\newcommand{\ie}{\emph{i.e. }}
\newcommand{\ps}[2]{\left\langle#1,#2\right\rangle}
\newcommand{\eqdef}{\overset{\text{def}}{=}}
\newcommand{\first}[2]{\left\lfloor #1 \right\rfloor_{#2}}
\newcommand{\embedalg}[2]{\Phi_{A_{#1}\hookrightarrow A_{#2}}}
\newcommand{\A}{\mathcal{A}}
\newcommand{\G}{\mathcal{G}}
\newcommand{\Gsym}{\G_{\text{sym}}}
\newcommand{\B}{\mathrm{B}}

\newcommand{\mutri}{\mu^{\textrm{tri}}}

\def\norm {\ensuremath{\mathcal{N}}}

% trick to make comments
\newcommand{\comment}[1]{}

\useoutertheme[subsection=false,shadow]{miniframes}
\useinnertheme{default}
\usefonttheme{serif}
\usepackage{palatino}

\setbeamerfont{title like}{shape=\scshape}
\setbeamerfont{frametitle}{shape=\scshape}
\setbeamertemplate{navigation symbols}{}
\setbeamertemplate{footline}[frame number]

\setbeamercolor*{lower separation line head}{bg=DeepSkyBlue4} 
\setbeamercolor*{page number in head/foot}{fg=gray} 

% \setbeamercolor*{normal text}{fg=black,bg=white} 
% \setbeamercolor*{alerted text}{fg=red} 
% \setbeamercolor*{example text}{fg=black} 
% \setbeamercolor*{structure}{fg=black} 
  
\setbeamercolor*{palette tertiary}{fg=black,bg=black!10} 
%\setbeamercolor*{palette quaternary}{fg=black,bg=black!10} 

\definecolor{mygreen}{rgb}{0.20,0.43,0.09}
\newcommand{\bib}[2]{\textcolor{blue}{[#1 '#2]}}
\newcommand{\fvb}[1]{\textcolor{violet}{\textbf{#1}}}
\newcommand{\frb}[1]{\textcolor{red}{\mathbf{#1}}}
\newcommand{\comp}[1]{\textcolor{purple}{$O(#1)$}}
\newcommand{\softcomp}[1]{\textcolor{purple}{$\tilde O(#1)$}}
\newcommand{\good}{\textcolor{mygreen}{\smiley{}}}
\newcommand{\bad}{\textcolor{red}{\frownie{}}}
\newcommand{\openquestion}{\includegraphics[scale=.03]{../logos/open-question.png}\fvb{Open
question: }}

% In order to have a slide dedicated to announcing the section
%\AtBeginSection[]
%{
%    \begin{frame}
%        \frametitle{Table of Contents}
%        \tableofcontents[currentsection]
%    \end{frame}
%}

\AtBeginSection[]{
  \begin{frame}
  \vfill
  \centering
  \begin{beamercolorbox}[sep=8pt,center,shadow=false,rounded=true]{title}
    \usebeamerfont{title}\secname\par%
  \end{beamercolorbox}
  \vfill
  \end{frame}
}

% So that alerted text is bold
% \setbeamerfont{alerted text}{series=\bfseries}

\usepackage{multicol}
\usepackage{enumerate}
\usepackage{enumitem}


\begin{document}
\begin{frame}
  \title{Évaluation 1}
  \author{Calcul mental}
\date{22 Novembre}
\titlepage
\end{frame}

\section{Calcul mental 1}

\begin{frame}
  \begin{center}
    Calculer
    \[
      \frac{3}{4}+\frac{1}{6}
    \]
  \end{center}~\\
  \begin{multicols}{2}
    \begin{enumerate}[label=(\Alph*)]
      \item $\frac{11}{12}$
      \item $\frac{17}{12}$
      \item $\frac{13}{12}$
      \item $\frac{9}{12}$
    \end{enumerate}
  \end{multicols}
\end{frame}

\begin{frame}
  \begin{center}
    Calculer
    \[
      (\sqrt3-1)^2
    \]
  \end{center}~\\
  \begin{multicols}{2}
    \begin{enumerate}[label=(\Alph*)]
      \item $3-2\sqrt3$
      \item $3+2\sqrt3$
      \item $4+2\sqrt3$
      \item $4-2\sqrt3$
    \end{enumerate}
  \end{multicols}
\end{frame}

\begin{frame}
  \begin{center}
    Développer
    \[
      (3-2x)^2
    \]
  \end{center}~\\
  \begin{multicols}{2}
    \begin{enumerate}[label=(\Alph*)]
      \item $9-12x+4x^2$
      \item $9-6x+4x^2$
      \item $9+6x+2x^2$
      \item $4-2x-3x^2$
    \end{enumerate}
  \end{multicols}
\end{frame}

\begin{frame}
  \begin{center}
    Soit $f$ la fonction définie sur $\mathbb{R}$ par
    \[
      f(x)=-x^2+2x-1.
    \]
    Combien vaut $f(4)$ ?
  \end{center}~\\
  \begin{multicols}{2}
    \begin{enumerate}[label=(\Alph*)]
      \item $7$
      \item $2$
      \item $-9$
      \item $-1$
    \end{enumerate}
  \end{multicols}
\end{frame}

\begin{frame}
  \begin{center}
    Factoriser
    \[
      4x^2-4x+1.
    \]
  \end{center}~\\
  \begin{multicols}{2}
    \begin{enumerate}[label=(\Alph*)]
      \item $(x-2)^2$
      \item $(2x-1)^2$
      \item $2(x-1)^2$
      \item $(\frac{x}{2}-2)^2$
    \end{enumerate}
  \end{multicols}
\end{frame}

\begin{frame}
  \begin{center}
    Simplifier 
    \[
      \sqrt{98}
    \]
  \end{center}~\\
  \begin{multicols}{2}
    \begin{enumerate}[label=(\Alph*)]
      \item $2\sqrt{49}$
      \item $2\sqrt7$
      \item $7\sqrt2$
      \item $49\sqrt{2}$
    \end{enumerate}
  \end{multicols}
\end{frame}

\begin{frame}
  \begin{center}
    Calculer
    \[
      30\text{\% de }690
    \]
  \end{center}~\\
  \begin{multicols}{2}
    \begin{enumerate}[label=(\Alph*)]
      \item $198$
      \item $207$
      \item $227$
      \item $189$
    \end{enumerate}
  \end{multicols}
\end{frame}

\begin{frame}
  \begin{center}
    Dériver
    \[
      f(x) = -2x^3 + 6x^2 - 12
    \]
  \end{center}~\\
  \begin{multicols}{2}
    \begin{enumerate}[label=(\Alph*)]
      \item $-3x^2+6x$
      \item $-6x^2+12x^2-12$
      \item $-6x^2+12x$
      \item $3x^2+12$
    \end{enumerate}
  \end{multicols}
\end{frame}

\begin{frame}
  \begin{center}
    Calculer
    \[
     \left( 0,8 \right)^2 
    \]
  \end{center}~\\
  \begin{multicols}{2}
    \begin{enumerate}[label=(\Alph*)]
      \item $0,16$
      \item $1,64$
      \item $6,4$
      \item $0,64$
    \end{enumerate}
  \end{multicols}
\end{frame}

\begin{frame}
  \begin{center}
    Calculer
    \[
      4-\left( \frac{1}{4}-1 \right)
    \]
  \end{center}~\\
  \begin{multicols}{2}
    \begin{enumerate}[label=(\Alph*)]
      \item $\frac{17}{4}$
      \item $\frac{19}{4}$
      \item $\frac{21}{4}$
      \item $\frac{23}{4}$
    \end{enumerate}
  \end{multicols}
\end{frame}

\end{document}
